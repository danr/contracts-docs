Our prototype contract checker is called \textbf{Halo}.
It uses GHC to parse, typecheck, and desugar a Haskell program, 
translates it into first order logic (exactly as in Section~\ref{ssect:trans-fol}), and
invokes a FOL theorem prover (Equinox, Z3, Vampire, etc) on the FOL formula.
The desugared Haskell program is expressed in GHC's intermediate language
call Core, an explicitly-typed variant of System F.  It is straightforward
to translate this into our language $\cal L$.

\subsection{Using the tool}

How does the user express contracts? We use HOAS and let the user define
contracts by primitive connectives, which can be extended by the user.
The connectives are defined as a GADT, where @Contract t@ is a 
a contract for a function of type @t@:
\begin{code}
data Contract t where
  (:->) :: Contract a -> (a -> Contract b) ->
           Contract (a -> b)
  Pred  :: (a -> Bool) -> Contract a
  CF    :: Contract a
  (:&:) :: Contract a -> Contract a -> Contract a
\end{code}

The connectives are @:->@ for dependent contract function space, @CF@
for crash-freedom, @Pred@ for predication, and
@:&:@ for conjunction. 
A useful derived connective is non-dependent function space:
\par {\small
\begin{code}
(-->) :: Contract a -> Contract b -> Contract (a->b)
c1 --> c2 = c1 :-> \_ -> c2
\end{code}
} \par
%As one would expect, @:->@ and @-->@ are right-associve.  We can
%create contract combinators that are always satisfied, and never
%satisfied:
%
%\begin{code}
%any :: Contract a
%any = Pred (\ _ -> True)
%
%never :: Contract a
%never = Pred (error "never!")
%\end{code}

A contract is always associated with a function, so we wrap a contract
together with its function in a @Statement@:

\begin{code}
data Statement where
    (:::) :: a -> Contract a -> Statement
\end{code}
For example, in our previous notation we might write the following
contract for @head@
$$
@head@ \in \CF \;\&\; \{@xs@ \mid @not (null xs)@ \} \rightarrow \CF
$$
Here is how we express the contract as a Haskell definition:
\begin{comment}
head (x:xs) = x
head []     = error "empty list"

not True = False    null [] = True
not False = True    null xs = False

f . g = \x -> f (g x)
\end{comment}
\begin{code}
c_head :: Statement
c_head = head ::: CF :&: Pred (not . null) --> CF
\end{code}
Upon saving this file to, say, @Head.hs@ and running @halo Head@, 
@halo@ translates the contract (and the supporting definitions) into
FOL, generates a TPTP file and invokes a theorem prover.
In fact @c_head@ 
is verified by all theorem provers we tried.

\subsection{Practical considerations}

To make the theorem prover work as fast as possible we trim the
theories to include only what is needed to prove a
property. Unnecessary function pointers, data types and definitions
for the current goal are not generated.

When proving a series of contracts, it is natural to do so in dependency order.
For example:
\begin{code}
  reverse :: [a] -> [a]
  reverse [] = []
  reverse (x:xs) = reverse xs ++ [x]

  reverse_c :: Statement
  reverse_c = reverse ::: CF --> CF
\end{code}
To prove this contract we must first prove that $@(++)@ \in \CF \rightarrow \CF$;
then we can prove @reverse@'s contract assuming the one for @(++)@.
At the moment we ask the programmer to specify which auxiliary contracts
are useful, via a second constructor in the @Statement@ type:
\begin{code}
  reverse_c = reverse ::: CF --> CF
              `Using` append_c
\end{code}

\subsection{Recursion}

We prove the contract for a recursive function 
using fixed point induction (Section~\ref{s:induction}). 
For recursive
functions, the tool then gives three files, one without induction, one
for the base case and step case. The typical situation is that the one
without induction does not pass because it lacks the induction
hypothesis, the base case always succeeds because that is what we
established our theory on: we need this to be admissible. Indeed, this
one does not need to be checked but it can serve as a good sanity
check of the tool. The step case can pass or fail, depending on if the
contract really holds, and if the induction hypothesis is strong
enough, and if we assume the right contracts.

One example of a recursive function in the Prelude is @foldr1@.

\begin{code}
foldr1          :: (a -> a -> a) -> [a] -> a
foldr1 f [x]    =  x
foldr1 f (x:xs) =  f x (foldr1 f xs)
foldr1 _ []     =  error "foldr1: empty list"
\end{code}

We can state that if @foldr1@ is applied to a crash free function, and
a non-empty list, then the result should be crash free as a contract:

\begin{code}
c_foldr = foldr1 ::: (CF --> CF --> CF) -->
                 CF :&: Pred (not . null) --> CF
\end{code}

Our tool proves this contract, but only when recursion is used.

\subsection{Dependent contracts}

Here is an example of a contract about @filter@,
where the result depends on an earlier argument, the predicate @p@:

\[
@filter@ \in (@p@ : \CF \rightarrow \CF) \rightarrow
             \CF \rightarrow \CF \;\&\; \{@ys@ \mid @all p ys@ \}
\]
This contract says that under good crash-free circumstances, the
all elements in the resulting list from filtering a list with @p@
satisfies @p@.
In our source-file syntax we use @(:->)@ to bind @p@.
\begin{code}
filter_all :: Statement
filter_all = filter ::: (CF --> CF) :-> \p ->
                        CF --> (CF & Pred (all p))
\end{code}
It is a little confusing since we then get two ``arrows'', one from
@:->@, and one from the @->@ in the lambda.
The contract can then proved by using fixed point induction (and not without).

\subsection{Higher order functions}

Our tool also deals with higher order functions containing
very higher-order functions. Consider this (real) function @withMany@ from the
@GHC@ component @Foreign.Util.Marshal@:
%\footnote{\url{http://hackage.haskell.org/packages/archive/base/latest/doc/html/Foreign-Marshal-Utils.html#v:withMany}}:

\begin{code}
withMany :: (a -> (b -> res) -> res)
         -> [a] -> ([b] -> res) -> res
withMany _       []     f = f []
withMany withFoo (x:xs) f = withFoo x (\x' ->
      withMany withFoo xs (\xs' -> f (x':xs')))
\end{code}

For @withMany@, our tool proves
$$
@withMany@\! \in \! (\CF\! \rightarrow\! (\CF \rightarrow \CF) \rightarrow \CF) \rightarrow \CF \rightarrow (\CF \! \rightarrow \! \CF) \rightarrow \CF
$$

% ------------------------ Omit ----------------------------
\begin{comment}
\subsection{A small case-study about invariants}

We consider a somewhat non-standard way of expressing propositional
logic formulae:

\begin{code}
data Formula = And [Formula]
             | Or  [Formula]
             | Neg (Formula)
             | Implies (Formula) (Formula)
             | Lit Bool
\end{code}

One invariant that we are particularily interested in is that we
should never have two consecutive negations, and that the lists of
@And@ and @Or@ are of length $\ge$ 2. We can express that as an ordinary
Haskell predicate:

\begin{code}
invariant :: Formula -> Bool
invariant f = case f of
  And xs      -> properList xs && all invariant xs
  Or xs       -> properList xs && all invariant xs
  Neg Neg{}   -> False
  Neg x       -> invariant x
  Implies x y -> invariant x && invariant y
  Lit x       -> True

properList :: [a] -> Bool
properList []  = False
properList [_] = False
properList _   = True
\end{code}

Now, we have a recursive function that negates formula:

\begin{code}
neg :: Formula -> Formula
neg (Neg f)         = f
neg (And fs)        = Or (map neg fs)
neg (Or fs)         = And (map neg fs)
neg (Implies f1 f2) = neg f2 `Implies` neg f1
neg (Lit b)         = Lit b
\end{code}

We make a combinator saying what it means to retain a predicate:

\begin{code}
retain :: (a -> Bool) -> Contract (a -> a)
retain p = Pred p :-> \x -> Pred (\r -> p x && p r)
\end{code}

\dr{TODO: explain this. This was DV's brilliant idea but I still don't
  fully understand it} Now, since @neg@ uses @map@, we need to show that
@map@ can retain the invariant. We use @all@, introduced above, for
this:

\begin{code}
map_invariant = map ::: retain invariant -->
                        retain (all invariant)
\end{code}

Explicitly spelling out the definition of @retain@ in the statement
above would be tedious and error-prone, so we see the benefit of being
able to express contracts as a DSL.

We can now express that @neg@ retains the invariant:

\begin{code}
neg_contr = neg ::: retain invariant
  `Using` map_invariant
\end{code}

We use @Using :: Statement -> Statement -> Statement@, another
constructor for @Statement@, which allows us to assume that other
contracts holds, when proving a complicated statement, thus
our Statement data type really looks like this:

\begin{code}
data Statement where
    (:::) :: a -> Contract a -> Statement
    Using :: Statement -> Statement -> Statement
\end{code}

For now, it's the user's responsibility to prove these assumed
contracts (for instance, with our tool!), but one can imagine a more
sophisticated front-end which does this automatically.  Note that
the assumption in @neg_contr@ is necessary. If we remove it, and
use the min-translation, we a finitely counter satisfiable theory.
\end{comment}
% ------------------------ End of omit ----------------------------


% \subsection{Example: shrink}
%
% Recall that @fromJust@ is the partial function @Maybe a -> a@, and consider
% this code:
%
% \begin{code}
% shrink :: (a -> a -> a) -> [Maybe a] -> a
% shrink op []     = error "Empty list!"
% shrink op [x]    = fromJust x
% shrink op (x:xs) = fromJust x `op` shrink op xs
% \end{code}
%
% Is this contract satisfied for it?
% \begin{code}
%     (CF --> CF --> CF) -->
%     (CF :&: Pred nonEmpty :&: Pred (all isJust)) --> CF
% \end{code}

\subsection{Experimental Results}


\newcommand{\timeout}{-}
\newcommand{\tot}{\multicolumn{2}{c}{\timeout}}
\newcommand{\tol}{\multicolumn{2}{c |}{\timeout}}

\begin{figure}

\begin{center}
\begin{unsrestab}

 & \multicolumn{4}{c | }{Equinox}
 & \multicolumn{4}{c | }{Z3}
 & \multicolumn{4}{c  }{Vampire}
 \\

Description
 & \multicolumn{2}{c}{min} & \multicolumn{2}{c |}{w/o}
 & \multicolumn{2}{c}{min} & \multicolumn{2}{c |}{w/o}
 & \multicolumn{2}{c}{min} & \multicolumn{2}{c }{w/o}
  \\

\hline

@ack@ CF          & 1&61  & \tol & 0&04 & 0&02 & 0&47 & 0&14 \\
@any@ morphism    & 10&40 & \tol & 0&07 & 0&04 & \tot & \tot \\
@filter@ @all@    & 3&24  & \tol & 0&05 & 0&04 & \tot & \tot \\
@concatMap@ CF    & 0&87  & \tol & 0&04 & 0&04 & 0&38 & 4&65 \\
@invariant@ CF    & 23&37 & \tol & 0&08 & \tol & \tot & \tot \\
@map@ invariant   & 22&82 & \tol & 0&17 & \tol & \tot & \tot \\
@neg@ invariant   & \tot  & \tol & 0&83 & \tol & \tot & \tot \\
@(++)@ inv. @all@ & \tot  & \tol & 0&08 & \tol & 2&70 & \tot \\
acc @exp@ CF      & 0&54  & \tol & 0&06 & 0&04 & 2&27 & 3&44 \\
@iterate@ CF      & 0&42  & 5&82 & 0&03 & 0&00 & 0&18 & 0&00 \\
@iterTree@ CF     & 1&85  & \tol & 0&03 & 0&00 & 2&90 & 0&01 \\
@repeat@ CF       & 0&10  & 0&05 & 0&03 & 0&00 & 0&01 & 0&00 \\
@risersBy@        & 26&18 & \tol & \tot & \tol & 7&80 & 0&82 \\
@shrink@          & 5&64  & \tol & 0&05 & 0&05 & 2&04 & \tot \\
@withMany@        & 21&51 & \tol & 0&05 & 0&01 & 3&96 & \tot \\
\end{unsrestab}
\end{center}

\caption{
  Theorem prover running time in seconds on some of the problems in the test suite
  on contracts that \emph{do} hold.
  Paradox had only timeouts and is not listed.
  }
  \label{fig:unsres}

\end{figure}

\begin{figure}
\begin{center}
\begin{satrestab}

 & \multicolumn{2}{c | }{Paradox}
 & \multicolumn{2}{c | }{Equinox}
 & \multicolumn{2}{c  }{Z3}
 \\

Description
 & \multicolumn{2}{c |}{min}
 & \multicolumn{2}{c |}{min}
 & \multicolumn{2}{c }{min}
  \\

\hline

@any@ morphism w/o @any@ CF    & 2&52 & \tol  & \tot \\
@concatMap@ CF w/o @(++)@ CF   & 0&09 & \tol  & \tot \\
@concatMap@ invariant weak     & 0&61 & \tol  & \tot \\
@neg@ invariant missing @map@  & 0&63 & \tol  & \tot \\
acc @exp@ CF w/o @(*)@ CF      & 0&15 & \tol  & \tot \\
@risersBy@ wrong precond.      & 7&07 & 9&14  & \tot \\
@risersBy@ wrong postcond.     & 3&14 & 19&70 & \tot \\
@shrink@ too lazy              & \tol & 7&16  & \tot \\
@head@ CF                      & 0&04 & 0&12  & 0&06 \\
@head@ CF with wrong precond.  & 0&37 & 0&44  & \tot \\

\end{satrestab}
\end{center}
\caption{Theorem prover running time in seconds on some of the problems in the test suite
  on contracts that \emph{does not} hold.
  Without using min we only got timeouts so it is not listed.
  Vampire had only timeouts and is not listed.
  }
  \label{fig:satres}
\end{figure}

% Z3 actually kills this :p


In Figure \ref{fig:unsres}, @shrink@ is basically
@foldr1 op . map fromJust@, and @invariant@ is for a data type
for predicate logic terms.

In Figure \ref{fig:satres}, the satisfiable problems (under the line) are
obtained by mutating the correct contract so it fails, i.e.  by
removing a necessary assumption or making the pre-conditions too weak
or post-conditions too strong.


\subsection{Contracts that do not hold}

In practice, programmers will often propose contracts that do not hold.
Unfortunately, in almost all such cases, all FOL provers will loop.
The prover is trying to prove that a theory is unsatisfiable.  A resolution
based prover (such as Vampire or eprover) will keep applying deduction rules, 
trying to find a contradiction, but no such contradiction exists.  A
model-based theorem prover will seek a finite counter-model to the theory, but
again no such counter-model exists. \spj{Koen please fix the wording here!}

It is obviouly unacceptable for the system to go into a loop if
the programmer writes a bogus contract, and we have promising
preliminary results based on so-called ``minimisation'', and 
finite counter-model generators such as Paradox \cite{koen}, but we 
leave this for (absolutely essential) future work.

