The implementation of the contracts checker has two components:

\begin{itemize}
    \item \textbf{Halo}, the Haskell to Logic translator, which takes
       GHC Core as input and translates function and data type
       definitions to first order logic as described in previous sections.

     \item \textbf{hcc}, the Haskell Contracts Checker, the user front
        end. Before giving halo the program to translate to FOL, hcc
        takes contracts, inputed in the Haskell source file as function
        definitions, identified by their type, and represents those in
        its own data type. Then we generate a theory accompanied with a
        conjecture for each contract, so the users can try them with
        their installed first order theorem provers.
\end{itemize}

How does the user input contracts? We use HOAS and let the user define
contracts by primitive connectives, which can be extended by the user.
The connectives are defined as a GADT:

\begin{code}
data Contract t where
  (:->) :: Contract a
        -> (a -> Contract b)
        -> Contract (a -> b)
  Pred  :: (a -> Bool) -> Contract a
  CF    :: Contract a
  (:&:) :: Contract a -> Contract a -> Contract a
\end{code}

The connectives are @:->@ for dependent contract function space, @CF@
for crash-freedom (no catchable bottoms), @Pred@ for predication, and
@:&:@ for conjunction. A notable difference from the notation in
\cite{xu+:contracts}, the user of our library explicitly writes @CF@.
A useful derived connective is non-dependent function space:

\begin{code}
(-->) :: Contract a -> Contract b -> Contract (a -> b)
c1 --> c2 = c1 :-> \_ -> c2
\end{code}

%As one would expect, @:->@ and @-->@ are right-associve.  We can
%create contract combinators that are always satisfied, and never
%satisfied:
%
%\begin{code}
%any :: Contract a
%any = Pred (\ _ -> True)
%
%never :: Contract a
%never = Pred (error "never!")
%\end{code}

A contract is always associated to a function, so we wrap a contract
together with its function to a @Statement@:

\begin{code}
data Statement where
    (:::) :: a -> Contract a -> Statement
\end{code}

This allows us to wrap a function with a contract.  We also see how
GADTs can ensure that our contracts are well-typed.  One intended use
for our tool is to, if you have a function @f@ and it is partial, you
can state under which precoditions it is not partial, and then our
tool proves that it will not crash if you only give it arguments
satisfying these conditions. An example in this fashion about the
@head@ function is from the \cite{xu+:contracts}:

\begin{code}
head (x:xs) = x
head []     = error "empty list"

null [] = True
null xs = False

not True = False
not False = True

f . g = \x -> f (g x)

head_contract =
    head ::: CF :&: Pred (not . null) --> CF
\end{code}

Upon saving this file to, say, @Head.hs@ and running @hcc Head@, a
TPTP file named @Head.head_contract.tptp@ is generated. This can then
be run with your favourite automated theorem prover, and the above
contracts is easily verified by for instance @eprover@.

However, if we remove the precondition of crash-freedom to @head@,
like this:

\begin{code}
head_contract_broken =
    head ::: Pred (not . null) --> CF
\end{code}

a theorem prover capable of giving finite counter-models, like @paradox@,
will give you a counter model, which, upon examination, will tell you
that its input argument is indeed not an empty list, but it is @:@ of
some bad arguments. Note that we do not get a counter example if we
give @hcc@ the debug flag @--no-min@ that turns off the min predicate
introduced earlier. The reason is that if we do not use min, we only
have infinite counter models because we need to have an injective, but
non-surjective function for @(:)@.

\subsection{Recursion}

We prove properties using fixed point induction. For recursive
functions, the tool then gives three files, one without induction, one
for the base case and step case. The typical situation is that the one
without induction does not pass because it lacks the induction
hypothesis, the base case always succeeds because that is what we
established our theory on: we need this to be admissible. Indeed, this
one does not need to be checked but it can serve as a good sanity
check of the tool. The step case can pass or fail, depending on if the
contract really holds, and if the induction hypothesis is strong
enough, and if we assume the right contracts.

\subsection{Higher order functions}

\begin{code}
all :: (a -> Bool) -> [a] -> Bool
all p []     = True
all p (x:xs) = p x && all p xs

filter :: (a -> Bool) -> [a] -> [a]
filter p [] = []
filter p (x:xs) | p x       = x : filter p xs
                | otherwise = filter p xs

filter_all_contr =
    filter ::: (CF --> CF) :-> \p ->
               CF :-> \xs ->
               (CF & Pred (all p))
\end{code}

This says that if filter takes a @p@ that is @CF --> CF@, and a
crash-free @xs@, then @filter p xs@ is @CF@ and satisfies @all p@,
i.e, the following returns @True@ or diverges:
@all p (filter p xs)@.
This contract is easily proved using fixed point induction.

% Note, however, that without induction we get a counter-example.

\subsection{Higher-higher order functions}

Our tool also deals with higher order functions containing
higher-order functions. Consider this function @withMany@ from the
@GHC@ component @Foreign.Util.Marshal@:
%\footnote{\url{http://hackage.haskell.org/packages/archive/base/latest/doc/html/Foreign-Marshal-Utils.html#v:withMany}}:

\begin{code}
-- Replicates a @withXXX@ combinator over a list of
-- objects, yielding a list of marshalled objects
--
withMany :: (a -> (b -> res) -> res)
         -- ^ withXXX combinator for one object
         -> [a]
         -- ^ storable objects
         -> ([b] -> res)
         -- ^ action on list of marshalled obj.s
         -> res
withMany _       []     f = f []
withMany withFoo (x:xs) f = withFoo x (\x2 ->
      withMany withFoo xs (\xs2 -> f (x2:xs2)))

\end{code}

For @withMany@, our tool proves
@(CF --> (CF --> CF) --> CF)@ @--> CF --> (CF --> CF) --> CF@.

\subsection{A small case-study about invariants}

We consider a somewhat non-standard way of expressing propositional
logic formulae:

\begin{code}
data Formula = And [Formula]
             | Or  [Formula]
             | Neg (Formula)
             | Implies (Formula) (Formula)
             | Lit Bool
\end{code}

One invariant that we are particularily interested in is that we
should never have two consecutive negations, and that the lists of
@And@ and @Or@ are of length $\ge$ 2. We can express that as an ordinary
Haskell predicate:

\begin{code}
invariant :: Formula -> Bool
invariant f = case f of
  And xs      -> properList xs && all invariant xs
  Or xs       -> properList xs && all invariant xs
  Neg Neg{}   -> False
  Neg x       -> invariant x
  Implies x y -> invariant x && invariant y
  Lit x       -> True

properList :: [a] -> Bool
properList []  = False
properList [_] = False
properList _   = True
\end{code}

Now, we have a recursive function that negates formula:

\begin{code}
neg :: Formula -> Formula
neg (Neg f)         = f
neg (And fs)        = Or (map neg fs)
neg (Or fs)         = And (map neg fs)
neg (Implies f1 f2) = neg f2 `Implies` neg f1
neg (Lit b)         = Lit b
\end{code}

We make a combinator saying what it means to retain a predicate:

\begin{code}
retain :: (a -> Bool) -> Contract (a -> a)
retain p = Pred p :-> \x -> Pred (\r -> p x && p r)
\end{code}

TODO: explain this. Now, since @neg@ uses @map@, we need to show that
@map@ can retain the invariant. We use @all@, introduced above, for
this:

\begin{code}
map_invariant = map ::: retain invariant
                    --> retain (all invariant)
\end{code}

Explicitly spelling out the definition of @retain@ in the statement
above would be tedious and error-prone, so we see the benefit of being
able to express contracts as a DSL.

Together with the statement @invariant ::: CF --> CF@, we
get that @neg@ retains the invariant thusly:

\begin{code}
neg_contr = neg ::: retain invariant
    `Using` map_invariant
    `Using` invariant_cf
\end{code}

We use @Using :: Statement -> Statement -> Statement@, another
constructor for @Statement@, which allows us to assume that other
contracts holds, when proving a complicated statement. For now, it's
the user's responsibility to prove these assumed contracts (for
instance, with our tool!), but one can imagine a more sophisticated
front-end which does this automatically.  Note that removing either
assumption yields theories that are satisfiable by 4-5 elements.

\subsection{Trimming}

We trim the theories as much as possible to only include exactly what
is needed to prove a property. All unnecessary function pointers, data
types and definitions are removed.

