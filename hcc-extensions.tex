\label{sect:induction}
%% \subsection{Induction}\label{sect:induction}

An important practical extension is the ability to prove contracts about recursive functions
using induction. For instance, we might want to prove that @length@ satisfies $\CF -> \CF$.
\begin{code}
length [] = Z 
length (x:xs) = S (length xs)
\end{code}
In the second case we need to show that the result of @length xs@ is crash-free but we do not 
have this information so the proof gets stuck, often resulting in the sat-solver looping. 

A naive approach would be to perform induction over the list argument of @length@ -- however
in Haskell datatypes may be lazy infinite streams and ordinary induction is not necessarily 
a valid proof principle. Fortunately, we can still appeal to {\em fixpoint induction}. The 
fixpoint induction sheme that we use for @length@ above would be to {\em assume} that the 
contract holds for the occurence some function @length_rec@ inside the body of its definition, 
and then try to prove it for the function:
\begin{code}
length [] = Z 
length (x:xs) = S (length_rec xs)
\end{code}

Formally, our induction scheme is:
\begin{definition}[Induction sheme]\label{def:induction}
To prove that $\dbrace{g} \in \dbrace{\Ct}$ for a function $g\;\as\;\ol{x:\tau} = e[g]$ (meaning $e$ contains
some occurences of $g$) we perform the following steps:
\begin{itemize*}
  \item Generate function symbols $g^{\circ}$, $g^{\bullet}$, and formula $\ctrans{}{}{g^{\circ} \in \Ct}$.
  \item Generate the theory formulas \[ \phi = \Th \land \ptrans{}{P \cup g^{\bullet}\;\as\;\ol{x:\tau}} = e[g^{\circ}] \] 
  \item Prove that the query $\phi \land \ctrans{}{}{g^{\circ} \in \Ct} \land \neg \ctrans{}{}{g^{\bullet} \in \Ct}$
        is unsatisfiable.
\end{itemize*}
%% If it is unsatisfiable then $\dbrace{f} \in \dbrace{\Ct}$.
\end{definition}

Why is this approach sound? The crucial step here is the fact that contracts are admissible predicates.
\begin{theorem}[Contract admissibility]
If $d_i \in \dbrace{\Ct}$ for all elements of a chain $d_1 \sqsubseteq d_2 \sqsubseteq \ldots$ then the limit of the chain 
$\sqcup d_i \in \dbrace{\Ct}$. Moreover, $\bot \in \dbrace{\Ct}$.
\end{theorem}
\begin{proof} By induction on the contract $\Ct$; for the $\CF$ case we get the result from Lemma~\ref{lem:cf-admissible}.
For the predicate case we get the result from the fact that the denotations of programs 
are continuous in $D_{\infty}$. The arrow case follows by induction.
\end{proof}

We can then prove the soundness of our induction scheme.
\begin{theorem} The induction scheme in Definition~\ref{def:induction} is correct. \end{theorem}
\begin{proof} We need to show that: 
$\dbrace{P}^{\infty}(g) \in \dbrace{\Ct}$ and hence, by admissibility it is enough to find
a chain whose limit is $\dbrace{P}^{\infty}(g)$ and such that every element is in $\dbrace{\Ct}$.
Let us consider the chain $\dbrace{P}^{k}(g)$ so that $\dbrace{P}^0(g) = \bot$ and 
$\dbrace{P}^{k+1}(g) = \dbrace{P}_{(\dbrace{P}^{k})}(g)$ whose limit is $\dbrace{P}^{\infty}(g)$. We 
know that $\bot \in \dbrace{\Ct}$ so, by using contract admissiblity, all we need to show is
that if $\dbrace{P}^{k}(g) \in \dbrace{\Ct}$ then $\dbrace{P}^{k+1}(g) \in \dbrace{\Ct}$. 

To show this, we can assume a model where the denotational interpretation ${\cal I}$ has been 
extended so that ${\cal I}(g^{\circ}) = \dbrace{P}^{k}(g)$ and ${\cal I}(g^{\bullet}) = \dbrace{P}^{k+1}(g)$.
By proving that the formula 
\[ \phi \land \ctrans{}{}{g^{\circ} \in \Ct} \land \neg \ctrans{}{}{g^{\bullet} \in \Ct} \] 
is unsatisfiable, since $\langle D_{\infty},{\cal I}\rangle \models \phi$ and 
$\langle D_{\infty},{\cal I}\rangle \models \ctrans{}{}{g^{\circ} \in \Ct}$, we learn 
that $\langle D_{\infty},{\cal I}\rangle \models \ctrans{}{}{g^{\bullet} \in \Ct}$, 
and hence $\dbrace{P}^{k+1}(g) \in \dbrace{\Ct}$.
\end{proof} 

Note that contract admissibility is {\em absolutely essential} for the soundness of our induction scheme, 
and is not a property that holds of every predicate on denotations. For instance consider the following 
Haskell definition:
\begin{code}
ones = 1 : ones
f (S x) = 1 : f x
f Z     = [0]
\end{code}
Let us try to check if the $\forall x @.@ @f@(x) \neq @ones@$ is true in 
the denotational model, using fixpoint induction. The case for $\bot$ holds, 
and so does the case for the @Z@ constructor. For the $@S@\;x$ case, we can 
assume that $@f@(x) \neq @ones@$ and we can easily prove that this implies that
$@f@(@S@\;x) \neq @ones@$. Nevertheless, the property is {\em not true} -- just pick 
a counterexample $\dbrace{@s@}$ where @s = S s@. What happened here is that the property 
is denotationally true of all the elements of the following chain
\[ \bot \sqsubseteq \injK{S}{\bot} \sqsubseteq \injK{S}{\injK{S}{\bot}} \sqsubseteq \ldots \] 
but is false in the limit of this chain. In other words $\leq$ is not admissible and our 
induction scheme is plain nonsense for non-admissible predicates. 

Finally, we have observed that for many practical cases, a straightforward generalization of our lemma above for 
mutually recursive definitions is required. Indeed, our tool performs mutual fixpoint induction when a recursive
group of functions is encountered. We leave it as future work to develop more advanced techniques such as 
strengthening of induction hypotheses or identifying more sophisticated induction schemes. 


