So far we have described the basic translation of the denotational semantics, programs, and contracts
to first-order logic. However, to enable verification of contracts in practice we must consider two 
important extensions, outlined in the rest of this section.

\subsection{Minimization of countermodels}\label{sect:minimization}

For a query of the form $\Th \land \Th_\lcfZ \land \neg \ctrans{}{P}{e \in \Ct}$, a SAT solver will search for
a model. When such a model exists, it will nclude tables for the function symbols in the formula. Notice that functions 
in FOL are total over the domain of the terms in the model. This means that function tables may be {\em infinite} if the 
terms in the model are infinite. Several (very useful!) axioms such as the discrimination axioms \textsc{AxDisjC} may in 
fact force the models to be infinite. For instance consider the following devinitions:
\begin{code}
length [] = Z
length (x:xs) = S (length xs)

isZero Z = True
isZero _ = False
\end{code}
Suppose that we would like to check that 
   \[ @length@ \in \CF -> \{ x \mid @isZero@\;x\} \]
which is a falsifiable contract.  A satisfiability-based checker 
will simply diverge trying to construct a counter model for the negation of the above query; we 
have confirmed that this is indeed the behaviour of several tools (Z3, Equinox, Eprover).
Indeed the table for @length@ is infinite since @[]@ is always disjoint from @Cons x x@ for 
any @x@ and @xs@. Even if there is a finitely representable infinite model there is always the 
possibility of the theorem prover searching in the ``wrong corner'' of the model for a 
counterexample with no success. 

From a practical point of view this is {\em not acceptable}: After all, there exists a very simple 
counterexample that demonstrates the problem, e.g. @[Z]@, and we only need the 
functions of our program to be defined on a {\em finite} number of values (those that appear 
during the evaluation of this problematic counterexample) to be able to demonstrate 
the problem. We simply {\em do not care} about values that a function may take outside 
the set of expressions that appear during the finite evaluation of a counterexample.

This is a challenge that we solve by modifying our axiomatization of the semantics 
and the translation of programs and contracts in a way that it can still admit 
the $\langle D_\infty,{\cal I}\rangle$ model, but also some {\em finite} model in 
the case that a counterexample exists. 

To achieve this effect, we introduce a predicate $min(\cdot)$ that, intuitively, is true
for the terms that have been evaluated during the execution of a counterexample. We use the
name $min$ because the purpose of this predicate is to minimize countermodels. 
We return to give a formal interpretation to $min(\cdot)$ later in this section but this 
intuition should suffice for explaining the modifications to the theory, and the program and 
contract translations. 

\begin{figure} 
{\small
\[\setlength{\arraycolsep}{1pt}
\begin{array}{c}
\ruleform{\Th_\infty^{min}} \\ \\ 
\begin{array}{lll}
 \textsc{AxDisjBU} & \formula{\bad \neq \unr} \\ 
 \textsc{AxDisjC} & \formula{\forall \oln{x}{n}\oln{y}{m} @.@} \\ 
                  & \formula{\;\;\highlight{min(K(\ol{x}))\;\lor\;min(J(\ol{y}))} =>
                                  K(\ol{x}){\neq}J(\ol{y})} \\
                  & \text{ for every } (K{:}\forall\as @.@ \oln{\tau}{n} -> T\;\as) \in \Sigma \\ 
                  & \text{ and } (J{:}\forall\as @.@ \oln{\tau}{m} -> S\;\as) \in \Sigma \\
 %% \textsc{AxDisjCUnr} & \formula{\forall \oln{x}{n} @.@ \highlight{\neg min(\unr)}} \\ 
 %%                  & \text{ for every } (K{:}\forall\as @.@ \oln{\tau}{n} -> T\;\as) \in \Sigma \\ \\
 \textsc{AxDisjCU} & \formula{\forall \oln{x}{n} @.@ \highlight{min(K(\ol{x}))} => K(\ol{x}) \neq unr} \\
 \textsc{AxDisjCB} & \formula{\forall \oln{x}{n} @.@ K(\ol{x}) \neq \bad} \\ 
 %% \textsc{AxDisjCBad} & \formula{\forall \oln{x}{n} @.@ K(\ol{x}) \neq \bad} \\ 
 %%                  & \text{ for every } (K{:}\forall\as @.@ \oln{\tau}{n} -> T\;\as) \in \Sigma \\ \\

 \textsc{AxPtr}  & \formula{\forall \oln{x}{n} @.@ \highlight{min(app(f_{ptr},\xs))} => f(\ol{x}) = app(f_{ptr},\xs)} \\
                 & \text{ for every } (f |-> \Lambda\as @.@ \lambda\oln{x{:}\tau}{n} @.@ u) \in P \\
 %% \textsc{AxAppB}  & \formula{\forall \oln{x}{n} @.@ K(\ol{x}) = app(\ldots (app(x_K,x_1),\ldots,x_n)\ldots)} \\
 %%                  & \text{ for every } (K{:}\forall\as @.@ \oln{\tau}{n} -> T\;\as) \in \Sigma \\
 \textsc{AxApp}  & \formula{\forall x, app(\bad,x) = \bad \; /\ \; app(\unr,x) = \unr}    \\ 
 \textsc{AxAppMin}& \formula{\highlight{\forall x, min(app(x,y)) => min(x)}} \\ 
 %% Not needed: we can always extend partial constructor applications to fully saturated and use AxAppC and AxDisjC
 %% \textsc{AxPartA} & \formula{\forall \oln{x}{n} @.@ app(\ldots (app(x_K,x_1),\ldots,x_n)\ldots) \neq \unr} \\
 %%                  & \formula{\quad\quad \land\; app(\ldots (app(x_K,x_1),\ldots,x_n)\ldots) \neq \bad} \\
 %%                  & \text{ for every } (K{:}\forall\as @.@ \oln{\tau}{m} -> T\;\as) \in \Sigma \text{ and } m > n \\
 %% \textsc{AxPartB} & \formula{\forall \oln{x}{n} @.@ app(f_{ptr},\xs) \neq \unr} \\
 %%                  & \formula{\quad\land\; app(f_{ptr},\xs) \neq \bad} \\
 %%                  & \formula{\quad\land\; \forall \oln{y}{k} @.@ app(f_{ptr},\xs) \neq K(\ol{y})} \\
 %%                  & \text{ for every } (f |-> \Lambda\as @.@ \lambda\oln{x{:}\tau}{m} @.@ u) \in P  \\
 %%                  & \text{ and every } (K{:}\forall\as @.@ \oln{\tau}{k} -> T\;\as) \in \Sigma \text{ and } m > n  \\ \\ 
 \textsc{AxInj}   & \formula{\forall \oln{y}{n} @.@ \highlight{min(K(\ys))}} \\ %% \;\land\; min(y_i)}} \\ 
                  & \formula{\quad\qquad\qquad => \sel{K}{i}(K(\ys)) = y_i} \\ 
                  & \text{for every } (K{:}\forall\as @.@ \oln{\tau}{n} -> T\;\as) \in \Sigma \text{ and } i \in 1..n 
\end{array}
%% \ruleform{\Th_\lcfZ^{min}} \\ \\ 
%% \begin{array}{lll} 
%%  \textsc{AxCfBU}  & \formula{\lcf{\unr} /\ \lncf{\bad}} \\
%%  \textsc{AxCfMin} & \formula{\highlight{\forall x @.@ \lcf{x} => min(x) \lor x = unr}} \\
%%  %% \textsc{AxCfB1}   & \formula{\forall \oln{x}{n} @.@ \bigwedge_i (\lcf{x_i}\lor \neg(min(x_i))} => \lcf{K(\ol{x})} \lor \neg(min(K(\ol{x}))) \\
%%  %%                   & \text{ for every } (K{:}\forall\as @.@ \oln{\tau}{n} -> T\;\as) \in \Sigma \\ 
%%  \textsc{AxCfC1} & \formula{\forall \oln{x}{n} @.@ \bigwedge\lcf{\ol{x}}} => \lcf{K(\ol{x})} \\
%%                  & \text{ for every } (K{:}\forall\as @.@ \oln{\tau}{n} -> T\;\as) \in \Sigma \\
%%  \textsc{AxCfC2} & \formula{\highlight{min(K(\oln{x}{n}))\land\neg\lcf{K(\oln{x}{n})}}} \\ 
%%                  & \formula{\quad\qquad\qquad \highlight{ => \bigvee_i (min(x_i)\land\neg\lcf{x_i})}}
%% \end{array}
\end{array}\]}
\caption{A theory with minimization}\label{fig:min-theory}
\end{figure}

Figure~\ref{fig:min-theory} presents a variation of $\Th_\infty^{min}$ which includes minimization.
In this figure, we have highlighted the parts where our theory differs compared to $\Th_\infty$ from
Figure~\ref{fig:prelude}. The first intersting axiom group is \rulename{AxDisjC} where instead of 
unconditionally asserting that two constructors are disjoint, we assert that they are disjoint if
one of the two values is a member of the $min(\cdot)$ predicate. Intuitively, evaluation of a 
counterexample has touched one of the two constructors (for instance, by performing a pattern matching
on a value of this constructor) and hence we should consider this constructor value different than any 
other one. Similarly \rulename{AxDisjCBU} asserts that a constructor value is not $\unr$ nor $\bad$ if 
it is an interesting value. This leaves the possibility of a model where all elements that are not in 
the $min(\cdot)$ set have been conflated to a single value. In \rulename{AxPtr}, if evaluation of a counterexample
has touched an appliation of $app(f_{ptr},\ol{x})$ then we should be able to derive the equality 
$f(\ol{x}) = app(f_{ptr},\ol{x})$, which will further allow us to gain knowledge about the definition of the function.
The intuition behind \rulename{AxAppMin} is easy: if the term $min(app(x,y))$ has appeared in the evaluation trace, 
we have definitely evaluated the argument and hence we have $min(x)$. Finally, the selector axiom group \rulename{AxInj}
has been modified to guard the constructor value to be in $min(\cdot)$ as one would expect. 


\begin{figure}\small
\[\begin{array}{c}
\ruleform{\utrans{\Sigma}{\Gamma}{t \sim u}^{min} = \formula{\phi}} \\ \\ 
\prooftree
   \begin{array}{c}
   \etrans{\Sigma}{\Gamma}{e} = \formula{t}
   \end{array}
   ----------------------------------------{DExp}
   \begin{array}{l} 
   \utrans{\Sigma}{\Gamma}{s \sim e }^{min} = \formula{\highlight{min(s)} => (s = t)} 
   \end{array}
   ~~~~~
  \begin{array}{l}
  \etrans{\Sigma}{\Gamma}{e} = \formula{t} \\
  %% constrs(\Sigma,T) = \ol{K} \\
  \text{for each branch}\;(K\;\oln{y}{l} -> e') \text{ it is } 
  %% \begin{array}{l}
  %%          (K{:}\forall \cs @.@ \oln{\sigma}{l} -> T\;\oln{c}{k}) \in \Sigma \text{ and }
           \etrans{\Sigma}{\Gamma,\ol{y}}{e'} = \formula{ t_K }
  %% \end{array}
  \end{array}
  ------------------------------------------{DCase}
  {\setlength{\arraycolsep}{1pt} 
  \begin{array}{l}
  \utrans{\Sigma}{\Gamma}{s \sim @case@\;e\;@of@\;\ol{K\;\ol{y} -> e'}}^{min} = \\
  \;\;\formula{ \begin{array}{l} 
     \highlight{min(s)} => \\
     \begin{array}{ll}
          ( & \highlight{min(t)}\;\land \\
            & (t = \bad => s = \bad)\;\land \\ 
            & (\forall \ol{y} @.@ t = K_1(\ol{y}) => s = t_{K_1})\;\land \ldots \land \\
            & (t{\neq}\bad\;\land\;t{\neq}K_1(\oln{{\sel{K_1}{i}}(t)}{})\;\land\;\ldots => s{=}\unr) \\
          )
%% (t = \bad /\ s = \bad)\;\lor\;(s = \unr)\;\lor \\
%%                                 \quad      \bigvee(t = K(\oln{{\sel{K}{i}}(t)}{}) \land
%%                                            s = t_K[\oln{\sel{K}{i}(t)}{}/\ol{y}])
                   \end{array}
     \end{array}}
  \end{array}}
  %% {       \setlength{\arraycolsep}{2pt} 
  %% \begin{array}{l}
  %% \utrans{\Sigma}{\Gamma}{s \sim @case@\;e\;@of@\;\ol{K\;\ol{y}{->}e'}} = \\
  %% \;\;\formula{
  %%      \begin{array}{l} (\highlight{s{=}\unr})\;\lor \\ 
  %%                           \;\; (\highlight{min(s) => min(t)}\;\land  \\
  %%                           \quad((t = \bad /\ s = \bad)\;\lor \\
  %%                           \quad\quad \bigvee(t = K(\oln{{\sel{K}{i}}(t)}{}) \land
  %%                                          s = t_K[\oln{\sel{K}{i}(t)}{}/\ol{y}])))
  %%                  \end{array}
  %%          }
  %% \end{array}}
\endprooftree
\end{array}\]
\caption{Program translation with minimization}\label{fig:min-def-trans-min}
\end{figure}

The translation of programs to accomodate minimization requires only modification 
to the $\utrans{}{\Gamma}{u}$ judgement, which now become $\utrans{}{\Gamma}{u}^{min}$. Its definition
is given in Figure~\ref{fig:min-def-trans-min}. Rule \rulename{DExp} is unfolding a function
defininion only if the result of the function is in the $min(\cdot)$ set. Rule \rulename{DCase}
has the same flavor. However if we have $min(s)$ then the focus of evaluation in the counterexample
will move on to the scrutinee of the case expression, and hence we get a $min(t)$ predicate, where the 
term $t$ is the FOL translation of the case scrutinee $e$.

\paragraph{Semantics of minimization}
\newcommand{\ThMin}{\Th_{\infty}^{min}}
\newcommand{\SDownarrow}{\downarrow}

Before diving to the more intricate details of minimization and its semantics, it is worth pointing out that 
extending ${\cal I}$ so that ${\cal I}(min)(d)$ holds for every $d : D_\infty$ creates an interpretation that 
validates $\ThMin$. 
\begin{theorem} $\langle D_\infty, {\cal I}\uplus min |-> \lambda d.true\rangle \models \ThMin$. \end{theorem} 

However, we are interested in finite models of the theory $\Th_\infty^{min}$ and in what 
follows we show how to construct a finite model of $\ThMin$, starting from an execution 
trace $P |- e \Downarrow v$ that satisfies certain conditions. 

%% Consider the graph $(G,E)$ induced by an execution trace 
%% $P |- e \Downarrow v$ with $G$ the set 
%%     \[ \{ e \mid P |- e \Downarrow v \text{in the trace}\} \cup 
%%            \{ @bad@ \} \cup \{ @bot@ \} \] 
%% where @bad@ and @bot@ are two distinguished elements (that will serve as the interpretations
%% of $\bad$ and $\unr$ respectively in this model. We also add blue edges $G$ between 
%% between the conclusion and the assumptions of any evaluation rule that has been used 
%% in the trace. Next we {\em complete} this graph so that for every node of the 
%% form $n : (e_1\;e_2)$, if $P |- e_2 \not\Downarrow$, a directed black edge is added 
%% from $e_1\;e_2$ to @bot@, else if $P |- e_2 \Downarrow v_2$ we add a black edge to 
%% from $n : (e_1\;e_2)$ to a new node $e_2$ and recursively build the graph
%% $P |- e_2 \Downarrow v_2$. Similarly for every node of the form $n : K[\taus](\oln{e})$.
%% This process is infinite but it has an infinite fixpoint by Tarski-Knaster since we are
%% continuously adding nodes and edges. 

%% Next we add red undirected edges, along the evaluation blue edges when the semantics
%% agree. We also add red edge for every two application nodes $n : e_1\;e_2$ and 
%% $n' : e_1'\;e_2'$ such that $n_1 : e_1$ and $n_1' : e_1'$ are in the reflexive transitive %% closure of red edges and $n_2 
%% we recurse on to
%% $\oln{e}$ 
%% %%  such that for every rule of the form 
%% %% $\frac{e_1 \Downarrow

Let us revisit the 
evaluation relation of Figure~\ref{fig:opsem} and let us refine it with the highlighted
parts in Figure~\ref{fig:opsem-strict}. The omitted rules are the same as in 
Figure~\ref{fig:opsem}, with the only difference is that they use $\SDownarrow$ instead
of $\Downarrow$.

\begin{figure}\small
\[\begin{array}{c}
\ruleform{P |- u \SDownarrow v} \\ \\
\prooftree
\begin{array}{c} \ \\ 
\end{array}
P |- \highlight{\ol{e} \SDownarrow \ol{v}} 
-------------------------------------{EValF}
P |- f^\ar[\taus]\;\oln{e}{m < \ar} \SDownarrow f^\ar[\taus]\;\oln{e}{m < \ar}
~~~~~
P |- \highlight{\ol{e} \SDownarrow \ol{v}}
-------------------------------------{EValC}
P |- K[\taus](\ol{e}) \SDownarrow K[\taus](\ol{e})
~~~~
\phantom{G}
-------------------------------------{EValB}
P |- @BAD@ \SDownarrow @BAD@
~~~~~
\begin{array}{c}
(f |-> \Lambda\ol{a} @.@ \lambda\oln{x{:}\tau}{m} @.@ u) \in P \\
P |- u[\ol{\tau}/\ol{a}][\ol{e}/\ol{x}] \SDownarrow v \quad
\highlight{P |- \ol{e} \SDownarrow \ol{v}}
\end{array}
-------------------------------------{EFun}
P |- f[\ol{\tau}]\;\oln{e}{m} \SDownarrow v
~~~~~
\begin{array}{c}  
P |- e_1 \SDownarrow v_1 \\ 
\highlight{P |- e_2 \SDownarrow v_2} \\
P |- v_1\;e_2 \SDownarrow w
\end{array}
------------------------------------------------{EApp}
P |- e_1\;e_2 \SDownarrow w
~~~~
\begin{array}{c}  \ \\ 
P |- e_1 \SDownarrow @BAD@ \\
\highlight{P |- e_2 \SDownarrow v_2}
\end{array}
------------------------------------------------{EBadApp}
P |- e_1\;e_2 \SDownarrow @BAD@
\endprooftree \\ \\ 
\text{ .. plus rules for @case@ .. } 
\end{array}\]
\caption{Strict operational semantics}\label{fig:opsem-strct}
\end{figure}

Observe that $\SDownarrow$ is a stricter version of $\Downarrow$, that 
can potentially diverge more often than $\Downarrow$ but cannot crash more often, 
as the results of evaluating expressions under constructors or arguments of 
applications are not used. In fact the following lemma is a straightforward induction.

\begin{lemma}
If $P |- e \SDownarrow v$ then $P |- e \Downarrow v$.
\end{lemma} 

Our finite model theorem is then the following:

\begin{theorem}\label{thm:finite-model} If $P |- e \SDownarrow v$ then there exists 
a finite set $S^{min}$ and an interpretation ${\cal I}^{min}$ such that 
$\langle S^{min},{\cal I}^{min}\rangle \models \ThMin$. Moreover 
$\langle S^{min},{\cal I}^{min}\rangle \models \Dtrans{}{P}^{min}$. 
\end{theorem}
\begin{proof} The reader who is interested in the precise 
construction of the model from the trace can consult the 
Appendix. 
\end{proof}

\paragraph{Minimization in contracts}
We have showed that, given a strict trace, our new theory admits a finite model. But what is its proving 
power? In order to prove a contract we may have to appeal to some axiom from the theory. But, alas, most
axioms in $\ThMin$ are guarded by $min(\cdot)$ predicates, effectively casting them unusable if we are 
trying to prove contracts that arise from the translation in Figure~\ref{fig:contracts-minless}.

The solution to this problem is to modify the translation of contracts as well, in order to assume 
or provide $min(\cdot)$ predicates that can allow axioms from $\ThMin$ to trigger. The modified 
contract translation as well as an axiomatization of $\Th_\lcfZ^{min}$ is in Figure~\ref{fig:min-typing}.

\dv{That's where I have reached. More tomorrow}


\begin{figure}\small
\[\begin{array}{c} 
\ruleform{\ctrans{\Sigma}{\Gamma}{e \in \Ct} = \formula{\phi}} \\ \\ 
\prooftree
  \begin{array}{c}
   \etrans{\Sigma}{\Gamma}{e} = \formula{t} \quad
   \etrans{\Sigma}{\Gamma,x}{e'} = \formula{t'}
  \end{array}
  ------------------------------------------{CBase}
  \begin{array}{l}
   \ctrans{\Sigma}{\Gamma}{e \in \{(x{:}\tau) \mid e' \}} = \highlight{min(t) => } \\
  %% \Sigma;\Gamma |- e \in \{(x{:}\tau \mid e' \}
   \quad \formula{\highlight{(min(t'[t/x])\;\land}} \\ 
   \quad \formula{\;\;((t{=}\unr) \lor (t'[t/x]{=}\unr) \lor (t'[t/x]{=}\True)))}
  \end{array}
  ~~~~~ 
  \begin{array}{c}
  \ctrans{\Sigma}{\Gamma,x}{x \notin \Ct_1} {=} \formula{\phi_1} \quad
  \ctrans{\Sigma}{\Gamma,x}{e\;x \in \Ct_2} {=} \formula{\phi_2}
  \end{array} 
  ------------------------------------------{CArr}
  \begin{array}{l} 
  \ctrans{\Sigma}{\Gamma}{e \in (x{:}\Ct_1) -> \Ct_2} = 
  \formula{\forall x @.@ \phi_1 \lor \phi_2}
  \end{array}
  ~~~~~
  \begin{array}{c}
  \ctrans{\Sigma}{\Gamma}{e \in \Ct_1} = \formula{ \phi_1} \quad
  \ctrans{\Sigma}{\Gamma}{e \in \Ct_2} = \formula{ \phi_2}
  \end{array}
  ------------------------------------------{CConj}
  \ctrans{\Sigma}{\Gamma}{e \in \Ct_1 \& \Ct_2} = \formula{ \phi_1 /\ \phi_2}
  ~~~~~
  \etrans{\Sigma}{\Gamma}{e} =  \formula{t}
  -------------------------------------------{CCf}
  \ctrans{\Sigma}{\Gamma}{e \in \CF} = \formula{\highlight{min(t) => \lcf{t}}}
 \endprooftree  \\ \\ 
\ruleform{\ctrans{\Sigma}{\Gamma}{e \notin \Ct} = \formula{\phi}} \\ \\ 
\prooftree
  \begin{array}{c}
   \etrans{\Sigma}{\Gamma}{e} = \formula{t} \quad
   \etrans{\Sigma}{\Gamma,x}{e'} = \formula{t'}
  \end{array}
  ------------------------------------------{NCBase}
  \begin{array}{l}
   \ctrans{\Sigma}{\Gamma}{e \notin \{(x{:}\tau) \mid e' \}} = \\
   \quad\formula{\highlight{min(t)\;\land\;min(t'[t/x])}\;\land } \\
   \quad\formula{\;\;((t{\neq}\unr) \land ((t'[t/x]{\neq}\unr) \land (t'[t/x]{\neq}\True)))}
  \end{array}
  ~~~~~ 
  \begin{array}{c}
  \ctrans{\Sigma}{\Gamma,x}{x \in \Ct_1} {=} \formula{\phi_1} \quad
  \ctrans{\Sigma}{\Gamma,x}{e\;x \notin \Ct_2} {=} \formula{\phi_2}
  \end{array} 
  ------------------------------------------{NCArr}
  \begin{array}{l} 
  \ctrans{\Sigma}{\Gamma}{e \notin (x{:}\Ct_1) -> \Ct_2} = 
  \formula{\exists x @.@ \phi_1 \land \phi_2}
  \end{array}
  ~~~~~
  \begin{array}{c}
  \ctrans{\Sigma}{\Gamma}{e \notin \Ct_1} = \formula{ \phi_1} \quad
  \ctrans{\Sigma}{\Gamma}{e \notin \Ct_2} = \formula{ \phi_2}
  \end{array}
  ------------------------------------------{NCConj}
  \ctrans{\Sigma}{\Gamma}{e \in \Ct_1 \& \Ct_2} = \formula{ \phi_1 \lor \phi_2}
  ~~~~~
  \etrans{\Sigma}{\Gamma}{e} =  \formula{t}
  -------------------------------------------{NCCf}
  \ctrans{\Sigma}{\Gamma}{e \notin \CF} = \formula{\highlight{min(t) \land \neg\lcf{t}}}
 \endprooftree \\ \\ 
\ruleform{\Th_\lcfZ^{min}} \\ \\ 
\begin{array}{lll} 
 \textsc{AxCfBU}  & \formula{\lcf{\unr} /\ \lncf{\bad}} \\
 \textsc{AxCfMin} & \formula{\highlight{\forall x @.@ \lcf{x} => min(x) \lor x = unr}} \\
 %% \textsc{AxCfB1}   & \formula{\forall \oln{x}{n} @.@ \bigwedge_i (\lcf{x_i}\lor \neg(min(x_i))} => \lcf{K(\ol{x})} \lor \neg(min(K(\ol{x}))) \\
 %%                   & \text{ for every } (K{:}\forall\as @.@ \oln{\tau}{n} -> T\;\as) \in \Sigma \\ 
 \textsc{AxCfC1} & \formula{\forall \oln{x}{n} @.@ \bigwedge\lcf{\ol{x}}} => \lcf{K(\ol{x})} \\
                 & \text{ for every } (K{:}\forall\as @.@ \oln{\tau}{n} -> T\;\as) \in \Sigma \\
 \textsc{AxCfC2} & \formula{\highlight{min(K(\oln{x}{n}))\land\neg\lcf{K(\oln{x}{n})}}} \\ 
                 & \formula{\quad\qquad\qquad \highlight{ => \bigvee_i (min(x_i)\land\neg\lcf{x_i})}}
\end{array}
\end{array}\]
\caption{Translation of contracts with minimization}\label{fig:min-typing}
\end{figure}


%% \begin{figure} 
%% {\small
%% \[\setlength{\arraycolsep}{1pt}
%% \end{array}\]}
%% \caption{Crash-freedom with minimization}\label{fig:min-theory}
%% \end{figure}



\subsubsection{min() as not unreachable}

TODO


\subsection{Induction}\label{sect:induction}


