\paragraph{Contracts that do not hold}
\label{ssect:countersat}

In practice, programmers will often propose contracts that do not hold.
For example, consider the following definitions:
\begin{code}
  length []     = Z
  length (x:xs) = S (length xs)

  isZero Z = True
  isZero _ = False
\end{code}
Suppose that we would like to check the (false) contract:
   \[ @length@ \in \CF -> \{ x \mid @isZero@\;x\} \]
\emph{A satisfiability-based checker
will simply diverge} trying to construct a counter model for the
negation of the above query; we have confirmed that this is indeed the
behaviour of several tools (Z3, Equinox, Eprover).  Why?  When a
counter-model exists, it will include tables for the function symbols
in the formula. Recall that functions in FOL are total over the domain
of the terms in the model. This means that function tables may be {\em
infinite} if the terms in the model are infinite. Several (very
useful!)  axioms such as the discrimination axioms \textsc{AxDisjC}
may in fact force the models to be infinite.

In our example, the table for @length@ is indeed infinite since @[]@ is
always disjoint from @Cons x xs@ for any @x@ and @xs@. Even if there
is a finitely-representable infinite model, the theorem prover may
search forever in the ``wrong corner'' of the model for a
counterexample.

From a practical point of view this is unfortunate; it is not
acceptable for the checker to loop when the programmer writes an
erroneous contract.  Tantalisingly, there exists a very simple
counterexample, e.g. @[Z]@, and that single small example is all the
programmer needs to see the falsity of the contract.

Addressing this problem is a challenging (but essential)
direction for future work, and we are currently
working on a modification of our theory that admits the denotational model, but
also permits {\em finite models} generated from counterexample traces.
%% These ideas are reminiscent to the techniques that the Nitpick \dv{IS this right?} tool
%% uses for generating finite counterexamples in Isabelle. \dv{Someone please check!}.
If the theory can guarantee the existence of a finite model in case of a counterexample,
a finite model checker such as Paradox~\cite{paradox} will be able find it.

%% It is obviouly unacceptable for the system to go into a loop if
%% the programmer writes a bogus contract, and we have promising
%% preliminary results based on so-called ``minimisation'', and
%% finite counter-model generators such as Paradox \cite{koen}, but we
%% leave this for (absolutely essential) future work.

\paragraph{A tighter correspondence to operational semantics?}

Earlier work gave a declarative specfication of contracts using
\emph{operational semantics} \cite{xu+:contracts}.  In this paper we have
instead used a \emph{denotational semantics} for contracts (Figure~\ref{f:den-sem-contracts}).
It is natural to ask whether or not the two semantics are identical.

From computational adequacy, Theorem~\ref{thm:adequacy} we can easily state
the following theorem:
\begin{corollary} Assume that $e$ and $\Ct$ contain no term variables and
assume that $\ctrans{}{\cdot}{e \in \{x \mid e_p\}} = \formula{\phi}$. It is the case
that $\langle D_\infty,{\cal I}\rangle \models \phi$ if and only iff either
$P \not|- e \Downarrow$ or $P \not|- e_p[e/x] \Downarrow$ or $P |- e_p[e/x] \Downarrow \True$. \end{corollary}
Hence, the operational and denotational semantics of \emph{predicate contracts} coincide.
However, the correspondence is not precise for \emph{dependent function contracts}.
Recall the operational definition of contract satisfaction for
a function contract:
\[\begin{array}{l}
   e \in (x{:}\Ct_1) -> \Ct_2 \text{ iff} \\
   \text{for all } e' \text{ such that } (e' \in \Ct_1) \text{ it is } e\;e' \in \Ct_2[e'/x]
\end{array}\]
The denotational specification (Figure~\ref{f:den-sem-contracts})
says that for all denotations $d'$ such that
$d' \in \dbrace{\Ct_1}$, it is the case that
$\dapp(\dbrace{e},d') \in \dbrace{\Ct_2}_{x |->d'}$.

Alas there are {\em more} denotations than images of terms in $D_{\infty}$,
and that breaks the correspondence. Consider the program:
\begin{code}
  loop = loop

  f :: (Bool -> Bool -> Bool) -> Bool
  f h = if (h True True) && (not (h False False))
        then if (h True loop) && (h loop True)
             then BAD else True
        else True
\end{code}
Also consider now this candidate contract for $f$:
\[ @f@ \in (\CF -> \CF -> \CF) -> \CF \]
Under the \emph{operational} definition of contract satisfaction,
@f@ indeed satisfies the contract.
To reach @BAD@ we have to pass both conditionals.
The first ensures that @h@ evaluates at least one of its
arguments, while the second will diverge if either argument is evaluated.
Hence @BAD@ cannot be reached, and the contract is satisfied.

However, \emph{denotationally} it is possible to have the classic
parallel-or function, $por$, defined as
follows:
\[\begin{array}{lcl}
  por\;\bot\;\bot & = & \bot \\
  por\;\bot\;\injKZ{True} & = & \injKZ{True} \\
  por\;\injKZ{True}\;\bot & = & \injKZ{True} \\
  por\;\injKZ{False}\;\injKZ{False} & = & \injKZ{False}
\end{array}\]
We have to define $por$ in the language of denotational semantics, because
we cannot write it in Haskell --- that is the point!  For convenience,
though, we use pattern matching notation instead of our
language of domain theory combinators.
The rest of the equations (for @BAD@ arguments) are induced by
monotonicity and we may pick whatever boolean value we like when both
arguments are @BAD@.

Now, this is denotationally a $\CF -> \CF -> \CF$ function, but it
will pass \emph{both} conditionals,
yielding @BAD@. Hence $app(@f@,por) = \injBad$, and @f@'s contract does not hold.
So we have a concrete case where an expression may satisfy its
contract operationally but not denotationally, because of the usual
loss of full abstraction: there are more tests than programs in the denotational world.
Due to contra-variance
we expect that the other inclusion will fail too.

This is not a serious problem in practice.
After all the two definitions mostly coincide, and
they precisely coincide in the base case.  At the end of the day, we
are interested in whether $@main@ \in \CF$, and we have proven that if
is crash-free denotationally, it is definitely crash-free in any
operationally-reasonable term.

Finally, is it possible to define an operational model for our FOL theory that interpreted
equality as contextual equivalence? Probably this could be made to work, although we believe
that the formal clutter from syntactic manipulation of terms could be worse than the current
denotational approach.


\paragraph{Polymorphic crash-freedom}

Observe that our axiomatisation of crash-freedom in Figure~\ref{fig:prelude}
includes only axioms for data constructors. In fact, our denotational interpretation
$\Fcf^{\infty}$ allows more axioms, such as:
\[\begin{array}{l}
    \forall x y @.@ \lcf{x} \land \lcf{y} => \lcf{app(x,y)}
\end{array}\]
This axiom is useful if we wish to give directly a $\CF$ contract to a value of
arrow type. For instance, instead of specifying that @map@ satisfies the contract
$(\CF -> \CF) -> \CF -> \CF$ one may want to say that it satisfies the contract
$\CF -> \CF -> \CF$. With the latter contract we need the previous axiom to be
able to apply the function argument of @map@ to a crash-free value and get a
crash-free result.

In some situations, the following axiom might be beneficial as well:
\[\begin{array}{l}
    (\forall \xs @.@ \lcf{f(\xs)}) => \lcf{f_{ptr}}
\end{array}\]
If the result of applying a function to any possible argument is crash-free then
so is the function pointer. This allows us to go in the inverse direction as before,
and pass a function pointer to a function that expects a $\CF$ argument. However notice
that this last axiom introduces a quantified assumption, which might lead to significant
efficiency problem.

Ideally we would like to say that $\dbrace{\CF} = \dbrace{\CF \rightarrow \CF}$,
but that is not quite true.  In particular,
\[\begin{array}{l}
   (\forall x @.@ \lcf{app(y,x)}) => \lcf{y}
\end{array}\]
is {\em not} valid in the denotational model. For instance consider the
value $\injK{K}{\injBad}$ for $y$. The left-hand side is going to always
be true, because the application is ill-typed and will yield $\bot$, but $y$
is not itself crash-free.


