We now turn our attention to contracts. The syntax of contracts
has appeared in Figure~\ref{fig:syntax} and includes base contracts
$\{ x \mid e \}$, arrow contructs $(x : \Ct_1) -> \Ct_2$, conjunctions
$\Ct_1 \& \Ct_2$ and crash-freedom $\CF$. Previous work~\cite{xu+:contracts} 
includes other constructs as well, but these constructs are enough to verify 
many programs and already demonstrate the interesting theoretical and practical problems.

We may also define the denotational semantics of contracts, below. We assume again 
that there is a program $P$, well-formed in a signature $\Sigma$. In the definition, 
$\rho$ is a semantic environment. 

\begin{definition}[Denotational semantics of contracts]
\[\begin{array}{l}
    \dbrace{x \mid e}_\rho(d) \text{ iff } \\
        \quad \unroll(d) = \bot \text{ or } 
        \unroll(\interp{e}{\dbrace{P}^\infty}{\rho,x|->d}) = \bot\;\text{ or } \\
        \quad \unroll(\interp{e}{\dbrace{P}^\infty}{\rho,x|->d}) = \ret(\inj{\True}(1)) \\ \\
    \dbrace{(x{:}\Ct_1) -> \Ct_2}_{\rho}(d) \text{ iff } \\
        \quad \text{for all } d_x \in D_\infty \\ 
        \quad\quad \text{if }
                     \dbrace{\Ct_1}_\rho(d_x)\text{ then }
                     \dbrace{\Ct_2}_{\rho,x|->d_x}(\dapp(d,d_x)) \\ \\ 
    \dbrace{\CF}_\rho(d) \text{ iff }  d \in \Fcf^{\infty} \\  \\
    \dbrace{\Ct_1 \& \Ct_2}_\rho(d) \text{ iff } 
       \dbrace{\Ct_1}_\rho(d) \text{ and } 
       \dbrace{\Ct_2}_\rho(d)
\end{array}\]
For a closed contract $\Ct$ we will use notation
$\dbrace{\Ct}$ for its denotation in the empty 
semantic environment.
\end{definition}



To the extend that in the end we are only interested in base contracts, giving a 
denotational semantics of full-higher-order contracts is not really interesting 
but we do this anyway. For a given denotation $d$, we define the 
predicate $\interp{\Ct}{\dbrace{P}^\infty}{\rho}(d)$ by recursion on the structure 
of the contract $\Ct$, such that:


\clearpage 
\subsection{Contracts in first-order logic}\label{sect:contracts-fol}



\begin{figure}\small
\[\begin{array}{c}
\ruleform{\Th{\Sigma}{P}^{\lcfZ}} \\ \\ 
\begin{array}{lll} 
 \textsc{AxCfA}   & \formula{\lcf{\unr} /\ \lncf{\bad}} \\
 \textsc{AxCfB}   & \formula{\forall \oln{x}{n} @.@ \lcf{K(\ol{x})} <=> \bigwedge\lcf{\ol{x}}} \\
                  & \text{ for every } (K{:}\forall\as @.@ \oln{\tau}{n} -> T\;\as) \in \Sigma
\end{array}
\end{array}\]
\caption{Prelude theory}\label{fig:prelude}
\end{figure}



\begin{figure}\small
\[\begin{array}{c} 
\ruleform{\ctrans{\Sigma}{\Gamma}{e \in \Ct} = \formula{\phi}} \\ \\ 
\prooftree
  \begin{array}{c}
   \etrans{\Sigma}{\Gamma}{e} = \formula{t} \quad
   \etrans{\Sigma}{\Gamma,x}{e'} = \formula{t'}
  \end{array}
  ------------------------------------------{CTransBase}
  \begin{array}{l}
   \ctrans{\Sigma}{\Gamma}{e \in \{(x{:}\tau) \mid e' \}} = \\
  %% \Sigma;\Gamma |- e \in \{(x{:}\tau \mid e' \}
  \;\;\formula{(t = \unr) \lor (t'[t/x] = \unr) \lor (t'[t/x] = \True)}
  \end{array}
  ~~~~~ 
  \begin{array}{c}
  \ctrans{\Sigma}{\Gamma,x}{x \in \Ct_1} {=} \formula{\phi_1} \quad
  \ctrans{\Sigma}{\Gamma,x}{e\;x \in \Ct_2} {=} \formula{\phi_2}
  \end{array} 
  ------------------------------------------{CTransArr}
  \begin{array}{l} 
  \ctrans{\Sigma}{\Gamma}{e \in (x{:}\Ct_1) -> \Ct_2} = 
  \formula{\forall x @.@ \neg \phi_1 \lor \phi_2} 
  \end{array}
  ~~~~~
  \begin{array}{c}
  \ctrans{\Sigma}{\Gamma}{e \in \Ct_1} = \formula{ \phi_1} \quad
  \ctrans{\Sigma}{\Gamma}{e \in \Ct_2} = \formula{ \phi_2}
  \end{array}
  ------------------------------------------{CTransConj}
  \ctrans{\Sigma}{\Gamma}{e \in \Ct_1 \& \Ct_2} = \formula{ \phi_1 /\ \phi_2}
  ~~~~~
  \etrans{\Sigma}{\Gamma}{e} =  \formula{t}
  -------------------------------------------{CTransCf}
  \ctrans{\Sigma}{\Gamma}{e \in \CF} = \formula{\lcf{t}}
 \endprooftree 
%% \\ \\ 
%% \ruleform{\Sigma;\Gamma |- e \notin \Ct \elab{ \phi} } \\ \\
%% \prooftree
%%   \begin{array}{c}
%%    \Sigma;\Gamma |- e : \tau \elab{ t}  \quad
%%    \Sigma;\Gamma,(x{:}\tau) |- e' : \Bool \elab{ t'}
%%   \end{array}
%%   ------------------------------------------{CNTransBase}
%%   \begin{array}{l}
%%   \Sigma;\Gamma |- e \notin \{(x{:}\tau) \mid e' \} 
%%   \elab{(t'[t/x] = \bad) \lor (t'[t/x] = \False)}
%%   \end{array}
%%   ~~~~~ 
%%   \Sigma;\cdot |- \Ct_1 : \tau
%%   ------------------------------------------{CNTransArr}
%%   \begin{array}{l} 
%%   \Sigma;\Gamma |- e \notin (x{:}\Ct_1) -> \Ct_2 
%%   \elab{\exists x @.@ (\Sigma;\Gamma,(x{:}\tau) |- x \in \Ct_1) /\ (\Sigma;\Gamma,(x{:}\tau) |- e\;x \notin \Ct_2)}
%%   \end{array}
%%   ~~~~~
%%   \begin{array}{c}
%%   \Sigma;\Gamma |- e \notin \Ct_1 \elab{ \phi_1} \quad
%%   \Sigma;\Gamma |- e \notin \Ct_2 \elab{ \phi_2}
%%   \end{array}
%%   ------------------------------------------{CNTransConj}
%%   \Sigma;\Gamma |- e \notin \Ct_1 \& \Ct_2 \elab{ \phi_1 \lor \phi_2}
%%   ~~~~
%%   \Sigma;\Gamma |- e : \tau \elab{ t}
%%   -------------------------------------------{CNTransCf}
%%   \Sigma;\Gamma |- e \notin \CF \elab{ \lncf{t}}
%%  \endprooftree
\end{array}\]
\caption{Baseline contract elaboration}\label{fig:typing}
\end{figure}


In this section we will attempt to ignore the higher-order case and just talk about 
base contracts. Let us use:

The following are true: 
\begin{lemma}[Base contract adequacy]\label{lem:base-contract-adequacy}
Assume that $\Sigma |- P$ and $fv(e) \subseteq dom(P)$, i.e. $e$ is closed.
If $\langle D_\infty,{\cal I}\rangle \models \ctrans{\Sigma}{\Delta}{e \in \CF}$ then $P |- e \in \CF$. If $\langle D_\infty,{\cal I}\rangle \models \ctrans{\Sigma}{\Delta}{e \in \{ x \mid e' \}}$ then $P |- e \in \{x \mid e' \}$.
\end{lemma}
{\bf DV: Generalize this to a notion of base contracts that includes conjuctions.}

In fact the above two statements hold if we extend the interpretation 
of crash-freedom in the model to contain elements from the function 
space as well. 

Because of the full-abstraction problems we have observed above it 
is not possible to state similar statements for arrow contracts. 



\subsection{Soundness of contract checking}\label{ssect:soundness}


\subsubsection{Invocation of a theorem prover}\label{sect:infocation}

Given a program $P$ with signature $\Sigma$, that is $\Sigma |- P$, we may define the theory
${\cal T}$ as follows:
     \[ \Th{\Sigma}{P}\;\land\;\Th{\Sigma}{P}^{\lcfZ}\;\land\;\dtrans{\Sigma}{P} \]
we know that $\langle D_\infty,{\cal I}\rangle \models {\cal T}$ from the previous sections. 
Assume below that $f$ is a function such that $f \in dom(P)$ and $fv(\Ct) \subseteq dom(P)$.

\begin{theorem}[Soundness]\label{thm:prover-soundess}
If ${\cal T}\;\land\;\neg(\ctrans{\Sigma}{P}{f \in \Ct})$ is 
unsatisfiable then $\langle D_\infty,{\cal I}\rangle \models \ctrans{\Sigma}{P}{f \in \Ct}$.
\end{theorem}
\begin{proof}
If there is no model for this formula (i.e. the theorem prover returns ``unsatisfiable'') then
its negation must be valid (true in all models), that 
is ${\cal T} -> \ctrans{\Sigma}{P}{f \in \Ct}$ is valid. By completeness
of first-order logic ${\cal T} |- \ctrans{\Sigma}{P}{f \in \Ct}$. This means in 
turn that all models of ${\cal T}$ validate $\ctrans{\Sigma}{P}{f \in \Ct}$. In particular 
for the denotational model we have that $\langle D_\infty,{\cal I}\rangle \models {\cal T}$ 
and hence $\langle D_\infty,{\cal I}\rangle \models \ctrans{\Sigma}{P}{f \in \Ct}$.
\end{proof}

\subsubsection{End-goal and incremental verification}\label{sect:incremental}

Assume that we are given a program $P$ with a function $f \in dom(P)$, for which we have 
already showed that $\langle D_\infty,{\cal I}\rangle \models \ctrans{\Sigma}{P}{f \in \Ct_f}$. 
Suppose next that we are presented with a ``next'' goal, to prove that 
$\langle D_\infty,{\cal I}\rangle \models \ctrans{\Sigma}{P}{h \in \Ct}$. 
We may consider the following three variations of how to do this:

\begin{itemize}
  \item Simply ask for the unsatisfiability of: 
    \[  \Th{\Sigma}{P}\;\land\;
        \Th{\Sigma}{P}^{\lcfZ}\;\land\;\dtrans{\Sigma}{P}\;\land\;\neg\ctrans{\Sigma}{P}{h \in \Ct_h} \] 
        The soundness of this query follows directly from Theorem~\ref{thm:prover-soundess} above.

  \item Ask for the unsatisfiability of:
    \[  \Th{\Sigma}{P}\;\land\;
        \Th{\Sigma}{P}^{\lcfZ}\;\land\;\dtrans{\Sigma}{P}\;\land\;\ctrans{\Sigma}{P}{f \in \Ct_f}\;\land\;\neg\ctrans{\Sigma}{P}{h \in \Ct_h}     \] 
        This query adds the {\em already proven} contract for $f$ in the theory. If this formula
        is unsatisfiable, then its negation is valid, and we know that the denotational model is 
        a model of the theory {\em and} of $\ctrans{\Sigma}{P}{f \in \Ct_f}$ and hence it must also
        be a model of $\ctrans{\Sigma}{P}{h \in \Ct_h}$. 
  \item Ask for the unsatisfiability of:
    \[  \Th{\Sigma}{P}\;\land\;
        \Th{\Sigma}{P}^{\lcfZ}\;\land\;\dtrans{\Sigma}{P \setminus f}\;\land\;\ctrans{\Sigma}{P}{f \in \Ct_f}\;\land\;
        \neg\ctrans{\Sigma}{P}{h \in \Ct_h}     \] 
        This query removes the axioms associated with the definition of $f$ since we may only have 
        its signature and contract available. Via a similar reasoning as before, such an invocation 
        is sound.
\end{itemize}

Our final goal is going to show that a program does not crash, that
is the final contract will be of the form $e \in \Ct$ where $\Ct$ is
going to be some {\em base contract}. Note that by base contract adequacy
(Lemma~\ref{lem:base-contract-adequacy}) if we manage to show a base contract 
denotationally, then the contract holds in operational terms.



\subsection{Denotational versus operational semantics for contracts}
TODO -- I have just dumpted material here. 

We have the rather obvious theorem below.

\begin{theorem}[Soundness and completeness for denotational semantics]
Assume a program $P$ with signature $\Sigma$, and expression $e$ and contract $\Ct$ 
such that $fv(e) \cup fv(\Ct) \subseteq dom(P)$. Then 
$\langle D_\infty,{\cal I}\rangle \models \ctrans{\Sigma}{P}{e \in \Ct}$ iff
$\interp{\Ct}{\dbrace{P}^{\infty}}{\cdot}(\interp{e}{\dbrace{P}^\infty}{\cdot})$.
\end{theorem}




\subsubsection{Contract satisfaction and crash-freedom}\label{sect:cf}

We would like to define a set of contract-satisfying denotations and also a set of contract-satisfying terms, 
characterized by $P |- e \in \Ct$, such that the following claim becomes true:

\begin{proposition} Assume that $\Sigma |- P$ and $fv(e) \subseteq dom(P)$, i.e. $e$ is closed.
Then: $\langle D_\infty,{\cal I}\rangle \models \ctrans{\Sigma}{\Delta}{e \in \Ct}$ iff $P |- e \in \Ct$.
\end{proposition}

Now there are several problems with coming up with a good definition of $P |- e \in \Ct$, 
which we elaborate in the following sections.

\subsubsection{Problem I: Crash-freedom} 

Ideally we would like to define crash-freedom {\em semantically} using the following 
strict bifunctor on admissible sets $S^{-},S^{+} \subseteq D_{\infty}$.
{\setlength{\arraycolsep}{2pt}
\[\begin{array}{rcl}
   F_{\lcfZ}(S^{-},S^{+}) & = & \{\;d\;\mid\;\unroll(d) \neq \ret(\inj{bad}(1))\;\land\; \\ 
                      &    & \quad \forall \ol{d} @.@ \unroll(d){=}\ret(\inj{K_1^\ar}\langle\oln{d}{\ar}\rangle) ==> \ol{d} \in S^{+} \} \\ 
                   & \cup & \ldots \\ 
                   & \cup & \{\;d\;\mid\;\unroll(d) \neq \ret(\inj{bad}(1))\;\land\; \\ 
                   &      & \quad \forall d_0 @.@ \unroll(d) = \ret(\inj{->}(d_0)) ==> \\ 
                   &      & \quad\quad \forall\;d' \in S^{-} ==> \dapp(d,d') \in S^{+} \}  \\
\end{array}\]}
The $\Fcf$ bifunctor has a negative and positive fixpoint, and by minimal invariance they coincide (one direction 
follows by Tarski-Knaster, the other can be inductively proved using the approximations on ever element of $D_{\infty}$ given
in Lemma~\ref{lem:min-inv-reqs} and the fact that the lub of the chain of $\rho_i$ is the identity and the fact that this 
functor preserves admissibility for the positive sets). Let us call this admissible set $\Fcf^{\infty} \subseteq D_{\infty}$.

We consider this predicate to be the ``ideal crash-freedom'' -- however it is very difficult to give a 1-1 operational
definition. The reason is that the $\Fcf$ functor quantifies in the function case over any $d'$ -- whereas in the operational
semantics it is only reasonable that we quantify over all terms (or over terms that do not contain @BAD@) In the absense of 
full abstraction of the domain (which is plausible, especially if we extend the language with other features) it is unclear 
what a corresponding predicate would look like in terms of operational semantics. 

We then go for a simpler predicate, which only characterizes crash-freedom for first-order terms, 
generate by the following functor on {\em admissible} sets of denotations:
{\setlength{\arraycolsep}{2pt}
\[\begin{array}{rcl}
   G_{\lcfZ}(S^{+}) & = & \{\;d\;\mid\; \unroll(d){=}\ret(\inj{K_1^\ar}\langle\oln{d}{\ar}\rangle) \land \ol{d} \in S^{+} \} \\ 
                  & \cup & \ldots \\ 
                  & \cup & \{\;\bot\;\}
\end{array}\]}
Notice that if $S$ is admissible then so is $G_{\lcfZ}(S)$. 

%% The $G_{\lcfZ}$ functor has a fixpoint and it is an admissible relation, and we will use its 
%% fixpoint $G_{\lcfZ}^\infty$, so now we need to say what $G_{\lcfZ|}^\infty$ means operationally. 
\begin{lemma} The functor $G_{\lcfZ}$ has a unique fixpoint $G_{\lcfZ}^\infty$ on admissible sets. \end{lemma}
\begin{proof} 
The intersection of admissible sets is admissible. Hence we have a complete join semi-lattice (which induces a 
complete lattice), so the monotone functor $G_{\lcfZ}$ does have a smallest and a greatest fixpoint call
it $G_{\lcfZ}^{min}$ and $G_{\lcfZ}^{max}$. Moreover this fixpoint will be an admissible relation. Now it must be 
that $G_{\lcfZ}^{min} \subseteq G_{\lcfZ}^{max}$ so we only show next that
also $G_{\lcfZ}^{max} \subseteq G_{\lcfZ}^{min}$. To do this we will show that:
\[ \forall i. d \in G_{\lcfZ}^{max} ==> \rho_i(d) \in G_{\lcfZ}^{min} \] 
by induction on $i$. For $i = 0$ it follows since $\rho_0(d) = \bot$. Let us assume 
that it holds for $i$, we need to show that $\rho_{i+1}(d) \in G_{\lcfZ}(G_{\lcfZ}^{min})$.
We know however that $d \in G_{\lcfZ}(G_{\lcfZ}^{max}$ and by simply case analysis and appealing
to the induction hypothesis we are done. Finally, by admissibility it must be that
$\sqcup\rho_i(d) \in G_{\lcfZ}^{min}$ and by Lemma~\ref{lem:min-inv-reqs} it
must be that $d \in G_{\lcfZ}^{min}$. This means that the two fixpoints coincide, 
hence there is only a unique fixpoint of $G_{\lcfZ}$, call it $G_{\lcfZ}^\infty$.
\end{proof} 

Now, we would like to define operationally the set of {\em crash-free} terms as a set $\Ecf$ of 
closed terms that satisfies:
{\setlength{\arraycolsep}{2pt}
\[\begin{array}{rcl}
   \Ecf & =    & \{ e \;\mid\; P |- e \Downarrow K[\taus](\ol{e}) /\ \ol{e} \in \Ecf \} \\
        & \cup & \ldots \\
        &      & \{ e \;\mid\; P \not|- e \Downarrow \} 
\end{array}\]}%
We do not know that the set $\Ecf$ exists, so we have to prove it. 
\begin{lemma}
There exists a largest set that satisfies the $\Ecf$ equation above.
\end{lemma} 
\begin{proof}
Define $\Ecf$ to be the set
\[ \{ e\;\mid\; \interp{e}{\dbrace{P}^\infty}{\cdot} \in G_{\lcfZ}^{\infty}\} \]
It is straightforward (by computational adequacy) to show that it satisfies the $\Ecf$ recursive
equation above. For uniqueness, assume any other set $E$ that satisfies the recursive equation
above. We can show that $\interp{E}{\dbrace{P}^\infty}{\cdot}$ is a
fixpoint of $G_{\lcfZ}$ and since there is only one such fixpoint, this is unique. So we have that:
\[\begin{array}{ll}
 e \in E & ==> \\ 
 \interp{e}{\dbrace{P}^\infty}{\cdot} \in \interp{E}{\dbrace{P}^\infty}{\cdot} & ==> \\
 \interp{e}{\dbrace{P}^\infty}{\cdot} \in G_{\lcfZ}^\infty & ==> \\
 e \in \Ecf 
\end{array}\] 
\end{proof}
%% \begin{lemma} 
%% If $e \in E$ and $\interp{e}{\dbrace{P}^\infty}{\cdot} = \interp{e'}{\dbrace{P}^\infty}{\cdot}$ then $e' in E$.
%% \end{lemma}
%% This relies on the fact that 
%% if $\interp{e}{\dbrace{P}^\infty}{\cdot} \in \interp{E}{\dbrace{P}^\infty}{\cdot}$ then $e \in E$. Why is 
%% that? Because the assumption means that 
%% $\interp{e}{\dbrace{P}^\infty}{\cdot} \in \{ d | \exists e' \in E /\ d = \interp{e'}{\dbrace{P}^\infty}{\cdot} \}$
%% and hence this means that there exists some $e' \in E $ such that 
%% $\interp{e}{\dbrace{P}^\infty}{\cdot} = \interp{e'}{\dbrace{P}^\infty}{\cdot}$ 
%% \end{proof} 

Let us extend the interpretation function above $\linterp{\cdot}$ so that: 
\[\begin{array}{rcl}
   \linterp{\lcfZ}  & = & G_{\lcfZ}^{\infty} 
\end{array}\]

\begin{theorem}
If $\Sigma |- P$ then we have that $\langle D_{\infty},{\cal I}\rangle \models \Th{\Sigma}{P}^{\lcfZ}$.
\end{theorem}

Notice that the axiom:
\[  \textsc{AxCfC}  \quad \formula{\forall x y @.@ \lcf{x} /\ \lcf{y} => \lcf{app(x,y)}} \]
is {\em not validated} by this interpretation of crash-freedom we have given. 


\subsubsection{Problem II: the absense of full-abstraction}

Unfortunately higher-orderness bites again. Having defined the set $\Ecf$ we might define formally
the predicate $P |- e \in \Ct$ where $fv(e) \subseteq dom(P)$ and $fv(\Ct) \subseteq dom(P)$ as 
follows:
{\setlength{\arraycolsep}{2pt}
\[\begin{array}{lcl}
    P |- e \in \{ x\;\mid\;e_p\} & <=> & P |- e \not\Downarrow \text{ or } P |- e_p[e/x] \not\Downarrow \text{ or} \\ 
                                 &     & P |- e_p[e/x] \Downarrow True \\
    P |- e \in (x{:}\Ct_1) -> \Ct_2 & <=> & 
                                 \text{for all } P' e' \text{ s.t. } fv(e') \subseteq dom(P{\uplus}P')  \\ 
                                   &   &  \text{it is } P\uplus P' |- e\;e' \in \Ct_2[e'/x] \\
    P |- e \in \Ct_1 \& \Ct_2 & <=> & P |- e \in \Ct_1 \text{ and } P |- e \in \Ct_2 \\
    P |- e \in \CF            & <=> & e \in \Ecf 
\end{array}\]}

Note we made the definition above well-scoped but not necessarily well-typed; let's ignore that for now (making everything
well-typed includes extra difficulties in the proof but hopefully not surmountable).

The interesting case is the case for arrow contracts, where we have extended the set of definitions $P$ with more 
definitions $P'$ -- that is to allow for tests $e'$ which can have arbitrary computational power, and not only those
that can be constructed in the current environment. That is expected the way we have set up things, so let us examine
what happens when we try to prove the proposition below:

\begin{proposition} Assume that $\Sigma |- P$ and $fv(e) \subseteq dom(P)$, i.e. $e$ is closed.
Then: $\langle D_\infty,{\cal I}\rangle \models \ctrans{\Sigma}{\Delta}{e \in \Ct}$ iff $P |- e \in \Ct$.
\end{proposition}

{\flushleft{\em Failed proof}:}
The base case and the case of $\CF$ follow from computational adequacy so we are good. However
let's try to prove the arrow case and in particular the $(<=)$ direction. 

Let us assume that for all $P'$ and $e'$ such that $fv(e') \subseteq dom(P\uplus P')$ it is the case that
$P |- e\;e' \in \Ct_2[e'/x]$. We need to show that $\langle D_\infty,{\cal I}\rangle$ is a model of the 
formula $\forall x. \ctrans{\Sigma}{x}{x \in \Ct_1} => \ctrans{\Sigma}{x}{e\;x \in \Ct_2}$. Let us fix
a denotation $d \in D_{\infty}$ and let us assume 
that $\langle D_{\infty},{\cal I} \rangle \models \ctrans{\Sigma}{x}{x \in \Ct_1}[d/x]$. However, this does not 
necessarily mean that we can find a closed $e'$ and $P'$, such 
that $\interp{e'}{\dbrace{P{\uplus}P'}^\infty}{\cdot} = d$ to be able to use the assumptions, unless some sort
of full-abstraction property is true. So we are stuck.

Here is a concrete counterexample, based on the lack of full-abstraction due to the {\em parallel or} function. 
Consider the program $P$ below:
\[\begin{array}{lcl}
f_\omega & |-> & f_\omega \\
f & |-> & \lambda (b{:}Bool) @.@ \lambda (h{:}Bool->Bool->Bool) @.@ \\
  &     & \quad @if@\;(h\;True\;b)\;\&\&\;(h\;b\;True)\;\&\& \\ 
  &     & \quad\qquad\qquad not\;(h\;False\;False)\;@then@ \\
  &     & \quad\quad @if@\;(h\;True\;f_\omega)\;\&\&\;(h\;f_\omega\;True)\;@then@\;@BAD@ \\
  &     & \quad\quad @else@\;True \\
  &     & \quad @else@\;True
\end{array}\]
Consider now the candidate contract for $f$ below: 
\[ \CF -> (\CF -> \CF -> \CF) -> \CF \]
Operationally we may assume a crash-free boolean as well as a function $h$ which is 
$\CF -> \CF -> \CF$. The first conditional ensures that the function behaves like an ``or'' function or 
diverges. However if we pass the first conditional, 
the second conditional will always diverge and hence the contract will be satisfied. 

However, denotationally it is possible to have a {\em monotone} function $por$ defined as follows:
\[\begin{array}{lcl}
  por\;\bot\;\bot & = & \bot \\ 
  por\;\bot\;True & = & True \\
  por\;True\;\bot & = & True \\ 
  por\;False\;False & = & False
\end{array}\] 
with the rest of the equations (for @BAD@ arguments) induced by monotonicity and whatever boolean value 
we like when both arguments are @BAD@. 

Now, this is denotationally a $\CF -> \CF -> \CF$ function, and it will pass the first conditional, but it will
also pass the second conditional, yielding @BAD@. Hence denotationally the contract for $f$ {\em does not hold}.

So we have a concrete case where the $<=$ direction fails. Because of contra-variance of arrow contracts, it is 
likely that the $=>$ direction is false as well. 


%% Now it may be the case that for all denotations that semantically satisfy a contract, these denotations {\em are} 
%% realizable by a term $e'$ and a context $P'$ but it is not entirely clear how to prove this (or if this is a good
%% idea). I am not sure if this is true either.
%% The other idea out of this situation is to compile the arrow contract differently by not quantifying over all 
%% denotations but rather some kind of {\em definable} denotations -- but I do not know how exactly to do this.


\paragraph{A way out of this?}
Well, if we restrict our higher-order tests to those that can be constructed from our signature then 
we may define the following:

{\setlength{\arraycolsep}{2pt}
\[\begin{array}{lcl}
    P |- e \in \{ x\;\mid\;e_p\} & <=> & P |- e \not\Downarrow \text{ or } P |- e_p[e/x] \not\Downarrow \text{ or} \\ 
                                 &     & P |- e_p[e/x] \Downarrow True \\
    P |- e \in (x{:}\Ct_1) -> \Ct_2 & <=> & 
                                 \text{for all } e' \text{ s.t. } fv(e') \subseteq dom(P)  \\ 
                                   &   &  \text{it is } P |- e\;e' \in \Ct_2[e'/x] \\
    P |- e \in \Ct_1 \& \Ct_2 & <=> & P |- e \in \Ct_1 \text{ and } P |- e \in \Ct_2 \\
    P |- e \in \CF            & <=> & e \in \Ecf 
\end{array}\]}
Notice that the difference with the previous version of $P |- e \in \Ct$ is that we {\em do not} extend the 
definitions $P'$ so we don't get the full power of higher-order tests. We show that {\em in the current signature
only} does the program satisfy the contract. 


Why did we do this change? Because denotationally this is not terribly hard to support -- instead of translating 
\[\begin{array}{l}
  \ctrans{\Sigma}{\Gamma}{e \in (x{:}\Ct_1) -> \Ct_2} =  
  \formula{\forall x @.@ \ctrans{\Sigma}{\Gamma,x}{x \in \Ct_1} => \ctrans{\Sigma}{\Gamma,x}{e\;x \in \Ct_2}}
\end{array}\] 
we use the following:
\[\begin{array}{l}
  \ctrans{\Sigma}{\Gamma}{e \in (x{:}\Ct_1) -> \Ct_2} = \\ 
  \qquad\qquad\quad 
\formula{\forall x @.@ \definable{x} \land \ctrans{\Sigma}{\Gamma,x}{x \in \Ct_1} => \ctrans{\Sigma}{\Gamma,x}{e\;x \in \Ct_2}}
\end{array}\] 
where $\definable{x}$ could be axiomatized as containing all terms 
made up of the functions in $P$, applications, and data constructors:

\[\begin{array}{lll} 
 \textsc{DefCons} & \formula{\forall \xs @.@ \definable{K(\xs)} <=> \definable{\xs}} \\
                        & \text{ for every } (K{:}\forall\as @.@ \oln{\tau}{n} -> T\;\as) \in \Sigma \\
 \textsc{DefFun}  & \formula{\definable{f_{ptr}}}  \\
                        & \text{ for every } (f |-> \Lambda\as @.@ \lambda\oln{x{:}\tau}{n} @.@ u) \in P \\
 \textsc{DefApp}  & \formula{\forall x y @.@ \definable{x}\land\definable{y} => \definable{app(x,y)}}
%% \formula{\bad \neq \unr}  \\ 
%%  \textsc{AxDisjB} & \formula{\forall \oln{x}{n}\oln{y}{m} @.@ K(\ol{x}) \neq J(\ol{y})} \\ 
%%                   & \text{ for every } (K{:}\forall\as @.@ \oln{\tau}{n} -> T\;\as) \in \Sigma \\ 
%%                   & \text{ and } (J{:}\forall\as @.@ \oln{\tau}{m} -> S\;\as) \in \Sigma \\
%%  \textsc{AxDisjC} & \formula{(\forall \oln{x}{n} @.@ K(\ol{x}) \neq \unr \land K(\ol{x}) \neq \bad)} \\ 
%%                   & \text{ for every } (K{:}\forall\as @.@ \oln{\tau}{n} -> T\;\as) \in \Sigma \\ \\
%%  \textsc{AxAppA}  & \formula{\forall \oln{x}{n} @.@ f(\ol{x}) = app(f_{ptr},\xs)} \\
%%                   & \text{ for every } (f |-> \Lambda\as @.@ \lambda\oln{x{:}\tau}{n} @.@ u) \in P \\
%%  %% \textsc{AxAppB}  & \formula{\forall \oln{x}{n} @.@ K(\ol{x}) = app(\ldots (app(x_K,x_1),\ldots,x_n)\ldots)} \\
%%  %%                  & \text{ for every } (K{:}\forall\as @.@ \oln{\tau}{n} -> T\;\as) \in \Sigma \\
%%  \textsc{AxAppC}  & \formula{\forall x, app(\bad,x) = \bad \; /\ \; app(\unr,x) = \unr}    \\ \\
%%  %% Not needed: we can always extend partial constructor applications to fully saturated and use AxAppC and AxDisjC
%%  %% \textsc{AxPartA} & \formula{\forall \oln{x}{n} @.@ app(\ldots (app(x_K,x_1),\ldots,x_n)\ldots) \neq \unr} \\
%%  %%                  & \formula{\quad\quad \land\; app(\ldots (app(x_K,x_1),\ldots,x_n)\ldots) \neq \bad} \\
%%  %%                  & \text{ for every } (K{:}\forall\as @.@ \oln{\tau}{m} -> T\;\as) \in \Sigma \text{ and } m > n \\
%%  \textsc{AxPartB} & \formula{\forall \oln{x}{n} @.@ app(f_{ptr},\xs) \neq \unr} \\
%%                   & \formula{\quad\land\; app(f_{ptr},\xs) \neq \bad} \\
%%                   & \formula{\quad\land\; \forall \oln{y}{k} @.@ app(f_{ptr},\xs) \neq K(\ol{y})} \\
%%                   & \text{ for every } (f |-> \Lambda\as @.@ \lambda\oln{x{:}\tau}{m} @.@ u) \in P  \\
%%                   & \text{ and every } (K{:}\forall\as @.@ \oln{\tau}{k} -> T\;\as) \in \Sigma \text{ and } m > n  \\ \\ 
%%  \textsc{AxInj}   & \formula{\forall \oln{y}{n} @.@ \sel{K}{i}(K(\ys)) = y_i} \\ 
%%                   & \text{for every } (K{:}\forall\as @.@ \oln{\tau}{n} -> T\;\as) \in \Sigma \text{ and } i \in 1..n \\ \\
%% \end{array} \\
%% \ruleform{\Th{\Sigma}{P}^{\lcfZ}} \\ \\ 
%% \begin{array}{lll} 
%%  \textsc{AxCfA}   & \formula{\lcf{\unr} /\ \lncf{\bad}} \\
%%  \textsc{AxCfB}   & \formula{\forall \oln{x}{n} @.@ \lcf{K(\ol{x})} <=> \bigwedge\lcf{\ol{x}}} \\
%%                   & \text{ for every } (K{:}\forall\as @.@ \oln{\tau}{n} -> T\;\as) \in \Sigma
\end{array}\]


In the model, $\definable{\cdot}$ should be possible to define, as a
predicate on denotations. The disadvantage to this approach is that
arrow contracts will only hold for whatever is in your context, not
arbitrary expressions, which might be what we want, but might not be
modular enough.

The other {\em potential} problem (i.e. I have not yet checked) might be in the 
proof of admissibility of induction. 

And yet another potential problem is that as we incrementally extend our signature 
with new function definitions (and possibly contracts) previously defined contracts
may no longer hold. This is pretty bad for modularity.

\paragraph{Yet another possible solution}

A solution that seems somewhat more modular is based on the observation that, 
during the evaluation of a program there exists a {\em set} of terms (maybe infinite) that can
appear as arguments to other terms or functions. Our idea is to guard the arrow contracts so that
we do not quantify over any possible term (or denotation, in the translation) but rather only 
those that may appear as {\em arguments} in some application. We translate arrow contract as 
follows:
\[\begin{array}{l}
  \ctrans{\Sigma}{\Gamma}{e \in (x{:}\Ct_1) -> \Ct_2} = \\ 
  \qquad\qquad\quad 
\formula{\forall x @.@ arg(x) \land \ctrans{\Sigma}{\Gamma,x}{x \in \Ct_1} => \ctrans{\Sigma}{\Gamma,x}{e\;x \in \Ct_2}}
\end{array}\] 
where $arg(x)$ ensures that $x$ is the denotation of a term that will be passed as an argument to $e$. We'd need to define
a similar predicate on the evaluation relation, call it $Arg(e)$ and modify the program translation to thread the $arg(\cdot)$
predicate through. 

