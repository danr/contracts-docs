Our account so far has been largely informal.  How can we be sure
that if the FOL prover says ``Yes!  This FOL formula is provable'', then
the corresponding $\theLang$ program indeed satisfies the claimed contract?

To prove this claim we take a denotational approach.
Most of what follows is an adaptation of well-known techniques to
our setting and there are no surprises --- we refer the reader
to~\cite{winskel} or~\cite{benton+:coq-domains} for a short and modern
exposition of the standard methodology.

\subsection{Technical preliminaries}

We will assume a program $P$, well-formed in a signature $\Sigma$, so
that $\Sigma |- P$.
Given a signature $\Sigma$ we define a strict
bi-functor $F$ on complete partial orders (cpos), below:
%% For a well-formed signature $\Sigma$, we define the strict bi-functor on cpos, below,
%% assuming that $K_1\ldots K_k$ are all the constructors in $\Sigma$:
\[\begin{array}{lclll}
  F(D^{-},D^{+}) & = & ( \quad{\prod_{\ar_1}{D^{+}}} & K_1^{\ar_1} \in \Sigma \\
               & + & \;\quad\ldots                    & \ldots \\
               & + & \;\quad{\prod_{\ar_k}{D^{+}}} & K_k^{\ar_k} \in \Sigma \\
               & + & \;\quad(D^{-} =>_c D^{+}) \\
               & + & \;\quad\unitcpo_{bad} \quad )_{\bot}
\end{array}\]
\spj{Why bi-functor? Why not just functor?}
The bi-functor $F$ is the lifting of a big sum: that sum consists of
(i) products, one for each possible constructor (even across different data types), (ii) the continuous
function space from $D^{-}$ to $D^{+}$, and (iii) a unit cpo to denote @BAD@ values.
The notation $\prod_{n}{D}$ abbreviates $n$-ary products of cpos (the unit cpo $\unitcpo$ if $n = 0$).
The product and sum constructions are standard, but note that we use their non-strict versions.
The notation $C =>_c D$ denotes the cpo
induced by the space of continuous functions from the cpo $C$ to the cpo $D$. We use
the notation $\unitcpo_{bad}$ to
denote a single-element cpo -- the $bad$ subscript is just there for readability.
The notation $D_\bot$ is {\em lifting}.
% , which is a monad, equipped with the following two continuous
% functions.
% \[\begin{array}{l}
%    \retK   : D =>_c D_\bot \\
%    \bindK_{f : D =>_c E_\bot} : D_\bot =>_c E_\bot
% \end{array}\]
% with the obvious definitions.

Observe that we have dropped all type information from the source
language. The elements of the products corresponding to data
constructors are simply $D^{+}$ (instead of more a precise description
from type information) and the return types of data constructors are
similarly ignored. This is not to say that a more type-rich
denotational semantics is not possible (or desirable even) but this
simple denotational semantics turns out to be sufficient for
formalisation and verification.

%% for $\lambda$-abstractions and @BAD@. Observe that we have
%% Moreover, the following continuous operations are defined:
%% \[\begin{array}{l}
%%    \curry_{f : D\times E =>_c F} : D =>_c (E =>_C F) \\
%%    \eval : (E =>_c D)\times E =>_c D
%% \end{array}\]
%% for any cpos $D, E, F$.

Now we can define $D_{\infty}$ as the solution to this recursive domain equation
$$D_{\infty} \approx F( D_{\infty}, D_{\infty})$$
We can show that $D_{\infty}$ exists using the
standard {\em embedding-projection} pairs methodology. Moreover, we define the
value domain $V_{\infty}$ thus:
    \[V_{\infty} = \begin{array}[t]{ll}
             \quad\;{\prod_{\ar_1}{D_{\infty}}} & K_1^{\ar_1} \in \Sigma \\
             \; + \;\ldots                    & \ldots \\
             \; + \;{\prod_{\ar_k}{D_{\infty}}} & K_k^{\ar_k} \in \Sigma \\
             \; + \;(D_{\infty} =>_c D_{\infty}) \\
             \; + \;\unitcpo_{bad} \quad
    \end{array}\]
The following continuous functions also exist:
% , each being the inverse of the
% other (i.e. composing to the identity function on the corresponding cpo):
\[\begin{array}{rcl}
  \retK   & : & D =>_c D_\bot \\
  \bindK_{f : D =>_c E_\bot} & : & D_\bot =>_c E_\bot \\
  \rollK & : & (V_{\infty})_\bot =>_c D_{\infty} \\
  \unrollK & : &D_{\infty} =>_c (V_{\infty})_\bot
\end{array}\]
However in what follows we will always elide these functions to reduce clutter.

To denote elements of $V_{\infty}$ we use the following notation.
\begin{itemize}
\item $\injK{K}{d_1, \ldots, d_n}$ denotes the injection of
the $n$-ary product of $D_{\infty}$ into the component of the sum
$V_{\infty}$ corresponding to the $n$-ary constructor $K$.
\item $\injFun{d}$ is the injection of
an element of $D_{\infty} =>_c D_{\infty}$ into the function component of $V_{\infty}$
\item $\injBad$ is the unit injection into $V_{\infty}$.
\end{itemize}

%% We summarize the (standard) construction needed for the proof of Lemma~\ref{lem:rec-solution} based on embedding-projection pairs,
%% because some of its details will be useful later. Consider the chain of cpos $D_i$ defined as:
%% \[\begin{array}{lcl}
%%    D_0 & = & \{\bot\} \\
%%    D_{i+1} & = & F_{\Sigma}(D_i,D_i)
%% \end{array}\]
%% and moreover consider the corresponding {\em embeddings} $e_i : D_i =>_c D_{i+1}$ and
%% {\em projections} $p_i : D_{i+1} =>_c D_i$ defined as:
%% \[\begin{array}{lcl}
%%    e_0 & = & \dlambda d @.@ \bot_{D_1} \\
%%    p_0 & = & \dlambda d @.@ \bot_{D_0} \\
%%    e_{i+1} & = & F_{\Sigma}(p_i,e_i) \\
%%    p_{i+1} & = & F_{\Sigma}(e_i,p_i)
%% \end{array}\]
%% The following is an easy fact to prove:
%% \begin{lemma}
%% For every $i$ and $x$ element of $D_i$ we have $p_i\cdot e_i(x) = x$. For
%% every $y$ element of $D_{i+1}$ we have that $e_i\ cdot p_i(y) \sqsubseteq y$.
%% \end{lemma}
%% Consider now the cpo defined by the carrier set
%%    \[ \{ x \in \Pi_{i \in \omega}D_i \;\mid\; x_n = p_n(x_{n+1}) \} \]
%% and the pointwise order induced by the order in each $D_i$, and $\bot$ element the
%% infinite tuple of the corresponding $\bot$ elements. This cpo {\em is} going to be the
%% set $D_{\infty}$. To prove this we need som more definitions. Let $j_{n,m} : D_n =>_c D_m$
%% be defined as:
%% \[\begin{array}{lcl}
%%    j_{n,m}(d) & = & \left\{\begin{array}{ll}
%%                              e_{m-1}\cdot\ldots \cdot e_n(d) & n < m \\
%%                              d                        & n = m \\
%%                              p_{n-1}\cdot\ldots \cdot p_n(d) & n > m
%%                           \end{array}\right.
%% \end{array}\]
%% and define $j_i : D_i =>_c D_\infty$ as:
%% \[\begin{array}{lcl}
%%    j_i(d) & = & \langle j_{i,0}(d),j_{i,1}(d),\ldots \\
%% \end{array}\]
%% We can easily show that $j_i$ and $\pi_i$ (the $i$-th projection from a tuple) form
%% an embedding-projection pair:
%% \begin{lemma}
%% For all $i$ and $x$ element of $D_{\infty}$ it is $j_i(\pi_i(x)) \sqsubseteq x$.
%% For all $y$ element of $D_{i}$ it is $\pi_i(j_i(y)) = y$.
%% \end{lemma}
%% The most important theorem is that the limit of $j_i\cdot\pi_i$ is the identity.
%% \begin{lemma}\label{lem:id-sqcup}
%% For all $x$ elements of $D_{\infty}$ it is $\sqcup(j_i\cdot\pi_i)(x) = x$.
%% \end{lemma}
%% \begin{proof} We have one direction by the previous lemma and least upper bounds.
%% So the hard direction is to show that $\sqcup(j_i\cdot\pi_i)(x) \sqsupseteq x$.
%% We know that:
%% \[       \pi_n(j_n(\pi_n(d))) = \pi_n(d) \]
%% by unfolding definitions and we know that $\sqcup(j_n\cdot\pi_n) \sqsupseteq j_n\cdot\pi_n$
%% since it is a least upper bound. By monotonicity we then get that
%% \[      \pi_n(\sqcup(j_n\cdot\pi_n)(d)) \sqsupseteq \pi_n(d) \]
%% but that holds for ever $n$ which means that:
%% \[       \sqcup(j_n\cdot\pi_n)(d) \sqsupseteq d \]
%% since $\sqsubseteq$ on $D_\infty$ is defined by the conjuction of the
%% pointwise $\sqsubseteq$ on each $D_i$.
%% \end{proof}
%% Now let us consider the chain
%% \[ F_{\Sigma}(D_0,D_0) \quad F_{\Sigma}(D_1,D_1) \quad \ldots \]
%% and the pointed cpo $F_{\Sigma}(D_{\infty},D_{\infty})$ we have that:
%% \[\begin{array}{lcl}
%%     F_{\Sigma}(p_i,e_i)   & : & F(D_i,D_i) =>_c F(D_{i+1},D_{i+1}) \\
%%     F_{\Sigma}(\pi_i,j_i) & : & F(D_i,D_i) =>_c F(D_{\infty},D_{\infty})
%% \end{array}\]
%% we will show that there exists an isomorphism $D_\infty \cong F_{\Sigma}(D_\infty,D_\infty)$.

%% \begin{lemma} Define functions $\roll = \sqcup(j_{i+1}\cdot F_{\Sigma}(j_i,\pi_i))$ and
%% $\unroll = \sqcup (F_{\Sigma}(\pi_i,j_i)\cdot \pi_{i+1})$. They form the required isomorphism
%% $D_\infty \cong F_{\Sigma}(D_\infty,D_\infty)$.
%% \end{lemma}
%% This lemma concludes the proof of Lemma~\ref{lem:rec-solution}.

%% The following fact will be extremely useful in establishing the existence of solutions
%% to recursive equations {\em over} the recursively defined domain via approximating the
%% denotations. Let us call $\rho_i = j_i\cdot\pi_i$.

%% \begin{theorem}\label{lem:min-inv-reqs} The following are true:
%% \begin{itemize}
%%    \item $\unroll \cdot \rho_{i+1} \cdot \roll = F_{\Sigma}(\rho_i,\rho_i)$
%%    \item $\sqcup\rho_i(d) = d$, for all elements $d$ of $D_{\infty}$.
%% \end{itemize}
%% \end{theorem}


\subsection{Denotational semantics of expressions and programs}

\begin{figure}
$$
\setlength{\arraycolsep}{2pt}
\begin{array}{c}
\begin{array}{rcl}
\multicolumn{3}{l}{\interp{e}{\cdot}{\cdot} : (\FVarCpo =>_c D_{\infty}) \times (\VarCpo =>_c D_{\infty}) =>_c D_{\infty}} \\
  \interp{x}{\sigma}{\rho} & = & \rho(x) \\
  \interp{f\;[\taus]}{\sigma}{\rho} & = & \sigma(f) \\
  \interp{K\;[\taus]\;(\ol{e})}{\sigma}{\rho} & = & \roll{\ret{\injK{K}{\ol{\interp{e}{\sigma}{\rho}}}}} \\
  \interp{e_1\;e_2}{\sigma}{\rho} & = & \dapp(\interp{e_1}{\sigma}{\rho}, \interp{e_2}{\sigma}{\rho}) \\
  \interp{@BAD@}{\sigma}{\rho} & = & \roll{\ret{\injBad}} \\[2mm]
\hline \\
\multicolumn{3}{l}{\interp{u}{\cdot}{\cdot} : (\FVarCpo =>_c D_{\infty}) \times (\VarCpo =>_c D_{\infty}) =>_c D_{\infty}} \\[1mm]
  \interp{e}{\sigma}{\rho} & = & \interp{e}{\sigma}{\rho} \\[1mm]
  \interp{@case@\;e\;@of@ \ol{ K\;\ys -> e_K}}{\sigma}{\rho}
          & = & \unroll{\interp{e_K}{\sigma}{\rho, \ol{y |-> d}}} \\
          & & \quad \text{if $\unroll{\interp{e}{\sigma}{\rho}} = \injK{K}{\ol{d}}$} \\
          & & \quad \text{and $K$ is a @case@ branch} \\
          & = & \injBad \quad \text{if}\; \unroll{\interp{e}{\sigma}{\rho}} = \injBad \\
          & = & \bot \quad \text{otherwise}
\end{array} \\
\hline \\
\begin{array}{rcl}
\multicolumn{3}{l}{\dbrace{P} : (\FVarCpo =>_c D_{\infty}) =>_c (\FVarCpo =>_c D_{\infty})}  \\[1mm]
\dbrace{P}_{\sigma} f & = & \roll{\ret{\injFun{\dlambda d_1 @.@ \ldots
       \roll{\ret{\injFun{\dlambda d_n @.@ \interp{u}{\sigma}{\ol{x |-> d}}}}}\ldots}}} \\
 && \quad \text{ if } (f\;\ol{a}\;\ol{x} = u) \in P \\
          & = & \bot \quad \text{otherwise}
\end{array}
\end{array}
$$
\caption{Denotational semantics of $\theLang$} \label{f:denot}
\end{figure}

Figure~\ref{f:denot} gives the denotational interpretations
of expressions $e$, right hand sides $u$, and programs $P$, in
terms of the domain-theoretic language and combinators we have defined.

First, the denumerable set of term variable names $x_1,\ldots$ induces a discrete
cpo $\VarCpo$  and the denumerable set of function variable names $f_1,\ldots$ induces a discrete
cpo $\FVarCpo$. We define, {\em semantic term environments} to be the cpo $(\VarCpo =>_c D_{\infty})$,
and {\em semantic function environments} to be the cpo $(\FVarCpo =>_c D_{\infty})$.

Figure~\ref{f:denot} defines the denotational semantics of expressions
$\dbrace{e}$ as a continuous map from these two environments to $D_{\infty}$.
It is entirely straightforward except for application, which depends on
the continuous function $\dapp : D_\infty \times D_\infty =>_c D_\infty$,
defined thus\footnote{
A small technical remark: we write
the definition with pattern matching notation
$\dapp(d,a)$ (instead of using $\pi_1$ for projecting
out $d$ and $\pi_2$ for projecting out $a$) but that is fine,
since $\times$ is not a lifted construction.
Also note that we are, as advertised, suppressing uses
of $\bindK$, $\rollK$, etc.
}:
{\setlength{\arraycolsep}{2pt}
\[\begin{array}{rcll}
  \dapp(d,a) & = & d_f(a)    & \text{if}\; d = \injFun{d_f} \\
             & = & \injBad & \text{if}\; d = \injBad \\
             & = & \bot    & \text{otherwise}
\end{array}\]}%
That is, application applies the payload $d_f$
if the function $d$ comes
from the appropriate component of $V_{\infty}$, propagates $\injBad$,
and otherwise returns $\bot$.  %% Indeed application, partial
%% application, and currying are all definable in cpos of continuous
%% functions, so we will be using $\lambda$-calculus notation for our
%% domain theory, as above.

The semantics of right-hand sides $\dbrace{u}$ is defined similarly.  The
semantics of a @case@ expression is the semantics of the matching branch,
if one exists. Otherwise, like application, it propagates $\injBad$.
In all other cases we return $\bot$, not $\injBad$;
all the missing cases can only be constructors
of different datatypes than the datatype that $K$ belongs to, because
all @case@ expressions are complete (Section~\ref{s:syntax}).
This treatment corresponds directly to our treatment of $unr$ in Section~\ref{s:case-fol}.

Finally, Figure~\ref{f:denot} gives the semantics of a program $P$, which should
be read recalling its syntax in Figure~\ref{fig:syntax}.
Since $\dbrace{P})$ is continuous, its limit exists and is an element of the
cpo $\FVarCpo =>_c D_{\infty}$.

\begin{definition}
We will refer to the limit of the $\dbrace{P}$ as $\dbrace{P}^{\infty}$ in what follows.
Moreover, to reduce notational overhead below, for a program with no free variables we
will use notation $\dbrace{e}$ to mean $\interp{e}{\dbrace{P}^\infty}{\cdot}$, and
$\dbrace{e}_\rho$ to mean $\interp{e}{\dbrace{P}^\infty}{\rho}$
\end{definition}

% ------------------ omit this -------------------
\begin{comment}
Types do not matter at all for our denotational semantics.
\begin{lemma}[Type irrelevance]
It is the case that $\interp{u}{\sigma}{\rho} = \interp{u[\ol{\tau}/\as]}{\sigma}{\rho}$
for any type substitution of variable $\as$ to types $\taus$.
%% and $\interp{e}{\sigma}{\rho} = \interp{e[\ol{\tau}/\as]}{\sigma}{\rho}$.
\end{lemma}
%% \begin{proof} Straightforward induction. \end{proof}
The following is an essential lemma for establishing the soundness of denotational semantics:
\begin{lemma}[Substitutivity]
If $\Sigma;\Delta,x{:}\tau |- e : \tau$ and $\rho$ is a semantic environment
and $\Sigma;\Delta |- e' : \tau'$ then
\[ \interp{e}{\sigma}{\rho,x |-> \interp{e'}{\sigma}{\rho}} = \interp{e[e'/x]}{\sigma}{\rho} \]
and if $\Sigma;\Delta,x{:}\tau |- u : \tau$ then
\[ \interp{u}{\sigma}{\rho,x |-> \interp{e'}{\sigma}{\rho}} = \interp{u[e'/x]}{\sigma}{\rho} \]
\end{lemma}
\end{comment}
% ------------------ omit this -------------------
Although we have not presented a formal operational semantics, we state the usual
soundness and adequacy results:
\begin{theorem}[Soundness and adequacy]\label{thm:adequacy}
Assume $\Sigma |- P$ and $u$ with no free term variables. Then (i)~if $P |- u \Downarrow v$ then $\dbrace{u} = \dbrace{v} $; and (ii)~if $\unroll{\dbrace{e}} \neq \bot$, then
$\exists v$ such that $P |- e \Downarrow v$.
\end{theorem}
\begin{comment}
The proof of this theorem is also standard domain theory so we only sketch the
high-level road-map: The proof proceeds by defining a {\em logical relation} between
semantics and syntax, via the use of a bi-functor on admissible relations between elements
of $D_\infty$ and closed expressions, and using minimal
invariance~\cite{pitts-rel-domains} to show that this
bi-functor has a fixpoint. Adequacy then follows from the {\em fundamental theorem} of this
logical relation, which asserts that every expression is related to its denotation.
\end{comment}
%% To do this we define a {\em logical relation} first between semantics
%% and syntax. Let $Rel \subseteq D_\infty \times Expr$ be the space of
%% {\em admissible} and {\em equality-respecting} relations between
%% denotations and closed (non-necessarily well-typed) terms. Some explanations:
%% \begin{itemize}
%%   \item $R \in Rel$ is {\em admissible} iff whenever
%%   $R(d_i,e)$ for every element of a chain $d_1\ldots$ then also $R(\sqcup_{\omega}d_i,e)$.
%%   \item $R \in Rel$ is {\em equality-respecting} iff for every
%%   $R(d,e)$ and $d' = d$ (according to the equality on $D_{\infty}$) it also is
%%   $R(d',e)$.
%% \end{itemize}

%% Let use define the following bi-functor on the space of $Rel$ relations:
%% {\setlength{\arraycolsep}{2pt}
%% \[\begin{array}{lcl}
%%    F_{P}(R^{-},R^{+}) & = & \{ (d,e)\;\mid\;\forall \ol{d} @.@ \unroll(d) = \ret(\inj{K_1^\ar}\langle\oln{d}{\ar}\rangle) ==> \\
%%                    &   & \quad \exists \oln{e}{\ar} @.@ P |- e \Downarrow K_1[\taus](\ol{e}) \land (d_i,e_i) \in R^{+} \} \\
%%                    & \cup & \ldots \\
%%                    & \cup & \{ (d,e)\;\mid\;\forall d_0 @.@ \unroll(d) = \ret(\inj{->}(d_0)) ==> \\
%%                    &   & \quad \exists v @.@ P |- e \Downarrow v \;\land \\
%%                    &  & \quad\quad \forall (d',e') \in R^{-} @.@ (\dapp(d,d'),v\;e') \in R^{+} \}  \\
%%                    & \cup & \{ (d,e)\;\mid\; \unroll(d) = \ret(\inj{bad}(1)) ==> \\
%%                    &   & \quad P |- e \Downarrow @BAD@ \}
%% \end{array}\]}

%% \begin{lemma} There exists negative and a positive fixpoint of $F_{P}$ and they coincide: let us call this
%% $F_{P}^\infty$ -- it is isomorphic to $F_{P}(F_P^\infty,F_P^\infty)$.
%% \end{lemma}
%% \begin{proof}
%% We can follow the standard roadmap described in the work of Pitts to show this, taking
%% advantage of the approximation on every element of $D_{\infty}$ given in
%% Lemma~\ref{lem:min-inv-reqs}.
%% \end{proof}

%% \begin{lemma}\label{lem:bot-in-fix}
%% $(\roll(\bot),e) \in F_{P}^\infty$. \end{lemma}

%% \begin{lemma}\label{lem:eval-respecting}
%% If $(d,e) \in F_{P}^\infty$ and $P |- e \Downarrow v$ then $(d,v) \in F_{P}^\infty$.
%% Moreover, if $(d,v) \in F_{P}^\infty$ and $P |- e \Downarrow v$ then $(d,e) \in F_{P}^\infty$.
%% \end{lemma}
%% \begin{proof}
%% For the first part,
%% if $(d,e) \in F_{P}^\infty$ then $(d,e) \in F_{P}(F_{P}^\infty,F_{P}^\infty)$.
%% By the definition of $F_{P}(\cdot,\cdot)$ and by rule \rulename{EVal} the
%% result follows. The second part is a similar case analysis.
%% \end{proof}

%% \begin{lemma}[Fundamental theorem for expressions]\label{lem:fund-thm-exp}
%% For all $\sigma$ such that $(\sigma(f),f\;[\taus]) \in F_P^\infty$ and
%% all $\rho$ and vectors of closed terms $\ol{e}$ such that $(\rho(x_i),e_i) \in F_P^\infty$
%% and all $e$ with free variables in $\ol{x}$ it must be the case
%% that $(\interp{e}{\sigma}{\rho},e[\ol{e}/\ol{x}]) \in F_P^\infty$.
%% \end{lemma}
%% \begin{proof} The proof is by induction on $e$.
%% \begin{itemize}
%%   \item Case $e = x_i$ for some $x_i \in \ol{x}$ follows by the assumptions.
%%   \item Case $e = f\;[\taus]$ for some $f$ follows by assumptions.
%%   \item Case $e = K^\ar[\taus](\oln{e'}{\ar})$. By induction hypothesis we
%%   have that for each $e'_i$ it is $(\interp{e'_i}{\sigma}{\rho},e'_i[\ol{e}/\ol{x}]) \in F_P^\infty$ and
%%   by using rule \rulename{EVal} we are done since
%%       \[ \interp{K^{\ar}[\taus](\oln{e'}{\ar})}{\sigma}{\rho} = \roll(\ret(\inj{K}(\langle\ol{\interp{e'_i}{\sigma}{\rho}}\rangle))) \]
%%   \item Case $e = @BAD@$ follows by unfolding definitions.
%%   \item Case $e = e_1\;e_2$. We need to show that
%%      \[ (\interp{e_1\;e_2}{\sigma}{\rho},e_1[\ol{e}/\ol{x}]\;e_2[\ol{e}/\ol{x}]) \in F_P^\infty \]
%%   By induction hypothesis we have that
%%   \begin{eqnarray}
%%      (\interp{e_1}{\sigma}{\rho},e_1[\ol{e}/\ol{x}]) \in F_P^\infty \label{eqn:e1} \\
%%      (\interp{e_2}{\sigma}{\rho},e_2[\ol{e}/\ol{x}]) \in F_P^\infty \label{eqn:e2}
%%   \end{eqnarray}
%%   Equation~\ref{eqn:e1} gives four cases: First, if $\interp{e_1}{\sigma}{\rho} = \roll(\bot)$ then we
%%   are done since $\dapp(\bot,\_) = \roll(\bot)$ and $(\roll(\bot), e_1\;e_2) \in F_P^\infty$ by Lemma~\ref{lem:bot-in-fix}.
%%   Second, if $\interp{e_1}{\sigma}{\rho} = \roll(\ret(\inj{K}(\langle\ol{d}\rangle)))$ for some constructor
%%   $K$ then $\dapp(\interp{e_1}{\sigma}{\rho},\_) = \roll(\bot)$ and by similar reasoning as above we are done.
%%   Third, if $\interp{e_1}{\sigma}{\rho} = \roll(\ret(\inj{bad}(1)))$ then it must be that $P |- e_1[\ol{e}/\ol{x}] \Downarrow @BAD@$ by
%%   induction hypothesis, and by rule \rulename{EBadApp} we know that $P |- e_1[\ol{e}/\ol{x}]\;e_2[\ol{e}/\ol{x}] \Downarrow @BAD@$ hence,
%%   by Lemma~\ref{lem:eval-respecting} we are done. The final case is the interesting one, where
%%   $\interp{e_1}{\sigma}{\rho} = \roll(\ret(\inj{->}(d_0)))$ in which case by induction hypothesis we know that
%%   $(d_0(\interp{e_2}{\sigma}{\rho}), v\;e_2[\ol{e}/\ol{x}]) \in F_P^\infty$ for $P |- e_1[\ol{e}/\ol{x}] \Downarrow v$. But we know that
%%   $v\;e_2[\ol{e}/\ol{x}]$ evaluates to a value {\em iff} $e_1[\ol{e}/\ol{x}]\;e_2[\ol{e}/\ol{x}]$ evaluates to a value and by
%%   Lemma~\ref{lem:eval-respecting} we are done.
%% \end{itemize}
%% \end{proof}


%% \begin{lemma}[Fundamental theorem for top-level expressions]\label{lem:fund-thm-case}
%% For all $\sigma$ such that $(\sigma(f),f\;[\taus]) \in F_P^\infty$ and
%% all $\rho$ and vectors of closed terms $\ol{e}$ such that $(\rho(x_i),e_i) \in F_P^\infty$
%% and all $u$ with free variables in $\ol{x}$ it must be the case
%% that $(\interp{u}{\sigma}{\rho},u[\ol{e}/\ol{x}]) \in F_P^\infty$.
%% \end{lemma}
%% \begin{proof} By induction on $u$. If $u$ is a term $e$ then we are immediately done
%% by Lemma~\ref{lem:fund-thm-exp}. If $u = @case@\;e\;@of@\;\ol{K\;\ys -> e'}$ then the
%% result follows by appealing to the induction hypothesis for $e$ and performing a case
%% analysis on $\interp{e}{\sigma}{\rho}$ -- in the interesting case we appeal further to
%% Lemma~\ref{lem:fund-thm-exp} for a matching $e_K$ and the evaluation-respecting lemma,
%% Lemma~\ref{lem:eval-respecting}.
%% \end{proof}

%% Finally, for the recursive functions environment $P$ we prove the following.
%% \begin{lemma} For any $f$, $(\dbrace{P}^\infty(f),f\;[\taus]) \in F_P^\infty$. \end{lemma}
%% \begin{proof}
%% Since $F_P^\infty$ is itself an {\em admissible} relation, we need only prove that:
%% \[ \forall i @.@ \forall f @.@ (\dbrace{P}^i(f),f\;[\taus]) \in F_P^\infty \]
%% which we do by induction on $i$. For $i = 0$ we are immediately done by Lemma~\ref{lem:bot-in-fix}.
%% Let us assume that the property is true for $i$. We must show it is true for $i+1$. That is,
%% we must show that $(\dbrace{P}^{i+1}(f),f\;[\taus]) \in F_P^\infty$. Hence, if $f$ has arity $n$, by
%% the definition of $\dbrace{P}$ and the definition of the logical relation it is enough to show that
%% for all $(\oln{d}{n},\oln{e}{n}) \in F_P^\infty$ it must be the case that
%% \[    (\interp{u}{\dbrace{P}^i}{\ol{x |-> d}}, u[\ol{e}/\ol{x}]) \in F_P^\infty \]
%% for $f |-> (\Lambda\as @.@ \lambda\oln{x{:}\tau}{n} @.@ u) \in P$. But that follows by
%% Lemma~\ref{lem:fund-thm-case}, since by induction hypothesis it is the case that for
%% every $f$ we have $(\dbrace{P}^i(f),f\;[\ol{\tau}]) \in F_P^\infty$.
%% \end{proof}

%% \begin{corollary}\label{cor:fund-thm-top}
%% For every closed expression $e$ in $P$ (not-necessarily well-typed) we have that
%% $(\interp{e}{\dbrace{P}^\infty}{\cdot}, e) \in F_P^\infty$.
%% \end{corollary}

%% From this corollary, adequacy follows by unfolding definitions.

%% \begin{corollary}[Model-based-reasoning]
%% If $\Sigma |- P$ and and $e_1$ contains no free term variables and $e_2$ contains no free term
%% variables.

%% $\Sigma;\cdot |- e_1 : \tau$ and $\Sigma;\cdot |- e_2 : \tau$,
%% then for every closed $e$ such that $\Sigma |- P$ and $\Sigma;\cdot |- e : \tau -> Bool$,
%% if $\interp{e_1}{\dbrace{P}^\infty}{\cdot} = \interp{e_2}{\dbrace{P}^\infty}{\cdot}$ then
%% $P |- e\;e_1 \Downarrow$ iff $P |- e\;e_2 \Downarrow$.
%% \end{corollary}
%% \begin{proof}
%% For one direction assume that $P |- e\;e_1 \Downarrow w$, hence by computational soundness it must be that
%% $\interp{e\;e_1}{\dbrace{P}^\infty}{\cdot} = \roll(\ret(d))$. By assumptions we must also
%% have that $\interp{e\;e_2}{\dbrace{P}^\infty}{\cdot} = \roll(\ret(d))$. By the fundamental theorem
%% we know that
%% \[ (\interp{e\;e_2}{\dbrace{P}^\infty}{\cdot}, e\;e_2) \in F_{P}^\infty \]
%% and hence $P |- e\;e_2 \Downarrow$. The other direction is symmetric.
%% \end{proof}

\subsection{Denotational semantics of contracts} \label{s:den-sem-contracts}

\begin{figure}
$$
\setlength{\arraycolsep}{1pt}
\begin{array}{rcl}
\multicolumn{3}{c}{
\ruleform{\dbrace{\Ct}_{\rho} \subseteq D_{\infty}} }
\\ \\
\dbrace{x \mid e}_\rho
  & =  & \{ d \mid \unroll{d} = \bot \, \lor \, \unroll{\dbrace{e}_{\rho,x|->d}} 
                \in \{ \ret{\injKZ{True} , \bot} \} \} 
\\[1em]
\dbrace{(x{:}\Ct_1) -> \Ct_2}_{\rho}
 & = & \{ d \mid
           \forall d' \!\in\! \dbrace{\Ct_1}_\rho.
           \dapp(d,d') \in \dbrace{\Ct_2}_{\rho,x|->d'}
           \}
\\[1em]
\dbrace{\Ct_1 \& \Ct_2}_\rho
 & = & \{ d | d \in \dbrace{\Ct_1}_\rho /\ d \in \dbrace{\Ct_2}_\rho \}
\\[1em]
\dbrace{\CF}_\rho & = &  \Fcf^{\infty}  \\
\multicolumn{3}{l}{\text{where}} \\
   F_{\lcfZ}^{\infty} & = & \{ \bot \} \\
                   & \cup & \{\;\injK{K}{\ol{d}} \mid K^n \in \Sigma,\; d_i \in F_{\lcfZ}^{\infty} \} \\
                   & \cup & \{\;\injFun{d} \mid \forall d' \in F_{\lcfZ}^{\infty}.\; d(d') \in F_{\lcfZ}^{\infty} \}
\end{array}
$$
\caption{Denotations of contracts} \label{f:den-sem-contracts}
\end{figure}

Now we are ready to say formally what it means for a function to satisfy a contract.
We define the semantics of a contract as the set of denotations that satisfy it:
\[              \dbrace{\Ct}_\rho \subseteq D_\infty  \]
where $\Ct$ is a contract with free term variables in the semantic environment $\rho$.
Figure~\ref{f:den-sem-contracts} gives the definition of this function.
A base contract $\{x \mid e\}$ is satisfied by $\bot$ or
or by a computation that causes the predicate $e$ to become $\bot$
or return \True\footnote{
In previous
work \cite{xu+:contracts} the base contract also
required {\em crash-freedom}.
We changed this choice only for reasons of taste; both choices
are equally straightforward technically.}.
%% The reason for introducing the
%% possibility of $\bot$ either for the expression or for the predicate
%% is associated with {\em admissibility} of induction, a topic that we
%% return to in Section~\ref{sect:induction}.
The denotation of an arrow contract, and of conjunction, are both straightforward.

The $\CF$ contract is a little harder. Intuitively an expression is crash-free iff it cannot
crash if plugged into an arbitrary crash-free context. Of course this is a
self-referential definition so how do we know it makes sense? The original paper
\cite{xu+:contracts} specified that an expression is crash free iff it
cannot crash when plugged into a context that {\em syntactically} does not contain the
@BAD@ value. This is a reasonable definition in the operational semantics world, but
here we can do better because we are working with elements of $D_\infty$. Using
techniques developed by Pitts~\cite{pitts-rel-domains} we can define crash-freedom denotationally as the greatest solution
\dr{The greatest solution should be $D^{\infty}$? Don't you mean the smallest solution?}
to the recursive equation for $F_{\lcfZ}^{\infty}$ in Figure~\ref{f:den-sem-contracts}. Technically,
since the equation involves mixed-variance recursion, to show that such a fixpoint exists we have
to use minimal invariance.
In addition we get the following, which will be useful later on for induction:
\begin{lemma}\label{lem:cf-admissible}
$\bot \in F_{\lcfZ}$ and $F_{\lcfZ}^{\infty}$ is admissible, that is if all elements of a chain are in
$F_{\lcfZ}^{\infty}$ then so is its limit.
\end{lemma}

%% Using the techniques developed by Pitts we may define a recursive predicate for crash-freedom,
%% using the following strict bi-functor on admissible sets of
%% denotations $S^{-},S^{+} \subseteq D_\infty$.
%% {\setlength{\arraycolsep}{2pt}
%% \[\begin{array}{rcl}
%%    F_{\lcfZ}(S^{-},S^{+}) & = & \{\;d\;\mid\;\unroll{d} \neq \ret{\inj{bad} 1} \text{ and } \\
%%                       &    & \quad \text{for all } \ol{d}, \\
%%                       &    & \quad\quad\text{ if } \unroll{d} = \ret{\inj{K_1^\ar}\langle\oln{d}{\ar}\rangle} \\
%%                       &    & \quad\quad\text{ then } \ol{d} \in S^{+} \} \\
%%                    & \cup & \ldots \\
%%                    & \cup & \{\;d\;\mid\;\unroll{d} \neq \ret{\inj{bad} 1} \text{ and } \\
%%                    &      & \quad \text{for all } d_0, \\
%%                    &      & \quad\quad\text{ if } \unroll{d} = \ret{\inj{->}(d_0)} \text{ then } \\
%%                    &      & \quad\quad\text{ for all }\;d' \in S^{-} \text{ it is } d_0(d') \in S^{+} \}  \\
%% \end{array}\]}
%% The intersection of admissible sets is admissible so the $\Fcf$ bi-functor has a negative and positive fixpoint, and by minimal
%% invariance they coincide (one direction follows by Tarski-Knaster, the other can be inductively proved using the fact that the
%% lub of the chain of embedding-projections is the identity and the fact that this
%% functor preserves admissibility for the covariant argument). Let us call this admissible set $\Fcf^{\infty} \subseteq D_{\infty}$.

%% We may now define the interpretation of $\CF$ contracts using this fixpoint:
%% \[     \dbrace{\CF}_\rho(d) \text{ iff }  d \in \Fcf^{\infty}    \]

%% %% We may also define the denotational semantics of contracts, below. We assume again
%% %% that there is a program $P$, well-formed in a signature $\Sigma$. In the definition,
%% %% $\rho$ is a semantic environment.

%% %% \begin{definition}[Denotational semantics of contracts]
%% %% \[\begin{array}{l}
%% %%     \dbrace{x \mid e}_\rho(d) \text{ iff } \\
%% %%         \quad \unroll(d) = \bot \text{ or }
%% %%         \unroll(\interp{e}{\dbrace{P}^\infty}{\rho,x|->d}) = \bot\;\text{ or } \\
%% %%         \quad \unroll(\interp{e}{\dbrace{P}^\infty}{\rho \uplus x|->d}) = \ret(\inj{\True} 1) \\ \\
%% %%     \dbrace{(x{:}\Ct_1) -> \Ct_2}_{\rho}(d) \text{ iff } \\
%% %%         \quad \text{for all } d_x \in D_\infty \\
%% %%         \quad\quad \text{if }
%% %%                      \dbrace{\Ct_1}_\rho(d_x)\text{ then }
%% %%                      \dbrace{\Ct_2}_{\rho,x|->d_x}(\dapp(d,d_x)) \\ \\
%% %%     \dbrace{\CF}_\rho(d) \text{ iff }  d \in \Fcf^{\infty} \\  \\
%% %%     \dbrace{\Ct_1 \& \Ct_2}_\rho(d) \text{ iff }
%% %%        \dbrace{\Ct_1}_\rho(d) \text{ and }
%% %%        \dbrace{\Ct_2}_\rho(d)
%% %% \end{array}\]
%% %% For a closed contract $\Ct$ we will use notation
%% %% $\dbrace{\Ct}$ for its denotation in the empty
%% %% semantic environment.
%% %% \end{definition}

%% %% To the extend that in the end we are only interested in base contracts, giving a
%% %% denotational semantics of full-higher-order contracts is not really interesting
%% %% but we do this anyway. For a given denotation $d$, we define the
%% %% predicate $\interp{\Ct}{\dbrace{P}^\infty}{\rho}(d)$ by recursion on the structure
%% %% of the contract $\Ct$, such that:


\subsection{Soundness of the logic translation}  \label{s:soundness}

We have developed a formal semantics for expressions as well as contracts, so it is time we
see how we can use this semantics to show that our translation to first-order logic is sound
with respect to this semantics.

Our plan is to give an interpretation (in the FOL sense of the term) to our 
translated FOL terms, using the carrier set $D_\infty$ as our model. 
Happily this is straightforward to do:
\[\begin{array}{rcl}
   \linterp{f(\ol{t})} & = & \dapp(\dbrace{f},\ol{\linterp{t}}) \\
   \linterp{app(t_1,t_2)}     & = & \dapp(\linterp{t_1}, \linterp{t_2}) \\
   \linterp{f_{ptr}}  & = & \dbrace{f} \\
   \linterp{K(\ol{t})} & = & \roll{\ret{\injK{K}{\ol{\linterp{t}}}}} \\
   \linterp{\sel{K}{i}(t)} & = &  d_i \quad \text{if}\; \linterp{t} = \injK{K}{\ol{d}} \\
                           & = & \bot \quad \text{otherwise} \\
  \linterp{unr}       & = & \bot \\
  \linterp{bad}       & = & \injBad
\end{array}\]

The essential soundness theorem that states that our interpretation makes sense is
the following.
\begin{theorem}[Interpretation respects denotations]\label{thm:interp-respect}
Assume that $\Sigma |- P$ and expression $e$ does not contain any free variables.
Then, if $\etrans{}{\cdot}{e} = t$ then $\linterp{t} = \dbrace{e}$.
\end{theorem}
The proof is an easy induction on the size of the term $e$.

Our soundness results are expressed with the following theorem

\begin{theorem}\label{thm:models-inf}
If $\Sigma |- P$ then $\langle D_{\infty},{\cal I}\rangle \models \Th \land \ptrans{\Sigma}{P}$
\end{theorem}
%% \begin{theorem}\label{thm:models-defs}
%% If $\Sigma |- P$ then $\langle D_{\infty},{\cal I}\rangle \models \dtrans{\Sigma}{P}$.
%% \end{theorem}
%% \begin{theorem}\label{thm:models-cf} $\langle D_\infty,{\cal I}\rangle \models \Th_\lcfZ$.
%% \end{theorem}
As a corollary we get our ``guiding principle'' from the introduction.
\begin{corollary}\label{cor:guiding-principle}
Assume that $\Sigma |- P$ and $e_1$ and $e_2$ contain no free term variables. The following
are true:
\begin{itemize*}
  \item $\dbrace{e_1} = \dbrace{e_2}$ iff ${\cal I}(\etrans{}{}{e_1}) = {\cal I}(\etrans{}{}{e_2})$.
  \item If $\Th \land \ptrans{\Sigma}{P} |- \etrans{}{}{e_1} = \etrans{}{}{e_2}$ then $\dbrace{e_1} = \dbrace{e_2}$.
\end{itemize*}
\end{corollary}
\begin{proof} The first part follows directly from Theorem~\ref{thm:interp-respect}.
For the second part the left-hand side implies that $\etrans{}{}{e_1}$ and $\etrans{}{}{e_2}$ are
equal in all models of $\Th \land \ptrans{\Sigma}{P}$, in particular (using Theorem~\ref{thm:models-inf})
by $\langle D_{\infty},{\cal I}\rangle$ and by the first part the case is finished.
\end{proof}


\begin{theorem}\label{thm:den-contr-satisfaction} Assume that $e$ and $\Ct$ contain no free
term variables. Then the FOL translation of the claim $e \in \Ct$ holds in the model
if and only if the denotation of $e$ is in the semantics of $\Ct$.  Formally:
$$\langle D_\infty,{\cal I}\rangle \models \ctrans{}{\cdot}{e \in \Ct}
  \;\; \Leftrightarrow \;\; \dbrace{e} \in \dbrace{\Ct}
$$
\end{theorem}

% What does the last result tell us about the actual execution of the program $e$? In this
% paper we have elided a discussion about the operational semantics but we discuss this
% in further detail in Section~\ref{sect:discussion}.

\paragraph{Completeness of axiomatisation}

The $D_\infty$ domain has a complex structure and there are many more facts that
hold about elements of $D_\infty$ that are not reflected in any of our axioms in $\Th$.
For instance, here are some admissible axioms that are valid:
\[\begin{array}{l}
    \formula{\forall \oln{x}{n} @.@ app(f_{ptr},\xs) \neq \unr}
    \formula{\land\; app(f_{ptr},\xs) \neq \bad} \\
    \formula{\quad\land\; \forall \oln{y}{k} @.@ app(f_{ptr},\xs) \neq K(\ol{y})} \\
    \text{ for every } (f\;\ol{a}\;\oln{x}{m} = u) \in P
    \text{ and } K \in \Sigma \text{ with } m > n
%% \\
%%     \text{ and every } (K{:}\forall\as @.@ \oln{\tau}{k} -> T\;\as) \in \Sigma \text{ and } m > n
\end{array}\]
These axioms assert that partial applications cannot be equated to
any constructor, $\bot$ nor $\injBad$. If the reader is worried that without a
complete formalisation of all equalities of $D_\infty$ it is impossible to prove any
programs correct, we would like to reassure them that that is not the case as we
shall see in the next section.

\paragraph{Lazy semantics simplifies the translation}
We have mentioned previously (Section~\ref{ssect:trans-exprs}) that the 
laziness of $\theLang$ helps in keeping the translation simple, and here we 
explain why. 

Whenever we use universal quantification in the logic, we really quantify over
{\em any} denotation, including $\bot$ and $\injBad$. In a call-by-name language, 
given a function @f x = True@, the axiom $\forall x @.@ f(x) = \injKZ{True}$ 
is true in the intended denotational model. However, in a call-by-value setting, $x$ 
is allowed to be interpreted as $\bot$. That means that the unguarded axiom is 
actually not true, because $f\,\bot \not= \injKZ{True}$.
Instead we need the following annoying variation:
\[  \forall x @.@ x \neq bad \land x \neq \unr => f(x) = t \]
Moreover, the axioms for the $app(\cdot,\cdot)$ combinator have to be modified
to perform checks that the argument is not $\bot$ or $\injBad$ before actually 
calling a function.   In a call-by-name language these guards are needed 
only for @case@ and the function part of $app$.

These complications lead to a more complex first-order theory in the call-by-value 
case.

%% value, including $\bot$. Very rarely do we have to add a condition that
%% a value is not equal to $\bot$. Only when a function is explicitly strict in a particular argument (by doing pattern matching) do we have to do this. Also, function application in the language corresponds directly to term construction in the logic.

%% To deal with a (pure) strict language, we would have two choices: Either (1) we explicitly make each function strict in all its arguments, by adding extra axioms for each function definition that express what happens when you apply the function to $\bot$, and by adding extra conditions to all the other function axioms that exclude $\bot$; or (2) we use a monadic translation (or some variant thereof), basically turning each application site into a definedness check of the arguments. Neither method leads to a simple first-order theory that is easy to reason about by automatic provers.

%% \spj{What else should we say here.  Something about recursion?  About multiple functions?}




%% \subsection{Denotational versus operational semantics for contracts}
%% TODO -- I have just dumpted material here.

%% We have the rather obvious theorem below.

%% \begin{theorem}[Soundness and completeness for denotational semantics]
%% Assume a program $P$ with signature $\Sigma$, and expression $e$ and contract $\Ct$
%% such that $fv(e) \cup fv(\Ct) \subseteq dom(P)$. Then
%% $\langle D_\infty,{\cal I}\rangle \models \ctrans{\Sigma}{P}{e \in \Ct}$ iff
%% $\interp{\Ct}{\dbrace{P}^{\infty}}{\cdot}(\interp{e}{\dbrace{P}^\infty}{\cdot})$.
%% \end{theorem}




%% \subsubsection{Contract satisfaction and crash-freedom}\label{sect:cf}

%% We would like to define a set of contract-satisfying denotations and also a set of contract-satisfying terms,
%% characterized by $P |- e \in \Ct$, such that the following claim becomes true:

%% \begin{proposition} Assume that $\Sigma |- P$ and $fv(e) \subseteq dom(P)$, i.e. $e$ is closed.
%% Then: $\langle D_\infty,{\cal I}\rangle \models \ctrans{\Sigma}{\Delta}{e \in \Ct}$ iff $P |- e \in \Ct$.
%% \end{proposition}

%% Now there are several problems with coming up with a good definition of $P |- e \in \Ct$,
%% which we elaborate in the following sections.

%% \subsubsection{Problem I: Crash-freedom}

%% Ideally we would like to define crash-freedom {\em semantically} using the following
%% strict bifunctor on admissible sets $S^{-},S^{+} \subseteq D_{\infty}$.
%% {\setlength{\arraycolsep}{2pt}
%% \[\begin{array}{rcl}
%%    F_{\lcfZ}(S^{-},S^{+}) & = & \{\;d\;\mid\;\unroll(d) \neq \ret(\inj{bad}(1))\;\land\; \\
%%                       &    & \quad \forall \ol{d} @.@ \unroll(d){=}\ret(\inj{K_1^\ar}\langle\oln{d}{\ar}\rangle) ==> \ol{d} \in S^{+} \} \\
%%                    & \cup & \ldots \\
%%                    & \cup & \{\;d\;\mid\;\unroll(d) \neq \ret(\inj{bad}(1))\;\land\; \\
%%                    &      & \quad \forall d_0 @.@ \unroll(d) = \ret(\inj{->}(d_0)) ==> \\
%%                    &      & \quad\quad \forall\;d' \in S^{-} ==> \dapp(d,d') \in S^{+} \}  \\
%% \end{array}\]}
%% The $\Fcf$ bifunctor has a negative and positive fixpoint, and by minimal invariance they coincide (one direction
%% follows by Tarski-Knaster, the other can be inductively proved using the approximations on ever element of $D_{\infty}$ given
%% in Lemma~\ref{lem:min-inv-reqs} and the fact that the lub of the chain of $\rho_i$ is the identity and the fact that this
%% functor preserves admissibility for the positive sets). Let us call this admissible set $\Fcf^{\infty} \subseteq D_{\infty}$.

%% We consider this predicate to be the ``ideal crash-freedom'' -- however it is very difficult to give a 1-1 operational
%% definition. The reason is that the $\Fcf$ functor quantifies in the function case over any $d'$ -- whereas in the operational
%% semantics it is only reasonable that we quantify over all terms (or over terms that do not contain @BAD@) In the absense of
%% full abstraction of the domain (which is plausible, especially if we extend the language with other features) it is unclear
%% what a corresponding predicate would look like in terms of operational semantics.

%% We then go for a simpler predicate, which only characterizes crash-freedom for first-order terms,
%% generate by the following functor on {\em admissible} sets of denotations:
%% {\setlength{\arraycolsep}{2pt}
%% \[\begin{array}{rcl}
%%    G_{\lcfZ}(S^{+}) & = & \{\;d\;\mid\; \unroll(d){=}\ret(\inj{K_1^\ar}\langle\oln{d}{\ar}\rangle) \land \ol{d} \in S^{+} \} \\
%%                   & \cup & \ldots \\
%%                   & \cup & \{\;\bot\;\}
%% \end{array}\]}
%% Notice that if $S$ is admissible then so is $G_{\lcfZ}(S)$.

%% %% The $G_{\lcfZ}$ functor has a fixpoint and it is an admissible relation, and we will use its
%% %% fixpoint $G_{\lcfZ}^\infty$, so now we need to say what $G_{\lcfZ|}^\infty$ means operationally.
%% \begin{lemma} The functor $G_{\lcfZ}$ has a unique fixpoint $G_{\lcfZ}^\infty$ on admissible sets. \end{lemma}
%% \begin{proof}
%% The intersection of admissible sets is admissible. Hence we have a complete join semi-lattice (which induces a
%% complete lattice), so the monotone functor $G_{\lcfZ}$ does have a smallest and a greatest fixpoint call
%% it $G_{\lcfZ}^{min}$ and $G_{\lcfZ}^{max}$. Moreover this fixpoint will be an admissible relation. Now it must be
%% that $G_{\lcfZ}^{min} \subseteq G_{\lcfZ}^{max}$ so we only show next that
%% also $G_{\lcfZ}^{max} \subseteq G_{\lcfZ}^{min}$. To do this we will show that:
%% \[ \forall i. d \in G_{\lcfZ}^{max} ==> \rho_i(d) \in G_{\lcfZ}^{min} \]
%% by induction on $i$. For $i = 0$ it follows since $\rho_0(d) = \bot$. Let us assume
%% that it holds for $i$, we need to show that $\rho_{i+1}(d) \in G_{\lcfZ}(G_{\lcfZ}^{min})$.
%% We know however that $d \in G_{\lcfZ}(G_{\lcfZ}^{max}$ and by simply case analysis and appealing
%% to the induction hypothesis we are done. Finally, by admissibility it must be that
%% $\sqcup\rho_i(d) \in G_{\lcfZ}^{min}$ and by Lemma~\ref{lem:min-inv-reqs} it
%% must be that $d \in G_{\lcfZ}^{min}$. This means that the two fixpoints coincide,
%% hence there is only a unique fixpoint of $G_{\lcfZ}$, call it $G_{\lcfZ}^\infty$.
%% \end{proof}

%% Now, we would like to define operationally the set of {\em crash-free} terms as a set $\Ecf$ of
%% closed terms that satisfies:
%% {\setlength{\arraycolsep}{2pt}
%% \[\begin{array}{rcl}
%%    \Ecf & =    & \{ e \;\mid\; P |- e \Downarrow K[\taus](\ol{e}) /\ \ol{e} \in \Ecf \} \\
%%         & \cup & \ldots \\
%%         &      & \{ e \;\mid\; P \not|- e \Downarrow \}
%% \end{array}\]}%
%% We do not know that the set $\Ecf$ exists, so we have to prove it.
%% \begin{lemma}
%% There exists a largest set that satisfies the $\Ecf$ equation above.
%% \end{lemma}
%% \begin{proof}
%% Define $\Ecf$ to be the set
%% \[ \{ e\;\mid\; \interp{e}{\dbrace{P}^\infty}{\cdot} \in G_{\lcfZ}^{\infty}\} \]
%% It is straightforward (by computational adequacy) to show that it satisfies the $\Ecf$ recursive
%% equation above. For uniqueness, assume any other set $E$ that satisfies the recursive equation
%% above. We can show that $\interp{E}{\dbrace{P}^\infty}{\cdot}$ is a
%% fixpoint of $G_{\lcfZ}$ and since there is only one such fixpoint, this is unique. So we have that:
%% \[\begin{array}{ll}
%%  e \in E & ==> \\
%%  \interp{e}{\dbrace{P}^\infty}{\cdot} \in \interp{E}{\dbrace{P}^\infty}{\cdot} & ==> \\
%%  \interp{e}{\dbrace{P}^\infty}{\cdot} \in G_{\lcfZ}^\infty & ==> \\
%%  e \in \Ecf
%% \end{array}\]
%% \end{proof}
%% %% \begin{lemma}
%% %% If $e \in E$ and $\interp{e}{\dbrace{P}^\infty}{\cdot} = \interp{e'}{\dbrace{P}^\infty}{\cdot}$ then $e' in E$.
%% %% \end{lemma}
%% %% This relies on the fact that
%% %% if $\interp{e}{\dbrace{P}^\infty}{\cdot} \in \interp{E}{\dbrace{P}^\infty}{\cdot}$ then $e \in E$. Why is
%% %% that? Because the assumption means that
%% %% $\interp{e}{\dbrace{P}^\infty}{\cdot} \in \{ d | \exists e' \in E /\ d = \interp{e'}{\dbrace{P}^\infty}{\cdot} \}$
%% %% and hence this means that there exists some $e' \in E $ such that
%% %% $\interp{e}{\dbrace{P}^\infty}{\cdot} = \interp{e'}{\dbrace{P}^\infty}{\cdot}$
%% %% \end{proof}

%% Let us extend the interpretation function above $\linterp{\cdot}$ so that:
%% \[\begin{array}{rcl}
%%    \linterp{\lcfZ}  & = & G_{\lcfZ}^{\infty}
%% \end{array}\]

%% \begin{theorem}
%% If $\Sigma |- P$ then we have that $\langle D_{\infty},{\cal I}\rangle \models \Th{\Sigma}{P}^{\lcfZ}$.
%% \end{theorem}

%% Notice that the axiom:
%% \[  \textsc{AxCfC}  \quad \formula{\forall x y @.@ \lcf{x} /\ \lcf{y} => \lcf{app(x,y)}} \]
%% is {\em not validated} by this interpretation of crash-freedom we have given.


%% \subsubsection{Problem II: the absense of full-abstraction}

%% Unfortunately higher-orderness bites again. Having defined the set $\Ecf$ we might define formally
%% the predicate $P |- e \in \Ct$ where $fv(e) \subseteq dom(P)$ and $fv(\Ct) \subseteq dom(P)$ as
%% follows:
%% {\setlength{\arraycolsep}{2pt}
%% \[\begin{array}{lcl}
%%     P |- e \in \{ x\;\mid\;e_p\} & <=> & P |- e \not\Downarrow \text{ or } P |- e_p[e/x] \not\Downarrow \text{ or} \\
%%                                  &     & P |- e_p[e/x] \Downarrow True \\
%%     P |- e \in (x{:}\Ct_1) -> \Ct_2 & <=> &
%%                                  \text{for all } P' e' \text{ s.t. } fv(e') \subseteq dom(P{\uplus}P')  \\
%%                                    &   &  \text{it is } P\uplus P' |- e\;e' \in \Ct_2[e'/x] \\
%%     P |- e \in \Ct_1 \& \Ct_2 & <=> & P |- e \in \Ct_1 \text{ and } P |- e \in \Ct_2 \\
%%     P |- e \in \CF            & <=> & e \in \Ecf
%% \end{array}\]}

%% Note we made the definition above well-scoped but not necessarily well-typed; let's ignore that for now (making everything
%% well-typed includes extra difficulties in the proof but hopefully not surmountable).

%% The interesting case is the case for arrow contracts, where we have extended the set of definitions $P$ with more
%% definitions $P'$ -- that is to allow for tests $e'$ which can have arbitrary computational power, and not only those
%% that can be constructed in the current environment. That is expected the way we have set up things, so let us examine
%% what happens when we try to prove the proposition below:

%% \begin{proposition} Assume that $\Sigma |- P$ and $fv(e) \subseteq dom(P)$, i.e. $e$ is closed.
%% Then: $\langle D_\infty,{\cal I}\rangle \models \ctrans{\Sigma}{\Delta}{e \in \Ct}$ iff $P |- e \in \Ct$.
%% \end{proposition}

%% {\flushleft{\em Failed proof}:}
%% The base case and the case of $\CF$ follow from computational adequacy so we are good. However
%% let's try to prove the arrow case and in particular the $(<=)$ direction.

%% Let us assume that for all $P'$ and $e'$ such that $fv(e') \subseteq dom(P\uplus P')$ it is the case that
%% $P |- e\;e' \in \Ct_2[e'/x]$. We need to show that $\langle D_\infty,{\cal I}\rangle$ is a model of the
%% formula $\forall x. \ctrans{\Sigma}{x}{x \in \Ct_1} => \ctrans{\Sigma}{x}{e\;x \in \Ct_2}$. Let us fix
%% a denotation $d \in D_{\infty}$ and let us assume
%% that $\langle D_{\infty},{\cal I} \rangle \models \ctrans{\Sigma}{x}{x \in \Ct_1}[d/x]$. However, this does not
%% necessarily mean that we can find a closed $e'$ and $P'$, such
%% that $\interp{e'}{\dbrace{P{\uplus}P'}^\infty}{\cdot} = d$ to be able to use the assumptions, unless some sort
%% of full-abstraction property is true. So we are stuck.

%% Here is a concrete counterexample, based on the lack of full-abstraction due to the {\em parallel or} function.
%% Consider the program $P$ below:
%% \[\begin{array}{lcl}
%% f_\omega & |-> & f_\omega \\
%% f & |-> & \lambda (b{:}Bool) @.@ \lambda (h{:}Bool->Bool->Bool) @.@ \\
%%   &     & \quad @if@\;(h\;True\;b)\;\&\&\;(h\;b\;True)\;\&\& \\
%%   &     & \quad\qquad\qquad not\;(h\;False\;False)\;@then@ \\
%%   &     & \quad\quad @if@\;(h\;True\;f_\omega)\;\&\&\;(h\;f_\omega\;True)\;@then@\;@BAD@ \\
%%   &     & \quad\quad @else@\;True \\
%%   &     & \quad @else@\;True
%% \end{array}\]
%% Consider now the candidate contract for $f$ below:
%% \[ \CF -> (\CF -> \CF -> \CF) -> \CF \]
%% Operationally we may assume a crash-free boolean as well as a function $h$ which is
%% $\CF -> \CF -> \CF$. The first conditional ensures that the function behaves like an ``or'' function or
%% diverges. However if we pass the first conditional,
%% the second conditional will always diverge and hence the contract will be satisfied.

%% However, denotationally it is possible to have a {\em monotone} function $por$ defined as follows:
%% \[\begin{array}{lcl}
%%   por\;\bot\;\bot & = & \bot \\
%%   por\;\bot\;True & = & True \\
%%   por\;True\;\bot & = & True \\
%%   por\;False\;False & = & False
%% \end{array}\]
%% with the rest of the equations (for @BAD@ arguments) induced by monotonicity and whatever boolean value
%% we like when both arguments are @BAD@.

%% Now, this is denotationally a $\CF -> \CF -> \CF$ function, and it will pass the first conditional, but it will
%% also pass the second conditional, yielding @BAD@. Hence denotationally the contract for $f$ {\em does not hold}.

%% So we have a concrete case where the $<=$ direction fails. Because of contra-variance of arrow contracts, it is
%% likely that the $=>$ direction is false as well.


%% %% Now it may be the case that for all denotations that semantically satisfy a contract, these denotations {\em are}
%% %% realizable by a term $e'$ and a context $P'$ but it is not entirely clear how to prove this (or if this is a good
%% %% idea). I am not sure if this is true either.
%% %% The other idea out of this situation is to compile the arrow contract differently by not quantifying over all
%% %% denotations but rather some kind of {\em definable} denotations -- but I do not know how exactly to do this.


%% \paragraph{A way out of this?}
%% Well, if we restrict our higher-order tests to those that can be constructed from our signature then
%% we may define the following:

%% {\setlength{\arraycolsep}{2pt}
%% \[\begin{array}{lcl}
%%     P |- e \in \{ x\;\mid\;e_p\} & <=> & P |- e \not\Downarrow \text{ or } P |- e_p[e/x] \not\Downarrow \text{ or} \\
%%                                  &     & P |- e_p[e/x] \Downarrow True \\
%%     P |- e \in (x{:}\Ct_1) -> \Ct_2 & <=> &
%%                                  \text{for all } e' \text{ s.t. } fv(e') \subseteq dom(P)  \\
%%                                    &   &  \text{it is } P |- e\;e' \in \Ct_2[e'/x] \\
%%     P |- e \in \Ct_1 \& \Ct_2 & <=> & P |- e \in \Ct_1 \text{ and } P |- e \in \Ct_2 \\
%%     P |- e \in \CF            & <=> & e \in \Ecf
%% \end{array}\]}
%% Notice that the difference with the previous version of $P |- e \in \Ct$ is that we {\em do not} extend the
%% definitions $P'$ so we don't get the full power of higher-order tests. We show that {\em in the current signature
%% only} does the program satisfy the contract.


%% Why did we do this change? Because denotationally this is not terribly hard to support -- instead of translating
%% \[\begin{array}{l}
%%   \ctrans{\Sigma}{\Gamma}{e \in (x{:}\Ct_1) -> \Ct_2} =
%%   \formula{\forall x @.@ \ctrans{\Sigma}{\Gamma,x}{x \in \Ct_1} => \ctrans{\Sigma}{\Gamma,x}{e\;x \in \Ct_2}}
%% \end{array}\]
%% we use the following:
%% \[\begin{array}{l}
%%   \ctrans{\Sigma}{\Gamma}{e \in (x{:}\Ct_1) -> \Ct_2} = \\
%%   \qquad\qquad\quad
%% \formula{\forall x @.@ \definable{x} \land \ctrans{\Sigma}{\Gamma,x}{x \in \Ct_1} => \ctrans{\Sigma}{\Gamma,x}{e\;x \in \Ct_2}}
%% \end{array}\]
%% where $\definable{x}$ could be axiomatized as containing all terms
%% made up of the functions in $P$, applications, and data constructors:

%% \[\begin{array}{lll}
%%  \textsc{DefCons} & \formula{\forall \xs @.@ \definable{K(\xs)} <=> \definable{\xs}} \\
%%                         & \text{ for every } (K{:}\forall\as @.@ \oln{\tau}{n} -> T\;\as) \in \Sigma \\
%%  \textsc{DefFun}  & \formula{\definable{f_{ptr}}}  \\
%%                         & \text{ for every } (f |-> \Lambda\as @.@ \lambda\oln{x{:}\tau}{n} @.@ u) \in P \\
%%  \textsc{DefApp}  & \formula{\forall x y @.@ \definable{x}\land\definable{y} => \definable{app(x,y)}}
%% %% \formula{\bad \neq \unr}  \\
%% %%  \textsc{AxDisjB} & \formula{\forall \oln{x}{n}\oln{y}{m} @.@ K(\ol{x}) \neq J(\ol{y})} \\
%% %%                   & \text{ for every } (K{:}\forall\as @.@ \oln{\tau}{n} -> T\;\as) \in \Sigma \\
%% %%                   & \text{ and } (J{:}\forall\as @.@ \oln{\tau}{m} -> S\;\as) \in \Sigma \\
%% %%  \textsc{AxDisjC} & \formula{(\forall \oln{x}{n} @.@ K(\ol{x}) \neq \unr \land K(\ol{x}) \neq \bad)} \\
%% %%                   & \text{ for every } (K{:}\forall\as @.@ \oln{\tau}{n} -> T\;\as) \in \Sigma \\ \\
%% %%  \textsc{AxAppA}  & \formula{\forall \oln{x}{n} @.@ f(\ol{x}) = app(f_{ptr},\xs)} \\
%% %%                   & \text{ for every } (f |-> \Lambda\as @.@ \lambda\oln{x{:}\tau}{n} @.@ u) \in P \\
%% %%  %% \textsc{AxAppB}  & \formula{\forall \oln{x}{n} @.@ K(\ol{x}) = app(\ldots (app(x_K,x_1),\ldots,x_n)\ldots)} \\
%% %%  %%                  & \text{ for every } (K{:}\forall\as @.@ \oln{\tau}{n} -> T\;\as) \in \Sigma \\
%% %%  \textsc{AxAppC}  & \formula{\forall x, app(\bad,x) = \bad \; /\ \; app(\unr,x) = \unr}    \\ \\
%% %%  %% Not needed: we can always extend partial constructor applications to fully saturated and use AxAppC and AxDisjC
%% %%  %% \textsc{AxPartA} & \formula{\forall \oln{x}{n} @.@ app(\ldots (app(x_K,x_1),\ldots,x_n)\ldots) \neq \unr} \\
%% %%  %%                  & \formula{\quad\quad \land\; app(\ldots (app(x_K,x_1),\ldots,x_n)\ldots) \neq \bad} \\
%% %%  %%                  & \text{ for every } (K{:}\forall\as @.@ \oln{\tau}{m} -> T\;\as) \in \Sigma \text{ and } m > n \\
%% %%  \textsc{AxPartB} & \formula{\forall \oln{x}{n} @.@ app(f_{ptr},\xs) \neq \unr} \\
%% %%                   & \formula{\quad\land\; app(f_{ptr},\xs) \neq \bad} \\
%% %%                   & \formula{\quad\land\; \forall \oln{y}{k} @.@ app(f_{ptr},\xs) \neq K(\ol{y})} \\
%% %%                   & \text{ for every } (f |-> \Lambda\as @.@ \lambda\oln{x{:}\tau}{m} @.@ u) \in P  \\
%% %%                   & \text{ and every } (K{:}\forall\as @.@ \oln{\tau}{k} -> T\;\as) \in \Sigma \text{ and } m > n  \\ \\
%% %%  \textsc{AxInj}   & \formula{\forall \oln{y}{n} @.@ \sel{K}{i}(K(\ys)) = y_i} \\
%% %%                   & \text{for every } (K{:}\forall\as @.@ \oln{\tau}{n} -> T\;\as) \in \Sigma \text{ and } i \in 1..n \\ \\
%% %% \end{array} \\
%% %% \ruleform{\Th{\Sigma}{P}^{\lcfZ}} \\ \\
%% %% \begin{array}{lll}
%% %%  \textsc{AxCfA}   & \formula{\lcf{\unr} /\ \lncf{\bad}} \\
%% %%  \textsc{AxCfB}   & \formula{\forall \oln{x}{n} @.@ \lcf{K(\ol{x})} <=> \bigwedge\lcf{\ol{x}}} \\
%% %%                   & \text{ for every } (K{:}\forall\as @.@ \oln{\tau}{n} -> T\;\as) \in \Sigma
%% \end{array}\]


%% In the model, $\definable{\cdot}$ should be possible to define, as a
%% predicate on denotations. The disadvantage to this approach is that
%% arrow contracts will only hold for whatever is in your context, not
%% arbitrary expressions, which might be what we want, but might not be
%% modular enough.

%% The other {\em potential} problem (i.e. I have not yet checked) might be in the
%% proof of admissibility of induction.

%% And yet another potential problem is that as we incrementally extend our signature
%% with new function definitions (and possibly contracts) previously defined contracts
%% may no longer hold. This is pretty bad for modularity.

%% \paragraph{Yet another possible solution}

%% A solution that seems somewhat more modular is based on the observation that,
%% during the evaluation of a program there exists a {\em set} of terms (maybe infinite) that can
%% appear as arguments to other terms or functions. Our idea is to guard the arrow contracts so that
%% we do not quantify over any possible term (or denotation, in the translation) but rather only
%% those that may appear as {\em arguments} in some application. We translate arrow contract as
%% follows:
%% \[\begin{array}{l}
%%   \ctrans{\Sigma}{\Gamma}{e \in (x{:}\Ct_1) -> \Ct_2} = \\
%%   \qquad\qquad\quad
%% \formula{\forall x @.@ arg(x) \land \ctrans{\Sigma}{\Gamma,x}{x \in \Ct_1} => \ctrans{\Sigma}{\Gamma,x}{e\;x \in \Ct_2}}
%% \end{array}\]
%% where $arg(x)$ ensures that $x$ is the denotation of a term that will be passed as an argument to $e$. We'd need to define
%% a similar predicate on the evaluation relation, call it $Arg(e)$ and modify the program translation to thread the $arg(\cdot)$
%% predicate through.

