
The initial contracts language is sufficient to express a wide variety
of properties, but it is easy to extend the language to be able to
declare more properties, and sometimes in a more straightforward way.
This section describes the new constructs.

\subsection{Parameterised Contracts and Local Assumptions}
Can this property that describes that @all p@ is a list homomorphism
be written as a contract?

$$\forall \; @p@ \; @xs@ \; @ys@ \; . \;
    @all p xs && all p ys@ = @all p (xs ++ ys)@$$

First of all, we need to ask ourselves which of the functions this
could be a contract for. A promising candidate seems to be @(++)@. So
will this do for a contract?

\[\begin{array}{rcl}
@(++)@ & \in & \{ @xs@ \mid @CF@ \; \& \; @any p xs@ \} \\
       & \to & \{ @ys@ \mid @CF@ \; \& \; @any p ys@ \} \\
       & \to & \{ @rs@ \mid @CF@ \; \& \; @any p rs@ \}
\end{array}\]

The problem here is that @p@ is a free variable, moreover, it is also
important that $@p@ \in @CF@ \to @CF@$. One could introduce a new function
which takes @p@ as an argument but ignores it as this:

\begin{code}
append_dummy p xs ys = xs ++ ys
\end{code}

Now, the contract can be expressed:

\[\begin{array}{rcl}
@append_dummy@ & \in & ( @p@ : @CF@ \to @CF@ ) \\
               & \to & \{ @xs@ \mid @CF@ \; \& \; @any p xs@ \} \\
               & \to & \{ @ys@ \mid @CF@ \; \& \; @any p ys@ \} \\
               & \to & \{ @rs@ \mid @CF@ \; \& \; @any p rs@ \}
\end{array}\]

This approach has several downsides:
\begin{enumerate}
  \item The @append_dummy@ function is not recursive, but this
  contract needs to be proved with fixed point induction on @(++)@,

  \item Annoying acrobatics involved in introducing a new function to
  get a new parameter,

  \item If we use this contract when proving another contract,
  chances are that @append_dummy@ is not going to be interesting, even
  though the property it expresses is. This could be a problem when
  using the min-translation.
\end{enumerate}

We will later argue that it is benefical to prove contracts for not
only one function, but for any expression. This solves the first entry
in the list above, and partially the second because then we could use
a lambda instead of a new top-level declaration.

However, by just extending the language of statements slightly we will
solve the latter two. We will allow quantification and assumptions in
statetements, so the contract can instead be written as this:

\[\begin{array}{rcl}
\forall \; @p@ \; & . & \; (@p@ \in @CF@ \to @CF@) => \\
@(++)@ & \in & \{ @xs@ \mid @CF@ \; \& \; @any p xs@ \} \\
       & \to & \{ @ys@ \mid @CF@ \; \& \; @any p ys@ \} \\
       & \to & \{ @rs@ \mid @CF@ \; \& \; @any p rs@ \}
\end{array}\]

Thus, statements now contain these two extra constructs:

\[\begin{array}{lrll}
  s,t & ::=  & e \in C                     & \text{Contracts} \\
      & \mid & \highlight{s => t}          & \text{Assumption} \\
      & \mid & \highlight{\forall x @.@ s} & \text{Quantification} \\
      & \mid & s \; \textsf{using} \; t    & \text{Reuse}
\end{array}\]

The translation of statements can be viewed in
Figure~\ref{fig:stmt-trans}.  Fixed point induction can now be
expressed quite elegantly: for a statement $s$ susceptible to fixed
point induciton over $f$, we now instead consider the statement
$s[f^\circ/f] => s[f^\bullet/f]$.

\begin{figure}\small
\setlength{\arraycolsep}{2pt}
\[\begin{array}{c}
\ruleform{\trs{v}{s} = \formula{\phi}} \\ \\
\begin{array}{lcl}
  \trs{-}{e \in C}         & = & \trc{e \notin C} \\
  \trs{+}{e \in C}         & = & \trc{e \in C} \\
  \trs{-}{\forall x @.@ s} & = & \exists x @.@ \trs{-}{s} \\
  \trs{+}{\forall x @.@ s} & = & \forall x @.@ \trs{+}{s} \\
  \trs{-}{s => t}          & = & \trs{+}{s} \land \trs{-}{t} \\
  \trs{+}{s => t}          & = & \trs{-}{s} \lor \trs{+}{t} \\
  \trs{-}{s \; \textsf{using} \; t} & = & \trs{-}{s} \land \trs{+}{t} \\
  \trs{+}{s \; \textsf{using} \; t} & = & \trs{+}{s} \\
\end{array}
\end{array}\]
\caption{
    The translation of statements in positive and negative
    position. Right-nested uses of $\textsf{using}$ are assumed to be
    removed.  \label{fig:stmt-trans}
}
\end{figure}

Another property describable with these statements is the one of
associativity as a contract:

\[\begin{array}{rcl}
\forall \; @z@ \; . \; @z@ : @CF@ => @(+)@
    & \in & ( @x@ \in @CF@ ) \to ( @y@ : @CF@ ) \to \\
    &     & \{ @r@ \mid @CF@ \; \& \; @r + z == x + (y + z)@ \}
\end{array}\]

\subsection{Expressions, not Functions}

If we lift the restriction that contracts is always accompanied by a
function to let contracts express properties about general
expressions, we get a richer language which can express this:

\begin{enumerate}
  \item Contracts that for functions that are partially applied:
    $$@map fromJust@ \in \{ @xs@ \mid @all isJust xs@ \} \to @CF@$$

  \item Repeated variables are allowed, allowing reflexivity and
    idempotence:
    \[\begin{array}{l}
    \forall \; @x@ \; . @x@ \in @CF@ => @x == x@ \in \{ @b@ \mid @CF@ \; \& \; @b@ \} \\
    \forall \; @x@ \; . @x@ \in @CF@ => @x && x@ \in \{ @b@ \mid @CF@ \; \& \; @b == x@ \}
    \end{array}\]

  \item Combined with assumptions, we can now express complex
      properties such as symmetry:

    \[\begin{array}{rcl}
    \forall \; @x@ & . & @x@ \in @CF@ => \\
    \forall \; @y@ & . & @y@ \in @CF@ => \\
                   &   & @x == y@ \in \{ @b@ \mid @CF@ \; \& \; @b@ \} => \\
                   &   & @y == x@ \in \{ @b@ \mid @CF@ \; \& \; @b@ \}    \\
    \end{array}\]

\end{enumerate}

Fixed point induction will be applied to the function at the top of
the rightmost (of $=>$s) statement if it is a recursive function. For
the symmetry example, this means that we would do fixed point induction
over @(==)@.

It turns out that contracts as they were before this section can now
be described in terms of explicit quantification and
assumptions. Indeed, the following two are equivalent:

\[\begin{array}{l}
@f@ \in (x_1 : C_1) \to \cdots \to (x_n : C_n) \to C
\\ \qquad\qquad <=> \\
\forall \; x_1 . x_1 \in C_1 => \cdots =>
\forall \; x_n . x_n \in C_n =>
    @f@ \; x_1 \; \cdots \; x_n \in C
\end{array}\]

It is up to the users of to use whichever they find more useful.

\subsection{First order equality in contracts}
\dr{This is not implemented, it is something to consider}

The earlier example with associativity of @(++)@ was defined in terms
of some equality function @(==)@. However, these are not very
convenient to prove with. We have to show that they constitute an
equivalence relation, but more serious is that we need to show that
they form a congruence over the functions we are interested in. The
equality in first order logic always has this property; it is
substitutive. And indeed, some properties we show hold up to its
equality, such as associativity:

\[\begin{array}{rcl}
\forall \{ @zs@ \} . @(++)@ & \in & \{ @xs@ \} \to \{ @ys@ \} \to \\
                            &     & \{ @rs@ \mid @rs ++ zs@ = @xs ++ (ys ++ zs)@ \}
\end{array}\]

How can we express this in our DSL? We make a new @Eq@ constructor for @Contract@:

$$@Eq :: (a -> Equality a) -> Contract a@$$,

and we make a new data type Equality:

\begin{code}
data Equality a where
    (:=:) :: a -> a -> Contract a
\end{code}

We now write the property as this:

\begin{code}
app_assoc :: [a] -> Statement
app_assoc zs = (++) :::
    Any :-> \xs -> Any :-> \ys -> Eq
        (\rs -> rs ++ zs :=: xs ++ (ys ++ zs))
\end{code}

Using induction, the step case goes through, but the base case is a
bit more problematic. We then have to prove that we have
$\bot @++ zs@ = @xs ++ (ys ++ zs)@$ for all @xs@, @ys@ and @zs@.
Dimitrios comes to the rescue and demands all contracts to hold in the
base case, so we really add a bottom fall-through for equality.

Formally, we extend contracts with a new construct
\[\begin{array}{lrll}
\multicolumn{3}{l}{\text{Contracts}} \\
 \Ct & ::=  & \cdots                 & \text{Previous constructs} \\
     & \mid & \formula{\{ x \mid e_1 = e_2 \}} & \text{Equality}
\end{array}\]

Translated as follows:

\[\begin{array}{l}
\ctrans{\Sigma}{\Gamma}{e \in \{x \mid e_1 = e_2 \}}
  = \; t{=}\unr \; \lor \; t_1[t/x]{=}t_2[t/x] \\
\quad \text{where} \;
   t = \etrans{\Sigma}{\Gamma}{e}, \;
   t_1 = \etrans{\Sigma}{\Gamma}{e_1} \; \text{and} \;
   t_2 = \etrans{\Sigma}{\Gamma}{e_2}
\end{array}\]

\paragraph{Suggested min translation of equality}

\[\begin{array}{l}
\ctrans{\Sigma}{\Gamma}{e \in \{x \mid e_1 = e_2 \}} \\
\quad = \; \formula{( min(t_1) \lor min(t_2) )} \\ %%  \formula{min(t) \; \land \;}
\quad => (\formula{min(t)} \; \land \; t{=}\unr) \; \lor \; t_1[t/x]{=}t_2[t/x] \\ \\
\ctrans{\Sigma}{\Gamma}{e \notin \{x \mid e_1 = e_2 \}} \\
\quad = \; \formula{(min(t) \; \land \; t{=}unr) \lor neq(t_1[t/x],t_2[t/x]))}
\end{array}\]

Where $neq$ is an apartness relation axiomatised in Figure~\ref{fig:neq-axioms}.

\dr{If removing min, we can also remove neq by replace it to $\neq$}

\begin{figure}
{\small
\[\setlength{\arraycolsep}{1pt}
\begin{array}{c}
 \ruleform{neq} \\ \\
\begin{array}{lll}
 \textsc{NeqIrrRefl} & \forall x @.@ \neg neq(x,x) \\
 \textsc{NeqSym}     & \forall x, y @.@ neq(x,y) => neq(y,x) \\
 \textsc{NeqTrans}   & \forall x, y, z @.@ neq(x,y) => (neq(x,z) \lor neq(y,z)) \\
 \textsc{NeqMin}     & \forall x, y @.@ neq(x,y) => (min(x) \land min(y)) \\
 \textsc{NeqDisj}    & \forall \oln{x}{n}\oln{y}{m} @.@ neq(K(\ol{x}),K(\ol{y})) => \bigvee_i neq(x_i,y_i) \\
                     & \text{ for every } (K{:}\forall\as @.@ \oln{\tau}{n} -> T\;\as) \in \Sigma \\
 \textsc{NeqApp}     & \forall f, g, x, y @.@ neq(app(f,x),app(g,y)) => \\
                     & (neq(f,g) \lor neq(x,y))
\end{array}
\end{array}\]}
\caption{An axiomatisation of neq
    \label{fig:neq-axioms}}
\end{figure}

\paragraph{Equality versus partially applied contracts}

With the exception from the extra $\bot$ guard, the expressibility of
equality and partially applied contracts indeed do overlap. The example
with @map fromJust@ above can be now instead be written:

$$@map@ \in \{ @f@ \mid @f@ = @fromJust@ \} \to \{ @xs@ \mid @all isJust xs@ \} \to @CF@$$

It is a bit clumsy, so we accept both versions.

\subsection{SMT 2.0 and triggers}
\dr{I don't know where this section will fit}
\paragraph{Support for primitive Integers}
