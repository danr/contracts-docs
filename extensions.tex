\dr{I just use this section for what I am currently working on but it
  should be spread across the article. The first subsections could be
  merged with the previous section.}
The language for expressing contracts in the previous section is a bit
too weak. Here are the the new ideas:

\subsection{Parameterised contracts}

Motivating example:

\[\begin{array}{rcl}
@(++)@ & \in & \{ @xs@ \mid @CF@ \} \to \{ @ys@ \mid @CF@ \} \to \\
       &     & \{ @rs@ \mid @CF@ \; \land \; @any p xs || any p ys <=> any p rs@ \}
\end{array}\]

This contract was not earlier expressible because @p@ occurs free. We
allow \emph{parameterised} contracts, and let the user write the above
contract as follows:

\begin{code}
any_app :: (a -> Bool) -> Statement
any_app p = given (p ::: CF --> CF) $
    (++) ::: CF :-> \xs -> CF :-> \ys ->
             CF :&: Pred (\rs ->
                any p xs || any p ys <=> any p rs)
\end{code}
% $
\dr{@Using@ is a bit annoying since the lambdas in @:->@ binds over
  them, so I used @given = flip Using@}

This is sound because we could make another function behaving exactly
like @(++)@, but with an extra argument that is @p@, that is just
passed along every recursive call. The above construction allows us
to avoid making such an unnatural gymnastic.

\dr{Interestingly, the above contract could be written somewhat
  similar if we had disjunction in our contracts. That makes me
  wonder, are coproducts admissible?}
\dr{Later note: I cannot
  remember how to express it as a disjunction any more}

We can now also express associativity properties:

\begin{code}
app_assoc :: [a] -> Statement
app_assoc zs = (++) :::
    CF :-> \xs -> CF :-> \ys -> CF :&: Pred
        (\rs -> rs ++ zs == xs ++ (ys ++ zs))
\end{code}

\subsection{Partially applied contracts}

Quite the opposite to parameterised contracts, we also allow
\emph{partially applied} contracts. An example:
$$@map fromJust@ \in \{ @xs@ \mid @all isJust xs@ \} \to @CF@$$

This is now simple to express in our language:

\begin{code}
unjust = map fromJust ::: Pred (all isJust) --> CF
\end{code}

Fixed point induction will be applied to the above contract as @map@
here has both these characteristics:
\begin{itemize}
   \item it is the root of the expression, and
   \item it is a recursive function.
\end{itemize}

What if we instead had @for :: [a] -> (a -> b) -> [b]@ defined
and we wanted to express the same contract? We surely want to
be able to write either of these:

\begin{code}
unjust_flip = flip for fromJust
                    ::: Pred (all isJust) --> CF

unjust_lambda = \xs -> for xs fromJust
                    ::: Pred (all isJust) --> CF
\end{code}

And, indeed, we allow both of these, and we regard @for@ as being on
top of the expression, susceptible for induction, on both examples
above. How? In @unjust_flip@, we inline @flip@ since it is a
non-recursive function that does not do case distinctions. Then we end
up with the same as in the second example, @unjust_lambda@. We strip off the leading
lambda(s), and then the top level @for@ is readily found.

\paragraph{Parameterised and partially applied contracts}

When combining these two techniques, we can express contracts
reflexivity for a concrete equivalence function:

\begin{code}
refl :: Nat -> Statement
refl x = given (x ::: CF) $
    x == x ::: CF :&: Pred (const True)
\end{code}
% $

Notice that the contract is for @x == x@, i.e. for the fully applied @(==)@.
Induction is used to prove this contract.

\subsection{First order equality in contracts}

The earlier example with associativity of @(++)@ was defined in terms
of some equality function @(==)@. However, these are not very
convenient to prove with. We have to show that they constitute an
equivalence relation, but more serious is that we need to show that
they form a congruence over the functions we are interested in. The
equality in first order logic always has this property; it is
substitutive. And indeed, some properties we show hold up to its
equality, such as associativity:

\[\begin{array}{rcl}
\forall \{ @zs@ \} . @(++)@ & \in & \{ @xs@ \} \to \{ @ys@ \} \to \\
                            &     & \{ @rs@ \mid @rs ++ zs@ = @xs ++ (ys ++ zs)@ \}
\end{array}\]

How can we express this in our DSL? We make a new @Eq@ constructor for @Contract@:

$$@Eq :: (a -> Equality a) -> Contract a@$$,

and we make a new data type Equality:

\begin{code}
data Equality a where
    (:=:) :: a -> a -> Contract a
\end{code}

We now write the property as this:

\begin{code}
app_assoc :: [a] -> Statement
app_assoc zs = (++) :::
    Any :-> \xs -> Any :-> \ys -> Eq
        (\rs -> rs ++ zs :=: xs ++ (ys ++ zs))
\end{code}

Using induction, the step case goes through, but the base case is a
bit more problematic. We then have to prove that we have
$\bot @++ zs@ = @xs ++ (ys ++ zs)@$ for all @xs@, @ys@ and @zs@.
Dimitrios comes to the rescue and demands all contracts to hold in the
base case, so we really add a bottom fall-through for equality.

Formally, we extend contracts with a new construct
\[\begin{array}{lrll}
\multicolumn{3}{l}{\text{Contracts}} \\
 \Ct & ::=  & \cdots                 & \text{Previous constructs} \\
     & \mid & \formula{\{ x \mid e_1 = e_2 \}} & \text{Equality}
\end{array}\]

Translated as follows:

\[\begin{array}{l}
\ctrans{\Sigma}{\Gamma}{e \in \{x \mid e_1 = e_2 \}}
  = \; t{=}\unr \; \lor \; t_1[t/x]{=}t_2[t/x] \\
\quad \text{where} \;
   t = \etrans{\Sigma}{\Gamma}{e}, \;
   t_1 = \etrans{\Sigma}{\Gamma}{e_1} \; \text{and} \;
   t_2 = \etrans{\Sigma}{\Gamma}{e_2}
\end{array}\]

\paragraph{Suggested min translation of equality}

\[\begin{array}{l}
\ctrans{\Sigma}{\Gamma}{e \in \{x \mid e_1 = e_2 \}} \\
\quad = \; \formula{( min(t_1) \lor min(t_2) )} \\ %%  \formula{min(t) \; \land \;}
\quad => (\formula{min(t)} \; \land \; t{=}\unr) \; \lor \; t_1[t/x]{=}t_2[t/x] \\ \\
\ctrans{\Sigma}{\Gamma}{e \notin \{x \mid e_1 = e_2 \}} \\
\quad = \; \formula{(min(t) \; \land \; t{=}unr) \lor neq(t_1[t/x],t_2[t/x]))}
\end{array}\]

Where $neq$ is an apartness relation axiomatised in Figure~\ref{fig:neq-axioms}.

\dr{If removing min, we can also remove neq by replace it to $\neq$}

\begin{figure}
{\small
\[\setlength{\arraycolsep}{1pt}
\begin{array}{c}
 \ruleform{neq} \\ \\
\begin{array}{lll}
 \textsc{NeqIrrRefl} & \forall x @.@ \neg neq(x,x) \\
 \textsc{NeqSym}     & \forall x, y @.@ neq(x,y) => neq(y,x) \\
 \textsc{NeqTrans}   & \forall x, y, z @.@ neq(x,y) => (neq(x,z) \lor neq(y,z)) \\
 \textsc{NeqMin}     & \forall x, y @.@ neq(x,y) => (min(x) \land min(y)) \\
 \textsc{NeqDisj}    & \forall \oln{x}{n}\oln{y}{m} @.@ neq(K(\ol{x}),K(\ol{y})) => \bigvee_i neq(x_i,y_i) \\
                     & \text{ for every } (K{:}\forall\as @.@ \oln{\tau}{n} -> T\;\as) \in \Sigma \\
 \textsc{NeqApp}     & \forall f, g, x, y @.@ neq(app(f,x),app(g,y)) => \\
                     & (neq(f,g) \lor neq(x,y))
\end{array}
\end{array}\]}
\caption{An axiomatisation of neq
    \label{fig:neq-axioms}}
\end{figure}

\paragraph{Equality versus partially applied contracts}

With the exception from the extra $\bot$ guard, the expressibility of
equality and partially applied contracts indeed do overlap. The example
with @map fromJust@ above can be now instead be written:

$$@map@ \in \{ @f@ \mid @f@ = @fromJust@ \} \to \{ @xs@ \mid @all isJust xs@ \} \to @CF@$$

It is a bit clumsy, so we accept both versions.

\subsection{SMT 2.0 and triggers}
\dr{I don't know where this section will fit}
\paragraph{Support for primitive Integers}
