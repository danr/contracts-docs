
By using the min translation, we enable the possibility for finite
models.  When a contract does not hold under some assumed other
contracts or the induction hypothesis, the theory is satisfiable. When
using the min translation this means we have good chances for finding
a finite countermodel. While we do not have a formal proof for this,
or for the limitiations, practical experience shows that the finite
model finder @paradox@ is very efficient on our problem.

A finite model has some finite domain of elements that we can write
$\mathbf{D} = \{\mathbf{1} , \mathbf{2} , \cdots , \mathbf{n}\}$ for a
domain of $n$ elements. We also have an interpretation for every
function and predicate in the theory. Or predicates are @min@ and
@CF@, but we have many uses for FOL functions (and constants), namely,
they correspond to:

\begin{itemize}
    \item Original functions a from the GHC Core
    \item Constructor
    \item Projections
    \item Pointers
    \item Skolem variables from translated contracts
    \item The @app@ table
\end{itemize}

We consider a subtraction example:

\begin{code}
    (-) :: Nat -> Nat -> Nat
    x      - Zero   = x
    Zero   - _      = error "Negative Nat!"
    Succ x - Succ y = x - y
\end{code}

Let's explore the counterexample we get with this contract:
$$@(-)@ \in \{ @CF@ -> @CF@ -> @CF@ \}$$

Two skolem constants, @x@ and @y@ are introduced for the first and
second argument respectively, for @(-)@.  Running @paradox@ on the
translated file with fixed point induction gives us a five element
countermodel. The relevant entries for the constants are:

\[\begin{array}{rcl}
@x@ & = & \mathbf{3} \\
@y@ & = & \mathbf{4}
\end{array}\]

How can we know what these elements $\mathbf{3}$ and $\mathbf{4}$ are?
A first glance at the entry for @min@ shows that $\mathbf{2}$ and
$\mathbf{5}$ are the only non-min values, so we ought to find some
constructor that correspond to $\mathbf{3}$ and $\mathbf{4}$. The
constructor tables for BAD, UNR, Zero and Succ look like this:

\[\begin{array}{lcl}
@BAD@ & = & \mathbf{1} \\
@UNR@ & = & \mathbf{2} \\
\\
@Zero@ & = & \mathbf{3} \\
\\
@Succ@(\mathbf{1}) & = & \mathbf{5} \\
@Succ@(\mathbf{2}) & = & \mathbf{2} \\
@Succ@(\mathbf{3}) & = & \mathbf{4} \\
@Succ@(\mathbf{4}) & = & \mathbf{5} \\
@Succ@(\mathbf{5}) & = & \mathbf{5} \\
\end{array}\]

Interesting! It seems like $@x@ = @Zero@$ and $@y@ = @Succ @\mathbf{3}$.
Futhermore, @UNR@ is $\mathbf{2}$, which was not in the @min@-set, so
we haven't ever started evaluating it (phew!). Notice, too, that @x@
and @y@ are not @BAD@. But let's have a look at the table for
@Succ@. There are three elements, $\mathbf{1}$, $\mathbf{4}$ and
$\mathbf{5}$, that go to $\mathbf{2}$ - can then this element be
@Succ@ of two different things? Naturally not, the missing piece is
the projection table (recall that constructors are not everywhere
injective when we do not have min). Here it is for @Succ@:
\[\begin{array}{lcl}
@Succ@_0(\mathbf{1}) & = & \mathbf{3} \\
@Succ@_0(\mathbf{2}) & = & \mathbf{3} \\
@Succ@_0(\mathbf{3}) & = & \mathbf{2} \\
@Succ@_0(\mathbf{4}) & = & \mathbf{3} \\
@Succ@_0(\mathbf{5}) & = & \mathbf{5} \\
\end{array}\]

Now we need to match up this table with the one for $@Succ@$. We
search for elements $x$ with $@Succ@_0(@Succ@(x)) = x$. Those
elements are the only we can say are really constructed with $@Succ@$.
Let's put this in a little table for our five elements:
\[\begin{array}{ccc}
x          & @Succ@(x)  & @Succ@_0(@Succ@(x)) \\
\mathbf{1} & \mathbf{5} & \mathbf{5} \\
\mathbf{2} & \mathbf{2} & \mathbf{3} \\
\mathbf{3} & \mathbf{4} & \mathbf{3} \\
\mathbf{4} & \mathbf{5} & \mathbf{5} \\
\mathbf{5} & \mathbf{5} & \mathbf{5} \\
\end{array}\]

In the example at hand, we get these two: $\mathbf{5} = @Succ@(\mathbf{5})$
and $\mathbf{3} = @Succ@(\mathbf{4})$. The latter is indeed @y@, and
we have that $@y@ = @Succ@(\mathbf{4}) = @Succ Zero@$, since
$\mathbf{4}$ is @Zero@, as seen above.

And that is our counterexample - the contract breaks by running
@Zero - Succ Zero@. In general, we can find representatives of
skolem variables with concrete types by this algorithm:

\paragraph{Finding representative constructors}
For each domain element $\mathbf{d}$, this is the procedure to see
which constructors it is equal to.  For a constructor @K@ with
arguments $\mathbf{d_1} \, \cdots \, \mathbf{d_n}$, check in the
tables for @K@, @K_1@, \ldots, @K_n@ if

\[\begin{array}{lcl}
@K@(\mathbf{d_1},\cdots,\mathbf{d_n}) & = & \mathbf{d}   \\
@K@_1(\mathbf{d})                    & = & \mathbf{d_1} \\
@K@_n(\mathbf{d})                    & = & \mathbf{d_n} \\
\end{array}\]

Then we know that $\mathbf{d} = @K @\mathbf{d_1} \, \cdots \, \mathbf{d_n}$.
After all such $\mathbf{d}$ has been determined, constructors and
projections functions are not needed any more.

\subsection{Types}

In the discussion with the subtraction example above, we did not
consider the types of elements - and while there were @Bool@s in the
theory, I did not say anything about them. But it turns out that our
element for @x@, namely $\mathbf{3}$, is also equal to @True@!  This
behaviour - indeed a bit weird at a first glance - is actually
expected since never add any axioms saying that constructors of
different types are unequal - such that $@True@ \neq @Zero@$. This is
deliberate - we will never need to tell these two values apart in a
proof.

So, when printing the values of skolem variables we need to consider
their type.  When printing a skolem variable $x$ in a type $\tau$,
we will only consider constructors of the correct type:

\[\begin{array}{rcl}
\mathsf{show}_{\tau}(x)
    & = & \{ @C@ \, (\mathsf{show}_{\tau_1}(x_1)) \, \cdots \, (\mathsf{show}_{\tau_n}(x_n)) \\
    &   & \mid @C@ : \tau_1 -> \cdots -> \tau_n -> \tau_r \\
    &   & , x = @C@ \, x_1 \, \cdots \, x_n \\
    &   & , \exists \sigma . \sigma\tau_r = \tau \} \\
\end{array}\]

The base case is nullary constructors, or to use another skolem
variable of the right type (but we should avoid printing @x@ = @x@).
The $\sigma$ is a substitution and allows us to pick a more specific
constructor, in particular, to allow us to use
$@UNR@ : \forall \alpha . \alpha$ when, say, a @Bool@ is expected.

If we cannot find a good representative for an element $\mathbf{d}$,
it is printed as a metavariable as @?d@. Futhermore, if $\mathbf{d}$
is not in the min set, it is simply written as $\cdots$.
