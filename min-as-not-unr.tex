
 In fact we may take
one step further and equate all the non-interesting values of the domain to $\bot$.

To achieve this effect, we update our Prelude theory axioms as follows:
{\small
\[\setlength{\arraycolsep}{1pt}
\begin{array}{c}
%% \ruleform{\Th{\Sigma}{P}} \\ \\
\begin{array}{lll}
 \textsc{AxDisjA} & \formula{\bad \neq \unr}  \\
 \textsc{AxDisjB} & \formula{\forall \oln{x}{n}\oln{y}{m} @.@} \\
                  & \formula{\;\;\highlight{K(\ol{x}){\neq}\unr\;\lor\;J(\ol{y}){\neq}\unr} =>
                                  K(\ol{x}){\neq}J(\ol{y})} \\
                  & \text{ for every } (K{:}\forall\as @.@ \oln{\tau}{n} -> T\;\as) \in \Sigma \\
                  & \text{ and } (J{:}\forall\as @.@ \oln{\tau}{m} -> S\;\as) \in \Sigma \\
 %% \textsc{AxDisjCUnr} & \formula{\forall \oln{x}{n} @.@ \highlight{\neg min(\unr)}} \\
 %%                  & \text{ for every } (K{:}\forall\as @.@ \oln{\tau}{n} -> T\;\as) \in \Sigma \\ \\
 \textsc{AxDisjCBad} & \formula{\forall \oln{x}{n} @.@ K(\ol{x}) \neq \bad} \\
                  & \text{ for every } (K{:}\forall\as @.@ \oln{\tau}{n} -> T\;\as) \in \Sigma \\ \\

 \textsc{AxAppA}  & \formula{\forall \oln{x}{n} @.@ f(\ol{x}) = app(f_{ptr},\xs)} \\
                  & \text{ for every } (f |-> \Lambda\as @.@ \lambda\oln{x{:}\tau}{n} @.@ u) \in P \\
 %% \textsc{AxAppB}  & \formula{\forall \oln{x}{n} @.@ K(\ol{x}) = app(\ldots (app(x_K,x_1),\ldots,x_n)\ldots)} \\
 %%                  & \text{ for every } (K{:}\forall\as @.@ \oln{\tau}{n} -> T\;\as) \in \Sigma \\
 \textsc{AxAppC}  & \formula{\forall x, app(\bad,x) = \bad \; /\ \; app(\unr,x) = \unr}    \\ \\
 %% Not needed: we can always extend partial constructor applications to fully saturated and use AxAppC and AxDisjC
 %% \textsc{AxPartA} & \formula{\forall \oln{x}{n} @.@ app(\ldots (app(x_K,x_1),\ldots,x_n)\ldots) \neq \unr} \\
 %%                  & \formula{\quad\quad \land\; app(\ldots (app(x_K,x_1),\ldots,x_n)\ldots) \neq \bad} \\
 %%                  & \text{ for every } (K{:}\forall\as @.@ \oln{\tau}{m} -> T\;\as) \in \Sigma \text{ and } m > n \\
 %% \textsc{AxPartB} & \formula{\forall \oln{x}{n} @.@ app(f_{ptr},\xs) \neq \unr} \\
 %%                  & \formula{\quad\land\; app(f_{ptr},\xs) \neq \bad} \\
 %%                  & \formula{\quad\land\; \forall \oln{y}{k} @.@ app(f_{ptr},\xs) \neq K(\ol{y})} \\
 %%                  & \text{ for every } (f |-> \Lambda\as @.@ \lambda\oln{x{:}\tau}{m} @.@ u) \in P  \\
 %%                  & \text{ and every } (K{:}\forall\as @.@ \oln{\tau}{k} -> T\;\as) \in \Sigma \text{ and } m > n  \\ \\
 \textsc{AxInj}   & \formula{\forall \oln{y}{n} @.@ \highlight{K(\ys) \neq \unr\;\land\; y_i \neq \unr}} \\
                  & \formula{\quad\qquad\qquad => \sel{K}{i}(K(\ys)) = y_i} \\
                  & \text{for every } (K{:}\forall\as @.@ \oln{\tau}{n} -> T\;\as) \in \Sigma \text{ and } i \in 1..n \\ \\
 \textsc{AxCfA}   & \formula{\lcf{\unr} /\ \lncf{\bad}} \\
 \textsc{AxCfB1}  & \formula{\forall \oln{x}{n} @.@ \bigwedge\lcf{\ol{x}}} => \lcf{K(\ol{x})} \\
                  & \text{ for every } (K{:}\forall\as @.@ \oln{\tau}{n} -> T\;\as) \in \Sigma \\
 \textsc{AxCfB2}  & \formula{\forall \oln{x}{n} @.@ \lcf{K(\ol{x})}\;\highlight{\land\;K(\ol{x}) \neq \unr} => \bigwedge\lcf{\ol{x}}} \\
                  & \text{ for every } (K{:}\forall\as @.@ \oln{\tau}{n} -> T\;\as) \in \Sigma
\end{array}
\end{array}\]}


\begin{figure}\small
\[\begin{array}{c}
\ruleform{\utrans{\Sigma}{\Gamma}{t \sim u} = \formula{\phi}} \\ \\
\prooftree
   \begin{array}{c} \ \\ \ \\
   \etrans{\Sigma}{\Gamma}{e} = \formula{t}
   \end{array}
   ----------------------------------------{TUTm}
   \begin{array}{l}
   \utrans{\Sigma}{\Gamma}{s \sim e } = \formula{(s = t) \lor \highlight{\neg min(s)}} \ \\ \ \\ \ \\
   \end{array}
   ~~~~~
  \begin{array}{l}
  \etrans{\Sigma}{\Gamma}{e} = \formula{t} \quad
  constrs(\Sigma,T) = \ol{K} \\
  \text{for each branch}\;(K\;\oln{y}{l} -> e') \\
  \begin{array}{l}
           (K{:}\forall \cs @.@ \oln{\sigma}{l} -> T\;\oln{c}{k}) \in \Sigma \text{ and }
           \etrans{\Sigma}{\Gamma,\ol{y}}{e'} = \formula{ t_K }
  \end{array}
  \end{array}
  ------------------------------------------{TUCase}
  {\setlength{\arraycolsep}{1pt}
  \begin{array}{l}
  \utrans{\Sigma}{\Gamma}{s \sim @case@\;e\;@of@\;\ol{K\;\ol{y} -> e'}} = \\
  \;\;\formula{ \begin{array}{l}
     \highlight{min(s)} => \\
     \begin{array}{ll}
          ( & \highlight{min(t)}\;\land \\
            & (t = \bad => s = \bad)\;\land \\
            & (\forall \ol{y} @.@ t = K_1(\ol{y}) => s = t_{K_1})\;\land \ldots \land \\
            & (t \neq \bad\;\land\;t \neq K_1(\oln{{\sel{K_1}{i}}(t)}{})\;\land\;\ldots => s = \unr) \\
          )
%% (t = \bad /\ s = \bad)\;\lor\;(s = \unr)\;\lor \\
%%                                 \quad      \bigvee(t = K(\oln{{\sel{K}{i}}(t)}{}) \land
%%                                            s = t_K[\oln{\sel{K}{i}(t)}{}/\ol{y}])
                   \end{array}
     \end{array}}
  \end{array}}
  %% {       \setlength{\arraycolsep}{2pt}
  %% \begin{array}{l}
  %% \utrans{\Sigma}{\Gamma}{s \sim @case@\;e\;@of@\;\ol{K\;\ol{y}{->}e'}} = \\
  %% \;\;\formula{
  %%      \begin{array}{l} (\highlight{s{=}\unr})\;\lor \\
  %%                           \;\; (\highlight{min(s) => min(t)}\;\land  \\
  %%                           \quad((t = \bad /\ s = \bad)\;\lor \\
  %%                           \quad\quad \bigvee(t = K(\oln{{\sel{K}{i}}(t)}{}) \land
  %%                                          s = t_K[\oln{\sel{K}{i}(t)}{}/\ol{y}])))
  %%                  \end{array}
  %%          }
  %% \end{array}}
\endprooftree
\end{array}\]
\caption{Minimality-enabled definition translation}\label{fig:min-def-trans-min}
\end{figure}



We will explain the modifications to the axiomatization in more detail in later sections.
%% In other words, we ensure that constructor applications are disjoint
%% only for values we are interested in. We will explain each axiom separately later.
%% Intuitively we wish to equate all terms that we are not interested in to $\unr$. We
%% can never be interested in $\unr$ in the intended model because that means that during
%% the evaluation of a term, which completed, we encountered a divergent term -- clearly a
%% contradiction!
What about function definitions? Figure~\ref{fig:etrans} has to be modified slightly as well,
as Figure~\ref{fig:min-def-trans} shows.

\begin{figure}\small
\[\begin{array}{c}
\ruleform{\utrans{\Sigma}{\Gamma}{t \sim u} = \formula{\phi}} \\ \\
\prooftree
   \begin{array}{c} \ \\ \ \\
   \etrans{\Sigma}{\Gamma}{e} = \formula{t}
   \end{array}
   ----------------------------------------{TUTm}
   \begin{array}{l}
   \utrans{\Sigma}{\Gamma}{s \sim e } = \formula{(s = t) \lor \highlight{s = \unr}} \ \\ \ \\ \ \\
   \end{array}
   ~~~~~
  \begin{array}{l}
  \etrans{\Sigma}{\Gamma}{e} = \formula{t} \quad
  constrs(\Sigma,T) = \ol{K} \\
  \text{for each branch}\;(K\;\oln{y}{l} -> e') \\
  \begin{array}{l}
           (K{:}\forall \cs @.@ \oln{\sigma}{l} -> T\;\oln{c}{k}) \in \Sigma \text{ and }
           \etrans{\Sigma}{\Gamma,\ol{y}}{e'} = \formula{ t_K }
  \end{array}
  \end{array}
  ------------------------------------------{TUCase}
  {\setlength{\arraycolsep}{1pt}
  \begin{array}{l}
  \utrans{\Sigma}{\Gamma}{s \sim @case@\;e\;@of@\;\ol{K\;\ol{y} -> e'}} = \\
  \;\;\formula{ \begin{array}{l}
     \highlight{s = \unr}\;\lor \\
     \begin{array}{ll}
          ( & \highlight{(t \neq \unr)}\;\land \\
            & (t = \bad => s = \bad)\;\land \\
            & (\forall \ol{y} @.@ t = K_1(\ol{y}) => s = t_{K_1})\;\land \ldots \land \\
            & (t = \bad\;\lor\;t = K_1(\oln{{\sel{K_1}{i}}(t)}{})\;\lor\;\ldots) \\
          )
%% (t = \bad /\ s = \bad)\;\lor\;(s = \unr)\;\lor \\
%%                                 \quad      \bigvee(t = K(\oln{{\sel{K}{i}}(t)}{}) \land
%%                                            s = t_K[\oln{\sel{K}{i}(t)}{}/\ol{y}])
                   \end{array}
     \end{array}}
  \end{array}}
  %% {       \setlength{\arraycolsep}{2pt}
  %% \begin{array}{l}
  %% \utrans{\Sigma}{\Gamma}{s \sim @case@\;e\;@of@\;\ol{K\;\ol{y}{->}e'}} = \\
  %% \;\;\formula{
  %%      \begin{array}{l} (\highlight{s{=}\unr})\;\lor \\
  %%                           \;\; (\highlight{min(s) => min(t)}\;\land  \\
  %%                           \quad((t = \bad /\ s = \bad)\;\lor \\
  %%                           \quad\quad \bigvee(t = K(\oln{{\sel{K}{i}}(t)}{}) \land
  %%                                          s = t_K[\oln{\sel{K}{i}(t)}{}/\ol{y}])))
  %%                  \end{array}
  %%          }
  %% \end{array}}
\endprooftree
\end{array}\]
\caption{Minimality-enabled definition translation}\label{fig:min-def-trans}
\end{figure}

%% \\ \\
%% \ruleform{ \Dtrans{\Sigma}{P} = \formula{\phi}} \\ \\
%% \prooftree
%%      \begin{array}{l}
%%        \text{for each} (f |-> \Lambda\oln{a}{n} @.@ \lambda\oln{x{:}\tau}{m} @.@ u) \in P \\
%%           \quad \utrans{\Sigma}{\ol{x}}{f(\ol{x}) \sim u} = \formula{\phi}
%%      \end{array}
%%      --------------------{TDefs}
%%      \Dtrans{\Sigma}{P} = \bigwedge_{P} \formula{\forall \ol{x} @.@ \phi}
%% \endprooftree

Now operationally we may instrument the evaluation relation to keep track of the set of
closed terms that appear during evaluation. The instrumented relation appears in
Figure~\ref{fig:opsem-instrumented}. Observe that if $P |- e \Downarrow w \curly S$ then
$S$ is a {\em finite set} of terms.


\begin{figure}\small
\[\begin{array}{c}
\ruleform{P |- e \Downarrow v \curly S} \\ \\
\prooftree
\begin{array}{c} \ \\
\end{array}
%% \begin{array}{c}
%% (f |-> \Lambda\ol{a} @.@ \lambda\oln{x{:}\tau}{m} @.@ u) \in P \\
%% P |- e_1 \Downarrow f\;[\taus]\;\oln{e}{m-1} \curly S_1 \\
%% P |- u[\ol{\tau}/\ol{a}][\ol{e},e_2/\ol{x}] \Downarrow w \curly S
%% \end{array}
%% ------------------------------------{EBeta}
%% P |- e_1\;e_2 \Downarrow w
  S = heads(v)
-------------------------------------{EVal}
P |- v \Downarrow v \curly S
~~~~
\begin{array}{c}
(f |-> \Lambda\ol{a} @.@ \lambda\oln{x{:}\tau}{m} @.@ u) \in P \\
P |- u[\ol{\tau}/\ol{a}][\ol{e}/\ol{x}] \Downarrow v \curly S_1 \\
S_2 = heads(f[\ol{\tau}]\;\oln{e}{m})
\end{array}
-------------------------------------{EFun}
P |- f[\ol{\tau}]\;\oln{e}{m} \Downarrow v \curly S_1 \cup S_2
~~~~~
\begin{array}{c}
P |- e_1 \Downarrow v_1 \curly S_1 \quad
P |- v_1\;e_2 \Downarrow w \curly S_2
\end{array}
------------------------------------------------{EApp}
P |- e_1\;e_2 \Downarrow w \curly S_1 \cup S_2 \cup \{ e_1\;e_2 \}
~~~~~
\begin{array}{c}
P |- e_1 \Downarrow @BAD@ \curly S
\end{array}
------------------------------------------------{EBadApp}
P |- e_1\;e_2 \Downarrow @BAD@ \curly S \cup \{ e_1\;e_2 \}
\endprooftree \\ \\
\ruleform{heads(e) = S} \\ \\
\begin{array}{lcl}
   heads(f\;[\ol{\tau}]) & = & \{ f\;[\ol{\tau}] \} \\
   heads(e_1\;e_2)       & = & \{ e_1\;e_2 \} \cup heads(e_1) \\
   heads(\_)            & = & \emptyset
\end{array} \\ \\
\ruleform{P |- u \Downarrow v \curly S} \\ \\
\prooftree
P |- e \Downarrow v \curly S
-------------------------------------{EUTm}
P |- e \Downarrow v \curly S
~~~~~
\begin{array}{c}
P |- e \Downarrow K_i[\ol{\sigma}_i](\ol{e}_i) \curly S_1 \quad
P |- e'_i[\ol{e}_i/\ol{y}_i] \Downarrow w \curly S_2
\end{array}
------------------------------------{ECase}
P |- @case@\;e\;@of@\;\ol{K\;\ol{y} -> e'} \Downarrow w \curly S_1 \cup S_2
~~~~~
\begin{array}{c}
P |- e \Downarrow @BAD@ \curly S \\
\end{array}
------------------------------------{EBadCase}
P |- @case@\;e\;@of@\;\ol{K\;\ol{y} -> e'} \Downarrow @BAD@ \curly S
%% \begin{array}{c}
%% (f |-> \Lambda\ol{a} @.@ \lambda\oln{x{:}\tau}{m} @.@ @case@\;e\;@of@\;\ol{K\;\ol{y} -> e'}) \in D \\
%% D |- e[\ol{\tau}/\ol{a}][\ol{e}/\ol{x}] \Downarrow @BAD@ \\
%% \end{array}
%% -------------------------------------{EBadCase}
%% D |- f[\ol{\tau}]\;\oln{e}{m} \Downarrow @BAD@
\endprooftree
\end{array}\]
\caption{Redex-instrumented operational semantics}\label{fig:opsem-instrumented}
\end{figure}