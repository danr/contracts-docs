\documentclass[preprint]{sigplanconf}

\usepackage{hcc-techreport}
\usepackage{ulem}

\begin{document} 
%% \preprintfooter{\textbf{--- DRAFT ---}}


%% \conferenceinfo{POPL'13,} {January 23--25, 2013, Rome, Italy.}
%% \CopyrightYear{2013}
%% \copyrightdata{978-1-4503-1832-7/13/01}

\renewcommand{\dv}[1]{\textcolor{red}{\bf \em DV: #1}}
%% \renewcommand{\spj}[1] {{\textcolor{red}{{SPJ: #1}}} }
%% \renewcommand{\kc}[1]{{\textcolor{magenta}{{KC: #1}}}}
%% \renewcommand{\dr}[1]{{\textcolor{magenta}{{DR: #1}}}}



\renewcommand{\textfraction}{0.1}
\renewcommand{\topfraction}{0.95}
\renewcommand{\dbltopfraction}{0.95}
\renewcommand{\floatpagefraction}{0.8}
\renewcommand{\dblfloatpagefraction}{0.8}

%%% Extra definitions -- move to hcc-techreport at some point (carefully to not break that!)

\newcommand{\Ct}{{\tt C}}   %% arbitrary contract
\newcommand{\Bt}{{\tt B}}   %% base contract
\newcommand{\Ft}{{ \tt F}}  %% first-order contract

\newcommand{\CF}{{\tt CF}}
\newcommand{\True}{\textit{True}}
\newcommand{\False}{\textit{False}}
\newcommand{\Bool}{\mathop{Bool}}
\newcommand{\ys}{\ol{y}}
\newcommand{\Th}[2]{{\cal T}_{#1,#2}}
\newcommand{\Ecf}{\textsc{Ecf}}
\newcommand{\oln}[2]{\ol{#1}^{#2}}
\newcommand{\tmar}[2]{\mathop{tmar}_{#1}(#2)}
\newcommand{\tyar}[2]{\mathop{tyar}_{#1}(#2)}
\newcommand{\ar}{n}
\newcommand{\lcf}[1]{\textsf{cf}(#1)}
\newcommand{\lcfZ}{\textsf{cf}}
\newcommand{\lncf}[1]{\neg\textsf{cf}(#1)}
\newcommand{\unr}{\mathop{unr}}
\newcommand{\bad}{\mathop{bad}}
\newcommand{\sel}[3]{\mathop{app(sel\_#1\!_{#2}, #3)}}
\newcommand{\ctrans}[3]{\{\!\!\{#3\}\!\!\}}
\newcommand{\etrans}[3]{\{\!\!\{{#3}\}\!\!\}}
\newcommand{\utrans}[3]{(#3 \sim \{\!\!\{#2\}\!\!\})}

% Get rid of this -- just temporay
\newcommand{\uutrans}[3]{\{\!\!\{#3\}\!\!\}}
\newcommand{\zs}{\ol{z}}

\newcommand{\red}{\rightsquigarrow}

\newcommand{\dtrans}[2]{\{\!\!\{#2\}\!\!\}}
\newcommand{\ptrans}[2]{\{\!\!\{#2\}\!\!\}}
%% Gadgets of domain theory 



\newcommand{\rollK}{\mathsf{roll}}
\newcommand{\unrollK}{\mathsf{unroll}}
\newcommand{\bindK}{\mathsf{bind}}
\newcommand{\retK}{\mathsf{ret}}

% \newcommand{\roll}[1]{\rollK(#1)}
% \newcommand{\unroll}[1]{\unrollK(#1)}
% \newcommand{\bind}[2]{\bindK_{#1}(#2)}
% \newcommand{\ret}[1]{\retK(#1)}
\newcommand{\roll}[1]{#1}
\newcommand{\unroll}[1]{#1}
\newcommand{\bind}[2]{#1(#2)}
\newcommand{\ret}[1]{#1}

\newcommand{\dlambda}{\mathsf{\lambda}}
\newcommand{\curry}{\mathsf{curry}}
\newcommand{\eval}{\mathsf{eval}}
\newcommand{\uncurry}{\mathsf{incurry}}
\newcommand{\dapp}{\mathsf{app}}

\newcommand{\injK}[2]{\mathsf{#1}(#2)}
\newcommand{\injKZ}[1]{\mathsf{#1}}
\newcommand{\injFun}[1]{\mathsf{Fun}(#1)}
\newcommand{\injBad}{\mathsf{Bad}}

\newcommand{\unitcpo}{{\sf{\bf 1}}}
\newcommand{\VarCpo}{\textit{Var}}
\newcommand{\FVarCpo}{\textit{FunVar}}
\newcommand{\interp}[3]{[\![#1]\!]_{\langle {#2},{#3}\rangle}}
\newcommand{\dbrace}[1]{[\![#1]\!]}
\newcommand{\linterp}[1]{{\cal I}(#1)}
\newcommand{\lassign}[1]{\mu(#1)}
\newcommand{\elab}[1]{\rightsquigarrow \formula{#1}}
\newcommand{\Fcf}{F_{\lcfZ}} 
\newcommand{\definable}[1]{{\mathop{def}}(#1)}
\newcommand{\curly}{\rightsquigarrow}
\newcommand{\Min}{\cal M}
\newcommand{\mlinterp}[1]{{\cal I}^{min}(#1)}

\renewcommand{\Th}{{\cal T}}

\newcommand{\theLang}{\lambda_{\sf HALO}}

\title{HALO2: HAskell to LOgic through Operational semantics}
\subtitle{Reconciling verification and model checking for functional programs}
%% \subtitle{A new approach to static contract checking for higher-order lazy programs}

\authorinfo{Dimitrios Vytiniotis \\ Simon Peyton Jones}
           {Microsoft Research}{}

\authorinfo{Dan Ros\'{e}n \\ Koen Claessen}
           {Chalmers University}{}
%% \authorinfo{Nathan Collins}
%%            {Portland State University}{}
\maketitle
\makeatactive

\begin{abstract}
\end{abstract}

\category{D.3}{Software}{Applicative (functional) languages}

\terms
verification, languages

\keywords
static contract checking, first-order logic


\section{Introduction}\label{s:intro}


\subsection{Syntax of $\theLang$} \label{s:syntax}

\begin{figure}
\[\begin{array}{l}
\begin{array}{lrll}
\multicolumn{4}{l}{\text{Programs, definitions, and expressions}} \\
P   & ::= & d_1 \ldots d_n \\
d   & ::= & f\; \ol{a} \; \ol{x\!:\!\tau} = u \\
u   & ::= & \multicolumn{2}{l}{e \; \mid \; @case@\;e\;@of@\;\ol{K\;\ol{y} -> e}} \\
e  & ::=  & x            & \text{Variables} \\
   & \mid & f[\ol{\tau}] & \text{Function variables} \\
   & \mid & K[\ol{\tau}](\ol{e}) & \text{Data constructors} \\
   & \mid & e\;e         & \text{Applications} \\
   & \mid & @BAD@        & \text{Runtime error} \\
\end{array}\\ \\
\begin{array}{lrll}
\multicolumn{3}{l}{\text{Syntax of closed values}} \\
 v,w & ::= & K^\ar[\ol{\tau}](\oln{e}{\ar}) \;\mid\; f^\ar[\ol{\tau}]\;\oln{e}{m < \ar}\;\mid\;@BAD@ \\ \\
\end{array}
\\
\begin{array}{lrll}
\multicolumn{3}{l}{\text{Contracts}} \\
 \Ct & ::=  & \{ x \mid e \}        & \text{Base contracts}  \\
     & \mid &  (x : \Ct_1) -> \Ct_2      & \text{Arrow contracts} \\
%%     & \mid & \Ct_1 \& \Ct_2             & \text{Conjunctions}   \\
%%     & \mid & \CF                        & \text{Crash-freedom}   \\
\end{array}
\\ \\
\begin{array}{lrll}
\multicolumn{3}{l}{\text{Types}} \\
\tau,\sigma & ::=  & T\;\taus & \text{Datatypes} \\
            & \mid & a \mid \tau -> \tau
\end{array}
\\ \\
\begin{array}{lrll}
\multicolumn{3}{l}{\text{Type environments and signatures}} \\
\Gamma & ::=  & \cdot \mid \Gamma,x \\
\Delta & ::=  & \cdot \mid \Delta,a \mid \Delta,x{:}\tau \\
\Sigma & ::=  & \cdot \mid \Sigma,T{:}n \mid \Sigma,f{:}\forall\ol{a} @.@ \tau \mid \Sigma,K^{\ar}{:}\forall\ol{a} @.@ \oln{\tau}{\ar} -> T\;\as
\end{array}
\\ \\
\begin{array}{lrll}
\multicolumn{3}{l}{\text{Auxiliary functions}} \\
%% constrs(\Sigma,T) & = & \{ K \mid (K{:}\forall \as @.@ \taus -> T\;\as) \in \Sigma \} \\
(\cdot)^{-}            & = & \cdot \\
(\Delta,a)^{-}         & = & \Delta^{-} \\
(\Delta,(x{:}\tau))^{-} & = & \Delta^{-},x
%% \tyar{D}{f} & = & n & \\
%%             & \multicolumn{3}{l}{\text{when}\; (f |-> \Lambda\oln{a}{n} @.@ \lambda\ol{x{:}\tau} @.@ u) \in D} \\
%% \tmar{D}{f} & = & n & \\
%%             & \multicolumn{3}{l}{\text{when}\; (f |-> \Lambda\ol{a} @.@ \lambda\oln{x{:}\tau}{n} @.@ u) \in D}
\end{array}
\end{array}\]
\caption{Syntax of $\theLang$ and its contracts}\label{fig:syntax}
\end{figure}

The syntax, and contracts are given in Figure~\ref{fig:syntax}. No
pointers, no selectors, no crash-freedom for now until we understand
the theory well enough.

\begin{figure*}\small
\[\begin{array}{c} 
\ruleform{P |- e \Downarrow v} \\ \\
\prooftree
\phantom{P}
-------------------------------------{EVal}
P |- v \Downarrow v
~~~~
\begin{array}{c}
(f |-> \Lambda\ol{a} @.@ \lambda\oln{x{:}\tau}{m} @.@ u) \in P \\
P |- u[\ol{\tau}/\ol{a}][\ol{e}/\ol{x}] \Downarrow v
\end{array}
-------------------------------------{EFun}
P |- f[\ol{\tau}]\;\oln{e}{m} \Downarrow v
~~~~
\begin{array}{c}  
P |- e_1 \Downarrow v_1 \quad
P |- v_1\;e_2 \Downarrow w
\end{array}
------------------------------------------------{EApp}
P |- e_1\;e_2 \Downarrow w
~~~~
\begin{array}{c}  
\phantom{P |- e_1 \Downarrow }
\end{array}
------------------------------------------------{EBadApp}
P |- @BAD@\;e_2 \Downarrow @BAD@
\endprooftree \\ \\ 
\ruleform{P |- u \Downarrow v} \\ \\
\prooftree
P |- e \Downarrow v
-------------------------------------{EUTm}
P |- e \Downarrow v
~~~~ 
\begin{array}{c}
P |- e \Downarrow K_i[\ol{\sigma}_i](\ol{e}_i) \quad K_i \in \ol{K} \quad
P |- e'_i[\ol{e}_i/\ol{y}_i] \Downarrow w
\end{array}
------------------------------------{ECase}
P |- @case@\;e\;@of@\;\ol{K\;\ol{y} -> e'} \Downarrow w
~~~~
\begin{array}{c}
P |- e \Downarrow @BAD@ \\
\end{array}
------------------------------------{EBadCase}
P |- @case@\;e\;@of@\;\ol{K\;\ol{y} -> e'} \Downarrow @BAD@
%% \begin{array}{c}
%% (f |-> \Lambda\ol{a} @.@ \lambda\oln{x{:}\tau}{m} @.@ @case@\;e\;@of@\;\ol{K\;\ol{y} -> e'}) \in D \\
%% D |- e[\ol{\tau}/\ol{a}][\ol{e}/\ol{x}] \Downarrow @BAD@ \\
%% \end{array}
%% -------------------------------------{EBadCase}
%% D |- f[\ol{\tau}]\;\oln{e}{m} \Downarrow @BAD@
\endprooftree
\end{array}\]
\caption{Operational semantics}\label{fig:opsem}
\end{figure*}

\begin{figure*}\small
\[\begin{array}{c} 
\ruleform{P |- e \downarrow v} \\ \\
v ::= \ldots \mid @UNR@ \\ \\ 
\prooftree
\phantom{P}
-------------------------------------{EVal}
P |- v \downarrow v
~~~~
\begin{array}{c}
(f |-> \Lambda\ol{a} @.@ \lambda\oln{x{:}\tau}{m} @.@ u) \in P \\
P |- u[\ol{\tau}/\ol{a}][\ol{e}/\ol{x}] \downarrow v
\end{array}
-------------------------------------{EFun}
P |- f[\ol{\tau}]\;\oln{e}{m} \downarrow v
~~~~
\begin{array}{c}  
P |- e_1 \downarrow v_1 \quad
P |- v_1\;e_2 \downarrow w
\end{array}
------------------------------------------------{EApp}
P |- e_1\;e_2 \downarrow w
~~~~
\begin{array}{c}  
\phantom{P |- e_1 \downarrow } 
\end{array}
------------------------------------------------{EBadApp}
P |- @BAD@\;e_2 \downarrow @BAD@
~~~~~
\begin{array}{c}  
\phantom{P |- e_1 \downarrow ...}
\end{array}
------------------------------------------------{EUnrApp}
P |- @UNR@\;e_2 \downarrow @UNR@
~~~~
\phantom{P}
---------------------------------------{EConApp}
P |- K[\ol{\taus}](\es)\;e \downarrow @UNR@

\endprooftree \\ \\ 
\ruleform{P |- u \downarrow v} \\ \\
\prooftree
P |- e \downarrow v
-------------------------------------{EUTm}
P |- e \downarrow v
~~~~ 
\begin{array}{c}
P |- e \downarrow K_i[\ol{\sigma}_i](\ol{e}_i) \quad K_i \in \ol{K} \quad
P |- e'_i[\ol{e}_i/\ol{y}_i] \downarrow w
\end{array}
------------------------------------{ECase}
P |- @case@\;e\;@of@\;\ol{K\;\ol{y} -> e'} \downarrow w
~~~~
\begin{array}{c}
P |- e \downarrow v \quad v \neq K_i(\ol{e}),@BAD@ %% K_i[\ol{\sigma}_i](\ol{e}_i) \quad
%% K_i \notin \ol{K}
\end{array}
------------------------------------{EUnrCase}
P |- @case@\;e\;@of@\;\ol{K\;\ol{y} -> e'} \downarrow @UNR@
~~~~~
\begin{array}{c}
P |- e \downarrow @BAD@ \\
\end{array}
------------------------------------{EBadCase}
P |- @case@\;e\;@of@\;\ol{K\;\ol{y} -> e'} \downarrow @BAD@
%% ~~~~
%% \begin{array}{c}
%% P |- e \downarrow @UNR@ \\
%% \end{array}
%% ------------------------------------{EScrUnrCase}
%% P |- @case@\;e\;@of@\;\ol{K\;\ol{y} -> e'} \downarrow @UNR@
%% \begin{array}{c}
%% (f |-> \Lambda\ol{a} @.@ \lambda\oln{x{:}\tau}{m} @.@ @case@\;e\;@of@\;\ol{K\;\ol{y} -> e'}) \in D \\
%% D |- e[\ol{\tau}/\ol{a}][\ol{e}/\ol{x}] \downarrow @BAD@ \\
%% \end{array}
%% -------------------------------------{EBadCase}
%% D |- f[\ol{\tau}]\;\oln{e}{m} \downarrow @BAD@
\endprooftree
\end{array}\]
\caption{Operational semantics with an explicit stuck value}\label{fig:opsem}
\end{figure*}



\begin{figure}
\[\begin{array}{c}
\begin{array}{lrll}
\multicolumn{3}{l}{\text{Terms}} \\
  s,t & ::=  & x                          & \text{Variables} \\
%       & \mid & f(\ol{t})                  & \text{Function applications} \\
      & \mid & K(\ol{t})                  & \text{Constructor applications} \\
%%      & \mid & \sel{K}{i}(t)              & \text{Constructor selectors} \\
      & \mid & f_{ptr} \mid app(t,s)       & \text{Pointers and application} \\
      & \mid & \unr \mid \bad             & \text{Unreachable, bad} \\ \\
\multicolumn{3}{l}{\text{Formulae}} \\
 \phi & ::=  & min(t)    & \text{Min-predicate} \\
%%      & \mid & \lncf{t}   & \text{Can provably cause crash} \\
      & \mid & t_1 = t_2  & \text{Equality} \\
      & \mid & \phi \land \phi \mid \phi \lor \phi \mid \neg \phi \\
      & \mid & \forall x @.@ \phi \mid \exists x @.@ \phi \\ \\
\end{array}
\\
\multicolumn{1}{l}{\text{Abbreviations}} \\
\begin{array}{rcl}
app(t,\oln{s}{n}) & = & (\ldots(app(t,s_1),\ldots s_n)\ldots) \\
\phi_1 \Rightarrow \phi_2 & = & \neg \phi_1 \lor \phi_2
\end{array}
\end{array}\]
\caption{Syntax of FOL}\label{fig:fol-image}
\end{figure}

The FOL language that we will use is in Figure~\ref{fig:fol-image}. 
Again nothing fancy, except for the $\min{t}$ predicate.

% ---------------------------------------------------
\begin{figure}
\setlength{\arraycolsep}{2pt}
\[\begin{array}{c}
\ruleform{\ptrans{\Sigma}{P} = \formula{\phi} } \\ \\ 
\ptrans{\Sigma}{\ol{d}} = \bigwedge \ol{\dtrans{\Sigma}{d}}
\\ \\
\ruleform{\dtrans{\Sigma}{d} = \formula{\phi}} \\ \\
\begin{array}{rcl}
  \dtrans{\Sigma}{f \;\ol{a}\;\ol{x{:}\tau} = u}
    & =     & \formula{\forall \ol{x} @.@\, \utrans{\sigma}{u}{app(f_{ptr},\xs)}} \\
%%    & \land & \formula{\forall \ol{x} @.@\, f(\ol{x}) = app(f_{ptr},\xs)} \\
\end{array}
\\ \\
\ruleform{\utrans{\Sigma}{u}{s} = \formula{\phi}} \\ \\
\begin{array}{rcl}
\utrans{\Sigma}{e}{s}
  & = & \formula{(s = \etrans{\Sigma}{\Gamma}{e})} \\
\multicolumn{3}{l}{\utrans{\Sigma}
    {@case@\;e\;@of@\;\ol{K\;\ol{y} -> e'}}{s}} \\
\multicolumn{3}{l}{
\quad
  \begin{array}[t]{rl}
    = & \formula{(\forall \ol{y} @.@ t = K_1(\ol{y}) => s = \etrans{\Sigma}{\Gamma}{e'_1})\;\land \ldots}  \\
    \land & t = \bad => s = \bad \\
    \land & \formula{(\forall \xs @.@ t{\neq}K_1(\xs)\;\land\;\ldots\; s \neq \bad => s=\unr)} \\
    \mbox{where} & t  =  \etrans{\Sigma}{\Gamma}{e}
 \end{array}
}
\end{array}
\\ \\
\ruleform{\etrans{\Sigma}{\Gamma}{e} = \formula{t} } \\ \\
\begin{array}{rcl}
\etrans{\Sigma}{\Gamma}{x} & = & \formula{x} \\
\etrans{\Sigma}{\Gamma}{f[\ol{\tau}]} & = & \formula{f_{ptr}} \\
\etrans{\Sigma}{\Gamma}{K[\ol{\tau}](\ol{e})} & = & \formula{K(\ol{\etrans{\Sigma}{\Gamma}{e}})} \\
\etrans{\Sigma}{\Gamma}{e_1\;e_2} & = & \formula{app(\etrans{\Sigma}{\Gamma}{e_1},
                                                     \etrans{\Sigma}{\Gamma}{e_2})}
%% rans{\Sigma}{\Gamma}{@BAD@} & = & \formula{\bad}
\end{array}
\\ \\
\ruleform{\ctrans{\Sigma}{\Gamma}{e \in \Ct} = \formula{\phi}} \\ \\
\begin{array}{rcl}
\ctrans{\Sigma}{\Gamma}{e \in \{x\;\mid\;e' \}}
  & = & \formula{t{=}\unr} \\
  & \lor & \formula{t'[t/x]{=}\unr} \\
  & \lor & \formula{t'[t/x]{=}\True} \\
  & \mbox{where} &
    \begin{array}[t]{rcl}
      t  & = & \etrans{\Sigma}{\Gamma}{e} \; \text{and} \; t' = \etrans{\Sigma}{\Gamma}{e'}
    \end{array}
\\
\ctrans{\Sigma}{\Gamma}{e \in (x{:}\Ct_1) -> \Ct_2}
  & = & \formula{\forall x @.@ \ctrans{\Sigma}{\Gamma,x}{x \in \Ct_1}
                          \Rightarrow \ctrans{\Sigma}{\Gamma,x}{e\;x \in \Ct_2}}
%% \\
%% \ctrans{\Sigma}{\Gamma}{e \in \Ct_1 \& \Ct_2}
%%    & = & \formula{ \ctrans{\Sigma}{\Gamma}{e \in \Ct_1} /\ \ctrans{\Sigma}{\Gamma}{e \in \Ct_2}}
%% \\
%% \ctrans{\Sigma}{\Gamma}{e \in \CF} & = & \formula{\lcf{\etrans{\Sigma}{\Gamma}{e}}}
\end{array}
\end{array}\]
\caption{Translation of programs and contracts to logic}
   \label{fig:etrans}\label{fig:contracts-minless}
\end{figure}
% ---------------------------------------------------
\begin{figure}
\setlength{\arraycolsep}{1pt}
\[\begin{array}{c}
\ruleform{\text{Theory}\,\Th} \\ \\ 
\begin{array}{ll}
%% \multicolumn{2}{l}{\text{Axioms for $bad$ and $unr$}} \\
 \textsc{AxAppBad}  & \formula{\forall x @.@ app(\bad,x){=}\bad}  \\
 \textsc{AxAppUnr}  & \formula{\forall x @.@ app(\unr,x){=}\unr}    \\
 \textsc{AxAppPart} & \formula{\forall \ys\xs @.@ app(f^n, \xs^{m < n}) \neq \bad,J(\ol{y}),\unr} \\
 \textsc{AxDisjBU} & \formula{\bad \neq \unr}  \\
\\
%% \multicolumn{2}{l}{\mbox{Axioms for data constructors}} \\
 \textsc{AxDisjC} & \formula{\forall \oln{x}{n}\oln{y}{m} @.@ K(\ol{x}) \neq J(\ol{y})} \\
                  & \text{ for every } (K{:}\forall\as @.@ \oln{\tau}{n} -> T\;\as) \in \Sigma \\
                  & \text{ and } (J{:}\forall\as @.@ \oln{\tau}{m} -> S\;\as) \in \Sigma \\
                  & \text{ and } K \neq J \\
 \textsc{AxDisjCBU} & \formula{\forall \oln{x}{n} @.@ K(\ol{x}) \neq \unr,\bad} \\
                  & \text{ for every } (K{:}\forall\as @.@ \oln{\tau}{n} -> T\;\as) \in \Sigma \\

%% DEATH TO INJECTIVITY
%% \textsc{AxInj}   & \formula{\forall \oln{y}{n} @.@ \sel{K}{i}(K(\ys)) = y_i} \\
%%                  & \text{for every } K^\ar \in \Sigma \text{ and } i \in 1..n \\
%% \\
%% \multicolumn{2}{l}{\mbox{Axioms for crash-freedom}} \\
%%  \textsc{AxCfC}  & \formula{\forall \oln{x}{n} @.@ \lcf{K(\ol{x})} <=> \bigwedge\lcf{\ol{x}}} \\
%%                  & \text{ for every } (K{:}\forall\as @.@ \oln{\tau}{n} -> T\;\as) \in \Sigma \\
 %% \textsc{AxCfBU} & \formula{\lcf{\unr} /\ \lncf{\bad}} \\
\end{array}
\end{array}\]
\caption{Theory $\Th$: axioms of the FOL constants}\label{fig:prelude} \label{fig:data-cons}
\end{figure}


\section{Minimization-based translation}

Before I even start with the minimization-based translation I want to
observe that I think nothing breaks (i.e. we do not become less
complete if we remove the injectivity axioms for constructors). Not
sure how exactly to prove this (because of the same issues that
that arise when proving completeness further down...)

The min-based translation is given in Figures~\ref{fig:min-etrans}
and~\ref{fig:min-prelude}. I use evaluation instead of equality
because this is the only way I've managed to produce an FOL model from
a trace -- the symmetry and congruence properties of equality made this
result impossible to prove for equality.

\begin{figure}
\setlength{\arraycolsep}{2pt}
\[\begin{array}{c}
\ruleform{\ptrans{\Sigma}{P}^M = \formula{\phi} } \\ \\ 
\ptrans{\Sigma}{\ol{d}}^M = \bigwedge \ol{\dtrans{\Sigma}{d}^M}
\\ \\
\ruleform{\dtrans{\Sigma}{d}^M = \formula{\phi}} \\ \\
\begin{array}{lcl}
  \dtrans{\Sigma}{f \;\ol{a}\;\ol{x{:}\tau} = u} & = &  \\ 
  \multicolumn{3}{l}{\qquad\qquad \formula{\forall\ol{x} @.@\, {\bf min(app(f_{ptr},\xs)) => } \utrans{\sigma}{u}{app(f_{ptr},\xs)}}} \\
%%     & \land & \formula{\forall \ol{x} @.@\, f(\ol{x}) = app(f_{ptr},\xs)} \\
\end{array} \\ \\ 
\ruleform{\utrans{\Sigma}{u}{s}^M = \formula{\phi}} \\ \\
\begin{array}{rcl}
\utrans{\Sigma}{e}{s}^M
  & = & \formula{(s \red \etrans{\Sigma}{\Gamma}{e})} \\
\multicolumn{3}{l}{\utrans{\Sigma}
    {@case@\;e\;@of@\;\ol{K\;\ol{y} -> e'}}{s}^M} \\
\multicolumn{3}{l}{
\quad
  \begin{array}[t]{rl}
    = & {\bf min(t)} \;  \land \\
      & \formula{\forall\ol{y} @.@ t \red K_1(\ol{y}) => s \red \etrans{\Sigma}{\Gamma}{e'_1}\;\land}  \\
      & \ldots \;\land \\
      & \formula{t \red \bad => s \red \bad \land}  \\
      & \formula{((\forall \xs @.@ t{\not\red}K_1(\xs)) \land \ldots \land t{\not\red}\bad => s{\red}\unr)} \\
    \mbox{where} & t  =  \etrans{\Sigma}{\Gamma}{e}
 \end{array}
}
\end{array}
\\ \\
\ruleform{\ctrans{\Sigma}{\Gamma}{e \in \Ct}^M = \formula{\phi}} \\ \\
\begin{array}{rcl}
\multicolumn{3}{l}{\ctrans{\Sigma}{\Gamma}{e \in \{x\;\mid\;e' \}}^M = }\\
\multicolumn{3}{l}{  \qquad ({\bf min(t)} \land t\red\unr)\;\lor} \\ 
\multicolumn{3}{l}{  \qquad({\bf min(t'[t/x])}\;\land\; (t'[t/x]{\red}\unr \lor t'[t/x]{=}\True))} \\
%%   &   & \;\lor\; \formula{t'[t/x]{=}\True}) \\
  \mbox{where} & &
      t = \etrans{\Sigma}{\Gamma}{e} \quad t' = \etrans{\Sigma}{\Gamma}{e'}
\\
\ctrans{\Sigma}{\Gamma}{e \in (x{:}\Ct_1) -> \Ct_2}^M
  & = & \formula{\forall x @.@ \ctrans{\Sigma}{\Gamma,x}{x \notin \Ct_1}^M \lor} \\
  &   & \qquad\qquad \formula{\ctrans{\Sigma}{\Gamma,x}{e\;x \in \Ct_2}}^M
%% \\
%% \ctrans{\Sigma}{\Gamma}{e \in \Ct_1 \& \Ct_2}
%%    & = & \formula{ \ctrans{\Sigma}{\Gamma}{e \in \Ct_1} /\ \ctrans{\Sigma}{\Gamma}{e \in \Ct_2}}
%% \\
%% \ctrans{\Sigma}{\Gamma}{e \in \CF} & = & \formula{\lcf{\etrans{\Sigma}{\Gamma}{e}}}
\end{array}
\\ \\
\ruleform{\ctrans{\Sigma}{\Gamma}{e \notin \Ct}^M = \formula{\phi}} \\ \\
\begin{array}{rcl}
\multicolumn{3}{l}{ \ctrans{\Sigma}{\Gamma}{e \notin \{x\;\mid\;e' \}}^M = } \\ 
\multicolumn{3}{l}{ \qquad {\bf min(t)\;\land\; min(t'[t/x])}\;\land} \\ 
\multicolumn{3}{l}{ \qquad (\formula{t \not \red \unr\;\land\; t'[t/x] \not\red \unr \;\land\; t'[t/x] \not\red \True})} \\
%% \multicolumn{3}{l}{ \qquad \land\; \formula{t'[t/x]{\not\red}\True}} \\
  \mbox{where} & & 
      t= \etrans{\Sigma}{\Gamma}{e} \;\; t' = \etrans{\Sigma}{\Gamma}{e'}
\\
\ctrans{\Sigma}{\Gamma}{e \notin (x{:}\Ct_1) -> \Ct_2}^M
  & = & \formula{\exists x @.@ \ctrans{\Sigma}{\Gamma,x}{x \in \Ct_1}^M \land} \\
  &   & \qquad\qquad \formula{\ctrans{\Sigma}{\Gamma,x}{e\;x \notin \Ct_2}}^M
%% \\
%% \ctrans{\Sigma}{\Gamma}{e \in \Ct_1 \& \Ct_2}
%%    & = & \formula{ \ctrans{\Sigma}{\Gamma}{e \in \Ct_1} /\ \ctrans{\Sigma}{\Gamma}{e \in \Ct_2}}
%% \\
%% \ctrans{\Sigma}{\Gamma}{e \in \CF} & = & \formula{\lcf{\etrans{\Sigma}{\Gamma}{e}}}
\end{array}
\end{array}\]
\caption{Translation to logic -- $min$ version}
   \label{fig:min-etrans}\label{fig:contracts-minfull}
\end{figure}


\begin{figure*}
\setlength{\arraycolsep}{3pt}
\[\begin{array}{c}
\ruleform{\text{Theory}\,\Th^M} \\ \\ 
\begin{array}{ll}
 \textsc{AxEqRefl}  & \forall x @.@ min(x) => x \red x \\
 \textsc{AxEqTrans} & \forall xyz @.@ x \red y \;\land\; y \red z => x \red z \\
 \textsc{AxEqMin}   & \forall xy @.@ min(x) \;\land\; x \red y => min(y) \\
 \textsc{AxAppMin}  & \formula{\forall x y @.@ min(app(x,y)) => min(x) } \\
 \textsc{AxEqCong}  & \forall x x' y @.@ min(app(x,y))\;\land\;x \red x' => app(x,y) \red app(x',y) \\ \\ 

 %% \textsc{AxDisjUnr}  & \forall x @.@ min(x) \land x \red \unr => x \not\red J(\ol{\sel{J}{i}(x)}) \land x \not\red \bad \\
 %% \textsc{AxDisjCon}  & \forall x\ys @.@ min(x) \land x \red K(\ys) => x \not\red J(\ol{\sel{J}{i}(x)}) \land x \not\red \bad \\
 %% \textsc{AxDisjApp}  & \forall x\ys @.@ min(x) \land x \red app(f_{ptr}^m,\ol{y}^{< m}) => x \not\red J(\ol{\sel{J}{i}(x)}) \land x \not\red \bad \\ \\ 

 \textsc{AxDisjUnr}  & \forall x\ys\zs @.@ min(x) \land x \red \unr => 
                         x \not\red J(\ys) \land x \not\red \bad \land x \not\red app(f_{ptr}^m,\ol{z}^{< m}) \\
 \textsc{AxDisjCon}  & \forall x\ys\xs\zs @.@ min(x) \land x \red K(\ys) => 
                         x \not\red J(\xs) \land x \not\red \bad \land x \not\red app(f_{ptr}^m,\ol{z}^{< m}) \\
 \textsc{AxDisjApp}  & \forall x\ys\xs @.@ min(x) \land x \red app(f_{ptr}^m,\ol{y}^{< m}) => x \not\red J(\xs) \land x \not\red \bad \\ \\ 

 \textsc{AxAppBad}   & \forall x @.@ min(app(\bad,x)) => app(\bad,x) \red \bad \\
 \textsc{AxAppUnr}   & \forall x @.@ min(app(\unr,x)) => app(\unr,x) \red \unr \\
 \textsc{AxAppCon}   & \forall \xs,y @.@ min(app(K(\xs),y)) => app(K(\xs),y) \red \unr 

 %% %% \multicolumn{2}{l}{\text{Axioms for $bad$ and $unr$}} \\
%%  \textsc{AxAppBad}  & \formula{\forall x @.@ app(\bad,x){=}\bad}  \\
%%  \textsc{AxAppUnr}  & \formula{\forall x @.@ app(\unr,x){=}\unr}    \\

%%  \textsc{AxAppPartUnr} & \formula{\forall \xs @.@ {\bf min(app(f^n, \xs^{m < n}))} =>} \\
%%                        & \qquad\qquad\quad app(f,\xs){\neq}\unr \\
%%  \textsc{AxDisjBU} & \formula{\bad \neq \unr}  \\
%% \\
%% %% \multicolumn{2}{l}{\mbox{Axioms for data constructors}} \\
%%  \textsc{AxDisjC} & \formula{\forall \oln{x}{n}\oln{y}{m} @.@ {\bf min(K(\ol{x})) \lor min(J(\ol{y})) => }} \\ 
%%                   & \qquad\qquad\quad K(\ol{x}){\neq}J(\ol{y}) \\
%%                   & \text{ for every } (K{:}\forall\as @.@ \oln{\tau}{n} -> T\;\as) \in \Sigma \\
%%                   & \text{ and } (J{:}\forall\as @.@ \oln{\tau}{m} -> S\;\as) \in \Sigma \\
%%                   & \text{ and } K \neq J \\
%%  \textsc{AxDisjCBU} & \formula{\forall \oln{x}{n} @.@ {\bf min(K(\ol{x})) =>} K(\ol{x}) \neq \unr,\bad} \\
%%                   & \text{ for every } (K{:}\forall\as @.@ \oln{\tau}{n} -> T\;\as) \in \Sigma \\
%%  \textsc{AxInj}   & \formula{\forall \oln{y}{n} @.@ {\bf min(K(\ys)) => } \sel{K}{i}(K(\ys)) = y_i} \\
%%                   & \text{for every } K^\ar \in \Sigma \text{ and } i \in 1..n \\
%% %% \\
%% %% \multicolumn{2}{l}{\mbox{Axioms for crash-freedom}} \\
%% %%  \textsc{AxCfC}  & \formula{\forall \oln{x}{n} @.@ \lcf{K(\ol{x})} <=> \bigwedge\lcf{\ol{x}}} \\
%% %%                  & \text{ for every } (K{:}\forall\as @.@ \oln{\tau}{n} -> T\;\as) \in \Sigma \\
%%  %% \textsc{AxCfBU} & \formula{\lcf{\unr} /\ \lncf{\bad}} \\
\end{array}
\end{array}\]
\caption{Theory $\Th^M$: axioms of the FOL constants -- $min$ version}\label{fig:min-prelude} \label{fig:data-cons}
\end{figure*}

\section{Soundness of min-translation}

Soundness of the $min$-enabled translation follows by a simple
model-theoretic argument, shown below.

\begin{theorem}[Soundness]
If $\Th^M,\ptrans{}{P}^M \vdash \neg \ctrans{}{}{e \notin C}^M$ then
   $\Th , \ptrans{}{P} \vdash \neg \ctrans{}{}{e \notin C}$.
\end{theorem}
\begin{proof}
Let us pick any model $M \models \Th,\ptrans{}{P}$. Consider the model 
$M^{\sharp}$ in which we interpret $min(\cdot)$ as everywhere true, and 
$\red$ as equality. It is an easy check then to establish the following:
\[   M^{\sharp} \models \Th^M \] 
and
\[   M^{\sharp} \models \ptrans{}{P} \]
Hence we get that $M^{\sharp} \models \neg \ctrans{}{}{e \notin C}^M$ 
by assumption. This is equivalent to $M \models \neg \ctrans{}{}{e \notin C}$,
by induction on $C$.
\end{proof}

\section{Completeness results and non-results}

For completeness we need a sequence of lemmas first.

\begin{lemma}[Guarded evaluation]\label{lem:guarded-eval}
If $P |- e \Downarrow v$ then $\Th^M, \ptrans{}{P}^M \vdash min(t) => t = t_v$
where $t = \etrans{}{}{e}$ and $t_v = \etrans{}{}{v}$.
\end{lemma}
\begin{proof} 
By induction on the evaluation of $e \Downarrow v$. We have the following
cases to consider:
\begin{itemize}
  \item Case \rulename{EVal}. In this case we have $P |- v \Downarrow v$ and the result follows by rule \rulename{AxEqRefl}.
  \item Case \rulename{EFun}. We have $P |- f[\ol{\tau}]\;\oln{e}{m} \Downarrow v$, given that 
         $(f |-> \Lambda\ol{a} @.@ \lambda\oln{x{:}\tau}{m} @.@ u) \in P$ and 
         $P |- u[\ol{\tau}/\ol{a}][\ol{e}/\ol{x}] \Downarrow v$. There are two cases to consider:
    \begin{itemize}
       \item Case $u = e_{rhs}$. In this case we have that $P |- u[\ol{\tau}/\ol{a}][\ol{e}/\ol{x}] \Downarrow v$.
       We have that $min(app(f_{ptr},\ol{t_e}))$ -- hence 
         \[ app(f_{ptr},\ol{t_e}) \red \etrans{}{}{u[\ol{\tau}/\ol{a}][\ol{e}/\ol{x}]} \] 
       Axiom \rulename{AxEqMin}, induction hypothesis, and \rulename{AxEqTrans} finish the case.

       \item Case $u = @case@\;e_{scr}\;@of@\;\ol{K\;\ol{y} -> e'}$. In this case we have the following two subcases:
       \begin{itemize}
         \item Case \rulename{EBadCase}. $P |- e_{scr}[\ol{\tau}/\ol{a}][\ol{e}/\ol{x}] \Downarrow @BAD@$.
               We have that $min(app(f_{ptr},\ol{t_e}))$, and therefore 
                    \[ min(\etrans{}{}{e_{scr}[\ol{\tau}/\ol{a}][\ol{e}/\ol{x}]}) \]
               By induction hypothesis this means 
                 \[ \etrans{}{}{e_{scr}[\ol{\tau}/\ol{a}][\ol{e}/\ol{x}]} \red \bad \] 
               and hence $app(f_{ptr},\ol{t_e}) \red \bad$.
         \item Case \rulename{ECase}. We have in this case that 
                \[ P |- e_{scr}[\ol{\tau}/\ol{a}][\ol{e}/\ol{x}] \Downarrow K_j(\ol{e_y}) \] and 
                 \[ P |- e'_j[\ol{\tau}/\ol{a}][\ol{e}/\ol{x}][\ol{e_y}/\ol{y}] \Downarrow w \]
               We have that $min(app(f_{ptr},\ol{t_e}))$, 
               and therefore:
               \[ min(\etrans{}{}{e_{scr}[\ol{\tau}/\ol{a}][\ol{e}/\ol{x}]}) \] 
               Hence, by induction hypothesis 
                     \[ \etrans{}{}{e_{scr}[\ol{\tau}/\ol{a}][\ol{e}/\ol{x}]} \red K_j(\etrans{}{}{\ol{e_y}}) \]
               Therefore 
                     \[ app(f_{ptr},\ol{t_e}) \red \etrans{}{}{e'_j[\ol{\tau}/\ol{a}][\ol{e}/\ol{x}][\ol{e_y}/\ol{y}]} \]
               Axiom \rulename{AxEqMin}, induction hypothesis, and \rulename{AxEqTrans} finish the case.
       \end{itemize}

    \end{itemize}

  \item Case \rulename{EApp}. In this case we have that $P |- e_1\;e_2 \Downarrow w$, given that 
    $P |- e_1 \Downarrow v_1$ and $P |- v_1\;e_2 \Downarrow w$. Assume $t_1$ and $t_2$, $t_{v_1}$ be the FOL translations.
    We have that $min(app(t_1,t_2))$, hence by \rulename{AxAppMin} we get $min(t_1)$. By induction hypothesis we get 
    that $t_1 \red t_{v_1}$. By \rulename{AxEqCong} we then get that $app(t_1,t_2) \red app(t_{v_1},t_2)$. By \rulename{AxEqMin}
    we get that $min(app(t_{v_1},t_2))$. By induction hypothesis then $app(t_{v_1},t_2) \red t_w$ and by \rulename{AxEqTrans}
    we get $app(t_1,t_2) \red t_w$ as required.

  \item Case \rulename{EBadApp}. We have that $P |- @BAD@\;e \Downarrow @BAD@$. We have 
        then that $min(app(bad,t_e))$ and hence by \rulename{AxAppBad} the case is finished.
\end{itemize} 
\end{proof}

\begin{lemma}[Guarded-Eq]\label{lem:guarded-eq}
   $\Th , \ptrans{}{P} \vdash \etrans{}{}{e} = True$ then
   $\Th^M , \ptrans{}{P}^M \vdash min(\etrans{}{}{e}) => \etrans{}{}{e} = True$.
\end{lemma}
\begin{proof}
By soundness of the $min$-less translation (proven in the POPL paper), we 
get that $\dbrace{e} = \injKZ{True}$, that is $P |- e \Downarrow \True$. 
By Lemma~\ref{lem:guarded-eval} the proof is finished.
\end{proof}


Surprisingly, Lemmas~\ref{lem:guarded-eq} and \ref{lem:guarded-eval} extend for the case of $\unr$.
We start with the following observation -- the $min$-less translation cannot {\em prove} divergence.
The intuition here is that is if an expression diverges then there surely exists a model that can 
equate it to $\unr$ (the canonical denotational model); however there exists {\em another} model
in which we can equate the expression to $\bad$!


\begin{lemma}[Divergence unprovability]\label{lem:divergence-unprovability}
If $t = \etrans{}{}{e}$ and $e$ diverges then there exists a model
${\cal M}$ such that ${\cal M} \models \Th \land \ptrans{}{P} \land (t \neq \unr)$.
\end{lemma}
\begin{proof} Consider the model $M_\downarrow$ where the universe consists of terms 
arranged in equivalence classes according to the following equivalence. Consider the extended
evaluation relation $P |- e \downarrow v$ with an explicit ``stuck'' value @UNR@. If $e$ diverges 
then $e \equiv @BAD@$, if $P |- e \downarrow v$ then $e = v$ (NB, $v$ contains the possibility of
@UNR@ and @BAD@). Close these axioms by reflexivity, symmetry, transitivity and congruence over 
applications and constructors and call this relation $\equiv$.

Interpret $app(\cdot,\cdot)$ as application.
%%  and every selector $\sel{K}{j}(\cdot)$ as a function
%% such that if $e \equiv K(\ol{e})$ according to the aforementioned equivalence then define 
%% $\sel{K}{j}(e) = e_j$; if $e \equiv @BAD@$ then $\sel{K}{j}(e) = @BAD@$; otherwise $\sel{K}{j}(e) = @UNR@$.

The following are true
\[ M_{\downarrow} \models \Th \land \ptrans{}{P} \]
Moreover in this model, since $e$ diverges, $e \equiv @BAD@ \neq @UNR@$.
\end{proof}

Note that using any {\em other} value than $@BAD@$ in the construction of Lemma~\ref{lem:divergence-unprovability}
is much harder to be made to work (if at all!), since if the scrutinee of a case expression diverges, 
allowing it to return a constructor from the patterns would result in the expression being equated to some 
ordinary (but unknown) value. By contrast letting it return $\bad$ allows the $\bad$ value to propagate 
through the case expression as it should -- since the case expression itself is divergent!

Lemma~\ref{lem:divergence-unprovability} ascertains that if we ever prove that $\etrans{}{}{e} = \unr$ then
$e$ does not diverge, but rather $e \not\Downarrow v$. 


\begin{lemma}[Guarded-Unr]\label{lem:guarded-unr}
If $t = \etrans{}{}{e}$ and 
   $\Th , \ptrans{}{P} \vdash t = \unr$ then 
   $\Th^M , \ptrans{}{P}^M \vdash min(t) => t = \unr$.
\end{lemma}
\begin{proof}
By Lemma~\ref{lem:divergence-unprovability} $e$ does not diverge, but rather evaluation gets ``stuck''. This 
means that $P |- e \downarrow @UNR@$ in the extended operational semantics. The result follows by repeating a 
similar proof as in Lemma~\ref{lem:guarded-eq} for the $\downarrow$ semantics.
\end{proof}

\subsection{Completeness attempts}

The lemmas shown in the previous section are very strong properties about proofs 
in the $min$-enabled theory simulating reduction of terms. We now continue with 
completeness results.

Let us introduce some special syntax, in order to facilitate the presentation;
in particular base contracts ($\Bt$) and first-order contracts ($\Ct$):
\[\begin{array}{lcl}
   \Ft   & ::= & (x_1:\Bt_1) -> \ldots -> (x_n:\Bt_n) -> \Bt \\ 
   \Bt   & ::= & \{ y \mid e \}
\end{array}\]

\begin{theorem}[Base completeness]\label{lem:base-completeness}
If $\Th, \ptrans{}{P} \vdash \neg \ctrans{}{}{e{\notin}\Bt}$ then
$\Th^M, \ptrans{}{P}^M \vdash \neg \ctrans{}{}{e{\notin}\Bt}^M$.
\end{theorem}
\begin{proof}
Let $\Bt = \{ y \mid p \}$. We know that
$\Th, \ptrans{}{P} \vdash \neg \ctrans{}{}{e{\notin}\Bt}$ and hence
{\bf we could} (by soundness) deduce that 
$ \dbrace{e} = \bot$ or $\dbrace{p[e/y]} = \bot$ or $\dbrace{p[e/y]} = \injKZ{True}$.

But that is {\bf unhelpful}, since the first and second disjunct assert that the 
expression may potentially diverge, and we know that our theories cannot prove divergence.

However, recall the $M_{\downarrow}$ model from the proof of Lemma~\ref{lem:divergence-unprovability}.
In this model we get that $e \equiv @UNR@$, or $p[e/y] \equiv @UNR@$, or $p[e/y] \equiv True$. This
means that either $P |- e \downarrow @UNR@$, or $P |- p[e/y] \downarrow @UNR@$, or 
$p[e/y] \downarrow @UNR@$. In any of these cases we have finite evaluation traces in hand
and repeating the proof of Lemma~\ref{lem:guarded-eq} for the $\downarrow$ semantics finishes 
the case.
\end{proof}


\paragraph{First order completeness?} Can we generalize the previous theorem to the first-order case?
\begin{quote}
If $\Th, \ptrans{}{P} \vdash \neg \ctrans{}{}{e{\notin}\Ft}$ then
$\Th^M, \ptrans{}{P}^M \vdash \neg \ctrans{}{}{e{\notin}\Ft}^M$.
\end{quote}
I do not know. Here is an attempted proof when $\Ft = \{ x \mid p \} -> \{ y \mid q \}$ 
to demonstrate the difficulties. In this case we have that:
\[\begin{array}{l}
     \Th,\ptrans{}{P} |- \forall x @.@  \phi(x) \lor \psi(x)  \\ 
     \begin{array}{ll}
       \text{where} & \phi(x) = x \neq \unr \land p[x] \neq \unr \land p[x] \neq True \\
                    & \psi(x) = app(e,x) = \unr \lor q[app(e,x)] = \unr,True
     \end{array}
\end{array}\]
Now we want to prove the corresponding $(.)^M$ statement. It suffices to deduce this for every logical term
in our signature $t$. However {\bf we have made sure that every logical term $t$ is the image of some actual Haskell term $e_t$} so it suffices to show
the $(.)^M$ version for some arbitrary $e_t$ argument. If we instantiate the previous statement with $e_t$ we get:
\[ \Th,\ptrans{}{P} |- \phi(e_t) \lor \psi(e_t) \]
Now, we have a small problem:
\begin{itemize}
   \item If we use the $D_{\infty}$ model then we can show that the left disjunct shows some finite evaluation traces so we can show the corresponding
         disjunct in the $(\cdot)^M$ world by our previous theorems. But the right disjunct has some infinite evaluation traces so we can't use the same proof method!
   \item If we use the $M_{\downarrow}$ model then we can show that the {\bf right} disjunct shows some finite evaluation traces, but the left allows the possibility
         for the expressions and predicates to be equated to @BAD@, which $M_{\downarrow}$ equates to non-termination so we don't have the corresponding finite evaluation
         traces there!
\end{itemize}
So the proof seems stuck even in the first-order case ...

I want to remark that I had previous attempts to prove the theorem by
(accidentally) using the claim that $\Th,\ptrans{}{P} |- e = unr \lor
p[e] = \unr \lor p[e] = True$ implies that either $\Th,\ptrans{}{P} |-
e = unr$ or $ \Th,\ptrans{}{P} |- p[e] = \unr$ or $\Th,\ptrans{}{P} |-
p[e] = True$, or, even worse, that $\Th,\ptrans{}{P} |- \phi(t) \lor
\psi(t)$ implies $\Th,\ptrans{}{P} |- \phi(t)$ or $\Th,\ptrans{}{P} |-
\psi(t)$! Once we get these, we may use the lemmas regarding
evaluation and proof in the $min$-system from the previous section to
finish the case. Sadly this is a non-trivial step (probably even
false).

%%    \item If $\Th, \ptrans{}{P} \vdash \neg \ctrans{}{}{e{\notin}\Ct}$ 
%%    then $\Th^M, \ptrans{}{P}^M \vdash \neg \ctrans{}{}{e{\notin}\Ct}^M$.
%%    \item If $\Th, \ptrans{}{P} \vdash \neg \ctrans{}{}{e{\in}\Ct}$ 
%%    then $\Th^M, \ptrans{}{P}^M \vdash \neg \ctrans{}{}{e{\in}\Ct}^M$.
%% \end{enumerate}
%% \end{theorem}
%% \begin{proof}
%% We prove the two parts by induction on the structure of the contract $\Ct$. Let us 
%% proceed for the first goal. We consider two cases and let $t = \etrans{}{}{e}$.
%% \begin{itemize}
%%   \item Case $\Ct = \{ y \mid p \}$. In this case we let $t_p = \etrans{}{}{p}$ and 
%%         we have that
%%         \begin{eqnarray*}
%%            \Th , \ptrans{}{P} \vdash \phi 
%%         \end{eqnarray*}
%%         where $\phi = t = \unr \lor t_p[t/y] = \unr
%%                                                 \lor t_p[t/y] = \True$.
%%         Let $\phi^M$ be $ t \red \unr \lor t_p[t/y] \red \unr
%%                                                 \lor t_p[t/y] \red \True$.
%%         We need to show that:
%%         \begin{eqnarray*}
%%           \Th^M, \ptrans{}{P}^M \vdash min(t) \land min(t_p[t/y]) => \phi^M
%%         \end{eqnarray*}
%%         This follows by Lemmas~\ref{lem:guarded-eq} and~\ref{lem:guarded-unr}.
%%   \item Case $\Ct = (x:\Ct_1) -> \Ct_2$. In this case we have that:
%%         \begin{eqnarray*}
%%           \Th, \ptrans{}{P} \vdash \neg (\exists x @.@ \phi[x])
%%         \end{eqnarray*}
%%         with $\phi[x] = \ctrans{\Sigma}{\Gamma,x}{x \in \Ct_1} 
%%                   \land \ctrans{\Sigma}{\Gamma,x}{e\;x \notin \Ct_2}$.
%%         We must show:
%%         \begin{eqnarray*}
%%           \Th^M, \ptrans{}{P}^M \vdash \neg (\exists x @.@ \phi^M[x])
%%         \end{eqnarray*}
%%         It suffices to show that for any arbitrary $t$ we have that
%%         \[ \Th^M, \ptrans{}{P}^M \vdash \neg\phi^M[t] \]
%%         However\footnote{This is very delicate, this back compiling {\bf is} 
%%         the reason I removed the selectors.} every $t$ is the image of a source term
%%         $\etrans{}{}{e_{arg}} = t$. Hence it suffices to show that:
%%         \[ \Th^M, \ptrans{}{P}^M \vdash \neg\ctrans{}{}{e_{arg} \in \Ct_1} \lor
%%                       \neg\ctrans{}{}{e\;e_{arg} \notin \Ct_2} \]
%%         This follows by induction hypothesis for both items.
%% \end{itemize}

%% For the second goal we proceed similarly: the arrow case follows by appealing
%% to the induction hypothesis (both cases again) so let us do the base case:
%% Let $\Ct = \{ y \mid p \}$ and $t_p = \etrans{}{}{p}$. We know that 
%% \[  \Th,\ptrans{}{P} \vdash t \neq \unr \land t_p[t/x] \neq \unr,\True \]
%% In this case, by adequacy of the $min$-less translation and the denotational
%% semantics it must be the case that:
%% $P |- e \Downarrow v_1$ and $P |- p[e/x] \Downarrow v_2$ such that $v_2 \neq True$.
%% Moreover, these $v_1$ and $v_2$ can only be of the forms @BAD@ or $K(\ol{e})$ or $f^m\;\ol{e}^{< m}$.

%% By Lemma~\ref{lem:guarded-eval} we know that:

%% \[ \Th^M,\ptrans{}{P}^M |- min(t) => t \red \etrans{}{}{v_1} \] 
%% and 
%% \[ \Th^M,\ptrans{}{P}^M |- min(t_p[t/x]) => t \red \etrans{}{}{v_2} \]

%% We need to show that $t \not\red \unr$. Assume by contradiction 
%% $t \red \unr$. The contradiction then follows by axiom \rulename{AxDisjUnr}.
%% Similarly we need to show that $t_p[t/x] \not\red \unr$. The argument is the
%% same as above. We additionally need to show that $t_p[t/x] \not\red \True$. If
%% $t_p[t/x] \red \True$ then \rulename{AxDisjCon} gives a 
%% contradiction.\footnote{A remark: actually if we assume that all the predicates start-off 
%% life as well-typed boolean functions, the above argument can be simplified because we 
%% know that $v_2$ can only be @False@, hence there is no need to use the somewhat elaborate
%% \rulename{AxDisjCon} axiom that also involves partial applications.}
%% \end{proof}

\paragraph{The problems with completeness for higher-order}

Completeness for higher-order only gets more tricky. I thought that the skeleton of the proof
would be an inductive argument by simultaneous induction that:
\begin{itemize}
  \item If $\Th,\ptrans{}{P} |- \neg \ctrans{}{}{e \notin \Ct}$ then 
           $\Th^M,\ptrans{}{P}^M |- \neg \ctrans{}{}{e \notin \Ct}^M$.
  \item If $\Th,\ptrans{}{P} |- \neg \ctrans{}{}{e \in \Ct}$ then 
           $\Th^M,\ptrans{}{P}^M |- \neg \ctrans{}{}{e \in \Ct}^M$.
\end{itemize}
Both base cases are provable by a similar argument as the the base completeness proved earlier.
However, for the arrow cases we have the same disjunction problem that we faced before. Let's
take the example when $\Ct = \Ct_1 -> \Ct_2$ and assume that:
\[ \Th,\ptrans{}{P} |- \neg (\exists x @.@ \ctrans{}{}{x \in \Ct_1} \land \ctrans{}{}{e x \notin \Ct_2}) \] 
that is:
\[ \Th,\ptrans{}{P} |- \forall x @.@ \neg\ctrans{}{}{x \in \Ct_1} \lor \neg\ctrans{}{}{e x \notin \Ct_2}) \]
By induction we do know that if 
$\Th,\ptrans{}{P} |- \neg\ctrans{}{}{e_t \in \Ct_1}$ then 
$\Th^M,\ptrans{}{P}^M |- \neg\ctrans{}{}{e_t \in \Ct_1}^M$, and moreover
that if $\Th,\ptrans{}{P} |- \neg\ctrans{}{}{e x \notin \Ct_2}$ then 
$\Th^M,\ptrans{}{P}^M |- \neg\ctrans{}{}{e x \notin \Ct_2}^M$. But we {\bf cannot use these} to deduce that:
\[ \Th^M,\ptrans{}{P}^M |- \forall x @.@ \neg\ctrans{}{}{x \in \Ct_1}^M \lor \neg\ctrans{}{}{e x \notin \Ct_2}^M) \]
If it were the case that $\Th,\ptrans{}{P} |- \neg\ctrans{}{}{e_t \in \Ct_1}$ or
 $\Th,\ptrans{}{P} |- \neg\ctrans{}{}{e x \notin \Ct_2})$ then we would be OK, but I do not believe that
this holds. This is really the same issue outlined before in the specific case for first-order contracts.

Now, some minor technical issues. Even if this case were to go through we'd still need to make sure that
every logical term is the image of some source Haskell term. But for the second statement, case for arrow
contracts again, we'd need to show some {\bf existential formula}. That is, if 
$\Th,\ptrans{}{P} |- \exists x @.@ \phi$ show that $\Th^M,\ptrans{}{P}^M |- \exists x @.@ \phi^M$ for some
formulae $\phi$ and $\phi^M$. The way that I know that we could possibly proceed here is to appeal to 
Herbrand's theorem exists a specific logical term $t$ such that $\Th,\ptrans{}{T} |- \phi[t]$. However, 
we can only appeal to Herbrand's theorem if our theory $\Th$ contains {\em only universal} statements. Sadly 
observe that our theories in their simplest form contain also existential statements arising 
from the last axiom that accompanies pattern matching.

The solution to these issues is to introduce the selector functions and make sure that they are in the 
image of a translation of a source program:
\begin{code}
  sel_K_i (K y1..yi...yn) = yi
  sel_K_i _ = BAD
\end{code}
We then modify the translation for pattern matching to:
\[ \begin{array}{c}
\ruleform{\utrans{\Sigma}{u}{s} = \formula{\phi}} \\ \\
\begin{array}{rcl}
\utrans{\Sigma}{e}{s}
  & = & \formula{(s = \etrans{\Sigma}{\Gamma}{e})} \\
\multicolumn{3}{l}{\utrans{\Sigma}
    {@case@\;e\;@of@\;\ol{K\;\ol{y} -> e'}}{s}} \\
\multicolumn{3}{l}{
\;\;
  \begin{array}[t]{rl}
    = & \formula{(\forall \ol{y} @.@ t = K_1(\ol{y}) => s = \etrans{\Sigma}{\Gamma}{e'_1})\;\land \ldots}  \\
    \land & t = \bad => s = \bad \\
    \land & \formula{{\bf t {\neq}K_1(\ol{\sel{K_1}{i}{t}}})\land\ldots t{\neq}\bad => s=\unr)} \\
    \mbox{where} & t  =  \etrans{\Sigma}{\Gamma}{e}
 \end{array}
}
\end{array}
\end{array}\]
In other words we have {\em replaced the statement that would give rise to an existential} with 
our own skolem function -- which happens to be expressible in source as well. In this way we can 
satisfy both requirements above. Interestingly, if one follows this approach, the axioms generated 
for the selector functions are of the (somewhat recursive) form:
\[\begin{array}{l}
   \forall x @.@ (\forall \ys @.@ x = K(\ys) => \sel{K}{i}{x} = y_i)\; \land \\
   \qquad\qquad  (\forall \ys @.@ x = J(\ys) => \sel{K}{i}{x} = \bad)\; \land \\
   \qquad\qquad \ldots \\ 
   \qquad\qquad (x = \bad => \sel{K}{i}{x} = \bad)\;\land  \\ 
   \qquad\qquad (x \neq K(\ol{\sel{K_1}{i}{x}}) \land \ldots \land x \neq \bad => \\ 
   \qquad\qquad\qquad \sel{K}{i}{x} = \unr)
\end{array}\]

Of course proving the theorem still fails on exactly the same place as
in the first-order case (the disjunction problem), so all of the above is only 
relevant if the disjunction problem can be surmounted.

\subsection{A different semantics that guarantees completeness}\label{ssect:bad-ok}

Consider a slightly different definition of the semantics of
contracts, as well as their translation to first-order logic.

\[\begin{array}{lcl}
  \dbrace{d \in \{ y \mid p\}}_\rho     & = & \dbrace{p}_{\rho,y |-> d} \neq \injKZ{False} \\ \\

  \dbrace{d \notin \{ y \mid p \}}_\rho & = & \dbrace{p}_{\rho,y |-> d} \neq \injKZ{True},\bot \\ \\

  \dbrace{d \in (x:\Ct_1) -> \Ct_2}_\rho & = & \text{for all} d', either \dbrace{d \notin \Ct_1} \\ 
                                        &   & \quad \text{or }\dbrace{\dapp(d,d') \in \Ct_2}_{\rho,x|->d'} \\

  \dbrace{d \notin (x:\Ct_1) -> \Ct_2}_\rho & = & \text{there exists} d', such that \dbrace{d \in \Ct_1} \\ 
                                        &   & \quad \text{and }\dbrace{\dapp(d,d') \notin \Ct_2}_{\rho,x|->d'} \\
\end{array}\]

The important difference with the ordinary version of contact satisfaction is 
that postconditions where the predicate returns @BAD@ are considered OK. 
Moreover if a precondition predicate crashes on an argument then this is again 
OK. So, predicates that return @BAD@ have this dual nature. Since we will effectively 
be proving negations of the $\notin$ relation, the following lemma is important.

\begin{lemma}\label{lem:in-notin}
If it is not the case that $\dbrace{d \notin \Ct}$ then $\dbrace{d \in \Ct}$.
\end{lemma}
\begin{proof}
By induction on the contract $\Ct$. We have the following cases:
\begin{itemize}
  \item Case $\Ct = \{ y \mid p \}$. Then we know that $\dbrace{p}_{\rho,y |-> d} = \injKZ{True}$
        or $\dbrace{p}_{\rho,y |-> d} = \bot$. This means that $\dbrace{p}_{\rho,y |-> d} \neq \injKZ{False}$.
  \item Case $\Ct = \Ct_1 -> \Ct_2$ (doing the non-depependent case for simplicity). We know that
        for every $d'$ either:
          \begin{itemize}
              \item (a) it is not the case that $\dbrace{d' \in \Ct_1}$, {\bf OR}
              \item (b) it is not the case that $\dbrace{\dapp(d,d') \in \Ct_2}$.
          \end{itemize}
        Pick a $d'$. We want to show that either
          \begin{itemize}
              \item (c) it is the case that $\dbrace{d' \notin \Ct_1}$, {\bf OR}
              \item (d) it is the case that $\dbrace{\dapp(d,d') \in \Ct_2}$.
          \end{itemize}
        If (a) is true then by the contrapositive of the induction hypothesis (c) holds.
        If (b) is true then by the induction hypotehsis (d) holds.
\end{itemize}
\end{proof}

Consider now the corresponding translation to first-order logic -- with and without $min$.

\[\begin{array}{c}
\ruleform{\ctrans{\Sigma}{\Gamma}{e \in \Ct} = \formula{\phi}} \\ \\
\begin{array}{rcl}
\ctrans{\Sigma}{\Gamma}{e \in \{x\;\mid\;e' \}}
  & = & \formula{t'[t/x]{\neq}\False} \\
  & \mbox{where} &
    \begin{array}[t]{l}
      t = \etrans{\Sigma}{\Gamma}{e} \; \text{and} \; t' = \etrans{\Sigma}{\Gamma}{e'}
    \end{array}
\\
\ctrans{\Sigma}{\Gamma}{e \notin \{x\;\mid\;e' \}}
  & = & \formula{t'[t/x]{\neq}\True,\unr} \\
  & \mbox{where} &
    \begin{array}[t]{l}
      t = \etrans{\Sigma}{\Gamma}{e} \; \text{and} \; t' = \etrans{\Sigma}{\Gamma}{e'}
    \end{array}
\\
\ctrans{\Sigma}{\Gamma}{e \in (x{:}\Ct_1) -> \Ct_2}
  & = & \formula{\forall x @.@ \ctrans{\Sigma}{\Gamma,x}{x \notin \Ct_1}
                          \lor \ctrans{\Sigma}{\Gamma,x}{e\;x \in \Ft_2}}
\\
\ctrans{\Sigma}{\Gamma}{e \notin (x{:}\Ct_1) -> \Ct_2}
  & = & \formula{\exists x @.@ \ctrans{\Sigma}{\Gamma,x}{x \in \Ct_1}
                          \land \ctrans{\Sigma}{\Gamma,x}{e\;x \in \Ct_2}}
\end{array}
\end{array}\]

In the $min$-world we have:

\[\begin{array}{c}
\ruleform{\ctrans{\Sigma}{\Gamma}{e \in \Ct}^M = \formula{\phi}} \\ \\
\begin{array}{rcl}
\ctrans{\Sigma}{\Gamma}{e \in \{x\;\mid\;e' \}}^M
  & = & min(t'[t/x]) \land \formula{t'[t/x]{\neq}\False} \\
  & \mbox{where} &
    \begin{array}[t]{l}
      t = \etrans{\Sigma}{\Gamma}{e} \; \text{and} \; t' = \etrans{\Sigma}{\Gamma}{e'}
    \end{array}
\\
\ctrans{\Sigma}{\Gamma}{e \notin \{x\;\mid\;e' \}}^M
  & = & min(t'[t/x]) \land \formula{t'[t/x]{\neq}\True,\unr} \\
  & \mbox{where} &
    \begin{array}[t]{l}
      t = \etrans{\Sigma}{\Gamma}{e} \; \text{and} \; t' = \etrans{\Sigma}{\Gamma}{e'}
    \end{array}
\\
\ctrans{\Sigma}{\Gamma}{e \in (x{:}\Ct_1) -> \Ct_2}^M
  & = & \formula{\forall x @.@ \ctrans{\Sigma}{\Gamma,x}{x \notin \Ct_1}^M
                          \lor \ctrans{\Sigma}{\Gamma,x}{e\;x \in \Ft_2}}^M
\\
\ctrans{\Sigma}{\Gamma}{e \notin (x{:}\Ct_1) -> \Ct_2}^M
  & = & \formula{\exists x @.@ \ctrans{\Sigma}{\Gamma,x}{x \in \Ct_1}^M
                          \land \ctrans{\Sigma}{\Gamma,x}{e\;x \in \Ct_2}}^M
\end{array}
\end{array}\]

%% (The negation of
%% $\notin$ translation is {\em not} equivalent to the $\in$ translation). 

Soundness is easy to prove.

\begin{theorem}[Soundness]
If  $\Th,\ptrans{}{P} |- \neg (e \notin \Ct)$ 
and $\dbrace{e} = d$ then $\dbrace{d \in \Ct}$.
Moreover, if $\Th^M,\ptrans{}{P}^M |- \neg \ctrans{}{}{e \notin \Ct}^M$
then $\Th^M,\ptrans{}{P}^M |- \neg \ctrans{}{}{e \notin \Ct}^M$.
\label{lem:in-notin}
\end{theorem}
\begin{proof}
For the first part, we know by basic soundness that it is not the case that 
$\dbrace{e \notin \Ct}$, and by Lemma~\ref{lem:in-notin}, we get that 
$\dbrace{e \in \Ct}$. The second part follows by the same model-theoretic
argument as we established soundness earlier (using an always-true $\min$
interpretation).
\end{proof}

Investigating the practical ramifications of this choice is to be 
explored -- however we observe  that under this translation 
we {\em are also guaranteed to get completeness:}

\begin{theorem}[Completeness]
If we use the selector-based theories (described above) then the following are true:
\begin{itemize}
  \item If $\Th,\ptrans{}{P} |- \neg \ctrans{}{}{e \notin \Ct}$ then 
           $\Th^M,\ptrans{}{P}^M |- \neg \ctrans{}{}{e \notin \Ct}^M$.
  \item If $\Th,\ptrans{}{P} |- \neg \ctrans{}{}{e \in \Ct}$ then 
           $\Th^M,\ptrans{}{P}^M |- \neg \ctrans{}{}{e \in \Ct}^M$.
\end{itemize}
\end{theorem}
\begin{proof}
\dv{Sketch still -- but pretty sure it works now!} By induction on 
the contract $\Ct$ one can observe that we only get formulas of a
mixed-prefix quantification $\forall\exists\ldots\forall\exists @.@
\phi$ where $\phi$ consists of conjunctions and disjunctions of
equations of the form $t = False$, $t = True$, $t = \unr$. To
eliminate the mixed prefix, we proceed as in previous proofs to
eliminate the universal quantifiers and we appeal to Herbrand's
theorem for eliminating the existentials -- since our theory is now a
purely universal one. Finally we end up knowing that we have proved a
statement of the form:
\[     \Th,\ptrans{}{P} |- \etrans{}{}{e} = False/True/\unr \lor \ldots \land \ldots \]
This means that in the $M_\downarrow$ model we have conjunctions or
disjunctions of finite evaluation traces, which in turn allows us to
use the guarded evaluation lemmas in the $min$-theories and finish the proof.
\end{proof}


This all is very good news because it asserts that we have 
not lost any proving power by moving to the $min$ world.

\dv{But we must explore the practicality of this approach. I think 
it puts more stress in having to have a special $cf(\cdot)$ 
predicate, because any argument for which the precondition 
crashes and the postcondition crashes will be a counterexample. 
So it is quite essential to have a CF contract too in practice.}

\section{Finite models}\label{sect:fin-mods}
\newcommand{\Univ}{{\cal U}}
\newcommand{\redop}{\longrightarrow^{\star}}

The unmodified original theory with an injectivity axiom {\em definitely} does
not enjoy finite models:
\[   \forall \xs\ys @.@ K(\xs) = K(\ys) => \xs = \ys \]
The reason is that discrimination axioms (\rulename{AxDisjC}) in
combination with injectivity axioms force any model of the theory to be infinite.

As we have seen, injectivity does not increase the ability to prove
equations between expressions and values.(\dv{Can I make this more
rigorous? Can I get a result for arbitrary contracts?}) Since it
hurts the finite model property we can safely drop it.

However, due to the presence of pattern matching, injectivity has not
been completely eliminated. One can write projection functions -- e.g.
\begin{code}
sel (S x) = x 
sel _     = @BAD@
\end{code}
Notice that if we have $S(x) = S(y)$ then 
$app(sel_{ptr},S(x)) = app(sel_{ptr},S(y))$ and by reduction $x = y$.

We've seen that the $min$-enabled theory is sound (\dv{Maybe complete too ...}) 
but does it enjoy finite models? Let us condider first simple base contracts. 
Assume that we know that
\begin{equation}
  \dbrace{e} \notin \dbrace{\{y \mid p\}} \label{fm:eq1} 
\end{equation}
Then this statement gives us evidence for two {\em finite execution traces}:
\begin{equation}
   e \Downarrow v  \quad\quad p[e/y] \Downarrow v \neq True \label{fm:eq2} 
\end{equation}
Given these two traces we will show next that the FOL formula 
\[ \Th^M \land \ptrans{}{P} \land \ctrans{}{}{e \notin \{ y \mid p \}} \] 
does have a {\em finite model}. By doing this we get a formal guarantee that 
if \ref{fm:eq1} holds then {\em there is} a finite model, so a finite model finder 
that systematically explores finite models will find one.

Here is the construction. 

\begin{definition}[Construction finite models from traces]
We start with a finite set of traces $S$ which are all finite and of the form $P |- e \Downarrow v$.
By trace I mean the conclusion of every subderivation in the derivation tree 
for $\Downarrow$, not just the final conclusion. This is equivalent to a collection 
of traces in the $\downarrow$ relation, such that no resulting value is equal to @UNR@.
\begin{itemize}
  \item Let the universe $\Univ$ be the set of terms and all their subterms that appear in $S$, @UNR@, and @BAD@
  \item Consider the following helper function on arbitrary terms:
        \[ \mu(e) = \text{if } e \in \Univ \text{ then } e \text{ else } @UNR@ \] 
        Define the interpretation of the logical terms as follows:
        \[\begin{array}{lcl} 
            I(K)   & = & \lambda\es \in \Univ @.@ \mu(K(\es)) \\
            I(app) & = & \lambda e_1,e_2 \in \Univ @.@ \mu(e_1\;e_2) \\
            I(f_{ptr}) & = & \mu(f) \\
            I(unr)  & = & @UNR@ \\
            I(bad)  & = & @BAD@ 
        \end{array}\]
  \item Define the interpretation of $min$ to be as follows:
        \[\begin{array}{lcl} 
           I(min) & = & \lambda e \in \Univ @.@ \\ 
                  &   & e \text{ was evaluated in } $S$ \text{ or } \\ 
                  &   & \text{some } e\;e_1\ldots e_n \text{ was evaluated in } S
        \end{array}\]
  \item Let us use notation $\redop$ for the many-step small-step evaluation 
        relation corresponding to $\downarrow$.
        Define the interpretation of $\red$ to be:
        \[ I(\red) = \lambda e_1,e_2 \in \Univ @.@ (e_1{\in}I(min)) \text{ and } (e_1{\redop}e_2) \]
\end{itemize} 
Let us call this model $M_S$.
\end{definition}


\begin{theorem} If $S$ is a finite collection of finite traces then $M_S \models \Th^M$.
\end{theorem}
\begin{proof}
We examine all the axioms of the theory.
\begin{itemize}
  \item \rulename{AxEqRefl}. We must show: 
             \[ M_S \models \forall x @.@ min(x) => x \red x \] 
        Pick a term $e_x \in \Univ$, assume $e_x \in I(min)$. Then clearly $e_x \in I(min)$ and $e_x \redop e_x$.
  \item \rulename{AxEqTrans}. We must show:
            \[ M_S \models \forall x y z @.@ x \red y \land y \red z => x \red z \]
        Pick terms $e_x,e_y,e_z \in \Univ$. Assume $e_x \in I(min)$ and $e_x \redop e_y$ and $e_y \in I(min)$ 
        and $e_y \redop e_z$. Then clearly $e_x \in I(min)$ and $e_x \redop e_z$.
  \item \rulename{AxEqMin}. We must show:
           \[ M_S \models \forall xy @.@ min(x) \land x \red y => min(y) \]
        Pick terms $e_x,e_y \in \Univ$ assume $e_x \in I(min)$ and $e_x \redop e_y$. Then it must be that
        $e_y \in I(min)$.\footnote{Interestingly, in this interpretation I do not need the $min(x)$ assumption 
        in \rulename{AxEqMin}. But I do need it in reflexivity.}
  \item \rulename{AxAppMin}. We must show:
          \[ M_S \models \forall x @.@ min(app(x,y)) => min(x) \]
        Pick $e_x,e_y \in \Univ$ assume $\mu(e_x\;e_y) \in I(min)$. Now, it must be the case that
        $\mu(e_x\;e_y) = e_x\;e_y$ because otherwise we would have $@UNR@ \in I(min)$, which is not the case
        since all traces evaluate to non-@UNR@ values. Therefore $e_x\;e_y \in I(min)$ which implies that 
        $e_x \in I(min)$.
  \item \rulename{AxEqCong}. We must show:
        \[\begin{array}{l}
               M_S \models \forall x x' y @.@ min(app(x,y))\;\land\;x \red x' \\ 
               \qquad\qquad\quad => app(x,y) \red app(x',y) 
        \end{array}\]
        Pick $e_x, e_x', e_y \in \Univ$ and assume that $\mu(e_x\;e_y) \in I(min)$. As before this means
        $e_x\;e_y \in I(min)$. We also know that $e_x \in I(min)$ and $e_x \redop e_x'$, hence by a simple
        property of evaluation $e_x\;e_y \redop e_x'\;e_y$. 
  \item \rulename{AxDisjUnr}. We must show:
        \[\begin{array}{l}
             M_S \models \forall x\ys\zs @.@ min(x) \land x \red \unr => \ldots 
        \end{array}\]
        Pick $e_x, \ol{e_x},\ol{e_y},\ol{e_z} \in \Univ$. Assume $e_x \in I(min)$ and $e_x \redop @UNR@$. This
        is a contradiction so this axiom becomes trivially true in this model.
  \item \rulename{AxDisjCon}. We must show:
        \[\begin{array}{l}
            M_S \models \forall x\ys\xs\zs @.@ min(x) \land x \red K(\ys) => \\
            \qquad\qquad\begin{array}{ll}
                         & x \not\red J(\xs) \\ 
                   \land & x \not\red \bad \\
                   \land & x \not\red app(f_{ptr}^m,\ol{z}^{< m})
                        \end{array}
        \end{array}\]
        Pick $e_x,\ol{e_x},\ol{e_y},\ol{e_z} \in \Univ$. Assume $e_x \in I(min)$ and $e_x \redop \mu(K(\ol{e_y}))$,
        It is impossible to have $\mu(K(\ol{e_y})) = @UNR@$ and therefore it must be that 
        $e_x \redop K(\ol{e_y})$. Now we must show three things:
        \begin{itemize}
          \item $e_x \not\redop \mu(J(\ol{e_x}))$. There are two cases -- either 
                $\mu(J(\ol{e_x})) = @UNR@$ or $\mu(J(\ol{e_x})) = J(\ol{e_x})$. In the first case we know that
                $e_x \not\redop @UNR@$ and in the second case the result follows by determinacy of evaluation.
          \item $e_x \not\redop @BAD@$. This follows by determinacy of evaluation.
          \item $e_x \not\redop \mu(\ldots\mu(\mu(f e_{z1}) e_{z2}) \ldots,e_{zk}))$. Now if the outer
                $\mu$ gives @UNR@ then we are done, else it is the identity. If the next inner gives @UNR@ 
                then we are done again because we'd have to have $e_x \redop @UNR@$ etc.
        \end{itemize}
   \item \rulename{AxDisjApp}. Similar argument as above, observing that if 
             $e_x \in I(min)$ and $e_x \redop \mu(\ldots\mu(\mu(f e_{y1}) e_{y2}) \ldots e_{yk})$ then all
             of the $\mu$ calls must be identities, otherwise we'd have $e_x \redop @UNR@$ which can't happen.
   \item \rulename{AxAppBad}. We must show that:
             \[ M_S \models \forall x @.@ min(app(\bad,x)) => app(\bad,x) \red \bad  \]
         Pick $e_x \in \Univ$. Assume $\mu(@BAD@ e_x)) \in I(min)$. Hence it is not @UNR@. Indeed
              $@BAD@\;e_x \redop @BAD@$.
   \item \rulename{AxAppUnr}. We must show that:
             \[ M_S \models \forall x @.@ min(app(\unr,x)) => app(\unr,x) \red \unr \]
         Pick $e_x \in \Univ$ assume $\mu(@UNR@\;e_x) \in I(min)$. This is a contradiction -- if $\mu$ is
         the identity then $@UNR@\;e_x \in I(min)$ which means $@UNR@ \in I(min)$; and if it is @UNR@ then
         we get again a contradiction.
   \item \rulename{AxAppCon}. We must show that:
             \[ \forall \xs,y @.@ min(app(K(\xs),y)) => app(K(\xs),y) \red \unr  \]
         Pick $\ol{e_x}, e_y \in \Univ$. Assume that $\mu(\mu(K(\ol{e_x})\;e_y)) \in I(min)$. This means that
         both $\mu$ functions must be identities; but even so $K(\ol{e_x})\;e_y \redop @UNR@$ so even then 
         it must be that $@UNR@ \in I(min)$, a contradiction.
\end{itemize}
\end{proof}

\begin{theorem} If $S$ is a finite collection of finite traces then $M_S \models \ptrans{}{P}^M$.
\end{theorem}
\begin{proof}
The program $P$ is a collection of definitions so we show that $M_S \models d$ for every $d \in P$. We have two
cases:
\begin{itemize}
  \item The definition is of the form 
                 \[ f\;\xs = e \]
        We must show that:
        \[     M_S \models \forall \xs @.@ min(app(f_{ptr},\xs) => app(f_{ptr},\xs) \red \etrans{}{}{e} \]
        Pick $\ol{e_x}$ and assume that $\mu(\ldots\mu(\mu(f) e_{x1}) e_{x2})\ldots) \in I(min)$. This means that all
        $\mu$ calls must give the identity -- otherwise we'd get $@UNR@ \in I(min)$. Subsequently we have that
        $f\;\ol{e_x} \in I(min)$ and hence all terms and subterms of $e[\ol{e_x}/\xs]$ appear in the universe
        $\Univ$. This
        means that $I[\ol{x |-> e_x}](\etrans{}{}{e}) = e[\ol{e_x}/\xs]$ and clearly 
        $f\;\ol{e_x} \redop e[\ol{e_x}/\xs]$ as required.
        
  \item The definition is of the form 
                 \[ f\;\xs = @case@\;e_s\;@of@\;\ol{K\;\ys -> e_K} \]
        We need to show the following:
        \[\begin{array}{l}
            \forall \xs @.@ min(app(f_{ptr},\xs) => \\
            \quad \begin{array}{lll}
             (a) &        & min(\etrans{}{}{e_s}) \\
             (b) & \land & (\forall\ys @.@ \etrans{}{}{e_s}{\red}K(\ys) => app(f_{ptr},\xs) \red \etrans{}{}{e_K}) \\
                 & \land & \ldots \\
             (c) & \land & (\etrans{}{}{e_s}\red\bad => app(f_{ptr},\xs) \red \bad) \\
             (d) & \land & ( (\exists \xs @.@ \etrans{}{}{e_s} \red K(\xs)))\;\lor\;\ldots \\
                 &       & \quad \lor (\etrans{}{}{e_s} \red \bad\;\lor\;app(f_{ptr},\xs) \red \unr 
                  \end{array}
        \end{array}\]
        Pick $\ol{e_x} \in \Univ$. Assume that $\mu(\ldots\mu(\mu(f) e_{x1}) e_{x2})\ldots) \in I(min)$. 
        This means that all $\mu$ calls must give the identity -- otherwise we'd get $@UNR@ \in I(min)$. 
        Subsequently we have that $f\;\ol{e_x} \in I(min)$ and hence all terms and subterms 
        of $e_s[\ol{e_x}/\xs]$ appear in the universe and in fact $e_s[\ol{e_x}/\xs] \in I(min)$ --- moreover
        $I[\ol{x |-> e_x}](\etrans{}{}{e_s})$ must be equal to $e_s[\ol{e_x}/\xs]$. So (a) is proved. 
        For (b), let us use $J$ for $I[\ol{x |-> e_x}]$ below. Let us pick:
        $\ol{e_y} \in \Univ$ and assume that $e_s[\ol{e_x}/\xs] \redop \mu(K(\ol{e_y}))$. It must be that
        this call to $\mu$ gives the identity -- hence $e_s[\ol{e_x}/\xs] \redop K(\ol{e_y})$. In this case
        we know that $f\;\ol{e_x} \redop [\ol{e_x}/\xs][\ol{e_y}/\ys]e_{K} = J[\ol{y |-> e_y}](\etrans{}{}{e_{K}})$ 
        since
        all the subterms of $[\ol{e_x}/\xs][\ol{e_y}/\ys]e_{K}$ appear in the universe. So (b) is proven.
        The case for (c) is simple.
        For (d) we know that -- since the trace is finite -- it must be the case that there exists a value $v$
        such that $e_s[\ol{e_x}/\xs] \redop v$. Let us perform inversion on the shape of $v$:
        \begin{itemize}
          \item $v = K(\es)$ contained in the patterns. Then $d$ holds trivially (first disjunct, the 
                corresponding $\mu$ call must be identity).
          \item $v = J(\es)$ where $J$ is not contained in the patterns. This is impossible to happen because
                the trace would return @UNR@, which it doesn't.
          \item $v$ is some partial application. Same as previous case.
          \item $v = @BAD@$. The corresponding disjunct in (d) holds. 
          \item $v = @UNR@$. This can't happen since in that case @UNR@ would have to be in the trace.
        \end{itemize}
\end{itemize}
\end{proof}


\begin{theorem} If $S \supseteq \{ p[e/y] \Downarrow w \neq True, e \Downarrow v \}$ 
                         then $M_S \models \ctrans{}{}{e \notin \{ y \mid p \}}^M$.
\end{theorem}
\begin{proof} This is a very simple proof. Call $\etrans{}{}{e} = t$, $\etrans{}{}{p} = t_p$ and 
$\etrans{}{}{v} = t_v$, $\etrans{}{}{w} = t_w$. By completenss results in the previous sections we 
know that:
 \[ \Th^M,\ptrans{}{P} |- min(t) \land min(t_p[t/y]) => t \red t_v \land t_p[t/y] \red t_w \]
By inversion on the shape of $v$ we know that it can only be a constructor application, partial 
application, or @BAD@. As a consequence, rule \rulename{AxDisjUNR} gives us that $t \not\red \unr$. 
Hence we have:
\[ \Th^M,\ptrans{}{P},min(t),min(t_p[t/y]) |- t \not\red \unr \land t_p[t/y] \red t_w \]
By inversion on the shape of $w$ we know that it can be constructor, application, partial 
application, or @BAD@. As a result by \rulename{AxDisjUnr} we know that:
\[ \Th^M,\ptrans{}{P},min(t),min(t_p[t/y]) |- t \not\red \unr \land t_p[t/y] \not\red \unr \]
However we know that if it is a particular constructor application then it is not $True$, hence
rule \rulename{AxDisjCon} asserts $t_p[t/y] \not\red True$; 
if it is a partial application then by rule \rulename{AxDisjApp} we know that it $t_p[t/y] \not\red True$;
if it is @BAD@ then rule \rulename{AxDisjCon} asserts again that $t_p[t/y] \not\red True$. In all cases:
\[ \Th^M,\ptrans{}{P},min(t),min(t_p[t/y]) |- t \not\red \unr \land t_p[t/y] \not\red \unr,True \]

But $M_S \models \Th^M,\ptrans{}{P}, min(t) \land min(t_p[t/y])$ and therefore 
     \[ M_S \models min(t) \land min(t_p[t/y]) \land t \not\red \unr \land t_p[t/y] \not\red True \]
that is
     \[M_S \models \ctrans{}{}{e \notin \{ y \mid p \}} \]
as required.
\end{proof}

%% \begin{proof}
%% We consider the term model of all terms and subterms that appear in $S$. We consider equality
%% to be the equality induced by derivations $e \Downarrow v$ (that is we set $e = v$) and close 
%% by reflexivity, transitivity and congruence. We create a table for $min$ in the following way:
%% for each equivalence class created, we add it to $min$ iff there exists a term $e_{L}$ in 
%% that class that has appeared in the left-hand side of a derivation $e_{L} \Downarrow \_$ 
%% in the set of traces. Let us call this model ${\cal M}_S$. 
%% \end{definition}
%% \begin{lemma} If $S \supseteq \{ p[e/y] \Downarrow False, e \Downarrow v \}$ is a finite
%% collection of traces, then: 
%% \[   {\cal M}_S \models \Th^M \land \ptrans{}{P} \land \ctrans{}{}{e \notin \{ y \mid p \}} \] 
%% \end{lemma}
%% \begin{proof} \dv{TODO! But I think it should be doable ... } \end{proof}

\subsection{First-order arrow contracts}

The story is not finished yet, what about arrow contracts $(x:\Ct_1) -> \Ct_2$. The problem
here that if we know that $e$ does not belong in $\dbrace{(x:\Ct_1) -> \Ct_2}$ then this mean
that there exists some denotation $d$ such that the application of $\dbrace{e}$ to $d$ does
not belong to $\Ct_2$ {\em and moreover} $d$ belongs in $\dbrace{\Ct_1}$. 

Assuming that $\Ct_1$ and $\Ct_2$ are both base contracts and that $d$ is the image of some 
actual term $e_d$ then we do get some finite traces from the fact that the application does 
not belong in $\Ct_2$ but the fact is that our model has to also be a model of:
\[       \ctrans{}{}{e_d \in \Ct_1} \]

This is problematic, because we have no ``finite witness'' that $e_d$ is in $\dbrace{\Ct_1}$, 
contrary to the negative case. In fact $e_d$ could diverge, or, even worse, the predicate 
associated with $\Ct_1$ could diverge yielding an infinite trace. 

From an infinite trace it can often be the case that a finite model can be constructed, if
the trace involves a finite number of new values. But in principle an infinite trace can
be causing the evaluation of even more and new terms, which excludes finite models. Here is 
an example:

\begin{code}
data Nat = S Nat | Z

pinf Z (S x) = pinf (S Z) (S (S x))
pinf (S g) (S x) = pinf (S (S g)) (S (S x))
pinf g Z     = False

isZero Z = True
isZero _ = False

f :: Nat -> Nat
f x = x

pinfBroken = f ::: Pred (pinf Z) --> Pred isZero
\end{code}

Notice in this case that there is a finitely representable
counterexample @s = S s@ -- however the predicate 
@pinf Z s@ loops in a fashion that evaluates ever growing terms! Indeed
the $min$-enabled translation {\em also} results in the model finder 
looping for this broken contract.

We acknowledge the existence of such problematic counterexamples but
we can at least characterize (conservatively, not tightly) some
situations where we definitely can guarantee a finite model (we use $\Bt$ below
for base contracts). Incidentally same problems should arise as well with the
translation from section~\ref{ssect:bad-ok}

\subsection{Characterization of when finite models exist}

For first-order arrow contracts we can show that finite models exist whenever
a ``finitistic'' counterexample exists -- that is, one where the precondition
is actually true (as opposed to divergent)

\begin{theorem}[Finite models for finitistic first-order semantics]
Assume a contract $\Ft = (x{:}\Bt_1) -> \Bt_2$ and $\Bt_1 = \{ y \mid q \}$. Assume that there 
exists $e$ such that $\dbrace{e} \neq \bot$ and $\dbrace{q[e/y]} = \injKZ{True}$. Then there
exists a finite model $M_S$ such that 
\[   M_S \models \Th^M,\ptrans{}{P},\ctrans{}{}{f \notin \Ft}^M \]
\end{theorem}
\begin{proof}
It is easy to see that in this case we have a finite number of traces in hand.
\end{proof}

%% Assume that there exists $e$ such that $\dbrace{e} \in \flat\dbrace{\Bt_1}$ but 
%% $f\;e \notin \dbrace{\Bt_2}_{x |-> \dbrace{e}}$. Then there exists a finite model $M_S$ such that
%% \[     M_S \models \Th^M,\ptrans{}{P},\ctrans{}{}{f \notin \Ft}^M
%% \end{theorem}
%% \begin{proof} Easy to show similarly to before, since we have a finite collection of finite traces. \end{proof}


\section{Higher-order arrow contracts}

So the question here is, when do we have to use the $min$ guard on arrow contracts, that is:
\[\begin{array}{l}
    \ctrans{}{}{e \in \Ct_1 -> \Ct_2}^M =? \\
      \qquad\quad \forall x @.@ {\bf min(e\;x) =>} 
                   \ctrans{}{}{x \notin \Ct_1}^M \lor \ctrans{}{}{e\;x \in \Ct_2}^M 
\end{array}\]


I {\em think} that finite models break without this -- for instance consider a predicate 
\begin{code}
   sel (S x) y = sel x
   sel _     y = y 
\end{code}
And consider a contract assumption: $h \in (x:True) -> \{ y \mid sel x \}$. Automatically 
we have became interested in all possible successors by adding this to our assumptions.

Our proposed solution is to guard such arrow contracts with $min$ guards too so that only if the 
$h$ function is called, we become interested in evaluating its postcondition on an actual argument.
Arguably we lose some proving power (here is a contrived example).

\begin{code}
flop Z = True
flop (S x) = flop x
flop_ok = flop ::: Pred recN --> Pred (\x -> x)

{-# NOINLINE glop #-}
glop Z     = flop (S Z)
glop (S x) = flop (S (S x))

glop_ok = glop ::: Pred recN --> Pred (\x -> x)
               `Using` flop_ok

{-# NOINLINE foo #-}
foo (S x) = flop (S x)
foo Z     = True

foo_ok = foo ::: Pred recN --> Pred (\x -> x)
             `Using` glop_ok
\end{code}
This is a rather contrived kind of incompleteness induced by $min$. 
Basically we are using a contract for an expression that we don't 
directly call but if were to reduce it a few steps down the way, this 
expression could result to an expression for which we are interested 
and hence the full power of the contract should be unleashed. So this
contract holds in the $min$-free world but fails to hold in the $min$-world.
Interestingly if we were to use the evaluation relation in the $min$ free
world, it'd fail there too.

However we are likely to have finite models much more often -- but not always if the call 
to $h$ creates a possibly infinite evaluation of the postcondition of $h$. If that is not 
the case then we have finite models again. In practice this is almost always the case from 
our experience with the testsuites.

\dv{I think this is the only way we can present the $min() =>$ guards in arrow contracts. 
It is extremely hard to give any formal guarantee either for completeness or finite models. I've
consistently failed trying to do this for a very long time.}

\section{Crash-freedom}
\dv{This section is still in flux}

In terms of practicality we quite need crash freedom. I think that the ``right'' 
axioms for crash freedom are simply the following:

\[\begin{array}{ll}
 \textsc{AxCfC}  & \formula{\forall \oln{x}{n} @.@ \lcf{K(\ol{x})} <=> \bigwedge\lcf{\ol{x}}} \\
                 & \text{ for every } (K{:}\forall\as @.@ \oln{\tau}{n} -> T\;\as) \in \Sigma \\
 \textsc{AxCfBU} & \formula{\lcf{\unr} /\ \lncf{\bad}}
\end{array}\]

In the $min$-world:

\[\begin{array}{ll}
 \textsc{AxCfRed}  & \forall x y @.@ min(x) \land x \red y => (\lcf{x} <=> \lcf{y}) \\
 \textsc{AxCfC1}   & \forall \xs @.@ min(K(\xs)) \land \lcf{K(\xs)} => \bigwedge(\lcf{x_i} \land min(x_i))\\
 \textsc{AxCfC2}  & \bigwedge\lcf{\ol{x}} => \lcf{K(\xs)} \\ 
 \textsc{AxCfBU} & \formula{\lcf{\unr} /\ \lncf{\bad}} \\
%% \textsc{AxCfNRed}  & \forall x y @.@ min(x) \land x \red y => (\nlcf{x} <=> \nlcf{y}) \\
 \textsc{AxCfNC}   & \forall \xs @.@ min(K(\xs)) \land \lncf{K(\xs)} => 
                                              \exists x_i @.@ min(x_i) \land \lcf{\ol{x}} \\
\end{array}\]

Moreover, in the $min$ world we add conjunctions with $min(t)$ 
both in the positive and the negative translation:
\[\begin{array}{lcl}
    \ctrans{}{}{e \in \CF}    & = & min(\etrans{}{}{e}) \land \lcf{\etrans{}{}{e}} \\ 
    \ctrans{}{}{e \notin \CF} & = & min(\etrans{}{}{e}) \land \lncf{\etrans{}{}{e}}
\end{array}\]

Soundness of the $min$ translation definitely still holds -- completeness I believe yes but 
we need a separate lemma:

\begin{lemma} The following is true
\begin{itemize}
  \item If $\Th,\ptrans{}{P} |- \lcf{\etrans{}{}{e}}$ then 
           $\Th^M,\ptrans{}{P}^M |- min(\etrans{}{}{e}) => \lcf{\etrans{}{}{e}}$.
\end{itemize}
\end{lemma}
\begin{proof}\dv{Sketch} The proof is done again via the $M_\downarrow$ model -- since there 
crashing and non-termination are identified the assumption has a finite collection of traces 
as witnesses and we can exploit that to simulate a reduction-based proof in the $min$-world.
\end{proof}

Restrictions of the finite models theorem should also hold for ``finitistically'' 
crash-free terms.


%% \section{Arrow contracts and recovering finite models}

%% Consider the following restriction on the usage of contracts.

%% \begin{definition}[Contract-strict functions]
%% Assume that all base contracts $\ol{x{:}\{y \mid q\}} -> \{ y \mid r \}$ attached to function $f$ are such
%% that the following condition holds: In any application $f\;\ol{e}$,
%% if $\dbrace{e_i} = \bot$ or $\dbrace{q_i}_{y |-> \dbrace{e_i}} = \bot$ then $\dbrace{f\;\ol{e}} = \bot$. That
%% is the function $f$ cannot be less strict than its precondition. 
%% \end{definition}

%% %% This discussion should go somewhere -- but where? 
%% %% An easy way 
%% %% to achieve this is to add the precondition as a Haskell conjunction in the postcondition. E.g. if:
%% %% \begin{code}
%% %% f ::: (x:{y | p}) -> {y | q} 
%% %% \end{code} 
%% %% we could transform it to:
%% %% \begin{code}
%% %% f ::: (x:{y | p}) -> {y | x && p x && q }
%% %% \end{code}
%% %% \end{definition}
%% %% which will ensure that if the argument diverges or the 
%% %% precondition diverges then the postcondition diverges. Of course using 
%% %% the second contract means that an argument which causes the precondition 
%% %% to diverge but fails the postcondition will never be a counterexample.
%% %% \begin{code}
%% %% f ::: {x | \_.loop } -> {y | \_.False }
%% %% \end{code}
%% %% Anything is a counterexample in the first semantics. There is no
%% %% counterexample if we add the looping conjuncts to the postcondition.

%% \begin{theorem}[Finite models for contract-strict functions]
%% If the function $f$ is contract-strict for contract $\Ct = (x:\{ y \mid p\}) -> \{ y \mid q \}$ and 
%% and there exists $e$ such that $\dbrace{e} \in \dbrace{\{y \mid p\}}$ but 
%% $\dbrace{f\;e} \notin \dbrace{\{y \mid q\}}_{x |-> \dbrace{e}}$ then there exists 
%% a finite model $M_s \models \Th^M,\ptrans{}{P}^M,\ctrans{}{}{f \notin \Ct}^M$.
%% \end{theorem}
%% \begin{proof} By contract strictness it must be the case that $P |- p[e/y] \Downarrow v$ 
%% and $P |- e \Downarrow v$, otherwise $e$ would not be a counterexample since the result or the 
%% postcondition would diverge. This means that we have in hand a finite collection of finite traces 
%% and we can repeat the construction of the previous section.
%% \end{proof}




%% Enforcing contract-strictness is a rather external condition so could we do something different that 
%% allows us to prove perhaps fewer things but internalizes this choice? Consider the following alternative
%% semantics for first order contracts:
%% \[\begin{array}{lcl}
%%   \dbrace{\{ y \mid p\}}_\rho & = & 
%%           \{ d \mid d = \bot \text{ or } \dbrace{p}_{\rho,y |-> d} \sqsubseteq \injKZ{True} \} \\ \\ 
%%   \flat\dbrace{\{ y \mid p\}}_\rho & = & 
%%           \{ d \mid \dbrace{p}_{\rho,y |-> d} = \injKZ{True} \} \\ \\

%%   \dbrace{(x:\Bt) -> \Ft}_\rho & = & \{ d \mid \text{for all } d' \in \flat\dbrace{\Bt}_{\rho} \\
%%                               &   & \qquad \text{ it is }\dapp(d,d') \in \dbrace{\Ft}_{\rho,x|->d'} \} 
%% \end{array}\]

%% Consider the corresponding changes to $\ctrans{}{}{e \in \Ft}$ 
%% and $\ctrans{}{}{e \in \Ft}^M$ and $\ctrans{}{}{e \notin \Ft}^M$ needed 
%% to reflect this semantics:

%% \[\begin{array}{c}
%% \ruleform{\ctrans{\Sigma}{\Gamma}{e \in \Ft} = \formula{\phi}} \\ \\
%% \begin{array}{rcl}
%% \ctrans{\Sigma}{\Gamma}{e \in \{x\;\mid\;e' \}}
%%   & = & \formula{t{=}\unr} \\
%%   & \lor & \formula{t'[t/x]{=}\unr} \\
%%   & \lor & \formula{t'[t/x]{=}\True} \\
%%   & \mbox{where} &
%%     \begin{array}[t]{rcl}
%%       t  & = & \etrans{\Sigma}{\Gamma}{e} \; \text{and} \; t' = \etrans{\Sigma}{\Gamma}{e'}
%%     \end{array}
%% \\
%% \flat\ctrans{\Sigma}{\Gamma}{e \in \{x\;\mid\;e' \}}
%%   & = & \formula{t'[t/x]{=}\True} \\
%%   & \mbox{where} &
%%     \begin{array}[t]{rcl}
%%       t  & = & \etrans{\Sigma}{\Gamma}{e} \; \text{and} \; t' = \etrans{\Sigma}{\Gamma}{e'}
%%     \end{array}
%% \\
%% \ctrans{\Sigma}{\Gamma}{e \in (x{:}\Bt_1) -> \Ft_2}
%%   & = & \formula{\forall x @.@ \flat\ctrans{\Sigma}{\Gamma,x}{x \in \Bt_1}
%%                           \Rightarrow \ctrans{\Sigma}{\Gamma,x}{e\;x \in \Ft_2}}
%% \end{array}
%% \end{array}\]

%% In the $min$-world we have:

%% \[\begin{array}{c}
%% \ruleform{\ctrans{\Sigma}{\Gamma}{e \in \Bt}^M = \formula{\phi}} \\ \\
%% \begin{array}{rcl}
%% \multicolumn{3}{l}{\ctrans{\Sigma}{\Gamma}{e \in \{x\;\mid\;e' \}}^M = }\\
%% \multicolumn{3}{l}{  \qquad({\bf min(t'[t/x])}\;\land\; (t'[t/x]{=}\True))} \\
%% %%   &   & \;\lor\; \formula{t'[t/x]{=}\True}) \\
%%   \mbox{where} & &
%%       t = \etrans{\Sigma}{\Gamma}{e} \quad t' = \etrans{\Sigma}{\Gamma}{e'}
%% %% \\
%% %% \ctrans{\Sigma}{\Gamma}{e \in (x{:}\Ct_1) -> \Ct_2}^M
%% %%   & = & \formula{\forall x @.@ \ctrans{\Sigma}{\Gamma,x}{x \notin \Ct_1}^M \lor} \\
%% %%   &   & \qquad\qquad \formula{\ctrans{\Sigma}{\Gamma,x}{e\;x \in \Ct_2}}^M
%% %% \\
%% %% \ctrans{\Sigma}{\Gamma}{e \in \Ct_1 \& \Ct_2}
%% %%    & = & \formula{ \ctrans{\Sigma}{\Gamma}{e \in \Ct_1} /\ \ctrans{\Sigma}{\Gamma}{e \in \Ct_2}}
%% %% \\
%% %% \ctrans{\Sigma}{\Gamma}{e \in \CF} & = & \formula{\lcf{\etrans{\Sigma}{\Gamma}{e}}}
%% \end{array} \\ \\ 
%% \ruleform{\ctrans{\Sigma}{\Gamma}{e \notin \Ft}^M = \formula{\phi}} \\ \\
%% \begin{array}{rcl}
%% \multicolumn{3}{l}{ \ctrans{\Sigma}{\Gamma}{e \notin \{x\;\mid\;e' \}}^M = } \\ 
%% \multicolumn{3}{l}{ \qquad {\bf min(t)\;\land\; min(t'[t/x])}\;\land} \\ 
%% \multicolumn{3}{l}{ \qquad (\formula{t \not \red \unr\;\land\; t'[t/x] \not\red \unr \;\land\; t'[t/x] \not\red \True})} \\
%% %% \multicolumn{3}{l}{ \qquad \land\; \formula{t'[t/x]{\not\red}\True}} \\
%%   \mbox{where} & & 
%%       t= \etrans{\Sigma}{\Gamma}{e} \;\; t' = \etrans{\Sigma}{\Gamma}{e'}
%% \\
%% \ctrans{\Sigma}{\Gamma}{e \notin (x{:}\Bt_1) -> \Ft_2}^M
%%   & = & \formula{\exists x @.@ \ctrans{\Sigma}{\Gamma,x}{x \in \Bt_1}^M \land} \\
%%   &   & \qquad\qquad \formula{\ctrans{\Sigma}{\Gamma,x}{e\;x \notin \Ft_2}}^M
%% %% \\
%% %% \ctrans{\Sigma}{\Gamma}{e \in \Ct_1 \& \Ct_2}
%% %%    & = & \formula{ \ctrans{\Sigma}{\Gamma}{e \in \Ct_1} /\ \ctrans{\Sigma}{\Gamma}{e \in \Ct_2}}
%% %% \\
%% %% \ctrans{\Sigma}{\Gamma}{e \in \CF} & = & \formula{\lcf{\etrans{\Sigma}{\Gamma}{e}}}
%% \end{array}
%% \end{array}\]



%% For this semantics and logic translation, it is easy to show that soundness still holds. 
%% Moreover the following results are true:


%% Very pleasantly, completeness with respect to the unmodified theory goes through as well.

%% \begin{theorem}[Completeness for finitistic first-order semantics]
%% Assume a contract $\Ft = (x{:}\Bt_1) -> \Bt_2$. If 
%% $\Th,\ptrans{}{P} |- \neg \ctrans{}{}{e \in \Ft}$ then $\Th^M,\ptrans{}{P}^M |- \neg \ctrans{}{}{e \in \Ft}^M$.
%% \end{theorem}
%% \begin{proof}
%% We proceed as in the failed proof previously 
%% \end{proof}





%% \newcommand{\fin}{\mathsf{FIN}}
%% \dv{After milions of discussions with Josh I am slowly starting to get a handle on the semantics of arrows.}


%% Maybe having contracts where the precondition excludes infinite traces is a {\em good thing} to do 
%% anyway. Instead of imposing an external condition might we modify the very same meaning of 
%% contracts?

%% \[\begin{array}{lcl}
%%    d \notin \{ y \mid p \}  &  = & d \neq \bot \text{ and } \dbrace{p}_{x |-> d} \neq \bot,\injKZ{True} \\
%%    d \notin \Ct_1 -> \Ct_2  & =  & \text{there exists } d' \in_{fin} \Ct_1 \text{ and } d(d') \notin \Ct_2 \\

%%    d \in_{\fin} \{ y \mid p \} & = & \dbrace{p}_{x |-> d} = \injKZ{True}  \\
%%    d \in_{\fin} \Ct_1 -> \Ct_2 & = & \text{for all } d', d' \notin \Ct_1 \text{ or } d(d') \in_{\fin} \Ct_2 \\ \\ 
  
%%    d \in \{ y \mid p \}    & = & d = \bot \text{ or } \dbrace{p}_{x |-> d} \sqsubseteq \injKZ{True} \\ 
%%    d \in \Ct_1 -> \Ct_2    & = & \text{forall } d', d' \in_{\fin} \Ct_1 \text{ implies } d(d') \in \Ct_2
%% \end{array}\] 
%% Something along these lines? What we lose? What we gain? What still holds? Not sure ...

%% \subsection{Higher-order arrow contracts}

%% The story of first-order arrow contracts is quite reasonable: as our
%% example shows it is quite pathological to have a counterexample that
%% is justified by an infinite trace in the precondition, and such that
%% that infinite trace has to evaluate infinitely many new terms.

%% For higher-order contracts \dv{As is the case when we try to prove things
%% by induction, or when we have already proven a contract about a function and
%% attempt to prove more contracts using it.} we still have completeness (by our
%% previous theorems) but we have lost the finite models property -- this time due
%% to the {\em quantifier} of an arrow contract for an argument.

%% Consider a counterexample for:
%%   \[ \ctrans{}{}{f \in (\Ct_1 -> \Ct_2) -> \Bc} \] 
%% It will consist of a program (a denotation actually, but for the sake of the discussion 
%% let's just say program) $e$ such that $\ctrans{}{}{e \in \Ct_1 -> \Ct_2}$ but such that 
%% $f(e)$ does not satisfy the $\Bc$ contract. This in turn means that {\em for all possible 
%% arguments} to $e$, $e'$, either they do not satisfy $\Ct_1$ or the application satisfies 
%% $\Ct_2$. As there can be infinitely many arguments to $e$ that need to be tested, this does
%% not seem to be a property that we should hope it have some finite justification.

%% Consider the concrete positive 
%% translation when $\Ct_1 -> \Ct_2$ is $\{ x \mid p\} -> \{y \mid q\}$ and let us call $g$ the
%% counterexample: 
%% \begin{eqnarray*}
%%   \forall x. (min(x) \land min(p(x) \land x \neq \unr \land p(x) = \False) \lor \\
%%   \quad  min(g(x)) \land min(q(g(x))) \land q(g(x)) = \True 
%% \end{eqnarray*}
%% Notice that I have not added cases for $q(g(x)) = \unr$ and $g(x) = \unr$, following the 
%% ``finitistic'' satisfaction definition above.

%% Imagine though that $p(x) = \True$ uniformly -- this implies in effect that: 
%% \begin{eqnarray*}
%%     \forall x. min(g(x)) \land min(q(g(x))) \land q(g(x)) = \True
%% \end{eqnarray*}

%% If $g$ is the identity function, in effect we have asserted that for 
%% $\forall x. min(x)$ -- losing finite models.

%% It appears that the quantifier $\forall x @.@ \ldots$ introduced by the positive translation
%% is over too many terms, {\em even if} we restrict the base contract satisfaction to have this
%% finitistic flavor we have described previously.

%% So, what is a good candidate for modifying the positive arrow translation which can recover
%% finite models? One good candidate is: 

%% \section{FM for the restricted arrow translation?}
%% Can we get some finitistic notion of satisfaction in this restricted
%% arrow world that can give us a guarantee for finite models?


\end{document}

\clearpage
\section{Soundness and completeness}

\begin{theorem}[Soundness]
If $\Th^M,\ptrans{}{P}^M \vdash \neg \ctrans{}{}{e \notin C}^M$ then
   $\Th , \ptrans{}{P} \vdash \neg \ctrans{}{}{e \notin C}$.
\end{theorem}

\begin{proof}
Let us pick any model ${\cal M}$ of $\Th , \ptrans{}{P}$. Then it is easy to 
confirm that the model ${\cal M}'$ which is just ${\cal M}$ with the predicate $min(\cdot)$ 
interpreted as everywhere true, is a model of $\Th^M , \ptrans{}{P}^M$. This in turn means 
that $M' \models \neg \ctrans{}{}{e \notin C}^M$, and it is an easy lemma to show that this 
implies $M \models \neg \ctrans{}{}{e \notin C}$.
\end{proof}

Next we show that this translation is no weaker than the canonical translation (NB: which of course is incomplete with respect to the denotational semantics).

\begin{lemma}[Guarded-Neq]\label{lem:guarded-neq}
If $\Th , \ptrans{}{P} \vdash \etrans{}{}{e} \neq \unr$ then
   $\Th^M , \ptrans{}{P}^M \vdash min(\etrans{}{}{e}) => \etrans{}{}{e} \neq \unr$.
\end{lemma}
\begin{proof}
By soundness of the $min$-less translation and adequacy of the denotational
semantics we get that $e \Downarrow v$ for some value $v$. By Lemma~\ref{lem:guarded-eval}
this means that $\Th^M, \ptrans{}{P}^M \vdash min(t) => t = t_v$ where $t = \etrans{}{}{e}$
and $t_v = \etrans{}{}{v}$. However since $v$ is a value form we know that $t_v \neq \unr$ 
in the $min$-theory.
\end{proof}

\begin{lemma}[Guarded-Eq]\label{lem:guarded-eq}
   $\Th , \ptrans{}{P} \vdash \etrans{}{}{e} = True$ then 
   $\Th^M , \ptrans{}{P}^M \vdash min(\etrans{}{}{e}) => \etrans{}{}{e} = True$.
\end{lemma}
\begin{proof}
By soundness of the $min$-less translation we get that 
$\dbrace{e} = \injKZ{True}$, that is $e \Downarrow \True$. By 
Lemma~\ref{lem:guarded-eval} the proof is finished.
\end{proof}

\begin{lemma}[Guarded eval]\label{lem:guarded-eval}
If $e \Downarrow v$ then $\Th^M, \ptrans{}{P}^M \vdash min(t) => t = t_v$
where $t = \etrans{}{}{e}$ and $t_v = \etrans{}{}{v}$.
\end{lemma}
\begin{proof} 
By induction on the evaluation of $e \Downarrow v$.
\end{proof}

\begin{lemma}[Guarded-Unr]\label{lem:guarded-unr}
If $t = \etrans{}{}{e}$ and 
   $\Th , \ptrans{}{P} \vdash t = \unr$ then 
   $\Th^M , \ptrans{}{P}^M \vdash min(t) => t = \unr$.
\end{lemma}
\begin{proof} (\dv{Sketch!}) This is a much more tricky proof. The problem is that 
soundness of the $min$-less translation and adequacy of the denotational semantics 
tells us that there is no $v$ such that $e \Downarrow v$. This in turn means two 
things:
\begin{itemize}
  \item Either $e$ genuinely diverges, or
  \item Evaluation of $e$ gets stuck -- because of some type error.
\end{itemize}
Now, in the second case we are conceptually done because we have a finite number of steps
until stuckness so we can extend the proof of Lemma~\ref{lem:guarded-eval} by using the 
$\downarrow$ operational semantics that includes {\em explicitly provable} unreachability 
(stuckness) as a special value @UNR@ (the logical translation is what you'd expect, 
translating @UNR@ to $\unr$). Denotationally we still associate $\unr$ to $\bot$.

What if $e$ actually diverges? Then I claim that Lemma~\ref{lem:divergence-unprovability}
is true, which finishes the case.
\end{proof}

\begin{lemma}[Divergence unprovability]\label{lem:divergence-unprovability}
If $t = \etrans{}{}{e}$ and $e$ diverges then there exists a model
${\cal M}$ such that ${\cal M} \models \Th \land \ptrans{}{P} \land (t \neq \unr)$.
\end{lemma}
\begin{proof} (\dv{Sketch!})
The intuition here is that if an expression diverges then we can equate 
it to $\bad$ {\em as well} in some model! Consider the following model: 
let the universe be the universe of terms arranged in equivalence classes accoring
to the following equivalence. If $e$ diverges then $e = bad$, if $e \downarrow v$ then
$e = v$ and if the evaluation of $e$ gets stuck then $e = unr$. Close by reflexivity,
symmetry, transitivity and congruence. I claim that this is a good model of the theory
and the program definitions, and moreover in this model no divergent term is equated to
$\unr$, but rather $\bad$.

\dv{NB: Using any {\em other} value than $\bad$ in the construction of Lemma~\ref{lem:divergence-unprovability}
is much harder to be made to work (if at all!), since if the scrutinee of a case expression diverges, allowing it 
to return a constructor from the patterns would result in the expression being equated to some ordinary (but unknown) 
value. By contrast letting it return $\bad$ allows the $\bad$ value to propagate through the case expression as
it should -- since the case expression itself is divergent!}

\end{proof}




\begin{theorem}[Completeness]
The following are true:
\begin{enumerate}
   \item If $\Th , \ptrans{}{P} \vdash \neg \ctrans{}{}{e{\notin}\Ct}$ 
   then $\Th^M , \ptrans{}{P}^M \vdash \neg \ctrans{}{}{e{\notin}\Ct}^M$.
   \item If $\Th , \ptrans{}{P} \vdash \neg \ctrans{}{}{e{\in}\Ct}$ 
   then $\Th^M , \ptrans{}{P}^M \vdash \neg \ctrans{}{}{e{\in}\Ct}^M$.
\end{enumerate}
\end{theorem}
\begin{proof}
We prove the two parts by induction on the structure of the contract $\Ct$. Let us 
proceed for the first goal. We consider two cases and let $t = \etrans{}{}{e}$.
\begin{itemize}
  \item Case $\Ct = \{ y \mid p \}$. In this case we let $t_p = \etrans{}{}{p}$ and 
        we have that
        \begin{eqnarray*}
           \Th , \ptrans{}{P} \vdash \phi 
        \end{eqnarray*}
        where $\phi = t \neq \unr \land t_p[t/y] \neq \unr
                                                \land t_p[t/y] \neq \True$.
        We need to show that:
        \begin{eqnarray*}
          \Th^M, \ptrans{}{P}^M \vdash min(t) \land min(t_p[t/y]) => \phi
        \end{eqnarray*}
        This follows by Lemmas~\ref{lem:guarded-eq} and~\ref{lem:guarded-neq}.
  \item Case $\Ct = (x:\Ct_1) -> \Ct_2$. In this case we have that:
        \begin{eqnarray*}
          \Th, \ptrans{}{P} \vdash \phi
        \end{eqnarray*}
        with $\phi = \neg \exists x @.@ \ctrans{\Sigma}{\Gamma,x}{x \in \Ct_1} 
                                  \land \ctrans{\Sigma}{\Gamma,x}{e\;x \notin \Ct_2}$.
        We must show:
        \begin{eqnarray*}
          \Th^M, \ptrans{}{P}^M \vdash \phi^M
        \end{eqnarray*}
        This follows by induction hypothesis for both items.
\end{itemize}

For the second goal we proceed similarly: the base case follows 
by Lemmas~\ref{lem:guarded-eq} and~\ref{lem:guarded-unr}.
\end{proof}

\subsection{Restricted arrow translation completeness}

For reasons related to the existence of finite models (that we will return to in 
Section~\ref{sect:fin-mods}, our previous translation of contracts is not 
satisfactory. For this reason we will instead use a {\em restricted arrow} translation 
for arrow contracts, given below:

\[\begin{array}{lcl}
\ctrans{\Sigma}{\Gamma}{e \in (x{:}\Ct_1) -> \Ct_2}^M
  & = & \forall x @.@ {\bf min(\etrans{}{}{e\;x})} => \\ 
  &   & \qquad \ctrans{\Sigma}{\Gamma,x}{x \notin \Ct_1}^M \lor 
                                \ctrans{\Sigma}{\Gamma,x}{e\;x \in \Ct_2}^M
\end{array}\]

This is still sound (very easy to carry over the soundness proof), but we will now
investigate its completeness. Completeness (if true!) might come out as a bit of surprise since, 
when doing a proof, the unrestricted version 
arising from $\ctrans{}{}{e \in \Ct_1 -> \Ct_2}^M$ can fire at any time, whereas 
the restricted version with the modified arrow case can fire {\em only} if we can 
prove that $min(\etrans{}{}{e\;e'})$ for some argument $e'$.

Lets call the variants of the $(\cdot)^M$ translations, which use the modified arrow
contract translation $(\cdot)^{M}_{\flat}$. What we would like to show is the following:

\[\begin{array}{c}
   \Th^M, \ptrans{}{P}^M \vdash \neg \ctrans{}{}{e \notin \Ct}^M \\ 
   \text{ implies }  \\ 
   \Th^M, \ptrans{}{P}^M \vdash \neg \ctrans{}{}{e \notin \Ct}^M_{\flat}
\end{array}\]

An alternative way to put is that if there exists a model 
\[ {\cal M} \models \Th^M \land \ptrans{}{P}^M \land \ctrans{}{}{e \notin \Ct}^M_{\flat} \]
then there exists some model ${\cal M}_{\sharp}$ such that:
\[ {\cal M}_{\sharp} \models \Th^M \land \ptrans{}{P}^M \land \ctrans{}{}{e \notin \Ct}^M \]

\begin{definition}[Construction of ${\cal M}_{\sharp}$]
Consider a model 
\[ {\cal M} \models \Th^M \land \ptrans{}{P}^M \land \ctrans{}{}{e \notin \Ct}^M_{\flat} \]
Consider the following construction: (i) take all elements $d \in dom(M)$ such that 
$d \notin min(M)$ and equate them to ${\cal I}(\unr)$. This is the domain of the new model ${\cal M}_{\sharp}$.
The $min$-table of the new model is the same as the old, union with $\{ {\cal I}(\unr)$.
Call this new structure ${\cal M}_{\sharp}$.
\end{definition}

\begin{theorem}\label{thm:model-sharp-one}
If ${\cal M} \models \Th^M \land \ptrans{}{P}^M$ then ${\cal M}_{\sharp} \models \Th^M \land \ptrans{}{P}^M$.
\end{theorem}
\begin{proof} \dv{This should be a quite involved proof, since I am bottoming-out several definitions and 
adding them to min. But I think it should work...}
\end{proof}

\newcommand{\Mod}{{\cal M}}
\newcommand{\ModSharp}{{\cal M}_{\sharp}}

\begin{theorem}\label{thm:model-sharp-two}
Assume that base contracts are {\bf \em strict in their predicates}.
Assume ${\cal M} \models \Th^M \land \ptrans{}{P}^M$. Then:
\begin{itemize}
  \item If ${\cal M} \models \ctrans{}{}{e \notin \Ct}^M_{\flat}$ then ${\cal M}_{\sharp} \models \ctrans{}{}{e \notin \Ct}^M$.
  \item If ${\cal M} \models \ctrans{}{}{e \in \Ct}^M_{\flat}$ then ${\cal M}_{\sharp} \models \ctrans{}{}{e \in \Ct}^M$.
\end{itemize}
\end{theorem}
\begin{proof}
We prove the two parts by induction on the structure of the contract $\Ct$. For the first part we have:
\begin{itemize}
 \item $\Ct = \{ y \mid p \}$. We have that $\Mod \models min(e) \land min(p(e)) \land e \neq \unr \land p(e) \neq \{ \unr,\True \}$.
       But that means that $e$ and $p(e)$ were already in the $min$ table so they will not be equated to $\unr$ in $\ModSharp$ (nor will
       do so any values that they reduce to), hence we get: 
         \[ \ModSharp \models min(e) \land min(p(e)) \land y \neq \unr \land p(e) \neq \{ \unr,\True \} \]

 \item $\Ct = (x:\Ct_1) -> \Ct_2$. Notice for convenience we use a more decomposed form of the contract.
       In this case we have that $\Mod \models \exists x @.@ \ctrans{}{}{x \in \Ct_1}^M_{\flat} \land \ctrans{}{}{e\;x \notin \Ct_2}^M_{\flat}$ 
       and the result simply follows by induction hypothesis for both parts.
\end{itemize}
For the second part we have again two cases: 
\begin{itemize}
 \item $\Ct = \{ y \mid p \}$. We have that $\Mod \models min(e) \land min(p(e)) \land (e = \unr \lor p(e) = \{ \unr,\True \})$.
       Since $e$ and $p(e)$ were already in the min-table to start with the result follows.

 \item $\Ct = (\ol{x{:}\Ct}) -> \{ y \mid p\}$. Notice for convenience we use a more decomposed form of the contract.
       In this case we have that
       \[\begin{array}{l}
           \Mod \models \forall \xs @.@ (\neg(min(e\;\xs))) \lor \ctrans{}{}{\ol{x \notin \Ct}}^M_{\flat} \lor \phi \\ \\ 
           \phi = min(e\;\xs) \land min(p(e\;\xs)) \land \\ 
           \qquad\quad (e\;\xs = \unr \lor p(e\;\xs) = \{ \unr,\True \}) 
       \end{array}\]
       We need to show that:
       \begin{eqnarray*} 
           \ModSharp \models \forall \xs @.@ \ctrans{}{}{\ol{x \notin \Ct}}^M \lor \phi \\
       \end{eqnarray*}
       Let us pick some $\xs$. Assume $min(e\;\xs)$ in $\Mod$, then the result follows by induction hypothesis. However,
       if $min(e\;\xs) \notin \Mod$ then we know that $e\;\xs = \unr$ in $\ModSharp$ and moreover $min(\unr)$. Now, if
       in $\Mod$ we had $min(p(e\;\xs))$ then we must also have had $min(e\;\xs)$ since $p(\cdot)$ is strict. On the
       other hand if we $p(e\;\xs) \notin min(\Mod)$ then definitely $p(e\;\xs) = \bot$ and we have $min(\unr)$ in $\ModSharp$.
\end{itemize}
\end{proof}



\section{Finite models}\label{sect:fin-mods}

We've seen that the $min$-enabled theory is sound and complete but does 
it enjoy finite models? Let us condider first simple base contracts. Assume
that we know that
\begin{equation}
  \dbrace{e} \notin \dbrace{\{y \mid p\}} \label{fm:eq1} 
\end{equation}
Then this statement gives us evidence for two concrete and finite execution {\em traces}:
\begin{equation}
   e \Downarrow v  \quad\quad p[e/y] \Downarrow False \label{fm:eq2} 
\end{equation}
Given these two traces we will show next that the FOL formula 
\[ \Th^M \land \ptrans{}{P} \land \ctrans{}{}{e \notin \{ y \mid p \}} \] 
does have a {\em finite model}. By doing this we get a formal guarantee that 
if \ref{fm:eq1} holds then {\em there is} a finite model, so a finite model finder 
that systematically explores finite models will find one.

Here is the construction. 

\begin{definition}[Construction finite models from traces]
We start with a set of traces $S \supseteq \{ p[e/y] \Downarrow False, e \Downarrow v \}$.
By trace I mean the conclusion of every subderivation in the derivation tree 
for $\Downarrow$, not just the final conclusion.

We consider the term model of all terms and subterms that appear in $S$. We consider equality
to be the equality induced by derivations $e \Downarrow v$ (that is we set $e = v$) and close 
by reflexivity, transitivity and congruence. We create a table for $min$ in the following way:
for each equivalence class created, we add it to $min$ iff there exists a term $e_{L}$ in 
that class that has appeared in the left-hand side of a derivation $e_{L} \Downarrow \_$ 
in the set of traces. Let us call this model ${\cal M}_S$. 
\end{definition}

\begin{lemma} If $S \supseteq \{ p[e/y] \Downarrow False, e \Downarrow v \}$ is a finite
collection of traces, then: 
\[   {\cal M}_S \models \Th^M \land \ptrans{}{P} \land \ctrans{}{}{e \notin \{ y \mid p \}} \] 
\end{lemma}
\begin{proof} \dv{TODO! But I think it should be doable ... } \end{proof}


\subsection{First-order arrow contracts}

The story is not finished yet, what about arrow contracts $(x:\Ct_1) -> \Ct_2$. The problem
here that if we know that $e$ does not belong in $\dbrace{(x:\Ct_1) -> \Ct_2}$ then this mean
that there exists some denotation $d$ such that the application of $\dbrace{e}$ to $d$ does
not belong to $\Ct_2$ {\em and moreover} $d$ belongs in $\dbrace{\Ct_1}$. 

Assuming that $\Ct_1$ and $\Ct_2$ are both base contracts and that $d$ is the image of some 
actual term $e_d$ then we do get some finite traces from the fact that the application does 
not belong in $\Ct_2$ but the fact is that our model has to also be a model of:
\[       \ctrans{}{}{e_d \in \Ct_1} \]

This is problematic, because we have no ``finite witness'' that $e_d$ is in $\dbrace{\Ct_1}$, 
contrary to the negative case. In fact $e_d$ could diverge, or, even worse, the predicate 
associated with $\Ct_1$ could diverge yielding an infinite trace. 

From an infinite trace it can often be the case that a finite model can be constructed, if
the trace involves a finite number of new values. But in principle an infinite trace can
be causing the evaluation of even more and new terms, which excludes finite models. Here is 
an example:

\begin{code}
data Nat = S Nat | Z

pinf Z (S x) = pinf (S Z) (S (S x))
pinf (S g) (S x) = pinf (S (S g)) (S (S x))
pinf g Z     = False

isZero Z = True
isZero _ = False

f :: Nat -> Nat
f x = x

pinfBroken = f ::: Pred (pinf Z) --> Pred isZero
\end{code}

Notice in this case that there is a finitely representable
counterexample @s = S s@ -- however the predicate 
@pinf Z s@ loops in a fashion that evaluates ever growing terms! Indeed
the $min$-enabled translation {\em also} results in the model finder 
looping for this broken contract.

We acknowledge the existence of such problematic counterexamples but
we can at least characterize (conservatively, not tightly) some
situations where we definitely can guarantee a finite model (we use $\Bc$ below
for base contracts).

\begin{lemma}
If $\dbrace{\ol{e}} \in \dbrace{\ol{\Bc}}$ finitistically, that is, excluding the possibility
for the predicate and $e$ to be $\bot$ \dv{(... alternatively, via having traces that, if 
infinite they only mention finite terms is a more relaxed condition)} but 
$\dbrace{e\;\ol{e}} \notin \dbrace{\Bc_r}$ then there exists
a finite model of 
\[ \Th^M \land \ptrans{}{P}^M \land \ctrans{}{}{e \notin (\ol{x}:\ol{\Bc}) -> \Bc_r}^M \]
\end{lemma}
\begin{proof} Should be similar to the previous theorem.\end{proof}


\paragraph{A condition on contracts that guarantees finitistic satisfaction}
Assume that every contract of the form $\{x \mid p\} -> \{ y \mid q \}$ that we try to 
prove about $f$ satisfies the following condition: if $d = \bot$ or $\dbrace{p}(d) = \bot$ then 
$\dbrace{f}(d) = \bot$ or $\dbrace{q}(\dbrace{f}(d)) = \bot$. If this is the case, then 
{\em any} counterexample to $f$ satisfying its contract {\em excludes} the possibility of 
infinite traces, by construction, since an infinite trace for an argument satisfying its 
contract would result in an infinite trace in the result predicate or function, casting this
particular argument {\em not} a counterexample! In simpler words, if we can make sure that 
the precondition of a function is no stricter than the function itself (or its return 
predicate) then we recover finite models.

\subsection{Higher-order arrow contracts}

The story of first-order arrow contracts is quite reasonable: as our
example shows it is quite pathological to have a counterexample that
is justified by an infinite trace in the precondition, and such that
that infinite trace has to evaluate infinitely many new terms.

For higher-order contracts \dv{As is the case when we try to prove things
by induction, or when we have already proven a contract about a function and
attempt to prove more contracts using it.} we still have completeness (by our
previous theorems) but we have lost the finite models property -- this time due
to the {\em quantifier} of an arrow contract for an argument.

Consider a counterexample for:
  \[ \ctrans{}{}{f \in (\Ct_1 -> \Ct_2) -> \Bc} \] 
It will consist of a program (a denotation actually, but for the sake of the discussion 
let's just say program) $e$ such that $\ctrans{}{}{e \in \Ct_1 -> \Ct_2}$ but such that 
$f(e)$ does not satisfy the $\Bc$ contract. This in turn means that {\em for all possible 
arguments} to $e$, $e'$, either they do not satisfy $\Ct_1$ or the application satisfies 
$\Ct_2$. As there can be infinitely many arguments to $e$ that need to be tested, this does
not seem to be a property that we should hope it have some finite justification.

Consider the concrete positive 
translation when $\Ct_1 -> \Ct_2$ is $\{ x \mid p\} -> \{y \mid q\}$ and let us call $g$ the
counterexample: 
\begin{eqnarray*}
  \forall x. (min(x) \land min(p(x) \land x \neq \unr \land p(x) = \False) \lor \\
  \quad  min(g(x)) \land min(q(g(x))) \land q(g(x)) = \True 
\end{eqnarray*}
Notice that I have not added cases for $q(g(x)) = \unr$ and $g(x) = \unr$, following the 
``finitistic'' satisfaction definition above.

Imagine though that $p(x) = \True$ uniformly -- this implies in effect that: 
\begin{eqnarray*}
    \forall x. min(g(x)) \land min(q(g(x))) \land q(g(x)) = \True
\end{eqnarray*}

If $g$ is the identity function, in effect we have asserted that for 
$\forall x. min(x)$ -- losing finite models.

It appears that the quantifier $\forall x @.@ \ldots$ introduced by the positive translation
is over too many terms, {\em even if} we restrict the base contract satisfaction to have this
finitistic flavor we have described previously.

So, what is a good candidate for modifying the positive arrow translation which can recover
finite models? One good candidate is: 

\section{FM for the restricted arrow translation?}
Can we get some finitistic notion of satisfaction in this restricted
arrow world that can give us a guarantee for finite models?

\end{document}


