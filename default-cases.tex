
Can we translate cases with branches that matches anything?  In
Haskell we typically see them as a pattern match on @_@, and in GHC
Core they are written as the first alternative and called @DEFAULT@,
but to be similar to Haskell, we will write them with an underscore
last instead. It could be beneficial to translate them because they
can save a lot of axioms, the archetypal example is going from
$O(n^2)$ to $O(n)$ cases for equality on a data type with $n$
constructors.

For now, we explore a simple example:

\begin{code}
    id x = case x of
        True -> True
        _    -> False
\end{code}

Since the case already had a default branch, instead of the usual to
put a @_ -> UNR@ branch, an @UNR -> UNR@ branch is inserted, making
the case ``look'' like this:

\begin{code}
    id x = case x of
        True -> True
        BAD  -> BAD
        UNR  -> UNR
        _    -> False
\end{code}

Now the translation to FOL seems straightforward, and for this code
it amounts to these four axioms:

\[\begin{array}{rl}
 & @id@(@True@) = @True@, \\
 & @id@(@BAD@) = @BAD@, \\
 & @id@(@UNR@) = @UNR@, \\
\forall x . &
    (x \neq @True@ \land
     x \neq @BAD@ \land
     x \neq @UNR@) =>  @id@(x) = @False@
\end{array}\]

Great, and a contract like $\{ @id@ \mid @CF@ -> @CF@ \}$ goes through
just fine.

Let's make our simple function a bit more complicated:

\begin{code}
    id2 x = case x of
        True -> True
        _    -> case x of
            False -> False
            _     -> error "Unreachable!"
\end{code}

It looks innocent enough, right? Let us see it with the @UNR@ and
@BAD@ glasses as well:

\begin{code}
    id2 x = case x of
        True -> True
        BAD  -> BAD
        UNR  -> UNR
        _    -> case x of
            False -> False
            BAD  -> BAD
            UNR  -> UNR
            _     -> error "Unreachable!''
\end{code}

When tranlating this declaration, we collect ``constraints'' of
inequalities in the default branches. Let's see what we have collected
in the branch that goes to the error in the code for @id2@ above.
From the first case, we know that @x@ is neither of @True@, @BAD@,
@UNR@ and the second case we also get to know that @x@ is not
@False@. So the final axiom for this branch looks like this:

\[\begin{array}{rl}
\forall x . &
    (x \neq @True@ \land
     x \neq @False@ \land
     x \neq @BAD@ \land
     x \neq @UNR@) =>  \\ & @id2@(x) = @BAD@
\end{array}\]

Here, the call to @error@ has been translated to @BAD@. Does the
contract $\{ @id2@ \mid @CF@ -> @CF@ \}$ go through now? \emph{No!}
The reason for this is that an untyped argument - let's say @Nothing@ -
will with this translation of @id2@ return @BAD@:

$$@id2@(@Nothing@) = @BAD@$$

Our system is built upon untyped arguments should go to @UNR@!  The
example above was after all written in Haskell rather than i GHC Core.
Then, does it occur in the GHC Core at all? \emph{Yes,} non-trivial
pattern-matches can actually be actually translated to casing on the
same variable several times. One simple example is this @or@ function:

\begin{code}
    or :: Bool -> Bool -> Bool
    or True  y    = True
    or x     True = True
    or False y    = False
\end{code}

After desugaring, we get this code in (prettified) Core:

\begin{code}
or :: Bool -> Bool -> Ok
or = \ (x :: Bool) (y :: Bool) ->
    case x of {
        True -> True
        _ -> case y of {
            True -> True
            _ -> case x of {
                False -> False
                _ -> patError "function or"
\end{code}

This Core snippet for @or@ exhibits exactly the same problem as @id2@:
it cases on @x@ twice, and throws @BAD@ on an untyped argument.  These
examples seems to suggest that the problem is local: that is, it's a
problem that we try to translate many cases in one go, and we should
instead transform the program so there is only one top level case at
each function. But alas, no, this is not the situation at
hand.

We revisit the @id2@ example, and
lift out the inner case
\footnote{This is written in in source Haskell, and with a data type
  with a a few more constructors than two, this function will have
  default branches analogous to above, even with core optimisations.
}:

\begin{code}
    id x = case x of
        True -> True
        _    -> id' x

    {-# NOINLINE id2 #-}
    id2 x = case x of
        False -> False
        _     -> error "Unreachable!"
\end{code}

By inserting @UNR -> UNR@ branches, we still get with an ill-typed
argument @x@, we have $@id x@ = @error "Unreachable!"@ = @BAD@$.
Unfortunately, this means that the obvious way to translate default
patterns is unsound in our system, and we have not found any
alternative ways to translate it sound and efficient.

However, there is still a lot of light in the tunnel. What we do
instead is simply ``unrolling'' the default branches.
If we have a branch @DEFAULT -> rhs@, then for each constructor @K@
that is not mentioned in the case, we insert a @K -> rhs@
branch. While this might sound devastating and leads to case
explosion, it turns out that for instance @z3@ is still very efficient
with this translation.






