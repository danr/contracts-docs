\documentclass[preprint,nocopyrightspace,draft]{sigplanconf}

\usepackage{hcc-techreport}

\begin{document} 
\preprintfooter{\textbf{--- DRAFT ---}}

\renewcommand{\textfraction}{0.1}
\renewcommand{\topfraction}{0.95}
\renewcommand{\dbltopfraction}{0.95}
\renewcommand{\floatpagefraction}{0.8}
\renewcommand{\dblfloatpagefraction}{0.8}

%%% Extra definitions -- move to hcc-techreport at some point (carefully to not break that!)
\newcommand{\Ct}{{\tt C}}
\newcommand{\CF}{{\tt CF}}
\newcommand{\True}{\textit{True}}
\newcommand{\False}{\textit{False}}
\newcommand{\Bool}{\mathop{Bool}}
\newcommand{\ys}{\ol{y}}
\newcommand{\Th}[2]{{\cal T}_{#1,#2}}
\newcommand{\Ecf}{\textsc{Ecf}}
\newcommand{\oln}[2]{\ol{#1}^{#2}}
\newcommand{\tmar}[2]{\mathop{tmar}_{#1}(#2)}
\newcommand{\tyar}[2]{\mathop{tyar}_{#1}(#2)}
\newcommand{\ar}{n}
\newcommand{\lcf}[1]{\textsf{cf}(#1)}
\newcommand{\lcfZ}{\textsf{cf}}
\newcommand{\lncf}[1]{\neg\textsf{cf}(#1)}
\newcommand{\unr}{\mathop{unr}}
\newcommand{\bad}{\mathop{bad}}
\newcommand{\sel}[2]{\mathop{sel\_#1\!_{#2}}}
\newcommand{\ctrans}[3]{{\cal C}\{\!\!\{#3\}\!\!\}}
\newcommand{\etrans}[3]{{\cal E}\{\!\!\{#3\}\!\!\}}
\newcommand{\utrans}[3]{{\cal U}(#3)\{\!\!\{#2\}\!\!\}}

% Get rid of this -- just temporay
\newcommand{\uutrans}[3]{{\cal U}\{\!\!\{#3\}\!\!\}}

\newcommand{\dtrans}[2]{{\cal D}\{\!\!\{#2\}\!\!\}}
\newcommand{\ptrans}[2]{{\cal P}\{\!\!\{#2\}\!\!\}}
%% Gadgets of domain theory 



\newcommand{\rollK}{\mathsf{roll}}
\newcommand{\unrollK}{\mathsf{unroll}}
\newcommand{\bindK}{\mathsf{bind}}
\newcommand{\retK}{\mathsf{ret}}

% \newcommand{\roll}[1]{\rollK(#1)}
% \newcommand{\unroll}[1]{\unrollK(#1)}
% \newcommand{\bind}[2]{\bindK_{#1}(#2)}
% \newcommand{\ret}[1]{\retK(#1)}
\newcommand{\roll}[1]{#1}
\newcommand{\unroll}[1]{#1}
\newcommand{\bind}[2]{#1(#2)}
\newcommand{\ret}[1]{#1}

\newcommand{\dlambda}{\mathsf{\lambda}}
\newcommand{\curry}{\mathsf{curry}}
\newcommand{\eval}{\mathsf{eval}}
\newcommand{\uncurry}{\mathsf{incurry}}
\newcommand{\dapp}{\mathsf{app}}

\newcommand{\injK}[2]{\mathsf{#1}(#2)}
\newcommand{\injKZ}[1]{\mathsf{#1}}
\newcommand{\injFun}[1]{\mathsf{Fun}(#1)}
\newcommand{\injBad}{\mathsf{Bad}}

\newcommand{\unitcpo}{{\sf{\bf 1}}}
\newcommand{\VarCpo}{\textit{Var}}
\newcommand{\FVarCpo}{\textit{FunVar}}
\newcommand{\interp}[3]{[\![#1]\!]_{\langle {#2},{#3}\rangle}}
\newcommand{\dbrace}[1]{[\![#1]\!]}
\newcommand{\linterp}[1]{{\cal I}(#1)}
\newcommand{\lassign}[1]{\mu(#1)}
\newcommand{\elab}[1]{\rightsquigarrow \formula{#1}}
\newcommand{\Fcf}{F_{\lcfZ}} 
\newcommand{\definable}[1]{{\mathop{def}}(#1)}
\newcommand{\curly}{\rightsquigarrow}
\newcommand{\Min}{\cal M}
\newcommand{\mlinterp}[1]{{\cal I}^{min}(#1)}

\renewcommand{\Th}{{\cal T}}

\title{HALO: Haskell to Logic through Denotational Semantics}
%% \subtitle{A new approach to static contract checking for higher-order lazy programs}

\authorinfo{Dan Ros\'{e}n \\ Koen Claessen}
           {Chalmers University}{}
\authorinfo{Dimitrios Vytiniotis \\ Simon Peyton Jones}
           {Microsoft Research}{}
\authorinfo{Nathan Collins}
           {Portland State University}{}
\maketitle
\makeatactive

\begin{abstract}
Despite the benefits of strong types and purity for reasoning about
programs, many bugs remain in well-typed and purely functional
code.  Programmers often introduce assertions to ensure that their
code is crash-free.  In this work we allow programmers to express
assertions as contracts, whose validity we can check statically.  Our main
contribution is a novel translation to first-order logic 
of both Haskell programs, and contracts written in Haskell,
all justified by denotational semantics.
\end{abstract}



\section{Introduction}\label{s:intro}
  Haskell programmers enjoy the benefits of strong static types and purity: 
static types eliminate many bugs early on in the development cycle, and purity 
simplifies equational reasoning about programs. Despite these benefits, however, 
bugs may still remain inside purely functional code and programs often
crash if applied to the wrong arguments. 

Consider this Haskell definition:
\begin{code}
  f xs = head (reverse (True : xs))
  g xs = head (reverse xs)
\end{code}
Both @f@ and @g@ are well typed (and hence do not ``go wrong'' in Milner's 
sense), but @g@ can crash (when applied to the empty list), whereas @f@ cannot.
To distinguish the two we need reasoning that goes well beyond 
that typically embodied in a standard type system.

One response to this challenge is to beef up the type system
path that leads to dependently typed programming.  Another alternative,
studied by many researchers~\cite{contracts-literature}, is to allow 
the programmer to annotate a function with one or more
\emph{contracts}, in addition to its normal type.
For example, we might write the following contract for @reverse@:
\begin{code}
  reverse ::: xs:CF -> { ys | null xs <=> null ys }
\end{code}
This asserts that if @reverse@ is applied to a crash-free (@CF@) argument list @xs@
then the result @ys@ will be empty (@null@) if and only if @xs@ is empty.
Notice that @null@ and @<=>@ are just ordinary Haskell functions, perhaps
written by the programmer, even though they appear inside contracts.
With this in hand we might hope to prove that @f@ satisfies the contract
\begin{code}
  f ::: CF -> CF
\end{code}
But how do we verify that @reverse@ and @f@ satisfy the claimed
contracts? Contracts are often tested dynamically~\cite{finder-felliesen}, but 
our plan here is different: we want to verify contracts \emph{statically} 
and \emph{automatically}. It should be clear that there is a good deal of logical reasoning to do,
and a now-popular approach is to hand off the task to a theorem
prover, a SMT solver or first-order logic prover.
With that in mind, we make the following new contributions:

\begin{itemize}
  \item We give a translation of Haskell programs to first-order logic (FOL) theories. 
        It turns out that lazy programs (as opposed to
        strict programs!) have a very natural translation into first-order logic.
        (Section~\ref{ssect:trans-fol}) 
  \item We give a translation of contracts to FOL formulae, and an axiomatization of 
        the language semantics in FOL. 
        (Section~\ref{s:contracts-fol})
  \item We show that if we can prove the formula that arises from a contract translation 
        for a given program, then the program does indeed satisfy this contract. Our proof
        uses the novel to our knowledge technique of using the denotational 
        semantics as a first-order model. (Section~\ref{ssect:denot})
  \item We show how to use this translation in practice for static contract checking with
        a SAT-solver (Section~\ref{sect:soundness}), 
        and how to prove goals by induction. (Section~\ref{sect:extensions})
\end{itemize}

We consider this work to be the first step towards practical contract checking 
for Haskell programs, that lays out the theoretical foundations for further engineering 
and experimentation. Nevertheless, we have already implemented a prototype for Haskell 
programs, that uses GHC as a front-end. We have evaluated the practicality of our approach 
on many examples, including lazy and higher-order programs, and we report this initial 
evaluation in Section~\ref{sect:implementation}.

To our knowledge no-one has considered the translation of lazy higher-order programs to 
first-order logic before in a provably sound way with respect to the denotational
semantics of programs. Furthemore, our approach to static contract checking is 
distinctively different to previous work: instead of wrapping and 
symbolic execution~\cite{xu+:contracts,Xu:2012:HCC:2103746.2103767}, 
we harness purity and laziness to directly use the denotational semantics
of programs and contracts and discharge the obligations with a SAT-solver.
We discuss similarities and differences compared to related work in Section~\ref{sect:related}.


%%   \item The translation
%% For this paper we focus on the translation, but to substantiate the practicality 
        
        
%% \end{itemize} 


%% \begin{itemize}
%% \item We show how to translate Haskell terms
%% into First Order Logic (FOL) (Section~\ref{ssect:trans-fol}).  
%% It may appear surprising that this 
%% is even possible, since Haskell is a higher order language.  Although
%% the basic idea of the translation is folklore in the community,
%% we believe that this paper is the first to explain it explicitly.

%% \item We also show how to translate \emph{contracts} into FOL
%%       (Section~\ref{s:contracts-fol}), 
%%       a translation that is rather less obvious.

%% \item We give a proof based on denotational semantics 
%% that if the FOL prover discharges a 
%% suitable theorem about the translated Haskell term and contract, 
%% then indeed the original Haskell term satisfies that contract (Section~\ref{s:xxx}).

%% %\item It is one thing to make a sound translation, and quite another
%% %to produce FOL terms that the FOL prover can actually prove anything
%% %about --- a common experience is that it goes out to lunch instead.  We
%% %describe a number of techniques that dramatically improve
%% %theorem-proving times, moving them from infeasible to feasible (Section\ref{xxx}).

%% \item For this paper we focus on the
%% translation, but we have also implemented a static contract checker
%% for Haskell itself, by using GHC as a front end.  We have evaluated
%% the practicality of this approach on many examples, including lazy and
%% higher-order programs, as we describe in Section~\ref{xxx}.  \spj{I'd like
%% to say something more substantial here.}
%% \end{itemize}





















% \section{Checking Haskell contracts in practice}\label{s:examples}
%   TODO

\subsection{A high-level overview of the system}\label{ssect:schematic}


\section{A higher-order lazy language and its contracts}\label{sect:language}
  \section{Syntax and semantics}
\label{s:intro}%
We will work with a polymorphic call-by-name calculus with algebraic data types, pattern matching, 
but where (i) polymorphism is only at top-level (a-la ML), (ii) functions are $\lambda$-lifted to 
the top-level and (iii) case expressions can only occur at top level. It is easy to convert source
Haskell to this intermediate form, which simplifies our translation and formal metatheory. The 
syntax of the language is given in Figure~\ref{fig:syntax}. 

We also define an erasure on $\Delta$ which gives $\Gamma$ environments, $\Delta^{-} = \Gamma$ below:
\[\begin{array}{lrll}
(\cdot)^{-}            & = & \cdot \\
(\Delta,a)^{-}         & = & \Delta^{-} \\
(\Delta,(x{:}\tau)^{-} & = & \Delta^{-},x 
\end{array}\]


\begin{figure}\small
\[\begin{array}{l} 
\begin{array}{lrll}
\multicolumn{3}{l}{\text{Types}} \\
\tau,\sigma & ::=  & T\;\taus & \text{Datatypes} \\ 
            & \mid & a \mid \tau -> \tau 
\end{array}\\ \\ 
\begin{array}{lrll}
\multicolumn{3}{l}{\text{Expressions}} \\
e  & ::=  & x            & \text{Variables} \\ 
   & \mid & f[\ol{\tau}] & \text{Function variables} \\ 
   & \mid & K[\ol{\tau}](\ol{e}) & \text{Data constructors} \\
   & \mid & e\;e         & \text{Applications} \\
   & \mid & @BAD@        & \text{Runtime error} \\ 
\end{array}\\ \\ 
\begin{array}{lrll}
\multicolumn{3}{l}{\text{Definitions}} \\
P   & ::= & @fix@\;(f_1^{\ar_1},...,f_k^{\ar_k})\;@=@ \\ 
    &     & \quad\quad\quad\quad (\Lambda\as_1 @.@ \lambda{\oln{x{:}\tau}{\ar_1}} @.@ u_1,...,
                                                \Lambda\as_k @.@ \lambda{\oln{x{:}\tau}{\ar_k}} @.@ u_k) \\
%% %% d   & ::= & f |-> \Lambda\ol{a} @.@ \lambda\ol{x{:}\tau} @.@ u \\ 
u   & ::= & e \mid @case@\;e\;@of@\;\ol{K\;\ol{y} -> e} \\ 
%%D   & ::= & \cdot \mid D,d \\
\end{array}\\ \\ 
\begin{array}{lrll}
\multicolumn{3}{l}{\text{Datatype declarations}} \\
dec & ::= & @data@\;T\;\ol{a} = \ol{K\;\taus}
\end{array}\\ \\ 
\begin{array}{lrll}
\multicolumn{3}{l}{\text{Type environments and signatures}} \\
\Gamma & ::=  & \cdot \mid \Gamma,x \\
\Delta & ::=  & \cdot \mid \Delta,a \mid \Delta,x{:}\tau \\
\Sigma & ::=  & \cdot \mid \Sigma,T{:}n \mid \Sigma,f{:}\forall\ol{a} @.@ \tau \mid \Sigma,K^{\ar}{:}\forall\ol{a} @.@ \oln{\tau}{\ar} -> @T@\;\as
\end{array}\\ \\
\begin{array}{lrll}
\multicolumn{3}{l}{\text{Auxiliary functions}} \\
constrs(\Delta,T) & = & \{ K \mid (K{:}\forall \as @.@ \taus -> T\;\as) \in \Delta \} \\
%% \tyar{D}{f} & = & n & \\ 
%%             & \multicolumn{3}{l}{\text{when}\; (f |-> \Lambda\oln{a}{n} @.@ \lambda\ol{x{:}\tau} @.@ u) \in D} \\
%% \tmar{D}{f} & = & n & \\ 
%%             & \multicolumn{3}{l}{\text{when}\; (f |-> \Lambda\ol{a} @.@ \lambda\oln{x{:}\tau}{n} @.@ u) \in D}
\end{array}\\ \\ 
\begin{array}{lrll}
\multicolumn{3}{l}{\text{Syntax of closed values}} \\
 v,w & ::= & K^\ar[\ol{\tau}](\oln{e}{\ar}) \mid f^\ar[\ol{\tau}]\;\oln{e}{m < \ar} \mid @BAD@ \\ \\ 
\end{array} \\
\begin{array}{lrll}
\multicolumn{3}{l}{\text{Typed contracts}} \\
\Ct & ::=  & \{ (x{:}\tau) \mid e \}    & \text{Base contracts} \\ 
  & \mid &  (x{:}\Ct_1) -> \Ct_2        & \text{Arrow contracts} \\ 
  & \mid & \Ct_1 \& \Ct_2               & \text{Conjunctions}    \\ 
  & \mid & \CF                          & \text{Crash-freedom}  \\
\end{array} \\
\end{array}\]
\caption{Syntax}\label{fig:syntax}
\end{figure}

%% \begin{figure}\small
%% \[\begin{array}{c} 
%% \ruleform{\Sigma;\Delta |- \tau } \\ \\ 
%% \prooftree
%%          a \in \Delta
%%         ----------------{TVar}
%%          \Sigma;\Delta |- a  
%%         ~~~~ 
%%         \Sigma;\Delta |- \tau_1 \quad \Sigma;\Delta |- \tau_2
%%         ---------------------{TArr}
%%         \Sigma;\Delta |- \tau_1 -> \tau_2 
%%         ~~~~~ 
%%         \begin{array}{c}
%%           (T{:}n) \in \Sigma  \quad
%%           \Sigma;\Delta |- \taus
%%         \end{array}
%%         -------------------------{TData}
%%         \Sigma;\Delta |- T\;\oln{\tau}{n} 
%% \endprooftree \\ \\ 
%% \ruleform{\Sigma;\Delta |- e : \tau} \\ \\
%% \prooftree
%%   (f{:}\forall\oln{a}{n} @.@ \tau) \in \Sigma \quad \Sigma;\Delta |- \taus 
%%   --------------------------------------{TFVar}
%%   \Sigma;\Delta |- f[\oln{\tau}{n}] : \tau[\taus/\as]
%%   ~~~~ 
%%   (x{:}\tau) \in \Delta 
%%   --------------------------------------{TVar}
%%   \Sigma;\Delta |- x : \tau
%%   ~~~~~ 
%%   (K{:}\forall\oln{a}{n} @.@ \taus -> T\;\as) \in \Sigma \quad
%%   \Sigma;\Delta |- \taus \quad
%%   \Sigma;\Delta |- \ol{e : \tau}
%%   --------------------------------------{TCon}
%%   \Sigma;\Delta |- K[\oln{\tau}{n}](\ol{e}) : T\;\taus
%%   ~~~~~
%%   \phantom{\Delta}
%%   --------------------------------------{TBad}
%%   \Sigma;\Delta |- @BAD@ : \tau
%%   ~~~~
%%   \Sigma;\Delta |- e_1 : \sigma -> \tau \;\;
%%   \Sigma;\Delta |- e_2 : \sigma
%%   --------------------------------------{TApp}
%%   \Sigma;\Delta |- e_1\;e_2 : \tau
%% \endprooftree \\ \\ 
%% \ruleform{\Sigma;\Delta |- u : \tau} \\ \\
%% \prooftree
%%    \Sigma;\Delta |- e : \tau
%%    ---------------------------{TUTm}
%%     \Sigma;\Delta |- e : \tau
%%    ~~~~~ 
%%   \begin{array}{l}
%%   \Sigma;\Delta |- e : T\;\oln{\tau}{k} \quad
%%   constrs(\Sigma,T) = \ol{K} \\
%%   \text{for each branch}\;(K\;\oln{y}{l} -> e') \\
%%   \begin{array}{l}
%%            (K{:}\forall \cs @.@ \oln{\sigma}{l}{->}T\;\oln{c}{k}) \in \Sigma \text{ and }
%%            \Sigma;\ol{a},\ol{x{:}\tau},\ol{y{:}\sigma[\taus/\cs]} |- e'{:}\tau
%%   \end{array}
%%   \end{array}
%%   ----------------------------------------{TUCase}
%%   \Sigma;\Delta |- @case@\;e\;@of@\;\ol{K\;\ol{y} -> e'}
%% \endprooftree \\ \\ 
%% \ruleform{ \Sigma |- P} \\ \\ 
%% \prooftree
%%    \begin{array}{l}
%%    \text{ for every} (f |-> \Lambda\oln{a}{n} @.@ \lambda\oln{x{:}\tau}{m} @.@ u) \in P \\
%%    \quad\quad (f{:}\forall\oln{a}{n} @.@ \oln{\tau}{m} -> \tau) \in \Sigma
%%    \text{ and }\Sigma;\as,\ol{x{:}\tau} |- u : \tau
%%    \end{array}
%%    ----------------------------------------------{TFDef}
%%    \Sigma |- P
%% \endprooftree 
%% \end{array}\]
%% \caption{Typing relation}\label{fig:typing}
%% \end{figure}

%% \begin{figure}\small
%% \[\begin{array}{c} 
%% \ruleform{D |- value(e)} \\ \\
%% \prooftree
%%    \begin{array}{c}
%%    \tmar{D}{f} > m
%%    \end{array}
%%    ------------------------------------------------{VFun}
%%     D |- value(f[\oln{\tau}{n}]\;\oln{e}{m})
%%    ~~~~~
%%    ------------------------------------------------{VCon}
%%     D |- value(K[\ol{\tau}](\ol{e}))
%%    ~~~~ 
%%    ------------------------------------------------{VBad}
%%    D |- value(@BAD@)
%% \endprooftree
%% \end{array}\]
%% \caption{Values}\label{fig:typing}
%% \end{figure}

\begin{figure*}\small
\[\begin{array}{c} 
\ruleform{P |- e \Downarrow v} \\ \\
\prooftree

-------------------------------------{EVal}
P |- v \Downarrow v
~~~~
\begin{array}{c}
(f |-> \Lambda\ol{a} @.@ \lambda\oln{x{:}\tau}{m} @.@ u) \in P \\
P |- u[\ol{\tau}/\ol{a}][\ol{e}/\ol{x}] \Downarrow v
\end{array}
-------------------------------------{EFun}
P |- f[\ol{\tau}]\;\oln{e}{m} \Downarrow v
~~~~
\begin{array}{c}  
P |- e_1 \Downarrow v_1 \quad
P |- v_1\;e_2 \Downarrow w
\end{array}
------------------------------------------------{EApp}
P |- e_1\;e_2 \Downarrow w
~~~~
\begin{array}{c}  
P |- e_1 \Downarrow @BAD@ 
\end{array}
------------------------------------------------{EBadApp}
P |- e_1\;e_2 \Downarrow @BAD@
\endprooftree \\ \\ 
\ruleform{P |- u \Downarrow v} \\ \\
\prooftree
P |- e \Downarrow v
-------------------------------------{EUTm}
P |- e \Downarrow v
~~~~ 
\begin{array}{c}
P |- e \Downarrow K_i[\ol{\sigma}_i](\ol{e}_i) \quad
P |- e'_i[\ol{e}_i/\ol{y}_i] \Downarrow w
\end{array}
------------------------------------{ECase}
P |- @case@\;e\;@of@\;\ol{K\;\ol{y} -> e'} \Downarrow w
~~~~
\begin{array}{c}
P |- e \Downarrow @BAD@ \\
\end{array}
------------------------------------{EBadCase}
P |- @case@\;e\;@of@\;\ol{K\;\ol{y} -> e'} \Downarrow @BAD@
%% \begin{array}{c}
%% (f |-> \Lambda\ol{a} @.@ \lambda\oln{x{:}\tau}{m} @.@ @case@\;e\;@of@\;\ol{K\;\ol{y} -> e'}) \in D \\
%% D |- e[\ol{\tau}/\ol{a}][\ol{e}/\ol{x}] \Downarrow @BAD@ \\
%% \end{array}
%% -------------------------------------{EBadCase}
%% D |- f[\ol{\tau}]\;\oln{e}{m} \Downarrow @BAD@
\endprooftree
\end{array}\]
\caption{Operational semantics}\label{fig:opsem}
\end{figure*}

\begin{lemma}[Subject reduction]
Assume $\Sigma |- P$ and $\Sigma;\cdot |- e : \tau$
If $P |- e \Downarrow w$ then $P |- value(w)$ and $\Sigma;\cdot |- w : \tau$.
\end{lemma}

\begin{lemma}[Value determinacy]
If $\Sigma;\cdot |- v : \tau$ and 
$\Sigma |- P$ and $P |- value(v)$ and $P |- v \Downarrow w$ then $ v = w $.
\end{lemma}

\begin{lemma}[Determinacy of evaluation]
If $\Sigma;\cdot |- e : \tau$ and 
$\Sigma |- P$ and $\Sigma;\cdot |- e \Downarrow v_1$ and $\Sigma;\cdot |- e \Downarrow v_2$ then
$v_1 = v_2$.
\end{lemma}

\begin{lemma}[Big-step soundness]
If $\Sigma;\cdot |- e : \tau$ and 
$\Sigma |- P$ and $\Sigma;\cdot |- e \Downarrow v$ then $P |- value(v)$ and $\Sigma;\cdot |- v : \tau$.
\end{lemma}




\subsection{Denotational semantics}\label{ssect:denot}

For a well-formed signature $\Sigma$, we define the strict bi-functor on cpos, below, 
assuming that $K_1\ldots K_k$ are all the constructors in $\Sigma$: 
\[\begin{array}{lclll}
  F_{\Sigma}(D^{-},D^{+}) & = & ( \quad{\prod_{\alpha_1}{D^{+}}} & K_1^{\ar_1} \in \Sigma \\
                      & + & \;\quad\ldots                    & \ldots \\
                      & + & \;\quad{\prod_{\alpha_k}{D^{+}}} & K_k^{\ar_k} \in \Sigma \\ 
                      & + & \;\quad(D^{-} =>_c D^{+}) \\
                      & + & \;\quad\unitcpo_{bad} \quad )_{\bot}
\end{array}\] 
The notation $\prod_{n}{D}$ abbreviates $n$-ary products ($\unitcpo$ if $n = 0$). The product
and sum constructions are standard, and not strict. The notation $C =>_c D$ denotes the cpo 
induced by the space of continus functions from the cpo $C$ to the cpo $D$. We use $\unitcpo_{bad}$ 
notation to simply denote a single-element cpo -- the $bad$ subscript is just there for readability. 

The notation $D_\bot$ is {\em lifting}, which is a monad, equipped with the following two continuous functions.
\[\begin{array}{l}
   \ret   : D =>_c D_\bot \\ 
   \bind_{f : D =>_c E_\bot} : D_\bot =>_c E_\bot
\end{array}\]
with the obvious definitions.

Moreover, the following continuous operations are defined:
\[\begin{array}{l} 
   \curry_{f : D\times E =>_c F} : D =>_c (E =>_C F) \\ 
   \eval : (E =>_c D)\times E =>_c D 
\end{array}\] 
for any cpos $D, E, F$.

\begin{lemma}\label{lem:rec-solution} 
There exists a solution to the domain-recursive equation induced by $F_{\Sigma}$, call it $D_{\infty}$.
Moreover, let a value-domain: $V_{\infty}$ be defined as:
    \[\begin{array}{ll}
             \quad\;{\prod_{\alpha_1}{D_{\infty}}} & K_1^{\ar_1} \in \Sigma \\
             \; + \;\ldots                    & \ldots \\
             \; + \;{\prod_{\alpha_k}{D_{\infty}}} & K_k^{\ar_k} \in \Sigma \\ 
             \; + \;(D_{\infty} =>_c D_{\infty}) \\
             \; + \;\unitcpo_{bad} \quad
    \end{array}\]
Then the following functions also exist, each being the inverse of the other (i.e. composing to the identity 
function on the appropriate cpo):
\[\begin{array}{l}
  \roll : (V_{\infty})_\bot =>_c D_{\infty} \\ 
  \unroll : D_{\infty} =>_c (V_{\infty})_\bot
\end{array}\] 
\end{lemma}

We summarize the (standard) construction needed for the proof of Lemma~\ref{lem:rec-solution} based on embedding-projection pairs, 
because some of its details will be useful later. Consider the chain of cpos $D_i$ defined as: 
\[\begin{array}{lcl}
   D_0 & = & \{\bot\} \\ 
   D_{i+1} & = & F_{\Sigma}(D_i,D_i)
\end{array}\]
and moreover consider the corresponding {\em embeddings} $e_i : D_i =>_c D_{i+1}$ and 
{\em projections} $p_i : D_{i+1} =>_c D_i$ defined as:
\[\begin{array}{lcl}
   e_0 & = & \dlambda d @.@ \bot_{D_1} \\
   p_0 & = & \dlambda d @.@ \bot_{D_0} \\
   e_{i+1} & = & F_{\Sigma}(p_i,e_i) \\
   p_{i+1} & = & F_{\Sigma}(e_i,p_i)
\end{array}\]
The following is an easy fact to prove:
\begin{lemma}
For every $i$ and $x$ element of $D_i$ we have $p_i\cdot e_i(x) = x$. For
every $y$ element of $D_{i+1}$ we have that $e_i\ cdot p_i(y) \sqsubseteq y$. 
\end{lemma}
Consider now the cpo defined by the carrier set 
   \[ \{ x \in \Pi_{i \in \omega}D_i \;\mid\; x_n = p_n(x_{n+1}) \} \] 
and the pointwise order induced by the order in each $D_i$, and $\bot$ element the 
infinite tuple of the corresponding $\bot$ elements. This cpo {\em is} going to be the 
set $D_{\infty}$. To prove this we need som more definitions. Let $j_{n,m} : D_n =>_c D_m$ 
be defined as:
\[\begin{array}{lcl}
   j_{n,m}(d) & = & \left\{\begin{array}{ll} 
                             e_{m-1}\cdot\ldots \cdot e_n(d) & n < m \\
                             d                        & n = m \\
                             p_{n-1}\cdot\ldots \cdot p_n(d) & n > m 
                          \end{array}\right.
\end{array}\]
and define $j_i : D_i =>_c D_\infty$ as:
\[\begin{array}{lcl}
   j_i(d) & = & \langle j_{i,0}(d),j_{i,1}(d),\ldots \\
\end{array}\]
We can easily show that $j_i$ and $\pi_i$ (the $i$-th projection from a tuple) form
an embedding-projection pair:
\begin{lemma}
For all $i$ and $x$ element of $D_{\infty}$ it is $j_i(\pi_i(x)) \sqsubseteq x$.
For all $y$ element of $D_{i}$ it is $\pi_i(j_i(y)) = y$.
\end{lemma}
The most important theorem is that the limit of $j_i\cdot\pi_i$ is the identity.
\begin{lemma}\label{lem:id-sqcup}
For all $x$ elements of $D_{\infty}$ it is $\sqcup(j_i\cdot\pi_i)(x) = x$.
\end{lemma}
\begin{proof} We have one direction by the previous lemma and least upper bounds.
So the hard direction is to show that $\sqcup(j_i\cdot\pi_i)(x) \sqsupseteq x$. 
We know that:
\[       \pi_n(j_n(\pi_n(d))) = \pi_n(d) \]
by unfolding definitions and we know that $\sqcup(j_n\cdot\pi_n) \sqsupseteq j_n\cdot\pi_n$
since it is a least upper bound. By monotonicity we then get that 
\[      \pi_n(\sqcup(j_n\cdot\pi_n)(d)) \sqsupseteq \pi_n(d) \] 
but that holds for ever $n$ which means that:
\[       \sqcup(j_n\cdot\pi_n)(d) \sqsupseteq d \] 
since $\sqsubseteq$ on $D_\infty$ is defined by the conjuction of the 
pointwise $\sqsubseteq$ on each $D_i$.
\end{proof}
Now let us consider the chain 
\[ F_{\Sigma}(D_0,D_0) \quad F_{\Sigma}(D_1,D_1) \quad \ldots \] 
and the pointed cpo $F_{\Sigma}(D_{\infty},D_{\infty})$ we have that:
\[\begin{array}{lcl}
    F_{\Sigma}(p_i,e_i)   & : & F(D_i,D_i) =>_c F(D_{i+1},D_{i+1}) \\ 
    F_{\Sigma}(\pi_i,j_i) & : & F(D_i,D_i) =>_c F(D_{\infty},D_{\infty})
\end{array}\] 
we will show that there exists an isomorphism $D_\infty \cong F_{\Sigma}(D_\infty,D_\infty)$.

\begin{lemma} Define functions $\roll = \sqcup(j_{i+1}\cdot F_{\Sigma}(j_i,\pi_i))$ and 
$\unroll = \sqcup (F_{\Sigma}(\pi_i,j_i)\cdot \pi_{i+1})$. They form the required isomorphism 
$D_\infty \cong F_{\Sigma}(D_\infty,D_\infty)$.
\end{lemma}
This lemma concludes the proof of Lemma~\ref{lem:rec-solution}.

The following fact will be extremely useful in establishing the existence of solutions
to recursive equations {\em over} the recursively defined domain via approximating the
denotations. Let us call $\rho_i = j_i\cdot\pi_i$. 

\begin{theorem}\label{lem:min-inv-reqs} The following are true:
\begin{itemize} 
   \item $\unroll \cdot \rho_{i+1} \cdot \roll = F_{\Sigma}(\rho_i,\rho_i)$ 
   \item $\sqcup\rho_i(d) = d$, for all elements $d$ of $D_{\infty}$.
\end{itemize}
\end{theorem}

\paragraph{Definability of application}
We may now {\em define} application $\dapp : D_\infty \times D_\infty =>_c D_\infty$ as: 
{\setlength{\arraycolsep}{2pt}
\[\begin{array}{rcll}
   \dapp & = & \multicolumn{2}{l}{\dlambda d @.@ \roll(\bind_g (\unroll (\pi_1(d))))} \\
   \text{ where } g & = &  [ & \bot : \prod_{\ar_1}{D_\infty} =>_c D_\infty =>_c (V_\infty)_\bot \\
                    &   &  , & \ldots \\
                    &   &  , & \bot : \prod_{\ar_k}{D_\infty} =>_c D_\infty =>_c (V_\infty)_\bot \\
                    &   &  , & \dlambda d' @.@ \unroll(d'(\pi_2(d))) \\
                    &   &  , & \dlambda b @.@ \dlambda d. \ret(\inj{bad}(b))\hspace{2pt} ] 
\end{array}\]}%
where we have used notation $\langle , \rangle$ to introduce pairs. and $[\ldots]$ to eliminate ($n$-ary) sums.
The projections $\pi_1$ and $\pi_2$ are the obvious continues projections from the binary product space of $D_{\infty}$. 
We use notation $\inj{K}$ to denote the continuous map that injects some $\alpha$-ary product of $D_{\infty}$ into the sum $V_{\infty}$.
We use notation $\inj{->}$ to denote the continuous injection of $(D_{\infty} =>_c D_{\infty})$ into $V_{\infty}$ and finally, 
$\inj{bad}$ for the unit injection into $V_{\infty}$.

We have now defined the domain-theoretic language and combinators that we will use. 
We proceed to give interpretations of terms and definitions. 

First, the denumerable set of term variable names $x_1,\ldots$ induces a discrete 
cpo $\VarCpo$  and the denumerable set of function variable names $f_1,\ldots$ induces a discrete 
cpo $\FVarCpo$. We define, {\em semantic term environments} to be the cpo $(\VarCpo =>_c D_{\infty})$, 
and {\em semantic function environments} to be the cpo $(\FVarCpo =>_c D_{\infty})$. 

Next we will define $[\![e]\!]$ as a continuous map: 
\[ 
    (\FVarCpo =>_c D_{\infty}) \times (\VarCpo =>_c D_{\infty}) =>_c D_{\infty}
\] 
Below, for a given term $e$ and semantic environments $\rho : \VarCpo =>_c D_{\infty}$ and 
$\sigma : \FVarCpo =>_c D_{\infty}$ we let: 
\[\begin{array}{rcll}
  \interp{x}{\sigma}{\rho} & = & \rho(x) \\ 
  \interp{f\;[\taus]}{\sigma}{\rho} & = & \sigma(f) \\
  \interp{K\;[\taus]\;(\ol{e})}{\sigma}{\rho} & = & \roll(\ret(\inj{K}(\langle\interp{\ol{e}}{\sigma}{\rho}\rangle))) \\ 
  \interp{e_1\;e_2}{\sigma}{\rho} & = & \dapp(\langle \interp{e_1}{\sigma}{\rho}, \interp{e_2}{\sigma}{\rho}\rangle) \\ 
  \interp{@BAD@}{\sigma}{\rho} & = & \roll(\ret(\inj{bad}1))
\end{array}\]
A small technical remark: we write this with pattern matching notation (instead of using $\pi_1$ for projecting 
out $\sigma$ and $\pi_2$ for projecting out $\rho$) but that is fine, since $\times$ is not a lifted construction. 

Proceeding in the same way for $[\![u]\!]$ we get:
\[\setlength{\arraycolsep}{2pt}
  \begin{array}{rcll}
  \interp{e}{\sigma}{\rho} & = & \multicolumn{2}{l}{\interp{e}{\sigma}{\rho}} \\ 
  \interp{@case@\;e\;@of@ \ol{ K\;\ys -> e_K}}{\sigma}{\rho} & = & \multicolumn{2}{l}{\roll(\bind_{g} (\unroll(\interp{e}{\sigma}{\rho})))} \\ \\ 
  \text{ where } g  & = & [ & h_{K_1} \\
                    &   & , & \ldots \\
                    &   & , & h_{K_k} \\
                    &   & , & \bot \\ 
                    &   & , & \dlambda b @.@ \ret(b) \hspace{2pt} ] \\ 
              h_{K} & =  & \multicolumn{2}{l}{\dlambda d @.@ \unroll(\interp{e_K}{\sigma}{\rho, \ol{y |-> \pi_i(d)}})} \\ 
                    &   & \multicolumn{2}{l}{\text{when } K \text{ is in the branches}} \\
              h_{K}  & = & \multicolumn{2}{l}{\bot } \\ 
                    &   & \multicolumn{2}{l}{\text{otherwise}}                  
\end{array}\]

Finally, for a definition $P$ we may define a continuous map:
\[ 
        [\![P]\!] : (\FVarCpo =>_c D_{\infty}) =>_c (\FVarCpo =>_c D_{\infty}) 
\]
Let $P$ be 
\[\begin{array}{l} 
     @fix@\;(f_1^{\ar_1},...,f_k^{\ar_k})\;@=@\; 
     (\lambda{\oln{x{:}\tau}{\ar_1}} @.@ u_1,...,
                   \Lambda\as_k @.@ \lambda{\oln{x{:}\tau}{\ar_k}} @.@ u_k) 
\end{array}\] 
Then
\[\begin{array}{l}  
   [\![P]\!]_{\sigma} f =  \\ 
     \qquad\text{ if } (f |-> \Lambda\as @.@ \lambda\oln{x:\tau}{n} @.@ u) \in P \\
     \qquad\text{ then } \\
     \qquad\quad\quad \roll(\ret(\inj{->}(\dlambda d_1 @.@ \ldots  \\
     \qquad\quad\quad\quad \roll(\ret(\inj{->}(\dlambda d_n @.@ \interp{u}{\sigma}{\ol{x |-> d}})))\ldots))) \\
     \qquad \text{ else } \bot
\end{array}\]

Since $[\![P]\!]$ is continuous, its limit exists and is an element of the cpo $\FVarCpo =>_c D_{\infty}$, 
we will call this $\dbrace{P}^{\infty}$ in what follows.


\begin{lemma}[Type irrelevance]
It is the case that $\interp{u}{\sigma}{\rho} = \interp{u[\ol{\tau}/\as]}{\sigma}{\rho}$ 
and $\interp{e}{\sigma}{\rho} = \interp{e[\ol{\tau}/\as]}{\sigma}{\rho}$.
\end{lemma}
\begin{proof} Straightforward induction. \end{proof}

\begin{lemma}[Substitutivity]
If $\Sigma;\Delta,x{:}\tau |- e : \tau$ and $\rho$ is a semantic environment 
and $\Sigma;\Delta |- e' : \tau'$ then 
\[ \interp{e}{\sigma}{\rho,x |-> \interp{e'}{\sigma}{\rho}} = \interp{e[e'/x]}{\sigma}{\rho} \]
and if $\Sigma;\Delta,x{:}\tau |- u : \tau$ then 
\[ \interp{u}{\sigma}{\rho,x |-> \interp{e'}{\sigma}{\rho}} = \interp{u[e'/x]}{\sigma}{\rho} \]
\end{lemma}

\begin{lemma}[Denotational semantics soundness]
Assume $\Sigma |- P$. 
\begin{itemize*} 
  \item If $\Sigma;\cdot |- e : \tau$ and $P |- e \Downarrow v$ then $\interp{e}{\dbrace{P}^{\infty}}{\cdot} = \interp{v}{\dbrace{P}^{\infty}}{\cdot}$.
  \item If $\Sigma;\cdot |- u : \tau$ and $P |- u \Downarrow v$ then $\interp{u}{\dbrace{P}^{\infty}}{\cdot} = \interp{v}{\dbrace{P}^{\infty}}{\cdot}$.
\end{itemize*} 
\end{lemma} 
\begin{proof} The two statements are proved simultaneously by induction on the height of the evaluation derivation, making use
of the type irrelevance lemma and substitutivity.
\end{proof}

Finally we are interested in proving the following lemma:

\begin{theorem}[Computational adequacy]
Assume $\Sigma |- P$ and $\Sigma;\cdot |- e : \tau$. 
If $\unroll(\interp{e}{\dbrace{P}^{\infty}}{\cdot}) = \ret(d)$ for some 
element $d$ of $V_{\infty}$ then there exists a $v$ such 
that $P |- e \Downarrow v$.
\end{theorem}

To do this we define a {\em logical relation} first between semantics 
and syntax. Let $Rel \subseteq D_\infty \times Expr$ be the space of 
{\em admissible} and {\em equality-respecting} relations between 
denotations and closed (non-necessarily well-typed) terms. Some explanations:
\begin{itemize}
  \item $R \in Rel$ is {\em admissible} iff whenever 
  $R(d_i,e)$ for every element of a chain $d_1\ldots$ then also $R(\sqcup_{\omega}d_i,e)$. 
  \item $R \in Rel$ is {\em equality-respecting} iff for every 
  $R(d,e)$ and $d' = d$ (according to the equality on $D_{\infty}$) it also is
  $R(d',e)$. 
\end{itemize}

Let use define the following bi-functor on the space of $Rel$ relations:
{\setlength{\arraycolsep}{2pt}
\[\begin{array}{lcl}
   F_{P}(R^{-},R^{+}) & = & \{ (d,e)\;\mid\;\forall \ol{d} @.@ \unroll(d) = \ret(\inj{K_1^\ar}\langle\oln{d}{\ar}\rangle) ==> \\
                   &   & \quad \exists \oln{e}{\ar} @.@ P |- e \Downarrow K_1[\taus](\ol{e}) \land (d_i,e_i) \in R^{+} \} \\ 
                   & \cup & \ldots \\ 
                   & \cup & \{ (d,e)\;\mid\;\forall d_0 @.@ \unroll(d) = \ret(\inj{->}(d_0)) ==> \\ 
                   &   & \quad \exists v @.@ P |- e \Downarrow v \;\land \\ 
                   &  & \quad\quad \forall (d',e') \in R^{-} @.@ (\dapp(d,d'),v\;e') \in R^{+} \}  \\
                   & \cup & \{ (d,e)\;\mid\; \unroll(d) = \ret(\inj{bad}(1)) ==> \\ 
                   &   & \quad P |- e \Downarrow @BAD@ \} 
\end{array}\]}

\begin{lemma} There exists negative and a positive fixpoint of $F_{P}$ and they coincide: let us call this
$F_{P}^\infty$ -- it is isomorphic to $F_{P}(F_P^\infty,F_P^\infty)$.
\end{lemma}
\begin{proof}
We can follow the standard roadmap described in the work of Pitts to show this, taking 
advantage of the approximation on every element of $D_{\infty}$ given in 
Lemma~\ref{lem:min-inv-reqs}.
\end{proof}

\begin{lemma}\label{lem:bot-in-fix}
$(\roll(\bot),e) \in F_{P}^\infty$. \end{lemma}

\begin{lemma}\label{lem:eval-respecting}
If $(d,e) \in F_{P}^\infty$ and $P |- e \Downarrow v$ then $(d,v) \in F_{P}^\infty$.
Moreover, if $(d,v) \in F_{P}^\infty$ and $P |- e \Downarrow v$ then $(d,e) \in F_{P}^\infty$.
\end{lemma}
\begin{proof}
For the first part, 
if $(d,e) \in F_{P}^\infty$ then $(d,e) \in F_{P}(F_{P}^\infty,F_{P}^\infty)$. 
By the definition of $F_{P}(\cdot,\cdot)$ and by rule \rulename{EVal} the 
result follows. The second part is a similar case analysis.
\end{proof}

\begin{lemma}[Fundamental theorem for expressions]\label{lem:fund-thm-exp}
For all $\sigma$ such that $(\sigma(f),f\;[\taus]) \in F_P^\infty$ and 
all $\rho$ and vectors of closed terms $\ol{e}$ such that $(\rho(x_i),e_i) \in F_P^\infty$ 
and all $e$ with free variables in $\ol{x}$ it must be the case 
that $(\interp{e}{\sigma}{\rho},e[\ol{e}/\ol{x}]) \in F_P^\infty$.
\end{lemma}
\begin{proof} The proof is by induction on $e$. 
\begin{itemize}
  \item Case $e = x_i$ for some $x_i \in \ol{x}$ follows by the assumptions.
  \item Case $e = f\;[\taus]$ for some $f$ follows by assumptions.
  \item Case $e = K^\ar[\taus](\oln{e'}{\ar})$. By induction hypothesis we 
  have that for each $e'_i$ it is $(\interp{e'_i}{\sigma}{\rho},e'_i[\ol{e}/\ol{x}]) \in F_P^\infty$ and 
  by using rule \rulename{EVal} we are done since 
      \[ \interp{K^{\ar}[\taus](\oln{e'}{\ar})}{\sigma}{\rho} = \roll(\ret(\inj{K}(\langle\ol{\interp{e'_i}{\sigma}{\rho}}\rangle))) \]
  \item Case $e = @BAD@$ follows by unfolding definitions.
  \item Case $e = e_1\;e_2$. We need to show that
     \[ (\interp{e_1\;e_2}{\sigma}{\rho},e_1[\ol{e}/\ol{x}]\;e_2[\ol{e}/\ol{x}]) \in F_P^\infty \] 
  By induction hypothesis we have that 
  \begin{eqnarray}
     (\interp{e_1}{\sigma}{\rho},e_1[\ol{e}/\ol{x}]) \in F_P^\infty \label{eqn:e1} \\ 
     (\interp{e_2}{\sigma}{\rho},e_2[\ol{e}/\ol{x}]) \in F_P^\infty \label{eqn:e2}
  \end{eqnarray}
  Equation~\ref{eqn:e1} gives four cases: First, if $\interp{e_1}{\sigma}{\rho} = \roll(\bot)$ then we 
  are done since $\dapp(\bot,\_) = \roll(\bot)$ and $(\roll(\bot), e_1\;e_2) \in F_P^\infty$ by Lemma~\ref{lem:bot-in-fix}.
  Second, if $\interp{e_1}{\sigma}{\rho} = \roll(\ret(\inj{K}(\langle\ol{d}\rangle)))$ for some constructor
  $K$ then $\dapp(\interp{e_1}{\sigma}{\rho},\_) = \roll(\bot)$ and by similar reasoning as above we are done.
  Third, if $\interp{e_1}{\sigma}{\rho} = \roll(\ret(\inj{bad}(1)))$ then it must be that $P |- e_1[\ol{e}/\ol{x}] \Downarrow @BAD@$ by
  induction hypothesis, and by rule \rulename{EBadApp} we know that $P |- e_1[\ol{e}/\ol{x}]\;e_2[\ol{e}/\ol{x}] \Downarrow @BAD@$ hence, 
  by Lemma~\ref{lem:eval-respecting} we are done. The final case is the interesting one, where
  $\interp{e_1}{\sigma}{\rho} = \roll(\ret(\inj{->}(d_0)))$ in which case by induction hypothesis we know that 
  $(d_0(\interp{e_2}{\sigma}{\rho}), v\;e_2[\ol{e}/\ol{x}]) \in F_P^\infty$ for $P |- e_1[\ol{e}/\ol{x}] \Downarrow v$. But we know that 
  $v\;e_2[\ol{e}/\ol{x}]$ evaluates to a value {\em iff} $e_1[\ol{e}/\ol{x}]\;e_2[\ol{e}/\ol{x}]$ evaluates to a value and by 
  Lemma~\ref{lem:eval-respecting} we are done.
\end{itemize}
\end{proof}


\begin{lemma}[Fundamental theorem for top-level expressions]\label{lem:fund-thm-case}
For all $\sigma$ such that $(\sigma(f),f\;[\taus]) \in F_P^\infty$ and 
all $\rho$ and vectors of closed terms $\ol{e}$ such that $(\rho(x_i),e_i) \in F_P^\infty$ 
and all $u$ with free variables in $\ol{x}$ it must be the case 
that $(\interp{u}{\sigma}{\rho},u[\ol{e}/\ol{x}]) \in F_P^\infty$.
\end{lemma}
\begin{proof} By induction on $u$. If $u$ is a term $e$ then we are immediately done
by Lemma~\ref{lem:fund-thm-exp}. If $u = @case@\;e\;@of@\;\ol{K\;\ys -> e'}$ then the 
result follows by appealing to the induction hypothesis for $e$ and performing a case 
analysis on $\interp{e}{\sigma}{\rho}$ -- in the interesting case we appeal further to
Lemma~\ref{lem:fund-thm-exp} for a matching $e_K$ and the evaluation-respecting lemma, 
Lemma~\ref{lem:eval-respecting}.
\end{proof}

Finally, for the recursive functions environment $P$ we prove the following.
\begin{lemma} For any $f$, $(\dbrace{P}^\infty(f),f\;[\taus]) \in F_P^\infty$. \end{lemma}
\begin{proof}
Since $F_P^\infty$ is itself an {\em admissible} relation, we need only prove that:
\[ \forall i @.@ \forall f @.@ (\dbrace{P}^i(f),f\;[\taus]) \in F_P^\infty \] 
which we do by induction on $i$. For $i = 0$ we are immediately done by Lemma~\ref{lem:bot-in-fix}.
Let us assume that the property is true for $i$. We must show it is true for $i+1$. That is, 
we must show that $(\dbrace{P}^{i+1}(f),f\;[\taus]) \in F_P^\infty$. Hence, if $f$ has arity $n$, by
the definition of $\dbrace{P}$ and the definition of the logical relation it is enough to show that
for all $(\oln{d}{n},\oln{e}{n}) \in F_P^\infty$ it must be the case that 
\[    (\interp{u}{\dbrace{P}^i}{\ol{x |-> d}}, u[\ol{e}/\ol{x}]) \in F_P^\infty \] 
for $f |-> (\Lambda\as @.@ \lambda\oln{x{:}\tau}{n} @.@ u) \in P$. But that follows by 
Lemma~\ref{lem:fund-thm-case}, since by induction hypothesis it is the case that for 
every $f$ we have $(\dbrace{P}^i(f),f\;[\ol{\tau}]) \in F_P^\infty$.
\end{proof} 

\begin{corollary}\label{cor:fund-thm-top}
For every closed expression $e$ in $P$ (not-necessarily well-typed) we have that 
$(\interp{e}{\dbrace{P}^\infty}{\cdot}, e) \in F_P^\infty$. 
\end{corollary}

From this corollary, adequacy follows by unfolding definitions.

\begin{corollary}[Model-based-reasoning]
If $\Sigma |- P$ and $\Sigma;\cdot |- e_1 : \tau$ and $\Sigma;\cdot |- e_2 : \tau$, 
then for every closed $e$ such that $\Sigma |- P$ and $\Sigma;\cdot |- e : \tau -> Bool$,
if $\interp{e_1}{\dbrace{P}^\infty}{\cdot} = \interp{e_2}{\dbrace{P}^\infty}{\cdot}$ then  
$P |- e\;e_1 \Downarrow$ iff $P |- e\;e_2 \Downarrow$. 
\end{corollary}
\begin{proof}
For one direction assume that $P |- e\;e_1 \Downarrow w$, hence by computational soundness it must be that 
$\interp{e\;e_1}{\dbrace{P}^\infty}{\cdot} = \roll(\ret(d))$. By assumptions we must also 
have that $\interp{e\;e_2}{\dbrace{P}^\infty}{\cdot} = \roll(\ret(d))$. By the fundamental theorem 
we know that 
\[ (\interp{e\;e_2}{\dbrace{P}^\infty}{\cdot}, e\;e_2) \in F_{P}^\infty \] 
and hence $P |- e\;e_2 \Downarrow$. The other direction is symmetric.
\end{proof}

\subsection{Denotational semantics in first-order logic}\label{ssect:denot-fol}

\begin{figure}
\[\begin{array}{c} 
\begin{array}{lrll}
\multicolumn{3}{l}{\text{Terms}} \\
  s,t & ::=  & f(\ol{t}) \mid K(\ol{t}) \mid x & \text{Functions, constructors, vars} \\
      & \mid & \sel{K}{i}                      & \text{Constructor selectors} \\ 
      & \mid & app(t,s)                        & \text{Application} \\
      & \mid & \unr \mid \bad                  & \text{Unreachable, bad} \\ \\ 
\multicolumn{3}{l}{\text{Formulae}} \\ 
 \phi & ::=  & \lcf{t}    & \text{Provably cannot cause crash} \\
%%      & \mid & \lncf{t}   & \text{Can provably cause crash} \\
      & \mid & t_1 = t_2  & \text{Equality} \\ 
      & \mid & \phi \land \phi \mid \phi \lor \phi \mid \neg \phi \\
      & \mid & \forall x @.@ \phi \mid \exists x @.@ \phi \\ \\ 
\multicolumn{3}{l}{\text{Abbreviations}} \\ 
\multicolumn{4}{l}{app(t,\oln{s}{n}) = (\ldots(app(t,s_1),\ldots s_n)\ldots)}
\end{array}
\end{array}\]
\caption{Image of translation into FOL}\label{fig:fol-image}
\end{figure}


\begin{figure}\small
\[\begin{array}{c} 
\ruleform{\etrans{\Sigma}{\Gamma}{e} = \formula{t} } \\ \\
\prooftree
  \begin{array}{c}
  (f{:}\forall\oln{a}{n} @.@ \tau) \in \Sigma
  \end{array}
  --------------------------------------{TFVar}
  \etrans{\Sigma}{\Gamma}{f[\oln{\tau}{n}]} = \formula{f_{ptr}}
  ~~~~ 
  x \in \Gamma 
  --------------------------------------{TVar}
  \etrans{\Sigma}{\Gamma}{x} = \formula{x}
  ~~~~~ 
  \begin{array}{c}
  (K{:}\forall\oln{a}{n} @.@ \ol{\tau} -> T\;\as) \in \Sigma \quad
  \ol{\etrans{\Sigma}{\Gamma}{e} = \formula{t}}
  \end{array}
  --------------------------------------{TCon}
  \etrans{\Sigma}{\Gamma}{K[\oln{\tau}{n}](\ol{e})} = \formula{K(\ol{t})}
  ~~~~~
  \phantom{\Gamma}
  --------------------------------------{TBad}
  \etrans{\Sigma}{\Gamma}{@BAD@} = \formula{\bad}
  ~~~~
  \etrans{\Sigma}{\Gamma}{e_1} = \formula{t_1} \quad 
  \etrans{\Sigma}{\Gamma}{e_2} = \formula{t_2}
  --------------------------------------{TApp}
  \etrans{\Sigma}{\Gamma}{e_1\;e_2} = \formula{app(t_1,t_2)}
\endprooftree \\ \\ 
\ruleform{\utrans{\Sigma}{\Gamma}{t \sim u} = \formula{\phi}} \\ \\ 
\prooftree
   \etrans{\Sigma}{\Gamma}{e} = \formula{t}
   ----------------------------------------{TUTm}
   \utrans{\Sigma}{\Gamma}{s \sim e } = \formula{s = t} 
   ~~~~~
  \begin{array}{l}
  \etrans{\Sigma}{\Gamma}{e} = \formula{t} \quad
  constrs(\Sigma,T) = \ol{K} \\
  \text{for each branch}\;(K\;\oln{y}{l} -> e') \\
  \begin{array}{l}
           (K{:}\forall \cs @.@ \oln{\sigma}{l} -> T\;\oln{c}{k}) \in \Sigma \text{ and }
           \etrans{\Sigma}{\Gamma,\ol{y}}{e'} = \formula{ t_K }
  \end{array}
  \end{array}
  ------------------------------------------{TUCase}
  {  \setlength{\arraycolsep}{2pt} 
  \begin{array}{l}
  \utrans{\Sigma}{\Gamma}{s \sim @case@\;e\;@of@\;\ol{K\;\ol{y} -> e'}} = \\
  \;\;\formula{ \begin{array}{l} 
          (t = \bad => s = bad)\;\land \\ 
          (\forall \ol{y} @.@ t = K_1(\ol{y}) => s = t_{K+1}\;\land \ldots \land \\
          (t \neq \bad\;\land\;t \neq K_1(\oln{{\sel{K_1}{i}}(t)}{})\;\land\;\ldots => s = \unr) 
%% (t = \bad /\ s = \bad)\;\lor\;(s = \unr)\;\lor \\
%%                                 \quad      \bigvee(t = K(\oln{{\sel{K}{i}}(t)}{}) \land
%%                                            s = t_K[\oln{\sel{K}{i}(t)}{}/\ol{y}])
                   \end{array}
           }
  \end{array}}
\endprooftree \\ \\ 
\ruleform{ \Dtrans{\Sigma}{P} = \formula{\phi}} \\ \\ 
\prooftree
     \begin{array}{l}       
       \text{for each} (f |-> \Lambda\oln{a}{n} @.@ \lambda\oln{x{:}\tau}{m} @.@ u) \in P \\ 
          \quad \utrans{\Sigma}{\ol{x}}{f(\ol{x}) \sim u} = \formula{\phi}
     \end{array}
     --------------------{TDefs}
     \Dtrans{\Sigma}{P} = \bigwedge_{P} \formula{\forall \ol{x} @.@ \phi}
\endprooftree 

\end{array}\]
\caption{Well-scoped elaboration to FOL terms}\label{fig:etrans}
\end{figure}


In addition to contract translation we have an axiomatization in Figure~\ref{fig:prelude}.


\begin{figure}\small
\[\begin{array}{c}
\ruleform{\Th{\Sigma}{P}} \\ \\ 
\begin{array}{lll} 
 \textsc{AxDisjA} & \formula{\bad \neq \unr}  \\ 
 \textsc{AxDisjB} & \formula{\forall \oln{x}{n}\oln{y}{m} @.@ K(\ol{x}) \neq J(\ol{y})} \\ 
                  & \text{ for every } (K{:}\forall\as @.@ \oln{\tau}{n} -> T\;\as) \in \Sigma \\ 
                  & \text{ and } (J{:}\forall\as @.@ \oln{\tau}{m} -> S\;\as) \in \Sigma \\
 \textsc{AxDisjC} & \formula{(\forall \oln{x}{n} @.@ K(\ol{x}) \neq \unr \land K(\ol{x}) \neq \bad)} \\ 
                  & \text{ for every } (K{:}\forall\as @.@ \oln{\tau}{n} -> T\;\as) \in \Sigma \\ \\
 \textsc{AxAppA}  & \formula{\forall \oln{x}{n} @.@ f(\ol{x}) = app(f_{ptr},\xs)} \\
                  & \text{ for every } (f |-> \Lambda\as @.@ \lambda\oln{x{:}\tau}{n} @.@ u) \in P \\
 %% \textsc{AxAppB}  & \formula{\forall \oln{x}{n} @.@ K(\ol{x}) = app(\ldots (app(x_K,x_1),\ldots,x_n)\ldots)} \\
 %%                  & \text{ for every } (K{:}\forall\as @.@ \oln{\tau}{n} -> T\;\as) \in \Sigma \\
 \textsc{AxAppC}  & \formula{\forall x, app(\bad,x) = \bad \; /\ \; app(\unr,x) = \unr}    \\ \\
 %% Not needed: we can always extend partial constructor applications to fully saturated and use AxAppC and AxDisjC
 %% \textsc{AxPartA} & \formula{\forall \oln{x}{n} @.@ app(\ldots (app(x_K,x_1),\ldots,x_n)\ldots) \neq \unr} \\
 %%                  & \formula{\quad\quad \land\; app(\ldots (app(x_K,x_1),\ldots,x_n)\ldots) \neq \bad} \\
 %%                  & \text{ for every } (K{:}\forall\as @.@ \oln{\tau}{m} -> T\;\as) \in \Sigma \text{ and } m > n \\
 \textsc{AxInj}   & \formula{\forall \oln{y}{n} @.@ \sel{K}{i}(K(\ys)) = y_i} \\ 
                  & \text{for every } (K{:}\forall\as @.@ \oln{\tau}{n} -> T\;\as) \in \Sigma \text{ and } i \in 1..n \\ \\
\end{array} \\
\ruleform{\Th{\Sigma}{P}^{\lcfZ}} \\ \\ 
\begin{array}{lll} 
 \textsc{AxCfA}   & \formula{\lcf{\unr} /\ \lncf{\bad}} \\
 \textsc{AxCfB}   & \formula{\forall \oln{x}{n} @.@ \lcf{K(\ol{x})} <=> \bigwedge\lcf{\ol{x}}} \\
                  & \text{ for every } (K{:}\forall\as @.@ \oln{\tau}{n} -> T\;\as) \in \Sigma
\end{array}
\end{array}\]
\caption{Prelude theory}\label{fig:prelude}
\end{figure}


For a given program $P$ and expression $e$, we will view the denotational 
semantics $\dbrace{P}^{\infty}$ and $\interp{e}{\dbrace{P}^\infty}{\cdot}$ as first-order 
logic {\em models}. We use the elements of $D_{\infty}$ as the underlying set. We define
an interpretation of functions on the image of translation into FOL as: 
\[\begin{array}{rcl}
   \linterp{f^{\ar}}  & = & \dlambda (d {:} \prod_{\ar}D_{\infty}) @.@ \dapp(\dbrace{P}^{\infty}(f),\oln{\pi_i(d)}{i \in 1..\ar}) \\ 
   \linterp{f_{ptr}}  & = & \dbrace{P}^{\infty}(f) \\ 
   \linterp{app}     & = & \dapp \\ 
   \linterp{K^{\ar}}  & = & \dlambda (d {:} \prod_{\ar}D_{\infty}) @.@ \roll(\ret(\inj{K}(d))) \\ 
   \linterp{\sel{K}{i}} & = & \dlambda (d {:} D_{\infty}) @.@ \roll(\bind_g(\unroll(d))) \\ 
     \text{where } g  & = & [\;\bot \\ 
                      &   & ,\;\dlambda d @.@ \unroll(\pi_i(d))  \quad (\text{case for constructor } K) \\ 
                      &   & ,\;\bot \\ 
                      &   & ,\;\ldots\\ 
                      &   & ,\;\bot\; ]
\end{array}\]

\begin{theorem}
If $\Sigma |- P$ then we have that $\langle D_{\infty},{\cal I}\rangle \models \Th{\Sigma}{P}$. 
\end{theorem} 
\begin{proof} It is straightforward to confirm that every axiom in $\Th{\Sigma}{P}$ is true in the model. \end{proof}


Notice that there exist more axioms that are validated by the denotational model. For instance 
\[\begin{array}{l}
    \formula{\forall \oln{x}{n} @.@ app(f_{ptr},\xs) \neq \unr} \\
    \formula{\quad\land\; app(f_{ptr},\xs) \neq \bad} \\
    \formula{\quad\land\; \forall \oln{y}{k} @.@ app(f_{ptr},\xs) \neq K(\ol{y})} \\
    \text{ for every } (f |-> \Lambda\as @.@ \lambda\oln{x{:}\tau}{m} @.@ u) \in P  \\
    \text{ and every } (K{:}\forall\as @.@ \oln{\tau}{k} -> T\;\as) \in \Sigma \text{ and } m > n 
\end{array}\]
asserts that closures cannot be equated to any constructor, $\bot$ nor $\inj{@BAD@}(1)$.

It is straightforward unfolding of definitions to show that:

\begin{theorem}
If $\Sigma |- P$ then $\langle D_{\infty},{\cal I}\rangle \models \dtrans{\Sigma}{P}$. 
\end{theorem}


\section{Proving soundness through denotational semantics}
   \label{sect:contracts}\label{ssect:denot}
  \begin{figure}\small
\[\begin{array}{c}
\ruleform{\Th{\Sigma}{P}^{\lcfZ}} \\ \\ 
\begin{array}{lll} 
 \textsc{AxCfA}   & \formula{\lcf{\unr} /\ \lncf{\bad}} \\
 \textsc{AxCfB}   & \formula{\forall \oln{x}{n} @.@ \lcf{K(\ol{x})} <=> \bigwedge\lcf{\ol{x}}} \\
                  & \text{ for every } (K{:}\forall\as @.@ \oln{\tau}{n} -> T\;\as) \in \Sigma
\end{array}
\end{array}\]
\caption{Prelude theory}\label{fig:prelude}
\end{figure}


To the extend that in the end we are only interested in base contracts, giving a 
denotational semantics of full-higher-order contracts is not really interesting 
but we do this anyway. For a given denotation $d$, we define the 
predicate $\interp{\Ct}{\dbrace{P}^\infty}{\rho}(d)$ by recursion on the structure 
of the contract $\Ct$, such that:

\[\begin{array}{l}
    \interp{\{x \mid e\}}{\dbrace{P}^\infty}{\rho}(d) \text{ iff } \\
        \quad \unroll(d) = \bot \text{ or } 
        \unroll(\interp{e}{\dbrace{P}^\infty}{\rho,x|->d}) = \bot\;\text{ or } \\
        \quad \unroll(\interp{e}{\dbrace{P}^\infty}{\rho,x|->d}) = \ret(\inj{\mathop{True}}(1)) \\ \\
    \interp{\dbrace{(x{:}\Ct_1) -> \Ct_2}{P}^\infty}{\rho}(d) \text{ iff } \\
        \quad \text{for all } d_x \in D_\infty \\ 
        \quad\quad \text{if }
                     \interp{\Ct_1}{\dbrace{P}^\infty}{\rho}(d_x)\text{ then }
                     \interp{\Ct_2}{\dbrace{P}^\infty}{\rho,x|->d_x}(\dapp(d,d_x)) \\ \\ 
    \interp{\CF}{\dbrace{P}^\infty}{\rho}(d) \text{ iff }  d \in \Fcf^{\infty} \\  \\ 
    \interp{\Ct_1 \& \Ct_2}{\dbrace{P}^\infty}{\rho}(d) \text{ iff } 
       \interp{\Ct_1}{\dbrace{P}^\infty}{\rho}(d) \text{ and } 
       \interp{\Ct_2}{\dbrace{P}^\infty}{\rho}(d)
\end{array}\]

\subsection{Contracts in first-order logic}\label{sect:contracts-fol}

\begin{figure}\small
\[\begin{array}{c} 
\ruleform{\ctrans{\Sigma}{\Gamma}{e \in \Ct} = \formula{\phi}} \\ \\ 
\prooftree
  \begin{array}{c}
   \etrans{\Sigma}{\Gamma}{e} = \formula{t} \quad
   \etrans{\Sigma}{\Gamma,x}{e'} = \formula{t'}
  \end{array}
  ------------------------------------------{CTransBase}
  \begin{array}{l}
   \ctrans{\Sigma}{\Gamma}{e \in \{(x{:}\tau) \mid e' \}} = \\
  %% \Sigma;\Gamma |- e \in \{(x{:}\tau \mid e' \}
  \;\;\formula{(t = \unr) \lor (t'[t/x] = \unr) \lor (t'[t/x] = \True)}
  \end{array}
  ~~~~~ 
  \begin{array}{c}
  \ctrans{\Sigma}{\Gamma,x}{x \in \Ct_1} {=} \formula{\phi_1} \quad
  \ctrans{\Sigma}{\Gamma,x}{e\;x \in \Ct_2} {=} \formula{\phi_2}
  \end{array} 
  ------------------------------------------{CTransArr}
  \begin{array}{l} 
  \ctrans{\Sigma}{\Gamma}{e \in (x{:}\Ct_1) -> \Ct_2} = 
  \formula{\forall x @.@ \neg \phi_1 \lor \phi_2} 
  \end{array}
  ~~~~~
  \begin{array}{c}
  \ctrans{\Sigma}{\Gamma}{e \in \Ct_1} = \formula{ \phi_1} \quad
  \ctrans{\Sigma}{\Gamma}{e \in \Ct_2} = \formula{ \phi_2}
  \end{array}
  ------------------------------------------{CTransConj}
  \ctrans{\Sigma}{\Gamma}{e \in \Ct_1 \& \Ct_2} = \formula{ \phi_1 /\ \phi_2}
  ~~~~~
  \etrans{\Sigma}{\Gamma}{e} =  \formula{t}
  -------------------------------------------{CTransCf}
  \ctrans{\Sigma}{\Gamma}{e \in \CF} = \formula{\lcf{t}}
 \endprooftree 
%% \\ \\ 
%% \ruleform{\Sigma;\Gamma |- e \notin \Ct \elab{ \phi} } \\ \\
%% \prooftree
%%   \begin{array}{c}
%%    \Sigma;\Gamma |- e : \tau \elab{ t}  \quad
%%    \Sigma;\Gamma,(x{:}\tau) |- e' : \Bool \elab{ t'}
%%   \end{array}
%%   ------------------------------------------{CNTransBase}
%%   \begin{array}{l}
%%   \Sigma;\Gamma |- e \notin \{(x{:}\tau) \mid e' \} 
%%   \elab{(t'[t/x] = \bad) \lor (t'[t/x] = \False)}
%%   \end{array}
%%   ~~~~~ 
%%   \Sigma;\cdot |- \Ct_1 : \tau
%%   ------------------------------------------{CNTransArr}
%%   \begin{array}{l} 
%%   \Sigma;\Gamma |- e \notin (x{:}\Ct_1) -> \Ct_2 
%%   \elab{\exists x @.@ (\Sigma;\Gamma,(x{:}\tau) |- x \in \Ct_1) /\ (\Sigma;\Gamma,(x{:}\tau) |- e\;x \notin \Ct_2)}
%%   \end{array}
%%   ~~~~~
%%   \begin{array}{c}
%%   \Sigma;\Gamma |- e \notin \Ct_1 \elab{ \phi_1} \quad
%%   \Sigma;\Gamma |- e \notin \Ct_2 \elab{ \phi_2}
%%   \end{array}
%%   ------------------------------------------{CNTransConj}
%%   \Sigma;\Gamma |- e \notin \Ct_1 \& \Ct_2 \elab{ \phi_1 \lor \phi_2}
%%   ~~~~
%%   \Sigma;\Gamma |- e : \tau \elab{ t}
%%   -------------------------------------------{CNTransCf}
%%   \Sigma;\Gamma |- e \notin \CF \elab{ \lncf{t}}
%%  \endprooftree
\end{array}\]
\caption{Baseline contract elaboration}\label{fig:typing}
\end{figure}


In this section we will attempt to ignore the higher-order case and just talk about 
base contracts. Let us use:

The following are true: 
\begin{lemma}[Base contract adequacy]\label{lem:base-contract-adequacy}
Assume that $\Sigma |- P$ and $fv(e) \subseteq dom(P)$, i.e. $e$ is closed.
If $\langle D_\infty,{\cal I}\rangle \models \ctrans{\Sigma}{\Delta}{e \in \CF}$ then $P |- e \in \CF$. If $\langle D_\infty,{\cal I}\rangle \models \ctrans{\Sigma}{\Delta}{e \in \{ x \mid e' \}}$ then $P |- e \in \{x \mid e' \}$.
\end{lemma}
{\bf DV: Generalize this to a notion of base contracts that includes conjuctions.}

In fact the above two statements hold if we extend the interpretation 
of crash-freedom in the model to contain elements from the function 
space as well. 

Because of the full-abstraction problems we have observed above it 
is not possible to state similar statements for arrow contracts. 



\subsection{Soundness of contract checking}\label{ssect:soundness}


\subsubsection{Invocation of a theorem prover}\label{sect:infocation}

Given a program $P$ with signature $\Sigma$, that is $\Sigma |- P$, we may define the theory
${\cal T}$ as follows:
     \[ \Th{\Sigma}{P}\;\land\;\Th{\Sigma}{P}^{\lcfZ}\;\land\;\dtrans{\Sigma}{P} \]
we know that $\langle D_\infty,{\cal I}\rangle \models {\cal T}$ from the previous sections. 
Assume below that $f$ is a function such that $f \in dom(P)$ and $fv(\Ct) \subseteq dom(P)$.

\begin{theorem}[Soundness]\label{thm:prover-soundess}
If ${\cal T}\;\land\;\neg(\ctrans{\Sigma}{P}{f \in \Ct})$ is 
unsatisfiable then $\langle D_\infty,{\cal I}\rangle \models \ctrans{\Sigma}{P}{f \in \Ct}$.
\end{theorem}
\begin{proof}
If there is no model for this formula (i.e. the theorem prover returns ``unsatisfiable'') then
its negation must be valid (true in all models), that 
is ${\cal T} -> \ctrans{\Sigma}{P}{f \in \Ct}$ is valid. By completeness
of first-order logic ${\cal T} |- \ctrans{\Sigma}{P}{f \in \Ct}$. This means in 
turn that all models of ${\cal T}$ validate $\ctrans{\Sigma}{P}{f \in \Ct}$. In particular 
for the denotational model we have that $\langle D_\infty,{\cal I}\rangle \models {\cal T}$ 
and hence $\langle D_\infty,{\cal I}\rangle \models \ctrans{\Sigma}{P}{f \in \Ct}$.
\end{proof}

\subsubsection{End-goal and incremental verification}\label{sect:incremental}

Assume that we are given a program $P$ with a function $f \in dom(P)$, for which we have 
already showed that $\langle D_\infty,{\cal I}\rangle \models \ctrans{\Sigma}{P}{f \in \Ct_f}$. 
Suppose next that we are presented with a ``next'' goal, to prove that 
$\langle D_\infty,{\cal I}\rangle \models \ctrans{\Sigma}{P}{h \in \Ct}$. 
We may consider the following three variations of how to do this:

\begin{itemize}
  \item Simply ask for the unsatisfiability of: 
    \[  \Th{\Sigma}{P}\;\land\;
        \Th{\Sigma}{P}^{\lcfZ}\;\land\;\dtrans{\Sigma}{P}\;\land\;\neg\ctrans{\Sigma}{P}{h \in \Ct_h} \] 
        The soundness of this query follows directly from Theorem~\ref{thm:prover-soundess} above.

  \item Ask for the unsatisfiability of:
    \[  \Th{\Sigma}{P}\;\land\;
        \Th{\Sigma}{P}^{\lcfZ}\;\land\;\dtrans{\Sigma}{P}\;\land\;\ctrans{\Sigma}{P}{f \in \Ct_f}\;\land\;\neg\ctrans{\Sigma}{P}{h \in \Ct_h}     \] 
        This query adds the {\em already proven} contract for $f$ in the theory. If this formula
        is unsatisfiable, then its negation is valid, and we know that the denotational model is 
        a model of the theory {\em and} of $\ctrans{\Sigma}{P}{f \in \Ct_f}$ and hence it must also
        be a model of $\ctrans{\Sigma}{P}{h \in \Ct_h}$. 
  \item Ask for the unsatisfiability of:
    \[  \Th{\Sigma}{P}\;\land\;
        \Th{\Sigma}{P}^{\lcfZ}\;\land\;\dtrans{\Sigma}{P \setminus f}\;\land\;\ctrans{\Sigma}{P}{f \in \Ct_f}\;\land\;
        \neg\ctrans{\Sigma}{P}{h \in \Ct_h}     \] 
        This query removes the axioms associated with the definition of $f$ since we may only have 
        its signature and contract available. Via a similar reasoning as before, such an invocation 
        is sound.
\end{itemize}

Our final goal is going to show that a program does not crash, that
is the final contract will be of the form $e \in \Ct$ where $\Ct$ is
going to be some {\em base contract}. Note that by base contract adequacy
(Lemma~\ref{lem:base-contract-adequacy}) if we manage to show a base contract 
denotationally, then the contract holds in operational terms.



\subsection{Denotational versus operational semantics for contracts}
TODO -- I have just dumpted material here. 

We have the rather obvious theorem below.

\begin{theorem}[Soundness and completeness for denotational semantics]
Assume a program $P$ with signature $\Sigma$, and expression $e$ and contract $\Ct$ 
such that $fv(e) \cup fv(\Ct) \subseteq dom(P)$. Then 
$\langle D_\infty,{\cal I}\rangle \models \ctrans{\Sigma}{P}{e \in \Ct}$ iff
$\interp{\Ct}{\dbrace{P}^{\infty}}{\cdot}(\interp{e}{\dbrace{P}^\infty}{\cdot})$.
\end{theorem}




\subsubsection{Contract satisfaction and crash-freedom}\label{sect:cf}

We would like to define a set of contract-satisfying denotations and also a set of contract-satisfying terms, 
characterized by $P |- e \in \Ct$, such that the following claim becomes true:

\begin{proposition} Assume that $\Sigma |- P$ and $fv(e) \subseteq dom(P)$, i.e. $e$ is closed.
Then: $\langle D_\infty,{\cal I}\rangle \models \ctrans{\Sigma}{\Delta}{e \in \Ct}$ iff $P |- e \in \Ct$.
\end{proposition}

Now there are several problems with coming up with a good definition of $P |- e \in \Ct$, 
which we elaborate in the following sections.

\subsubsection{Problem I: Crash-freedom} 

Ideally we would like to define crash-freedom {\em semantically} using the following 
strict bifunctor on admissible sets $S^{-},S^{+} \subseteq D_{\infty}$.
{\setlength{\arraycolsep}{2pt}
\[\begin{array}{rcl}
   F_{\lcfZ}(S^{-},S^{+}) & = & \{\;d\;\mid\;\unroll(d) \neq \ret(\inj{bad}(1))\;\land\; \\ 
                      &    & \quad \forall \ol{d} @.@ \unroll(d){=}\ret(\inj{K_1^\ar}\langle\oln{d}{\ar}\rangle) ==> \ol{d} \in S^{+} \} \\ 
                   & \cup & \ldots \\ 
                   & \cup & \{\;d\;\mid\;\unroll(d) \neq \ret(\inj{bad}(1))\;\land\; \\ 
                   &      & \quad \forall d_0 @.@ \unroll(d) = \ret(\inj{->}(d_0)) ==> \\ 
                   &      & \quad\quad \forall\;d' \in S^{-} ==> \dapp(d,d') \in S^{+} \}  \\
\end{array}\]}
The $\Fcf$ bifunctor has a negative and positive fixpoint, and by minimal invariance they coincide (one direction 
follows by Tarski-Knaster, the other can be inductively proved using the approximations on ever element of $D_{\infty}$ given
in Lemma~\ref{lem:min-inv-reqs} and the fact that the lub of the chain of $\rho_i$ is the identity and the fact that this 
functor preserves admissibility for the positive sets). Let us call this admissible set $\Fcf^{\infty} \subseteq D_{\infty}$.

We consider this predicate to be the ``ideal crash-freedom'' -- however it is very difficult to give a 1-1 operational
definition. The reason is that the $\Fcf$ functor quantifies in the function case over any $d'$ -- whereas in the operational
semantics it is only reasonable that we quantify over all terms (or over terms that do not contain @BAD@) In the absense of 
full abstraction of the domain (which is plausible, especially if we extend the language with other features) it is unclear 
what a corresponding predicate would look like in terms of operational semantics. 

We then go for a simpler predicate, which only characterizes crash-freedom for first-order terms, 
generate by the following functor on {\em admissible} sets of denotations:
{\setlength{\arraycolsep}{2pt}
\[\begin{array}{rcl}
   G_{\lcfZ}(S^{+}) & = & \{\;d\;\mid\; \unroll(d){=}\ret(\inj{K_1^\ar}\langle\oln{d}{\ar}\rangle) \land \ol{d} \in S^{+} \} \\ 
                  & \cup & \ldots \\ 
                  & \cup & \{\;\bot\;\}
\end{array}\]}
Notice that if $S$ is admissible then so is $G_{\lcfZ}(S)$. 

%% The $G_{\lcfZ}$ functor has a fixpoint and it is an admissible relation, and we will use its 
%% fixpoint $G_{\lcfZ}^\infty$, so now we need to say what $G_{\lcfZ|}^\infty$ means operationally. 
\begin{lemma} The functor $G_{\lcfZ}$ has a unique fixpoint $G_{\lcfZ}^\infty$ on admissible sets. \end{lemma}
\begin{proof} 
The intersection of admissible sets is admissible. Hence we have a complete join semi-lattice (which induces a 
complete lattice), so the monotone functor $G_{\lcfZ}$ does have a smallest and a greatest fixpoint call
it $G_{\lcfZ}^{min}$ and $G_{\lcfZ}^{max}$. Moreover this fixpoint will be an admissible relation. Now it must be 
that $G_{\lcfZ}^{min} \subseteq G_{\lcfZ}^{max}$ so we only show next that
also $G_{\lcfZ}^{max} \subseteq G_{\lcfZ}^{min}$. To do this we will show that:
\[ \forall i. d \in G_{\lcfZ}^{max} ==> \rho_i(d) \in G_{\lcfZ}^{min} \] 
by induction on $i$. For $i = 0$ it follows since $\rho_0(d) = \bot$. Let us assume 
that it holds for $i$, we need to show that $\rho_{i+1}(d) \in G_{\lcfZ}(G_{\lcfZ}^{min})$.
We know however that $d \in G_{\lcfZ}(G_{\lcfZ}^{max}$ and by simply case analysis and appealing
to the induction hypothesis we are done. Finally, by admissibility it must be that
$\sqcup\rho_i(d) \in G_{\lcfZ}^{min}$ and by Lemma~\ref{lem:min-inv-reqs} it
must be that $d \in G_{\lcfZ}^{min}$. This means that the two fixpoints coincide, 
hence there is only a unique fixpoint of $G_{\lcfZ}$, call it $G_{\lcfZ}^\infty$.
\end{proof} 

Now, we would like to define operationally the set of {\em crash-free} terms as a set $\Ecf$ of 
closed terms that satisfies:
{\setlength{\arraycolsep}{2pt}
\[\begin{array}{rcl}
   \Ecf & =    & \{ e \;\mid\; P |- e \Downarrow K[\taus](\ol{e}) /\ \ol{e} \in \Ecf \} \\
        & \cup & \ldots \\
        &      & \{ e \;\mid\; P \not|- e \Downarrow \} 
\end{array}\]}%
We do not know that the set $\Ecf$ exists, so we have to prove it. 
\begin{lemma}
There exists a largest set that satisfies the $\Ecf$ equation above.
\end{lemma} 
\begin{proof}
Define $\Ecf$ to be the set
\[ \{ e\;\mid\; \interp{e}{\dbrace{P}^\infty}{\cdot} \in G_{\lcfZ}^{\infty}\} \]
It is straightforward (by computational adequacy) to show that it satisfies the $\Ecf$ recursive
equation above. For uniqueness, assume any other set $E$ that satisfies the recursive equation
above. We can show that $\interp{E}{\dbrace{P}^\infty}{\cdot}$ is a
fixpoint of $G_{\lcfZ}$ and since there is only one such fixpoint, this is unique. So we have that:
\[\begin{array}{ll}
 e \in E & ==> \\ 
 \interp{e}{\dbrace{P}^\infty}{\cdot} \in \interp{E}{\dbrace{P}^\infty}{\cdot} & ==> \\
 \interp{e}{\dbrace{P}^\infty}{\cdot} \in G_{\lcfZ}^\infty & ==> \\
 e \in \Ecf 
\end{array}\] 
\end{proof}
%% \begin{lemma} 
%% If $e \in E$ and $\interp{e}{\dbrace{P}^\infty}{\cdot} = \interp{e'}{\dbrace{P}^\infty}{\cdot}$ then $e' in E$.
%% \end{lemma}
%% This relies on the fact that 
%% if $\interp{e}{\dbrace{P}^\infty}{\cdot} \in \interp{E}{\dbrace{P}^\infty}{\cdot}$ then $e \in E$. Why is 
%% that? Because the assumption means that 
%% $\interp{e}{\dbrace{P}^\infty}{\cdot} \in \{ d | \exists e' \in E /\ d = \interp{e'}{\dbrace{P}^\infty}{\cdot} \}$
%% and hence this means that there exists some $e' \in E $ such that 
%% $\interp{e}{\dbrace{P}^\infty}{\cdot} = \interp{e'}{\dbrace{P}^\infty}{\cdot}$ 
%% \end{proof} 

Let us extend the interpretation function above $\linterp{\cdot}$ so that: 
\[\begin{array}{rcl}
   \linterp{\lcfZ}  & = & G_{\lcfZ}^{\infty} 
\end{array}\]

\begin{theorem}
If $\Sigma |- P$ then we have that $\langle D_{\infty},{\cal I}\rangle \models \Th{\Sigma}{P}^{\lcfZ}$.
\end{theorem}

Notice that the axiom:
\[  \textsc{AxCfC}  \quad \formula{\forall x y @.@ \lcf{x} /\ \lcf{y} => \lcf{app(x,y)}} \]
is {\em not validated} by this interpretation of crash-freedom we have given. 


\subsubsection{Problem II: the absense of full-abstraction}

Unfortunately higher-orderness bites again. Having defined the set $\Ecf$ we might define formally
the predicate $P |- e \in \Ct$ where $fv(e) \subseteq dom(P)$ and $fv(\Ct) \subseteq dom(P)$ as 
follows:
{\setlength{\arraycolsep}{2pt}
\[\begin{array}{lcl}
    P |- e \in \{ x\;\mid\;e_p\} & <=> & P |- e \not\Downarrow \text{ or } P |- e_p[e/x] \not\Downarrow \text{ or} \\ 
                                 &     & P |- e_p[e/x] \Downarrow True \\
    P |- e \in (x{:}\Ct_1) -> \Ct_2 & <=> & 
                                 \text{for all } P' e' \text{ s.t. } fv(e') \subseteq dom(P{\uplus}P')  \\ 
                                   &   &  \text{it is } P\uplus P' |- e\;e' \in \Ct_2[e'/x] \\
    P |- e \in \Ct_1 \& \Ct_2 & <=> & P |- e \in \Ct_1 \text{ and } P |- e \in \Ct_2 \\
    P |- e \in \CF            & <=> & e \in \Ecf 
\end{array}\]}

Note we made the definition above well-scoped but not necessarily well-typed; let's ignore that for now (making everything
well-typed includes extra difficulties in the proof but hopefully not surmountable).

The interesting case is the case for arrow contracts, where we have extended the set of definitions $P$ with more 
definitions $P'$ -- that is to allow for tests $e'$ which can have arbitrary computational power, and not only those
that can be constructed in the current environment. That is expected the way we have set up things, so let us examine
what happens when we try to prove the proposition below:

\begin{proposition} Assume that $\Sigma |- P$ and $fv(e) \subseteq dom(P)$, i.e. $e$ is closed.
Then: $\langle D_\infty,{\cal I}\rangle \models \ctrans{\Sigma}{\Delta}{e \in \Ct}$ iff $P |- e \in \Ct$.
\end{proposition}

{\flushleft{\em Failed proof}:}
The base case and the case of $\CF$ follow from computational adequacy so we are good. However
let's try to prove the arrow case and in particular the $(<=)$ direction. 

Let us assume that for all $P'$ and $e'$ such that $fv(e') \subseteq dom(P\uplus P')$ it is the case that
$P |- e\;e' \in \Ct_2[e'/x]$. We need to show that $\langle D_\infty,{\cal I}\rangle$ is a model of the 
formula $\forall x. \ctrans{\Sigma}{x}{x \in \Ct_1} => \ctrans{\Sigma}{x}{e\;x \in \Ct_2}$. Let us fix
a denotation $d \in D_{\infty}$ and let us assume 
that $\langle D_{\infty},{\cal I} \rangle \models \ctrans{\Sigma}{x}{x \in \Ct_1}[d/x]$. However, this does not 
necessarily mean that we can find a closed $e'$ and $P'$, such 
that $\interp{e'}{\dbrace{P{\uplus}P'}^\infty}{\cdot} = d$ to be able to use the assumptions, unless some sort
of full-abstraction property is true. So we are stuck.

Here is a concrete counterexample, based on the lack of full-abstraction due to the {\em parallel or} function. 
Consider the program $P$ below:
\[\begin{array}{lcl}
f_\omega & |-> & f_\omega \\
f & |-> & \lambda (b{:}Bool) @.@ \lambda (h{:}Bool->Bool->Bool) @.@ \\
  &     & \quad @if@\;(h\;True\;b)\;\&\&\;(h\;b\;True)\;\&\& \\ 
  &     & \quad\qquad\qquad not\;(h\;False\;False)\;@then@ \\
  &     & \quad\quad @if@\;(h\;True\;f_\omega)\;\&\&\;(h\;f_\omega\;True)\;@then@\;@BAD@ \\
  &     & \quad\quad @else@\;True \\
  &     & \quad @else@\;True
\end{array}\]
Consider now the candidate contract for $f$ below: 
\[ \CF -> (\CF -> \CF -> \CF) -> \CF \]
Operationally we may assume a crash-free boolean as well as a function $h$ which is 
$\CF -> \CF -> \CF$. The first conditional ensures that the function behaves like an ``or'' function or 
diverges. However if we pass the first conditional, 
the second conditional will always diverge and hence the contract will be satisfied. 

However, denotationally it is possible to have a {\em monotone} function $por$ defined as follows:
\[\begin{array}{lcl}
  por\;\bot\;\bot & = & \bot \\ 
  por\;\bot\;True & = & True \\
  por\;True\;\bot & = & True \\ 
  por\;False\;False & = & False
\end{array}\] 
with the rest of the equations (for @BAD@ arguments) induced by monotonicity and whatever boolean value 
we like when both arguments are @BAD@. 

Now, this is denotationally a $\CF -> \CF -> \CF$ function, and it will pass the first conditional, but it will
also pass the second conditional, yielding @BAD@. Hence denotationally the contract for $f$ {\em does not hold}.

So we have a concrete case where the $<=$ direction fails. Because of contra-variance of arrow contracts, it is 
likely that the $=>$ direction is false as well. 


%% Now it may be the case that for all denotations that semantically satisfy a contract, these denotations {\em are} 
%% realizable by a term $e'$ and a context $P'$ but it is not entirely clear how to prove this (or if this is a good
%% idea). I am not sure if this is true either.
%% The other idea out of this situation is to compile the arrow contract differently by not quantifying over all 
%% denotations but rather some kind of {\em definable} denotations -- but I do not know how exactly to do this.


\paragraph{A way out of this?}
Well, if we restrict our higher-order tests to those that can be constructed from our signature then 
we may define the following:

{\setlength{\arraycolsep}{2pt}
\[\begin{array}{lcl}
    P |- e \in \{ x\;\mid\;e_p\} & <=> & P |- e \not\Downarrow \text{ or } P |- e_p[e/x] \not\Downarrow \text{ or} \\ 
                                 &     & P |- e_p[e/x] \Downarrow True \\
    P |- e \in (x{:}\Ct_1) -> \Ct_2 & <=> & 
                                 \text{for all } e' \text{ s.t. } fv(e') \subseteq dom(P)  \\ 
                                   &   &  \text{it is } P |- e\;e' \in \Ct_2[e'/x] \\
    P |- e \in \Ct_1 \& \Ct_2 & <=> & P |- e \in \Ct_1 \text{ and } P |- e \in \Ct_2 \\
    P |- e \in \CF            & <=> & e \in \Ecf 
\end{array}\]}
Notice that the difference with the previous version of $P |- e \in \Ct$ is that we {\em do not} extend the 
definitions $P'$ so we don't get the full power of higher-order tests. We show that {\em in the current signature
only} does the program satisfy the contract. 


Why did we do this change? Because denotationally this is not terribly hard to support -- instead of translating 
\[\begin{array}{l}
  \ctrans{\Sigma}{\Gamma}{e \in (x{:}\Ct_1) -> \Ct_2} =  
  \formula{\forall x @.@ \ctrans{\Sigma}{\Gamma,x}{x \in \Ct_1} => \ctrans{\Sigma}{\Gamma,x}{e\;x \in \Ct_2}}
\end{array}\] 
we use the following:
\[\begin{array}{l}
  \ctrans{\Sigma}{\Gamma}{e \in (x{:}\Ct_1) -> \Ct_2} = \\ 
  \qquad\qquad\quad 
\formula{\forall x @.@ \definable{x} \land \ctrans{\Sigma}{\Gamma,x}{x \in \Ct_1} => \ctrans{\Sigma}{\Gamma,x}{e\;x \in \Ct_2}}
\end{array}\] 
where $\definable{x}$ could be axiomatized as containing all terms 
made up of the functions in $P$, applications, and data constructors:

\[\begin{array}{lll} 
 \textsc{DefCons} & \formula{\forall \xs @.@ \definable{K(\xs)} <=> \definable{\xs}} \\
                        & \text{ for every } (K{:}\forall\as @.@ \oln{\tau}{n} -> T\;\as) \in \Sigma \\
 \textsc{DefFun}  & \formula{\definable{f_{ptr}}}  \\
                        & \text{ for every } (f |-> \Lambda\as @.@ \lambda\oln{x{:}\tau}{n} @.@ u) \in P \\
 \textsc{DefApp}  & \formula{\forall x y @.@ \definable{x}\land\definable{y} => \definable{app(x,y)}}
%% \formula{\bad \neq \unr}  \\ 
%%  \textsc{AxDisjB} & \formula{\forall \oln{x}{n}\oln{y}{m} @.@ K(\ol{x}) \neq J(\ol{y})} \\ 
%%                   & \text{ for every } (K{:}\forall\as @.@ \oln{\tau}{n} -> T\;\as) \in \Sigma \\ 
%%                   & \text{ and } (J{:}\forall\as @.@ \oln{\tau}{m} -> S\;\as) \in \Sigma \\
%%  \textsc{AxDisjC} & \formula{(\forall \oln{x}{n} @.@ K(\ol{x}) \neq \unr \land K(\ol{x}) \neq \bad)} \\ 
%%                   & \text{ for every } (K{:}\forall\as @.@ \oln{\tau}{n} -> T\;\as) \in \Sigma \\ \\
%%  \textsc{AxAppA}  & \formula{\forall \oln{x}{n} @.@ f(\ol{x}) = app(f_{ptr},\xs)} \\
%%                   & \text{ for every } (f |-> \Lambda\as @.@ \lambda\oln{x{:}\tau}{n} @.@ u) \in P \\
%%  %% \textsc{AxAppB}  & \formula{\forall \oln{x}{n} @.@ K(\ol{x}) = app(\ldots (app(x_K,x_1),\ldots,x_n)\ldots)} \\
%%  %%                  & \text{ for every } (K{:}\forall\as @.@ \oln{\tau}{n} -> T\;\as) \in \Sigma \\
%%  \textsc{AxAppC}  & \formula{\forall x, app(\bad,x) = \bad \; /\ \; app(\unr,x) = \unr}    \\ \\
%%  %% Not needed: we can always extend partial constructor applications to fully saturated and use AxAppC and AxDisjC
%%  %% \textsc{AxPartA} & \formula{\forall \oln{x}{n} @.@ app(\ldots (app(x_K,x_1),\ldots,x_n)\ldots) \neq \unr} \\
%%  %%                  & \formula{\quad\quad \land\; app(\ldots (app(x_K,x_1),\ldots,x_n)\ldots) \neq \bad} \\
%%  %%                  & \text{ for every } (K{:}\forall\as @.@ \oln{\tau}{m} -> T\;\as) \in \Sigma \text{ and } m > n \\
%%  \textsc{AxPartB} & \formula{\forall \oln{x}{n} @.@ app(f_{ptr},\xs) \neq \unr} \\
%%                   & \formula{\quad\land\; app(f_{ptr},\xs) \neq \bad} \\
%%                   & \formula{\quad\land\; \forall \oln{y}{k} @.@ app(f_{ptr},\xs) \neq K(\ol{y})} \\
%%                   & \text{ for every } (f |-> \Lambda\as @.@ \lambda\oln{x{:}\tau}{m} @.@ u) \in P  \\
%%                   & \text{ and every } (K{:}\forall\as @.@ \oln{\tau}{k} -> T\;\as) \in \Sigma \text{ and } m > n  \\ \\ 
%%  \textsc{AxInj}   & \formula{\forall \oln{y}{n} @.@ \sel{K}{i}(K(\ys)) = y_i} \\ 
%%                   & \text{for every } (K{:}\forall\as @.@ \oln{\tau}{n} -> T\;\as) \in \Sigma \text{ and } i \in 1..n \\ \\
%% \end{array} \\
%% \ruleform{\Th{\Sigma}{P}^{\lcfZ}} \\ \\ 
%% \begin{array}{lll} 
%%  \textsc{AxCfA}   & \formula{\lcf{\unr} /\ \lncf{\bad}} \\
%%  \textsc{AxCfB}   & \formula{\forall \oln{x}{n} @.@ \lcf{K(\ol{x})} <=> \bigwedge\lcf{\ol{x}}} \\
%%                   & \text{ for every } (K{:}\forall\as @.@ \oln{\tau}{n} -> T\;\as) \in \Sigma
\end{array}\]


In the model, $\definable{\cdot}$ should be possible to define, as a
predicate on denotations. The disadvantage to this approach is that
arrow contracts will only hold for whatever is in your context, not
arbitrary expressions, which might be what we want, but might not be
modular enough.

The other {\em potential} problem (i.e. I have not yet checked) might be in the 
proof of admissibility of induction. 

And yet another potential problem is that as we incrementally extend our signature 
with new function definitions (and possibly contracts) previously defined contracts
may no longer hold. This is pretty bad for modularity.

\paragraph{Yet another possible solution}

A solution that seems somewhat more modular is based on the observation that, 
during the evaluation of a program there exists a {\em set} of terms (maybe infinite) that can
appear as arguments to other terms or functions. Our idea is to guard the arrow contracts so that
we do not quantify over any possible term (or denotation, in the translation) but rather only 
those that may appear as {\em arguments} in some application. We translate arrow contract as 
follows:
\[\begin{array}{l}
  \ctrans{\Sigma}{\Gamma}{e \in (x{:}\Ct_1) -> \Ct_2} = \\ 
  \qquad\qquad\quad 
\formula{\forall x @.@ arg(x) \land \ctrans{\Sigma}{\Gamma,x}{x \in \Ct_1} => \ctrans{\Sigma}{\Gamma,x}{e\;x \in \Ct_2}}
\end{array}\] 
where $arg(x)$ ensures that $x$ is the denotation of a term that will be passed as an argument to $e$. We'd need to define
a similar predicate on the evaluation relation, call it $Arg(e)$ and modify the program translation to thread the $arg(\cdot)$
predicate through. 



\subsection{Contract checking as satisfiability}\label{sect:soundness}
  Having established the soundness of our translation, it is time 
we see in this section how we can use this sound translation to verify a program. 
The following theorem is then true:

\begin{theorem}[Soundness]\label{thm:prover-soundness}
Assume that $e$ and $\Ct$ contain only function symbols from $P$ and no free term variables.
Let $\Th_{all} = \Th\;\land\;\ptrans{}{P}$. 
If $\Th_{all} \land \neg\ctrans{\Sigma}{P}{e \in \Ct}$ is unsatisfiable 
then $\langle D_\infty,{\cal I}\rangle \models \ctrans{\Sigma}{P}{e \in \Ct}$ and 
consequently $\dbrace{e} \in \dbrace{\Ct}$.
\end{theorem}
\begin{proof}
If there is no model for this formula then its negation must be valid (true in all models), that 
is $ \neg \Th_{all} \lor \ctrans{\Sigma}{P}{e \in \Ct}$ is valid. By completeness
of first-order logic $\Th_{all} |- \ctrans{\Sigma}{P}{e \in \Ct}$. This means 
that all models of $\Th_{all}$ validate $\ctrans{\Sigma}{P}{f \in \Ct}$. In particular, 
for the denotational model we have that $\langle D_\infty,{\cal I}\rangle \models \Th_{all}$ 
and hence $\langle D_\infty,{\cal I} \rangle \models \ctrans{\Sigma}{P}{e \in \Ct}$. 
Theorem~\ref{thm:den-contr-satisfaction} finishes the proof.
\end{proof}

Hence, to verify a program $e$ satisfies a contract $\Ct$ we need to do the following:
\begin{itemize*}
  \item Generate formulae for the theory $\Th\;\land\;\ptrans{}{P}$
  \item Generate the negation of a contract translation: $\neg\ctrans{\Sigma}{P}{e \in \Ct}$
  \item Ask a FOL theorem prover for a model for the conjunction of the above formulae
\end{itemize*}

\paragraph{Incremental verification}

Theorem~\ref{thm:prover-soundness} gives us a way to check that an expression satisfies a 
contract. Assume that we are given a program $P$ with a function $f \in dom(P)$, for which 
we have already shown that $\langle D_\infty,{\cal I}\rangle \models \ctrans{\Sigma}{P}{f \in \Ct_f}$.
Suppose next that we are presented with a ``next'' goal, to prove that 
$\langle D_\infty,{\cal I}\rangle \models \ctrans{\Sigma}{P}{h \in \Ct_h}$. 
We may consider the following variations of how to do this:

\begin{itemize*}
  \item Ask for the unsatisfiability of: 
    \[  \Th\; \land \; \ptrans{\Sigma}{P} \; \land \;\neg\ctrans{\Sigma}{P}{h \in \Ct_h} \]
        The soundness of this query follows from Theorem~\ref{thm:prover-soundness} above.

  \item Ask for the unsatisfiability of:
    \[  \Th \; \land\; \ptrans{\Sigma}{P} \; \land \; \ctrans{\Sigma}{P}{f \in \Ct_f} \land \neg \ctrans{\Sigma}{P}{h \in \Ct_h} \]
        This query adds the {\em already proven} contract for $f$ to the theory. If this formula
        is unsatisfiable, then its negation is valid, and we know that the denotational model is 
        a model of the theory {\em and} of $\ctrans{\Sigma}{P}{f \in \Ct_f}$ and hence it must also
        be a model of $\ctrans{\Sigma}{P}{h \in \Ct_h}$.
  \item Ask for the unsatisfiability of:
    \[  \Th \; \land \; 
        \ptrans{\Sigma}{P \setminus f} \; \land \; 
        \ctrans{\Sigma}{P}{f \in \Ct_f} \; \land \;
        \neg\ctrans{\Sigma}{P}{h \in \Ct_h} \]
        This query removes the axioms associated with the \emph{definition} of $f$, leaving
        only its \emph{contract} available.  This makes the proof of $h$'s contract
        insensitive to changes in $f$'s implementation.
        Via a similar reasoning as before, such an invocation 
        is sound as well.
\end{itemize*}

Incremental verification essentially implies that our approach does not require a
whole-program analysis: once a contract is proved about a function, it can be assumed,
even if the definition of the function is not exported, and subsequently used to prove
contracts of other functions that may call it. Exporting only the contract but not a 
definition of a function can be beneficial for efficiency reasons when verifying further
goals; the drawback is that the exported contracts might not be precise enough for proving
these further goals. Our framework supports any of those strategies.


%% Our final goal is going to show that a program does not crash, that
%% is the final contract will be of the form $e \in \Ct$ where $\Ct$ is
%% going to be some {\em base contract}. Note that by base contract adequacy
%% (Lemma~\ref{lem:base-contract-adequacy}) if we manage to show a base contract 
%% denotationally, then the contract holds in operational terms.

%% \clearpage


\section{Essential extensions}\label{sect:extensions}
  \label{sect:induction}
%% \subsection{Induction}\label{sect:induction}

An important practical extension is the ability to prove contracts about recursive functions
using induction. For instance, we might want to prove that @length@ satisfies $\CF -> \CF$.
\begin{code}
  length []     = Z 
  length (x:xs) = S (length xs)
\end{code}
In the second case we need to show that the result of @length xs@ is crash-free but we do not 
have this information so the proof gets stuck, often resulting in the FOL-solver looping. 

A naive approach would be to perform induction over the list argument of @length@ -- however
in Haskell datatypes may be lazy infinite streams and ordinary induction is not
a valid proof principle. Fortunately, we can still appeal to {\em fixpoint induction}. The 
fixpoint induction scheme that we use for @length@ above would be to {\em assume} that the 
contract holds for the occurence of 
some function @length_rec@ inside the body of its definition, 
and then try to prove it for the function:
\begin{code}
  length []     = Z 
  length (x:xs) = S (length_rec xs)
\end{code}

Formally, our induction scheme is:
\begin{definition}[Induction scheme]\label{def:induction}
To prove that $\dbrace{g} \in \dbrace{\Ct}$ for a function 
$g\;\as\;\ol{x{:}\tau} = e[g]$ (meaning $e$ contains
some occurrences of $g$), we perform the following steps:
\begin{itemize*}
  \item Generate function symbols $g^{\circ}$, $g^{\bullet}$
  \item Generate the theory formula \[ \phi = \Th \land 
             \ptrans{}{P \cup g^{\bullet}\;\as\;\ol{x{:}\tau} = e[g^{\circ}]} \] 
  \item Prove that the query $\phi \land \ctrans{}{}{g^{\circ} \in \Ct} \land \neg \ctrans{}{}{g^{\bullet} \in \Ct}$
        is unsatisfiable.
\end{itemize*}
%% If it is unsatisfiable then $\dbrace{f} \in \dbrace{\Ct}$.
\end{definition}

Why is this approach sound? The crucial step here is the fact that contracts are admissible predicates.
\begin{theorem}[Contract admissibility]
If $d_i \in \dbrace{\Ct}$ for all elements of a chain $d_1 \sqsubseteq d_2 \sqsubseteq \ldots$ then the limit of the chain 
$\sqcup d_i \in \dbrace{\Ct}$. Moreover, $\bot \in \dbrace{\Ct}$.
\end{theorem}
\begin{proof} By induction on the contract $\Ct$; for the $\CF$ case we get the result from Lemma~\ref{lem:cf-admissible}.
For the predicate case we get the result from the fact that the denotations of programs 
are continuous in $D_{\infty}$. The arrow case follows by induction.
\end{proof}

We can then prove the soundness of our induction scheme.
\begin{theorem} The induction scheme in Definition~\ref{def:induction} is correct. \end{theorem}
\begin{proof} We need to show that: 
$\dbrace{P}^{\infty}(g) \in \dbrace{\Ct}$ and hence, by admissibility it is enough to find
a chain whose limit is $\dbrace{P}^{\infty}(g)$ and such that every element is in $\dbrace{\Ct}$.
Let us consider the chain $\dbrace{P}^{k}(g)$ so that $\dbrace{P}^0(g) = \bot$ and 
$\dbrace{P}^{k+1}(g) = \dbrace{P}_{(\dbrace{P}^{k})}(g)$ whose limit is $\dbrace{P}^{\infty}(g)$. We 
know that $\bot \in \dbrace{\Ct}$ so, by using contract admissiblity, all we need to show is
that if $\dbrace{P}^{k}(g) \in \dbrace{\Ct}$ then $\dbrace{P}^{k+1}(g) \in \dbrace{\Ct}$. 

To show this, we can assume a model where the denotational interpretation ${\cal I}$ has been 
extended so that ${\cal I}(g^{\circ}) = \dbrace{P}^{k}(g)$ and ${\cal I}(g^{\bullet}) = \dbrace{P}^{k+1}(g)$.
By proving that the formula 
\[ \phi \land \ctrans{}{}{g^{\circ} \in \Ct} \land \neg \ctrans{}{}{g^{\bullet} \in \Ct} \] 
is unsatisfiable, since $\langle D_{\infty},{\cal I}\rangle \models \phi$ and 
$\langle D_{\infty},{\cal I}\rangle \models \ctrans{}{}{g^{\circ} \in \Ct}$, we get
$\langle D_{\infty},{\cal I}\rangle \models \ctrans{}{}{g^{\bullet} \in \Ct}$, 
and hence $\dbrace{P}^{k+1}(g) \in \dbrace{\Ct}$.
\end{proof} 

Note that contract admissibility is {\em absolutely essential} for the
soundness of our induction scheme, and is not a property that holds of
every predicate on denotations. For example, consider the following
Haskell definition:
\begin{code}
  ones = 1 : ones
  f (S x) = 1 : f x
  f Z     = [0]
\end{code}
Let us try to check if the $\forall x @.@ @f@(x) \neq @ones@$ is true in 
the denotational model, using fixpoint induction. Importanttly, by $\neq$ here we 
really mean the negation of logical equality -- {\em not} a Haskell
function that computes a @Bool@. The case for $\bot$ holds, 
and so does the case for the @Z@ constructor. For the $@S@\;x$ case, we can 
assume that $@f@(x) \neq @ones@$ and we can easily prove that this implies that
$@f@(@S@\;x) \neq @ones@$. Nevertheless, the property is {\em not true} -- just pick 
a counterexample $\dbrace{@s@}$ where @s = S s@. What happened here is that the property 
is denotationally true of all the elements of the following chain
\[ \bot \sqsubseteq \injK{S}{\bot} \sqsubseteq \injK{S}{\injK{S}{\bot}} \sqsubseteq \ldots \] 
but is false in the limit of this chain. In other words $\neq$ is not admissible and our 
induction scheme is plain nonsense for non-admissible predicates. 

Finally, we have observed that for many practical cases, a
straightforward generalization of our lemma above for mutually
recursive definitions is required. Indeed, our tool performs mutual
fixpoint induction when a recursive group of functions is
encountered. We leave it as future work to develop more advanced
techniques such as strengthening of induction hypotheses or
identifying more sophisticated induction schemes.




\section{Implementation and practical experience}\label{sect:implementation}
  Our prototype contract checker is called \textbf{Halo}.
It uses GHC to parse, typecheck, and desugar a Haskell program,
translates it into first order logic (exactly as in Section~\ref{ssect:trans-fol}), and
invokes a FOL theorem prover (Equinox, Z3, Vampire, etc) on the FOL formula.
The desugared Haskell program is expressed in GHC's intermediate language
called Core~\cite{Sulzmann:2007:SFT:1190315.1190324}, an explicitly-typed 
variant of System F.  It is straightforward
to translate Core into our language $\theLang$.

\subsection{Expressing contracts in Haskell}

How does the user express contracts?  We write contracts in Haskell
itself, using higher-order abstract syntax and a GADT, in 
a manner reminiscent 
of the work on {\em typed contracts} for functional 
programming~\cite{Hinze:2006:TCF:2100071.2100093}:
\begin{code}
data Contract t where
  (:->) :: Contract a 
        -> (a -> Contract b)
        -> Contract (a -> b)
  Pred  :: (a -> Bool) -> Contract a
  CF    :: Contract a
  (:&:) :: Contract a -> Contract a -> Contract a
\end{code}
A value of type @Contract t@ is a 
a contract for a function of type @t@.
The connectives are @:->@ for dependent contract function space, @CF@
for crash-freedom, @Pred@ for predication, and
@:&:@ for conjunction. 
One advantage of writing contracts as Haskell terms is that
we can use Haskell itself to build new contract combinators.
For example, a useful derived connective is non-dependent function space:
\par {\small
\begin{code}
(-->) :: Contract a -> Contract b -> Contract (a -> b)
c1 --> c2 = c1 :-> (\_ -> c2)
\end{code}
} \par
%As one would expect, @:->@ and @-->@ are right-associve.  We can
%create contract combinators that are always satisfied, and never
%satisfied:
%
%\begin{code}
%any :: Contract a
%any = Pred (\ _ -> True)
%
%never :: Contract a
%never = Pred (error "never!")
%\end{code}

A contract is always associated with a function, 
so we pair the two in a @Statement@:
\begin{code}
  data Statement where
      (:::) :: a -> Contract a -> Statement
\end{code}
In our previous mathematical notation we might write the following
contract for @head@:
$$
@head@ \in \CF \;\&\; \{@xs@ \mid @not (null xs)@ \} \rightarrow \CF
$$
Here is how we express the contract as a Haskell definition:
\begin{comment}
head (x:xs) = x
head []     = error "empty list"

not True = False    null [] = True
not False = True    null xs = False

f . g = \x -> f (g x)
\end{comment}
\begin{code}
c_head :: Statement
c_head = head ::: CF :&: Pred (not . null) --> CF
\end{code}
If we put this definition in a file @Head.hs@, together with the supporting
definitions of @head@, @not@, and @null@, 
then we can run @halo Head.hs@. 
The @halo@ program translates the contract and the supporting function
definitions into
FOL, generates a TPTP~\footnote{A  widely supported format for FOL~\citep{SS98}.} file, 
and invokes a theorem prover.
And indeed @c_head@ is verified by all theorem provers we tried.

For recursive functions @halo@ uses fixpoint induction, as
described in Section~\ref{sect:induction}.

% ------------------------ Omit ----------------------------
\begin{comment}
\subsection{Recursion}

We prove the contract for a recursive function
using fixed point induction (Section~\ref{s:induction}).
For recursive
functions, the tool then gives three TPTP files, one that can be used to try to 
prove the contract without induction, one for the base case and step case. The
typical situation is that the one without induction is not provable, because it 
lacks the appropriate induction hypothesis. The base case always succeeds 
(because $\bot$ satisfies every contract as we have seen in Section~\ref{s:induction})
so it really only serves as a sanity check of the tool. 
The induction step case may pass or fail, depending on if the
contract really holds, and if the induction hypothesis is strong
enough, and if we have assumed the right contracts for functions that may be used
in the body of the function that we are working on.

One example of a recursive function in the Prelude is @foldr1@.

\begin{code}
foldr1          :: (a -> a -> a) -> [a] -> a
foldr1 f [x]    =  x
foldr1 f (x:xs) =  f x (foldr1 f xs)
foldr1 _ []     =  error "foldr1: empty list"
\end{code}

We can state that if @foldr1@ is applied to a crash free function, and
a non-empty list, then the result should be crash free as a contract:
\begin{code}
c_foldr = foldr1 ::: (CF --> CF --> CF) 
                 --> CF :&: Pred (not . null) --> CF
\end{code}
Our tool proves this contract, but only when recursion is used.
\end{comment}
% ------------------------ End Omit ----------------------------

\subsection{Practical considerations}

To make the theorem prover work as fast as possible we trim the
theories to include only what is needed to prove a
property. Unnecessary function pointers, data types and definitions
for the current goal are not generated.

When proving a series of contracts, it is natural to do so in dependency order.
For example:
\begin{code}
  reverse :: [a] -> [a]
  reverse [] = []
  reverse (x:xs) = reverse xs ++ [x]

  reverse_cf :: Statement
  reverse_cf = reverse ::: CF --> CF
\end{code}
To prove this contract we must first prove that
$@(++)@ \in \CF \rightarrow \CF \rightarrow \CF$;
then we can prove @reverse@'s contract assuming the one for @(++)@.
At the moment, @halo@ asks the programmer to specify which auxiliary contracts
are useful, via a second constructor in the @Statement@ type:
\begin{code}
  reverse_cf = reverse ::: CF --> CF
                       `Using` append_cf
\end{code}


\subsection{Dependent contracts}

@halo@ can prove dependent contracts.
For example:
$$
@filter@ \in (@p@ : \CF \rightarrow \CF) \rightarrow
             \CF \rightarrow \CF \;\&\; \{@ys@ \mid @all p ys@ \}
$$
This contract says that under suitable assumptions of crash-freedom,
the result of @filter@ is both crash-free and satisfies @all p@.
Here @all@ is a standard Haskell function, and @p@ is the functional
argument itself.

In our source-file syntax we use @(:->)@ to bind @p@.
\begin{code}
filter_all :: Statement
filter_all = 
  filter ::: (CF --> CF) :-> \p ->
                CF --> (CF :&: Pred (all p))
\end{code}
The contract looks slightly confusing since it uses two ``arrows'', one from @:->@,
and one from the @->@ in the lambda.  This contract is proved by
applying fixed point induction.

\subsection{Higher order functions}

Our tool also deals with (very) higher order functions.
Consider this function @withMany@, taken from the
library @Foreign.Util.Marshal@:
%\footnote{\url{http://hackage.haskell.org/packages/archive/base/latest/doc/html/Foreign-Marshal-Utils.html#v:withMany}}:

\begin{code}
withMany :: (a -> (b -> res) -> res)
         -> [a] -> ([b] -> res) -> res
withMany _       []     f = f []
withMany withFoo (x:xs) f = withFoo x (\x' ->
      withMany withFoo xs (\xs' -> f (x':xs')))
\end{code}

For @withMany@, our tool proves

\[ @withMany@ \in \begin{array}[t]{l} (\CF -> (\CF -> \CF) -> \CF) -> \\
                                      \quad\quad (\CF -> (\CF -> \CF) -> \CF)
                  \end{array}\] 

% ------------------------ Omit ----------------------------
\begin{comment}
\subsection{A small case-study about invariants}

We consider a somewhat non-standard way of expressing propositional
logic formulae:

\begin{code}
data Formula = And [Formula]
             | Or  [Formula]
             | Neg (Formula)
             | Implies (Formula) (Formula)
             | Lit Bool
\end{code}

One invariant that we are particularly interested in is that we
should never have two consecutive negations, and that the lists of
@And@ and @Or@ are of length $\ge$ 2. We can express that as an ordinary
Haskell predicate:

\begin{code}
invariant :: Formula -> Bool
invariant f = case f of
  And xs      -> properList xs && all invariant xs
  Or xs       -> properList xs && all invariant xs
  Neg Neg{}   -> False
  Neg x       -> invariant x
  Implies x y -> invariant x && invariant y
  Lit x       -> True

properList :: [a] -> Bool
properList []  = False
properList [_] = False
properList _   = True
\end{code}

Now, we have a recursive function that negates formula:

\begin{code}
neg :: Formula -> Formula
neg (Neg f)         = f
neg (And fs)        = Or (map neg fs)
neg (Or fs)         = And (map neg fs)
neg (Implies f1 f2) = neg f2 `Implies` neg f1
neg (Lit b)         = Lit b
\end{code}

We make a combinator saying what it means to retain a predicate:

\begin{code}
retain :: (a -> Bool) -> Contract (a -> a)
retain p = Pred p :-> \x -> Pred (\r -> p x && p r)
\end{code}

\dr{TODO: explain this. This was DV's brilliant idea but I still don't
  fully understand it} Now, since @neg@ uses @map@, we need to show that
@map@ can retain the invariant. We use @all@, introduced above, for
this:

\begin{code}
map_invariant = map ::: retain invariant -->
                        retain (all invariant)
\end{code}

Explicitly spelling out the definition of @retain@ in the statement
above would be tedious and error-prone, so we see the benefit of being
able to express contracts as a DSL.

We can now express that @neg@ retains the invariant:

\begin{code}
neg_contr = neg ::: retain invariant
  `Using` map_invariant
\end{code}

We use @Using :: Statement -> Statement -> Statement@, another
constructor for @Statement@, which allows us to assume that other
contracts holds, when proving a complicated statement, thus
our Statement data type really looks like this:

\begin{code}
data Statement where
    (:::) :: a -> Contract a -> Statement
    Using :: Statement -> Statement -> Statement
\end{code}

For now, it's the user's responsibility to prove these assumed
contracts (for instance, with our tool!), but one can imagine a more
sophisticated front-end which does this automatically.  Note that
the assumption in @neg_contr@ is necessary. If we remove it, and
use the min-translation, we a finitely counter satisfiable theory.
\end{comment}
% ------------------------ End of omit ----------------------------


% \subsection{Example: shrink}
%
% Recall that @fromJust@ is the partial function @Maybe a -> a@, and consider
% this code:
%
% \begin{code}
% shrink :: (a -> a -> a) -> [Maybe a] -> a
% shrink op []     = error "Empty list!"
% shrink op [x]    = fromJust x
% shrink op (x:xs) = fromJust x `op` shrink op xs
% \end{code}
%
% Is this contract satisfied for it?
% \begin{code}
%     (CF --> CF --> CF) -->
%     (CF :&: Pred nonEmpty :&: Pred (all isJust)) --> CF
% \end{code}

\subsection{Experimental Results}


\newcommand{\timeout}{-}
\newcommand{\tot}{\multicolumn{2}{c}{\timeout}}
\newcommand{\tol}{\multicolumn{2}{c |}{\timeout}}

\begin{figure}

\begin{center}
\begin{unsrestab}

 & \multicolumn{4}{c | }{Equinox}
 & \multicolumn{4}{c | }{Z3}
 & \multicolumn{4}{c  }{Vampire}
 \\

Description
 & \multicolumn{2}{c}{min} & \multicolumn{2}{c |}{w/o}
 & \multicolumn{2}{c}{min} & \multicolumn{2}{c |}{w/o}
 & \multicolumn{2}{c}{min} & \multicolumn{2}{c }{w/o}
  \\

\hline

@ack@ CF          & 1&61  & \tol & 0&04 & 0&02 & 0&47 & 0&14 \\
@any@ morphism    & 10&40 & \tol & 0&07 & 0&04 & \tot & \tot \\
@filter@ @all@    & 3&24  & \tol & 0&05 & 0&04 & \tot & \tot \\
@concatMap@ CF    & 0&87  & \tol & 0&04 & 0&04 & 0&38 & 4&65 \\
@invariant@ CF    & 23&37 & \tol & 0&08 & \tol & \tot & \tot \\
@map@ invariant   & 22&82 & \tol & 0&17 & \tol & \tot & \tot \\
@neg@ invariant   & \tot  & \tol & 0&83 & \tol & \tot & \tot \\
@(++)@ inv. @all@ & \tot  & \tol & 0&08 & \tol & 2&70 & \tot \\
acc @exp@ CF      & 0&54  & \tol & 0&06 & 0&04 & 2&27 & 3&44 \\
@iterate@ CF      & 0&42  & 5&82 & 0&03 & 0&00 & 0&18 & 0&00 \\
@iterTree@ CF     & 1&85  & \tol & 0&03 & 0&00 & 2&90 & 0&01 \\
@repeat@ CF       & 0&10  & 0&05 & 0&03 & 0&00 & 0&01 & 0&00 \\
@risersBy@        & 26&18 & \tol & \tot & \tol & 7&80 & 0&82 \\
@shrink@          & 5&64  & \tol & 0&05 & 0&05 & 2&04 & \tot \\
@withMany@        & 21&51 & \tol & 0&05 & 0&01 & 3&96 & \tot \\
\end{unsrestab}
\end{center}

\caption{
  Theorem prover running time in seconds on some of the problems in the test suite
  on contracts that \emph{do} hold.
  Paradox had only timeouts and is not listed.
  }
  \label{fig:unsres}

\end{figure}

\begin{figure}
\begin{center}
\begin{satrestab}

 & \multicolumn{2}{c | }{Paradox}
 & \multicolumn{2}{c | }{Equinox}
 & \multicolumn{2}{c  }{Z3}
 \\

Description
 & \multicolumn{2}{c |}{min}
 & \multicolumn{2}{c |}{min}
 & \multicolumn{2}{c }{min}
  \\

\hline

@any@ morphism w/o @any@ CF    & 2&52 & \tol  & \tot \\
@concatMap@ CF w/o @(++)@ CF   & 0&09 & \tol  & \tot \\
@concatMap@ invariant weak     & 0&61 & \tol  & \tot \\
@neg@ invariant missing @map@  & 0&63 & \tol  & \tot \\
acc @exp@ CF w/o @(*)@ CF      & 0&15 & \tol  & \tot \\
@risersBy@ wrong precond.      & 7&07 & 9&14  & \tot \\
@risersBy@ wrong postcond.     & 3&14 & 19&70 & \tot \\
@shrink@ too lazy              & \tol & 7&16  & \tot \\
@head@ CF                      & 0&04 & 0&12  & 0&06 \\
@head@ CF with wrong precond.  & 0&37 & 0&44  & \tot \\

\end{satrestab}
\end{center}
\caption{Theorem prover running time in seconds on some of the problems in the test suite
  on contracts that \emph{does not} hold.
  Without using min we only got timeouts so it is not listed.
  Vampire had only timeouts and is not listed.
  }
  \label{fig:satres}
\end{figure}

% Z3 actually kills this :p


We have run @halo@ on a collection of mostly-small tests, 
some of which can be 
viewed in Figure~\ref{fig:unsres}. Our full testsuite and tables can be downloaded from
\url{https://github.com/danr/contracts/blob/master/tests/BigTestResults.md}.
\dv{Is this right? How many examples did we evaluate on? 10? 20? 300? is it important?}
The test cases include:
\begin{itemize*}
  \item Crash-freedom of standard functions
    (@(++)@, @iterate@, @concatMap@).

  \item Crash-freedom of functions with more complex recursive patterns
        (Ackermann's function, functions with accumulators).

%  \item A library for predicate logic terms with some smart constructors
%        retaining an invariant,
%
%        % \dr{Everything failed from this without min so I just skip it}

  \item Partial functions given appropriate preconditions
        (@foldr1@, @head@, @fromJust@).

  \item The @risers@ example from Catch~\citep{Mitchell:2008:PBE:1411286.1411293}.

  \item Some non-trivial post-conditions, such as the example above with @filter@ and @all@,
        and also $@any p xs || any p ys@ = @any p (xs ++ ys)@$.
\end{itemize*}

We tried four theorem provers, Equinox, Z3, Vampire and E, and gave
them 60 seconds for each problem. For our problems, Z3 seems to be the most
successful. To give an idea of the sizes of the FOL problems, we additionally 
include the number of FOL axioms that are associated with each verification task in 
Figure~\ref{fig:unsres}. This number ranges between 35-180 but does not appear to be 
directly associated to the running time.


\section{Discussion}\label{sect:discussion}
  \paragraph{Contracts that do not hold}
\label{ssect:countersat}

In practice, programmers will often propose contracts that do not hold. Our tool 
will generate the necessary formulae and search for a counter-model. When such a model 
exists, it will include tables for the function symbols in the formula. Recall that
functions in FOL are total over the domain of the terms in the model. This means that function 
tables may be {\em infinite} if the terms in the model are infinite. Several (very useful!) 
axioms such as the discrimination axioms \textsc{AxDisjC} may in fact force the models to be 
infinite. For instance consider the following definitions:
\begin{code}
  length [] = Z
  length (x:xs) = S (length xs)

  isZero Z = True
  isZero _ = False
\end{code}
Suppose that we would like to check that 
   \[ @length@ \in \CF -> \{ x \mid @isZero@\;x\} \]
which is false.  A satisfiability-based checker 
will simply diverge trying to construct a counter model for the negation of the above query; we 
have confirmed that this is indeed the behaviour of several tools (Z3, Equinox, Eprover).
Indeed the table for @length@ is infinite since @[]@ is always disjoint from @Cons x xs@ for 
any @x@ and @xs@. Even if there is a finitely-representable infinite model, 
the theorem prover may search forever in the ``wrong corner'' of the model for a 
counterexample. 

From a practical point of view this is unfortunate; it is not
acceptable for the checker to loop when the programmer writes an
erroneous contract.
Tantalisingly, there exists a very simple 
counterexample, e.g. @[Z]@, and that single small
example is all the programmer needs to see the falsity of the contract.

Addressing this problem a challenging (but essential) 
direction for future work, and we are currently 
working on a modification of our theory that admits the denotational model, but 
also permits {\em finite models} generated from counterexample traces.
%% These ideas are reminiscent to the techniques that the Nitpick \dv{IS this right?} tool
%% uses for generating finite counterexamples in Isabelle. \dv{Someone please check!}.
If the theory can guarantee the existence of a finite model in case of a counterexample,
a finite model checker such as Paradox~\cite{paradox} will be able find it.

%% It is obviouly unacceptable for the system to go into a loop if
%% the programmer writes a bogus contract, and we have promising
%% preliminary results based on so-called ``minimisation'', and
%% finite counter-model generators such as Paradox \cite{koen}, but we
%% leave this for (absolutely essential) future work.

\paragraph{A tighter correspondence to operational semantics?}

From computational adequacy, Theorem~\ref{thm:adequacy} we can easily state
the following theorem: 
\begin{corollary} Assume that $e$ and $\Ct$ contain no term variables and 
assume that $\ctrans{}{\cdot}{e \in \{x \mid e_p\}} = \formula{\phi}$. It is the case 
that $\langle D_\infty,{\cal I}\rangle \models \phi$ if and only iff either
$P \not|- e \Downarrow$ or $P \not|- e_p[e/x] \Downarrow$ or $P |- e_p[e/x] \Downarrow \True$. \end{corollary}

Hence, the operational and the denotational semantics of base predicate contracts coincide.
It is interesting to see whether our shift to denotational semantics -- at least compared
to previous work on checking Haskell contracts~\cite{xu+:contracts} -- has any other consequences? 
Does the above correspondence break for higher-order contracts?

The answer is that this precise correspondence indeed breaks. 
Recall that the operational definition of contract satisfaction for a contract $\Ct_1 -> \Ct_2$ is the following: 
\[\begin{array}{l} 
   e \in (x{:}\Ct_1) -> \Ct_2 \text{ iff} \\
   \text{for all } e' \text{ such that } (e' \in \Ct_1) \text{ it is } e\;e' \in \Ct_2[e'/x]
\end{array}\] 
The denotational one requires that for all denotations $d'$ such that
$d' \in \dbrace{\Ct_1}$ it is the case that 
$\dapp(\dbrace{e},d') \in \dbrace{\Ct_2}_{x |->d'}$. 
 
Alas there are {\em more} denotations than images of terms in $D_{\infty}$ and the correspondence breaks. Consider the program:
\[\begin{array}{lcl}
f_\omega = f_\omega \\
f (b{:}Bool) (h{:}Bool->Bool->Bool) = \\ 
\begin{array}{lll}
  &     & \quad @if@\;(h\;True\;b)\;\&\&\;(h\;b\;True)\;\&\& \\ 
  &     & \quad\qquad\qquad not\;(h\;False\;False)\;@then@ \\
  &     & \quad\quad @if@\;(h\;True\;f_\omega)\;\&\&\;(h\;f_\omega\;True)\;@then@\;@BAD@ \\
  &     & \quad\quad @else@\;True \\
  &     & \quad @else@\;True
\end{array}
\end{array}\]
Also consider now the candidate contract for $f$ below (and assume that we have managed to equate the denotational and the operational definitions of crash-freedom):
\[ \CF -> (\CF -> \CF -> \CF) -> \CF \]
Operationally we may assume a crash-free boolean as well as a function $h$ which is 
$\CF -> \CF -> \CF$. The first conditional ensures that the function behaves like an ``or'' function or 
diverges. However if we pass the first conditional, 
the second conditional will always diverge and hence the contract will be satisfied. 

However, denotationally it is possible to have a {\em monotone} function $por$ defined as follows (for convenience, 
we are using pattern matching notation instead of our language of domain theory combinators):
\[\begin{array}{lcl}
  por\;\bot\;\bot & = & \bot \\ 
  por\;\bot\;True & = & True \\
  por\;True\;\bot & = & True \\ 
  por\;False\;False & = & False
\end{array}\] 
The rest of the equations (for @BAD@ arguments) are induced by monotonicity and we may pick whatever boolean value 
we like when both arguments are @BAD@. 

Now, this is denotationally a $\CF -> \CF -> \CF$ function, and it will pass the first conditional, but it will
also pass the second conditional, yielding @BAD@. Hence denotationally the contract for $f$ {\em does not hold}.

So we have a concrete case where an expression may satisfy its
contract operationally but not denotationally, because there are more
tests than programs in the denotational world. Due to contra-variance
we expect that the other inclusion will fail too. This is not a catastrophic 
problem for our reasoning principles about programs -- after all the two 
definitions will mostly coincide and they will precisely coincide in the base case. 
In the end of the day we will be interested on whether a program crashes or not and 
if we have proven that it is crash-free denotationally, it is definitely crash-free in 
any operationally reasonable term. 

Finally, was it possible to define an operational model for our FOL theory that interpreted
equality as contextual equivalence? Probably this could be made to work, although we believe
that the formal clutter from syntactic manipulation of terms could be worse than the current
denotational approach. 


\paragraph{Polymorphic crash-freedom}

Observe that our axiomatisation of crash-freedom in Figure~\ref{fig:prelude} 
includes only axioms for data constructors. In fact, our denotational interpretation
$\Fcf^{\infty}$ allows more axioms, such as:
\[\begin{array}{l}
    \forall x y @.@ \lcf{x} \land \lcf{y} => \lcf{app(x,y)}
\end{array}\] 
This axiom is useful if we wish to give directly a $\CF$ contract to a value of 
arrow type. For instance, instead of specifying that @map@ satisfies the contract
$(\CF -> \CF) -> \CF -> \CF$ one may want to say that it satisfies the contract
$\CF -> \CF -> \CF$. With the latter contract we need the previous axiom to be 
able to apply the function argument of @map@ to a crash-free value and get a 
crash-free result. 

In some situations, the following axiom might be beneficial as well:
\[\begin{array}{l}
    (\forall \xs @.@ \lcf{f(\xs)}) => \lcf{f_{ptr}}
\end{array}\]
If the result of applying a function to any possible argument is crash-free then 
so is the function pointer. This allows us to go in the inverse direction as before, 
and pass a function pointer to a function that expects a $\CF$ argument. However notice
that this last axiom introduces a quantified assumption, which might lead to significant
efficiency problem.

On a final note, observe that the variation of the latter axiom for the 
$app(\cdot,\cdot)$ function
\[\begin{array}{l}
   (\forall x @.@ \lcf{app(y,x)}) => \lcf{y}
\end{array}\]
is {\em not} valid in the denotational model. For instance consider the
value $\injK{K}{\injBad}$ for $y$. The left-hand side is going to always 
be true, as the application will yield $\bot$, but $y$ is not crash-free 
itself.




\section{Related work}\label{sect:related}
  There are practically very few tools for the automatic 
verification of {\em lazy and higher-order} functional programs.
Furthermore, our approach of translating directly the denotational semantics of 
programs is not something that appears to be well-explored in the literature. 
Below we present some related work on the verification of functional programs. 

Catch~\cite{Mitchell:2008:PBE:1411286.1411293} is one of the very few tools that 
have been evaluated in large scale and address the verification of lazy Haskell 
programs. Using static analysis, Catch can detect pattern match failures, and hence 
prove that a program cannot crash. Some specification which describes the set of 
constructors that are expected as arguments to each function might be required. 
necessary for the analysis to succeed. Our aim with this work is to achieve similar and
moreove be in a position to also assert functional correctness specification.

Liquid Types~\cite{Rondon:2008:LT:1375581.1375602} has been an influential 
approach to call-by-value functional program verification. In Liquid Types, 
contracts are written as refinements in a fixed language of predicates (which may 
include recursive predicates) and the extracted conditions are discharged using an 
SMT-solver. Because the language of predicates is fixed, predicate abstraction can 
very effectively {\em infer} precise refinements, even for recursive functions, and 
hence the annotation burden is very low. In our case, since the language of predicates
is the very same programming language with the same semantics {\em by design}, inference
of function specifications is harder. The other important deviation is that liquid types
requires all {\em uses} of a function to satisfy its precondition whereas in the semantics
that we have chosen, bad uses are allowed but the programmer gets no guarantees back.
\dv{Todo: Andrey Rybalchenko ``sausage factory''}

Rather different to Liquid Types, allowing refinements to be written 
in the very same programming language that programs are written, is the 
Dminor~\cite{Bierman+:subtyping} approach. Contrary to our case however, in Dminor
the expressions that refine type must be pure, that is, terminating and having a unique
denotation (e.g. not dependent on the store). Driven from a typing relation that includes
logic entailment judgements, verification conditions are extracted and discharged automatically using Z3. 
Similar in spirit, other dependent type systems such 
as Fstar~\cite{fstar} extract verification conditions which can be discharged 
using automated tools 
or interactive theorem provers. Hybrid type systems such as Sage~\cite{Knowles+:sage}
will also attempt to prove as many of the goals statically and defer the rest as runtime
goals.

Boogie~\cite{boogie} is a verification back end which supports procedures as well as 
pure functions, and axiomatization of theories and could potentially be used as the 
backend of our translation as well. Boogie verify programs written in the BoogiePL 
programming languages using Z3. Recent work on performing induction on-top of an 
induction-free SMT solver~\cite{Leino:2012:AIS:2189257.2189278} proposes a ``tactic''
for encoding induction schemes as first-order queries, which is reminiscent of the way
that we propose to perform induction.

The recent work on the Leon system~\cite{Suter:2011:SMR:2041552.2041575} presents
an effective approach to the verification of {\em first-order} and {\em call-by-value} 
recursive functional programs which appears to be very efficient in practice: it works
by extending SMT with recursive programs and ``control literals'' that guide the pattern
matching search for a counter-model, and is guaranteed to find a model if one exists 
(whereas that is not the case in our system, as we discussed earlier). It does not include
a $\CF$-analogous predicate and is call-by-value, though no special treatment of the $\bot$ 
value nor pattern match failures seem to be in the scope of that project, which, unsurprisingly
yields a very fast verification framework for partial functional correctness. 

First-order logic has been used as a target for higher-order languages in other verification contexts as well.
The interactive theorem prover Isabelle for many years has had the opportunity to use
automated first-order provers to discharge proof obligations. This work has recently culminated into the tool
Sledgehammer \cite{Sledgehammer}, which not only uses first-order provers, but also SMT solvers as back-ends.
There has been a version of the dependently typed programming language Agda in which
proof obligations could be sent to an automatic first-order prover \cite{AgdaFOL}. Both of these use typed translations from a typed higher-order language of well-founded definitions to first-order logic. The work in this area that perhaps comes closest to ours in that they deal with a lazy, general recursive language with partial functions is by \citet{TypeTheoryFOL}, who use Agda as a logical framework to reason about general recursive functional programs, and combine interaction in Agda with automated proofs in first-order logic.

The previous work on static contract checking for Haskell~\cite{xu+:contracts} 
was based on {\em wrapping}. A term was effectively wrapped
with an appropriately pushed contract test, and symbolic exection or aggressive inlining was used to show that @BAD@ values could
never be reached in this wrapped term. 
In follow-up work, Xu~\cite{Xu:2012:HCC:2103746.2103767} proposes a variation for a 
{\em call-by-value} language this time which performs symbolic execution alongside with
keeping a ``logicization'' of the program which can be used to eliminate paths that can
provably not generate @BAD@ value using a theorem prover. The ``logicization'' of a 
program has a similar spirit to our translation to logic but it is unclear what is the
model that is being used to prove the soundness of this translation, nor its 
axiomatization. Furthermore the logicization of programs is dependent on if 
the resulting formula is going to be used as a goal or assumption in a proof. We believe
that the direct approach proposed in this paper, which is to directly encode the semantics
of programs and contracts might be simpler. That said, symbolic execution as proposed
in~\cite{Xu:2012:HCC:2103746.2103767} has the significant advantage of querying a 
theorem prover on many small goals as symbolic execution proceeds, instead of a 
single verification goal in the end. We have some ideas about how to break large 
contract negation queries to smaller ones, following the symbolic evaluation of 
a function, and we plan to integrate this methodology in our tool as well.

%% \begin{itemize}
%%   \item Contracts in general (Findler Felleisen etc)
%%   \item Xu's 2009
%%   \item Xu's PEPM 2012: Very related
%%   \item Minimization/finite models? Isabelle? (Jasmin's thesis?)
%%   \item Yann Regis-Giannas 
%%   \item Xeno (equalities), Hipspec 
%%   \item Higher-order model checking
%%   \item Triggers
%%   \item Our approach is reminiscent of appraches from the 80's/90's but which?
%%   \item Treatment of @BAD@ as in Extensible Extensions paper (maybe just a comment is neededed inline)
%%   \item More stuff that Koen knows about??????????
%% \end{itemize}
 


\section{Future work}\label{sect:future}
  % Integers
% SMT 2.0
% Printing countermodels
% (Typeclasses)

There are several avenues for future work.  In terms of our
implementation, we would like to add support for primitive data types,
such as @Integer@, using theorem provers such as @T-SPASS@ to deal
with the @tff@ (typed first-order arithmetic) extension TPTP. Another
way is to also generate theories in the SMT 2.0 format, understood by
Z3, which has support for integer arithmetic and more. As mentioned in
Section~\ref{ssect:countersat} we have ideas how to generate finite
counter examples for contracts that do not hold, and how they should
be presented to the user. It would also be interesting to see if
\emph{triggers} in SMT 2.0 could also be used to support that goal.
Another important direction is finding ways to split our big
verification goals into smaller ones that can be proven significantly
faster. Finally, we would like to investigate whether we can
automatically strengthen contracts to be used as induction hypotheses
in inductive proofs, deriving information from failed attempts.

%A way of presenting countermodels given by @paradox@ in an easily
%understandable way for the user would be helpful.
% quote Reasoning with Triggers?


\paragraph{Acknowledgements}
Thanks to Richard Eisenberg for some helpful feedback.

\bibliographystyle{plainnat}
\bibliography{hcc-popl}

\appendix 

\section{A finite model construction}\label{sect:finite-model-proof}


We now present the proof of Theorem~\ref{thm:finite-model}.
Consider the alternative presentation of $\SDownarrow$ below.
%% \begin{figure}\small
\[\begin{array}{c}
\ruleform{P |- u \SDownarrow v} \\ \\
\prooftree
\begin{array}{c}
P |- e_1 \SDownarrow f^\ar[\taus]\;\oln{e}{\ar-1} \quad
(f |-> \Lambda\as @.@ \oln{x{:}\tau}{\ar} @.@ u) \in P \\
P |- u[\ol{e},e_2/\xs] \SDownarrow w \quad \highlight{P |- e_2 \SDownarrow v_2}
\end{array}
-------------------------------------{EValF}
P |- e_1\;e_2 \SDownarrow w 
~~~~~
 \ar \geq 1
----------------------------{FVal}
P |- f^\ar[\taus] \SDownarrow f 
~~~~ 
(f{|->}\Lambda\as @.@ u) \in P \quad P |- u{\SDownarrow}v
-------------------------------------{FCaf}
P |- f^0[\taus] \SDownarrow v 
~~~~~
\begin{array}{c}
P |- e_1 \SDownarrow f^\ar[\taus]\;\oln{e}{< \ar-1} \quad
\highlight{P |- e_2 \SDownarrow v_2 }
\end{array}
-------------------------------------{EValP}
P |- e_1\;e_2 \SDownarrow f^\ar[\taus]\;\oln{e}{< \ar-1}\;e_2
~~~~~
\begin{array}{c} 
P |- e_1 \SDownarrow @BAD@ \quad
\highlight{P |- e_2 \SDownarrow v_2}
\end{array}
------------------------------------------------{EBadApp}
P |- e_1\;e_2 \SDownarrow @BAD@
~~~~~ 
P |- \highlight{\ol{e} \SDownarrow \ol{v}}
-------------------------------------{EValC}
P |- K[\taus](\ol{e}) \SDownarrow K[\taus](\ol{e})
~~~~
\phantom{G}
-------------------------------------{EValB}
P |- @BAD@ \SDownarrow @BAD@
\endprooftree \\ \\ 
\text{ ... plus rules for @case@ ... } 
\end{array}\]
%% \caption{Semi-strict operational semantics}\label{fig:opsem-semi}
%% \end{figure}

Let us consider a derivation tree $D_{0} @::@ P e \SDownarrow v$ and let us consider the set:
\[ S_0 = \{ e' \mid \exists D @.@ D @::@ P |- e'{\SDownarrow}w \text{ and } D{\sqsubseteq}D_0 \} \cup \{ @BAD@ \} \]
where with the notation $D_1 \sqsubseteq D_2$ we mean that $D_1$ is a sub-derivation of 
the derivation $D_2$. 

Let us create the set $S^{min}$ as follows:
\[         S^{min} = S_0 \cup \{ \bot \} \] 
where $\bot$ is a distinguished element. We refer to the elements of $S^{min}$ as $\mu$ (either and $e$ or $\bot$).
%% The edges of the graph are created with the following 
%% two rules:
%% \begin{itemize*}
%%    \item For every derivation rule of the form 
%%          \[\begin{array}{c}\prooftree
%%                D_1 @::@ e_1 \SDownarrow v_1 \ldots D_n @::@ e_1 \SDownarrow v_1
%%                ---------------------------------------{}
%%                D @::@ e \SDownarrow v  
%%          \endprooftree\end{array}\] 
%%          in $D_{0}$ then we add edges that connect $D$ to each of $e_1 \ldots e_n$.
%%    \item For every application node $D @::@ e_1\;e_2$ we add a {\em directed edge} from 
%%          the node to the node of $e_2$ (which, by the evaluation relation $\SDownarrow$ must also exist
%%          in the graph and by determinacy of $\SDownarrow$ is unique).
%%    \item For every data constructor node $D @::@ K[\taus](\ol{e})$ we add {\em labelled directed edges}
%%          from the node to the node of each $e_i \in \ol{e}$. The labels on the edges are just the indices.
%% \end{itemize*}

Consider the following equivalence relation on elements of $S^{min}$: 
\[\begin{array}{c}\prooftree
            P |- e \SDownarrow v
      ---------------------------{EqEval}
             e \equiv v 
      ~~~~ 
      \begin{array}{c}
           e_1 \equiv e_1' \quad e_2 \equiv e_2' 
      \end{array}
      ---------------------------------------------------{AppCong}
          e_1\;e_2 \equiv e_1' e_2' 
      ~~~~~ 
           e_i \equiv e_i'
      ---------------------------------------------------{ConCong}
           K[\taus](\ol{e}) \equiv K[\taus'](\ol{e}')
      ~~~~~
         \phantom{e}
      ---------------------------------------------------{Refl}
        \mu \equiv \mu
      ~~~~ 
        e_1 \equiv e_2
      ------------------------{Sym}
        e_2 \equiv e_1
      ~~~~\hspace{-5pt}
        e_1 \equiv e_2 \;\; e_2 \equiv e_3
      \hspace{-2pt}------------------------{Trans}
        e_1 \equiv e_3
\endprooftree\end{array}\]

The following lemma is true:
\begin{lemma}\label{lem:equiv-shapes} If $e_1 \equiv e_2$ then:
\begin{itemize*}
  \item If $P |- e_1 \SDownarrow f^\ar[\taus]\;\oln{e_1}{m}$ with $m < \ar$ then $P |- e_2 \SDownarrow f^\ar[\taus]\;\oln{e_2}{m}$ and $e_{1i} \equiv e_{2i}$.
  \item If $P |- e_1 \SDownarrow K[\taus](\ol{e_1})$ then $P |- e_2 \SDownarrow K[\taus'](\ol{e_2})$ and and $e_{1i} \equiv e_{2i}$.
\end{itemize*}
\end{lemma}
In what follows we will use the set $S^{min}$ as the carrier set of our first-order model, we will 
interpret equality as $\equiv$ and the first-order language of our signature as follows.


\[\setlength{\arraycolsep}{2pt}
\begin{array}{rcl}
   \mlinterp{f^{\ar}}(\mu_1,\ldots,\mu_n) & = & 
       \multicolumn{1}{l}{\text{If there exists } (e_1\;e_2) \in S^{min}} \\
   & & \multicolumn{1}{l}{\text{such that } e_1{\SDownarrow}f\;[\taus]\;\oln{e}{\ar-1}} \\
   & & \multicolumn{1}{l}{\text{and } \mu_i \equiv e_i} \\
   & & \multicolumn{1}{l}{\text{and } \mu_{\ar} \equiv e_2 \text{ then } (e_1\;e_2) \text{ else } \bot} \\
   \mlinterp{f_{ptr}} & = & \text{If there exists } f \in S^{min} \\
                        &   & \text{then } f \text{ else } \bot \\
  \mlinterp{app}(\mu_1,\mu_2) & = & \text{If there exists } (e_1\;e_2) \in S^{min} \\ 
                         &   & \text{such that } \mu_1 \equiv e_1 \\ 
                         &   & \text{and } \mu_2 \equiv e_2 \text{ then } (e_1\;e_2) \text{ else } \bot \\
  \mlinterp{K^\ar}(\mu_1,\ldots,\mu_\ar) & = & \text{If there exists } (K[\taus](\ol{e})) \in S^{min} \\
                                    &  & \text{such that } \mu_i' \equiv e_i \text{ then } (K[\taus](\ol{e})) \\
                                    &  & \text{else } \bot \\
  \mlinterp{\sel{K}{i}}(\mu) & = & \text{If there exists } (K[\taus](\ol{e})) \in S^{min} \\ 
                             &   & \text{such that } \mu \equiv (K[\taus](\ol{e})) \text{ then } e_i \\ 
                             &   & \text{else } \bot \\ 
%% \dapp(\dbrace{f},\oln{d}{\ar}) \\ 
%%    \linterp{app}(d_1,d_2)     & = & \dapp(d_1,d_2) \\
%%    \linterp{f_{ptr}}  & = & \dbrace{f} \\
%%    \linterp{K^{\ar}}(d_1,\ldots,d_\ar) & = & \roll(\ret(\inj{K}\langle d_1,\ldots,d_\ar\rangle)) \\ 
%%    \linterp{\sel{K}{i}}(d) & = &  \roll(\bind_g(\unroll(d))) \\ 
%%      \text{where } g  & = & [\;\bot \\ 
%%                       &   & ,\;\dlambda d @.@ \unroll(\pi_i(d))  \quad (\text{case for K}) \\ 
%%                       &   & ,\;\bot \\
%%                       &   & ,\;\ldots\\
%%                       &   & ,\;\bot\; ] \\
  \mlinterp{bad}       & = & @BAD@ \\
  \mlinterp{unr}       & = & \bot \\ \\ \\ 
  \mlinterp{min}(\mu)  & = & \mu \neq \bot \\ 
  \mlinterp{cf}(\mu)   & = & \text{There exists } e \text{ such that } \mu = e \\ 
                       &   & \text{and } \dbrace{e} \in F_\lcfZ^{\infty} 
\end{array}\]

First of all, we have to prove that $\mlinterp{\cdot}$ above is a function and not a relation. 
This is easy to do by observing that the interpretation can only return 
different terms that are nevertheless equated by $\equiv$. Secondly, we must prove that the interpreted 
functions are congruent over $\equiv$. That is also straightforward.

\begin{lemma}\label{lem:min-interp}
If $e \in S^{min}$ and $\etrans{}{\cdot}{e} = t$ then $\mlinterp{t} \equiv e$.
\end{lemma}
\begin{proof} Straightforward induction over the structure of $e$ and by observing 
that in any term $e \in S^{min}$, all structural subterms of $e$ are also in $S^{min}$.
\end{proof}

Moreover, we can show that $\langle S^{min},{\cal I}^{min}\rangle$ is a model of $\ThMin$, by showing that
it validates all the axioms. 
\begin{theorem}
$\langle S^{min},{\cal I}^{min}\rangle \models \ThMin$.
\end{theorem}
\begin{proof} We have to show that each of the axioms in $\ThMin$ is valid in this model. The only interesting 
axioms is \rulename{AxAppMin}:
\[ \forall \ol{x} @.@ min(app(f_{ptr},\xs)) => f(\ol{x}) = app(f_{ptr},\xs) \]
Let us pick elements $\mu_1\ldots\mu_n \in S^{min}$. Since $min(app(f_{ptr},\xs))$ it must be that they are actually
all expressions, $e_1\ldots e_n$. Moreover we have the following chain for the left-hand-side: 
\[\begin{array}{lcl}
     f & \equiv & e_1^{\star} \\ 
     e^{\star}_1\;e_1 & \equiv & e^{\star}_2 \\ 
     e^{\star}_2\;e_2 & \equiv & e^{\star}_3 \\ 
              & \ldots & 
\end{array}\] 
Such that the left-hand-side is $e^{\star}_n\;e_n$. Now, by Lemma~\ref{lem:equiv-shapes} 
we have that $e^{\star}_1 \SDownarrow f$, $e^{\star}_2 \SDownarrow f\;e_1'$ for some $e_1 \equiv e_1'$ and so on, 
so that eventually we have that $e^{\star}_n \SDownarrow f\;e_1'\ldots e_{n-1}'$ for equivalent $e_i \equiv e_i'$. 
Applying the interpretation of $f$ for the right-hand-side finishes the case.
\end{proof} 


\end{document}

%% \begin{abstract}
%% The Glasgow Haskell Compiler is an optimizing
%% compiler that expresses and manipulates first-class equality proofs in
%% its intermediate language.  We describe a simple, elegant technique that
%% exploits these equality proofs to support \emph{deferred type errors}.
%% The technique requires us to treat equality proofs as possibly-divergent
%% terms; we show how to do so without losing either soundness or
%% the zero-overhead cost model that the programmer expects.
%% \end{abstract}

%% \category{D.3.3}{Language Constructs and Features}{Abstract data types}
%% \category{F.3.3}{Studies of Program Constructs}{Type structure}

%% \terms{Design, Languages}

%% \keywords{Type equalities, Deferred type errors, System FC}

\section{Denotational semantics}


%% \begin{lemma}[Evaluation preserves equality]
%% If $\Sigma;\cdot |- e : \tau \rightsquigarrow t$ and 
%%    $\Sigma |- D \rightsquigarrow \phi_{\Sigma,D}$ and 
%%    $D |- e \Downarrow w$ then
%%    $\Sigma;\cdot |- w : \tau \rightsquigarrow s$ and $\Th{\Sigma}{D} /\ \phi_{\Sigma,D} |- t = s$.
%% \end{lemma}
%% \begin{proof} By induction on the evaluation $\Sigma |- e \Downarrow w$. \end{proof}


%% \begin{lemma}[Logic deduces sound value equalities]
%% Assume that $\Sigma;\cdot |- w : \tau \rightsquigarrow t$ and 
%% $D |- value(w)$ and $\Sigma |- D \rightsquigarrow \phi_{\Sigma,D}$. 
%% Then
%% \begin{enumerate*} 
%%   \item If $\Th{\Sigma}{D} /\ \phi_{\Sigma,D} |- t = \bad$ then $w = @BAD@$.
%%   \item If $\Th{\Sigma}{D} /\ \phi_{\Sigma,D} |- t = K(\ol{t})$ then $w = K[\taus](\ol{e})$, such 
%%         that $\Sigma;\cdot |- \ol{e : \tau} \rightsquigarrow \ol{s}$, and $\Th{\Sigma}{D} /\ \phi_{\Sigma,D} |- \ol{t = s}$.
%%   \item $\Th{\Sigma}{D} /\ \phi_{\Sigma,D} |- t \neq \unr$.
%% \end{enumerate*}
%% \end{lemma}
%% \begin{proof}
%% The proof of all three cases is by inversion on the $D |- value(w)$ derivation, 
%% apealling to the disjointness axioms.
%% %% \begin{enumerate*}
%% %%   \item By inversion on the $D |- value(w)$ derivation. In the case of \rulename{VBad} we are done.
%% %%   The case of \rulename{VFun} cannot happen, by the axiom set \rulename{AxPartB}. The case of \rulename{VCon} 
%% %%   cannot happen either: If the application is saturated then \rulename{AxDisjC} shows it is impossible; if it
%% %%   is not saturated we can always extend it and use \rulename{AxAppC} and \rulename{AxDisjC}. 
%% %%   \item Again by inversion on $D |- value(w)$ derivation. The case of \rulename{VBad} is easy. The case for 
%% %%   \rulename{VCon} follows by injectivity of constructors. The case of \rulename{VFun} can't happen by 
%% %%   \rulename{AxPartB}.
%% %%   \item Direct inversion on $D |- value(w)$, and using disjointness axioms.
%% %% \end{enumerate*} 
%% \end{proof}
 
%% Basic soundness will be stated as follows.
%% \begin{theorem}
%% If we have that
%% \begin{enumerate*} 
%%   \item $\Sigma;\cdot |- e : \tau$ and $\Sigma;\cdot |- \Ct : \tau$
%%   \item $\Sigma |- D \rightsquigarrow \phi_{\Sigma,D}$
%%   \item $\Sigma;\cdot |- e \in \Ct \rightsquigarrow \phi$
%% \end{enumerate*}
%% and $\Th{\Sigma}{D} /\ \phi_{\Sigma,D} /\ \neg \phi$ is unsatisfiable then $\Sigma;D |- e \in \Ct$.
%% \end{theorem}
%% \begin{proof}
%%  {\bf TODO}
%% \end{proof}

%% A remark: a formula $\phi$ is unsatisfiable iff $\neg \phi$ is valid in FOL. Hence, if 
%% $\Th{\Sigma}{D} /\ \phi_{\Sigma,D} /\ \neg \phi$ is unsatisfiable then 
%% $\neg (\Th{\Sigma}{D} /\ \phi_{\Sigma,D}) \lor \phi$ must be valid, and by completeness of FOL, 
%% $\Th{\Sigma}{D} /\ \phi_{\Sigma,D} |- \phi$.  

\section{Denotational semantics as FOL models} 



%% \begin{figure}\small
%% \[\begin{array}{c} 
%% %% \ruleform{ \dtrans{\Sigma}{d} = \formula{\phi} } \\ \\
%% %% \prooftree
%% %%   \begin{array}{c}
%% %%   (f{:}\forall\oln{a}{n} @.@ \oln{\tau}{m} -> \tau) \in \Sigma \quad 
%% %%   \etrans{\Sigma}{\ol{a},\ol{x{:}\tau}}{e} = \formula{t}
%% %%   \end{array}
%% %%   -------------------------------------------------------------------{TFDef}
%% %%   \dtrans{\Sigma}{(f |-> \Lambda\oln{a}{n} @.@ \lambda\oln{x{:}\tau}{m} @.@ e)} =  \formula{ (\forall x @.@ f(\oln{x}{m}) = t) }
%% %%   ~~~~~ 
%% %%   \begin{array}{l}
%% %%   (f{:}\forall\oln{a}{n} @.@ \oln{\tau}{m} -> \tau) \in \Sigma \quad 
%% %%   \etrans{\Sigma}{\ol{a},\ol{x{:}\tau}}{e} = \formula{t} \\
%% %%   constrs(\Sigma,T) = \ol{K} \\
%% %%   \text{for each branch}\;(K\;\oln{y}{l} -> e') \\
%% %%   \quad \begin{array}{l}
%% %%            (K{:}\forall \cs @.@ \oln{\sigma}{l} -> T\;\oln{c}{k}) \in \Sigma \\
%% %%            \etrans{\Sigma}{\ol{a},\ol{x{:}\tau},\ol{y{:}\sigma[\taus/\cs]}}{e'} = \formula{ t_K }
%% %%         \end{array}
%% %%   \end{array}
%% %%   -------------------------------------------------------------------{TCaseDef}
%% %%   \begin{array}{l}
%% %%    \dtrans{\Sigma}{(f |-> \Lambda\oln{a}{n} @.@ \lambda\oln{x{:}\tau}{m} @.@ @case@\;e\;@of@\;\ol{K\;\ol{y} -> e'})} = \\
%% %%    \quad \formula{ \begin{array}{lll} \forall \oln{x}{m} @.@ & \hspace{-7pt} (t = \bad /\ f(\ol{x}) = \bad)\; \lor \\ 
%% %%                                                                     & \hspace{-7pt}(f(\ol{x}) = \unr)\;\lor \\ 
%% %%                                                                     & \hspace{-7pt}(\bigvee(t = K(\oln{{\sel{K}{i}}(t)}{i\in 1..l})\;/\ \\
%% %%                                                                     & \hspace{-5pt}\quad f(\ol{x}) = t_K[\oln{\sel{K}{i}(t)}{i\in 1..l}/\ol{y}]))
%% %%                                                  \end{array}
%% %%                         }
%% %% \end{array}
%% %% \endprooftree  \\ \\ 
%% \end{array}\]
%% \caption{Definition elaboration to FOL}\label{fig:typing}
%% \end{figure}




{\bf DV: So basically this is Simon's strategy of side-stepping the lack of full abstraction
and the associated problems with it: In the end of the day we only care about base contracts,
in fact really only about the contract ``is this program crash-free'', so we don't have to make
a big fuss about higher-order contracts and their operational semantics. We have to motivate it
carefully and also be clear that for the intellectually curious reader who really wants to know what statement we have proved for a function contract when the prover says ``unsat'' we might want to give a full definition of the denotational meaning of contracts including both base and higher-order. I think we do not have the time luxury to look for more elaborate solutions (such as definable denotations and all that crazy stuff) to match the operational and the denotational semantics for higher-order contracts. Fullstop.}


\section{Minimizing countermodels}



\section{Min as unreachable}

 In fact we may take
one step further and equate all the non-interesting values of the domain to $\bot$.

To achieve this effect, we update our Prelude theory axioms as follows:
{\small
\[\setlength{\arraycolsep}{1pt}
\begin{array}{c}
%% \ruleform{\Th{\Sigma}{P}} \\ \\ 
\begin{array}{lll} 
 \textsc{AxDisjA} & \formula{\bad \neq \unr}  \\ 
 \textsc{AxDisjB} & \formula{\forall \oln{x}{n}\oln{y}{m} @.@} \\ 
                  & \formula{\;\;\highlight{K(\ol{x}){\neq}\unr\;\lor\;J(\ol{y}){\neq}\unr} =>
                                  K(\ol{x}){\neq}J(\ol{y})} \\
                  & \text{ for every } (K{:}\forall\as @.@ \oln{\tau}{n} -> T\;\as) \in \Sigma \\ 
                  & \text{ and } (J{:}\forall\as @.@ \oln{\tau}{m} -> S\;\as) \in \Sigma \\
 %% \textsc{AxDisjCUnr} & \formula{\forall \oln{x}{n} @.@ \highlight{\neg min(\unr)}} \\ 
 %%                  & \text{ for every } (K{:}\forall\as @.@ \oln{\tau}{n} -> T\;\as) \in \Sigma \\ \\
 \textsc{AxDisjCBad} & \formula{\forall \oln{x}{n} @.@ K(\ol{x}) \neq \bad} \\ 
                  & \text{ for every } (K{:}\forall\as @.@ \oln{\tau}{n} -> T\;\as) \in \Sigma \\ \\

 \textsc{AxAppA}  & \formula{\forall \oln{x}{n} @.@ f(\ol{x}) = app(f_{ptr},\xs)} \\
                  & \text{ for every } (f |-> \Lambda\as @.@ \lambda\oln{x{:}\tau}{n} @.@ u) \in P \\
 %% \textsc{AxAppB}  & \formula{\forall \oln{x}{n} @.@ K(\ol{x}) = app(\ldots (app(x_K,x_1),\ldots,x_n)\ldots)} \\
 %%                  & \text{ for every } (K{:}\forall\as @.@ \oln{\tau}{n} -> T\;\as) \in \Sigma \\
 \textsc{AxAppC}  & \formula{\forall x, app(\bad,x) = \bad \; /\ \; app(\unr,x) = \unr}    \\ \\
 %% Not needed: we can always extend partial constructor applications to fully saturated and use AxAppC and AxDisjC
 %% \textsc{AxPartA} & \formula{\forall \oln{x}{n} @.@ app(\ldots (app(x_K,x_1),\ldots,x_n)\ldots) \neq \unr} \\
 %%                  & \formula{\quad\quad \land\; app(\ldots (app(x_K,x_1),\ldots,x_n)\ldots) \neq \bad} \\
 %%                  & \text{ for every } (K{:}\forall\as @.@ \oln{\tau}{m} -> T\;\as) \in \Sigma \text{ and } m > n \\
 %% \textsc{AxPartB} & \formula{\forall \oln{x}{n} @.@ app(f_{ptr},\xs) \neq \unr} \\
 %%                  & \formula{\quad\land\; app(f_{ptr},\xs) \neq \bad} \\
 %%                  & \formula{\quad\land\; \forall \oln{y}{k} @.@ app(f_{ptr},\xs) \neq K(\ol{y})} \\
 %%                  & \text{ for every } (f |-> \Lambda\as @.@ \lambda\oln{x{:}\tau}{m} @.@ u) \in P  \\
 %%                  & \text{ and every } (K{:}\forall\as @.@ \oln{\tau}{k} -> T\;\as) \in \Sigma \text{ and } m > n  \\ \\ 
 \textsc{AxInj}   & \formula{\forall \oln{y}{n} @.@ \highlight{K(\ys) \neq \unr\;\land\; y_i \neq \unr}} \\ 
                  & \formula{\quad\qquad\qquad => \sel{K}{i}(K(\ys)) = y_i} \\ 
                  & \text{for every } (K{:}\forall\as @.@ \oln{\tau}{n} -> T\;\as) \in \Sigma \text{ and } i \in 1..n \\ \\
 \textsc{AxCfA}   & \formula{\lcf{\unr} /\ \lncf{\bad}} \\
 \textsc{AxCfB1}  & \formula{\forall \oln{x}{n} @.@ \bigwedge\lcf{\ol{x}}} => \lcf{K(\ol{x})} \\
                  & \text{ for every } (K{:}\forall\as @.@ \oln{\tau}{n} -> T\;\as) \in \Sigma \\ 
 \textsc{AxCfB2}  & \formula{\forall \oln{x}{n} @.@ \lcf{K(\ol{x})}\;\highlight{\land\;K(\ol{x}) \neq \unr} => \bigwedge\lcf{\ol{x}}} \\
                  & \text{ for every } (K{:}\forall\as @.@ \oln{\tau}{n} -> T\;\as) \in \Sigma
\end{array}
\end{array}\]}


\begin{figure}\small
\[\begin{array}{c}
\ruleform{\utrans{\Sigma}{\Gamma}{t \sim u} = \formula{\phi}} \\ \\ 
\prooftree
   \begin{array}{c} \ \\ \ \\ 
   \etrans{\Sigma}{\Gamma}{e} = \formula{t}
   \end{array}
   ----------------------------------------{TUTm}
   \begin{array}{l} 
   \utrans{\Sigma}{\Gamma}{s \sim e } = \formula{(s = t) \lor \highlight{\neg min(s)}} \ \\ \ \\ \ \\ 
   \end{array}
   ~~~~~
  \begin{array}{l}
  \etrans{\Sigma}{\Gamma}{e} = \formula{t} \quad
  constrs(\Sigma,T) = \ol{K} \\
  \text{for each branch}\;(K\;\oln{y}{l} -> e') \\
  \begin{array}{l}
           (K{:}\forall \cs @.@ \oln{\sigma}{l} -> T\;\oln{c}{k}) \in \Sigma \text{ and }
           \etrans{\Sigma}{\Gamma,\ol{y}}{e'} = \formula{ t_K }
  \end{array}
  \end{array}
  ------------------------------------------{TUCase}
  {\setlength{\arraycolsep}{1pt} 
  \begin{array}{l}
  \utrans{\Sigma}{\Gamma}{s \sim @case@\;e\;@of@\;\ol{K\;\ol{y} -> e'}} = \\
  \;\;\formula{ \begin{array}{l} 
     \highlight{min(s)} => \\
     \begin{array}{ll}
          ( & \highlight{min(t)}\;\land \\
            & (t = \bad => s = \bad)\;\land \\ 
            & (\forall \ol{y} @.@ t = K_1(\ol{y}) => s = t_{K_1})\;\land \ldots \land \\
            & (t \neq \bad\;\land\;t \neq K_1(\oln{{\sel{K_1}{i}}(t)}{})\;\land\;\ldots => s = \unr) \\
          )
%% (t = \bad /\ s = \bad)\;\lor\;(s = \unr)\;\lor \\
%%                                 \quad      \bigvee(t = K(\oln{{\sel{K}{i}}(t)}{}) \land
%%                                            s = t_K[\oln{\sel{K}{i}(t)}{}/\ol{y}])
                   \end{array}
     \end{array}}
  \end{array}}
  %% {       \setlength{\arraycolsep}{2pt} 
  %% \begin{array}{l}
  %% \utrans{\Sigma}{\Gamma}{s \sim @case@\;e\;@of@\;\ol{K\;\ol{y}{->}e'}} = \\
  %% \;\;\formula{
  %%      \begin{array}{l} (\highlight{s{=}\unr})\;\lor \\ 
  %%                           \;\; (\highlight{min(s) => min(t)}\;\land  \\
  %%                           \quad((t = \bad /\ s = \bad)\;\lor \\
  %%                           \quad\quad \bigvee(t = K(\oln{{\sel{K}{i}}(t)}{}) \land
  %%                                          s = t_K[\oln{\sel{K}{i}(t)}{}/\ol{y}])))
  %%                  \end{array}
  %%          }
  %% \end{array}}
\endprooftree
\end{array}\]
\caption{Minimality-enabled definition translation}\label{fig:min-def-trans-min}
\end{figure}



We will explain the modifications to the axiomatization in more detail in later sections.
%% In other words, we ensure that constructor applications are disjoint
%% only for values we are interested in. We will explain each axiom separately later. 
%% Intuitively we wish to equate all terms that we are not interested in to $\unr$. We 
%% can never be interested in $\unr$ in the intended model because that means that during
%% the evaluation of a term, which completed, we encountered a divergent term -- clearly a 
%% contradiction!
What about function definitions? Figure~\ref{fig:etrans} has to be modified slightly as well, 
as Figure~\ref{fig:min-def-trans} shows.

\begin{figure}\small
\[\begin{array}{c}
\ruleform{\utrans{\Sigma}{\Gamma}{t \sim u} = \formula{\phi}} \\ \\ 
\prooftree
   \begin{array}{c} \ \\ \ \\ 
   \etrans{\Sigma}{\Gamma}{e} = \formula{t}
   \end{array}
   ----------------------------------------{TUTm}
   \begin{array}{l} 
   \utrans{\Sigma}{\Gamma}{s \sim e } = \formula{(s = t) \lor \highlight{s = \unr}} \ \\ \ \\ \ \\ 
   \end{array}
   ~~~~~
  \begin{array}{l}
  \etrans{\Sigma}{\Gamma}{e} = \formula{t} \quad
  constrs(\Sigma,T) = \ol{K} \\
  \text{for each branch}\;(K\;\oln{y}{l} -> e') \\
  \begin{array}{l}
           (K{:}\forall \cs @.@ \oln{\sigma}{l} -> T\;\oln{c}{k}) \in \Sigma \text{ and }
           \etrans{\Sigma}{\Gamma,\ol{y}}{e'} = \formula{ t_K }
  \end{array}
  \end{array}
  ------------------------------------------{TUCase}
  {\setlength{\arraycolsep}{1pt} 
  \begin{array}{l}
  \utrans{\Sigma}{\Gamma}{s \sim @case@\;e\;@of@\;\ol{K\;\ol{y} -> e'}} = \\
  \;\;\formula{ \begin{array}{l} 
     \highlight{s = \unr}\;\lor \\
     \begin{array}{ll}
          ( & \highlight{(t \neq \unr)}\;\land \\
            & (t = \bad => s = \bad)\;\land \\ 
            & (\forall \ol{y} @.@ t = K_1(\ol{y}) => s = t_{K_1})\;\land \ldots \land \\
            & (t = \bad\;\lor\;t = K_1(\oln{{\sel{K_1}{i}}(t)}{})\;\lor\;\ldots) \\ 
          )
%% (t = \bad /\ s = \bad)\;\lor\;(s = \unr)\;\lor \\
%%                                 \quad      \bigvee(t = K(\oln{{\sel{K}{i}}(t)}{}) \land
%%                                            s = t_K[\oln{\sel{K}{i}(t)}{}/\ol{y}])
                   \end{array}
     \end{array}}
  \end{array}}
  %% {       \setlength{\arraycolsep}{2pt} 
  %% \begin{array}{l}
  %% \utrans{\Sigma}{\Gamma}{s \sim @case@\;e\;@of@\;\ol{K\;\ol{y}{->}e'}} = \\
  %% \;\;\formula{
  %%      \begin{array}{l} (\highlight{s{=}\unr})\;\lor \\ 
  %%                           \;\; (\highlight{min(s) => min(t)}\;\land  \\
  %%                           \quad((t = \bad /\ s = \bad)\;\lor \\
  %%                           \quad\quad \bigvee(t = K(\oln{{\sel{K}{i}}(t)}{}) \land
  %%                                          s = t_K[\oln{\sel{K}{i}(t)}{}/\ol{y}])))
  %%                  \end{array}
  %%          }
  %% \end{array}}
\endprooftree
\end{array}\]
\caption{Minimality-enabled definition translation}\label{fig:min-def-trans}
\end{figure}

%% \\ \\ 
%% \ruleform{ \Dtrans{\Sigma}{P} = \formula{\phi}} \\ \\ 
%% \prooftree
%%      \begin{array}{l}       
%%        \text{for each} (f |-> \Lambda\oln{a}{n} @.@ \lambda\oln{x{:}\tau}{m} @.@ u) \in P \\ 
%%           \quad \utrans{\Sigma}{\ol{x}}{f(\ol{x}) \sim u} = \formula{\phi}
%%      \end{array}
%%      --------------------{TDefs}
%%      \Dtrans{\Sigma}{P} = \bigwedge_{P} \formula{\forall \ol{x} @.@ \phi}
%% \endprooftree 

Now operationally we may instrument the evaluation relation to keep track of the set of 
closed terms that appear during evaluation. The instrumented relation appears in 
Figure~\ref{fig:opsem-instrumented}. Observe that if $P |- e \Downarrow w \curly S$ then 
$S$ is a {\em finite set} of terms.


\begin{figure}\small
\[\begin{array}{c} 
\ruleform{P |- e \Downarrow v \curly S} \\ \\ 
\prooftree
\begin{array}{c} \ \\ 
\end{array}
%% \begin{array}{c}
%% (f |-> \Lambda\ol{a} @.@ \lambda\oln{x{:}\tau}{m} @.@ u) \in P \\
%% P |- e_1 \Downarrow f\;[\taus]\;\oln{e}{m-1} \curly S_1 \\ 
%% P |- u[\ol{\tau}/\ol{a}][\ol{e},e_2/\ol{x}] \Downarrow w \curly S
%% \end{array}
%% ------------------------------------{EBeta}
%% P |- e_1\;e_2 \Downarrow w 
  S = heads(v)
-------------------------------------{EVal}
P |- v \Downarrow v \curly S
~~~~
\begin{array}{c}
(f |-> \Lambda\ol{a} @.@ \lambda\oln{x{:}\tau}{m} @.@ u) \in P \\
P |- u[\ol{\tau}/\ol{a}][\ol{e}/\ol{x}] \Downarrow v \curly S_1 \\ 
S_2 = heads(f[\ol{\tau}]\;\oln{e}{m}) 
\end{array}
-------------------------------------{EFun}
P |- f[\ol{\tau}]\;\oln{e}{m} \Downarrow v \curly S_1 \cup S_2
~~~~~
\begin{array}{c}  
P |- e_1 \Downarrow v_1 \curly S_1 \quad
P |- v_1\;e_2 \Downarrow w \curly S_2
\end{array}
------------------------------------------------{EApp}
P |- e_1\;e_2 \Downarrow w \curly S_1 \cup S_2 \cup \{ e_1\;e_2 \}
~~~~~
\begin{array}{c}  
P |- e_1 \Downarrow @BAD@ \curly S 
\end{array}
------------------------------------------------{EBadApp}
P |- e_1\;e_2 \Downarrow @BAD@ \curly S \cup \{ e_1\;e_2 \} 
\endprooftree \\ \\ 
\ruleform{heads(e) = S} \\ \\ 
\begin{array}{lcl}
   heads(f\;[\ol{\tau}]) & = & \{ f\;[\ol{\tau}] \} \\
   heads(e_1\;e_2)       & = & \{ e_1\;e_2 \} \cup heads(e_1) \\
   heads(\_)            & = & \emptyset 
\end{array} \\ \\
\ruleform{P |- u \Downarrow v \curly S} \\ \\
\prooftree
P |- e \Downarrow v \curly S 
-------------------------------------{EUTm}
P |- e \Downarrow v \curly S 
~~~~~
\begin{array}{c}
P |- e \Downarrow K_i[\ol{\sigma}_i](\ol{e}_i) \curly S_1 \quad
P |- e'_i[\ol{e}_i/\ol{y}_i] \Downarrow w \curly S_2 
\end{array}
------------------------------------{ECase}
P |- @case@\;e\;@of@\;\ol{K\;\ol{y} -> e'} \Downarrow w \curly S_1 \cup S_2
~~~~~
\begin{array}{c}
P |- e \Downarrow @BAD@ \curly S \\
\end{array}
------------------------------------{EBadCase}
P |- @case@\;e\;@of@\;\ol{K\;\ol{y} -> e'} \Downarrow @BAD@ \curly S
%% \begin{array}{c}
%% (f |-> \Lambda\ol{a} @.@ \lambda\oln{x{:}\tau}{m} @.@ @case@\;e\;@of@\;\ol{K\;\ol{y} -> e'}) \in D \\
%% D |- e[\ol{\tau}/\ol{a}][\ol{e}/\ol{x}] \Downarrow @BAD@ \\
%% \end{array}
%% -------------------------------------{EBadCase}
%% D |- f[\ol{\tau}]\;\oln{e}{m} \Downarrow @BAD@
\endprooftree
\end{array}\]
\caption{Redex-instrumented operational semantics}\label{fig:opsem-instrumented}
\end{figure}

\subsection{The intended min-imal model}

Our goal is then going to be to establish the following result, stated in non-technical terms:
\begin{quote}
If there exists a counterexample to a contract, then the negation of the contract-translation
formula is satisfiable not only on $\langle D_\infty,{\cal I}\rangle$ but it also has a {\em finite} 
model. That finite model is a model of our minimality-enabled theory. 
\end{quote}

We start unfolding the story. For a given program $P$ in a signature $\Sigma$ we have already 
shown how to construct $D_\infty$ and how to give interpretations ${\cal I}$ to a first-order 
vocabulary. Let us assume that the program and signature contains a polymorpic $undefined$ 
function, for convenience $undefined |-> udefined$. This is a realistic assumption to make 
(e.g. it comes in the standard Haskell prelude).

Assume now that we are given a formula $\phi$ defined as: 
\[  \phi = \ctrans{\Sigma}{\cdot}{e \in \Ct_1 -> \ldots \Ct_n -> @B@} \] 
for @B@ a base contract. Assume moreover that there exist $\oln{e}{n}$, closed for the
program $P$, such that for each $e_i$ it is true that:
\[\interp{\Ct_i}{\dbrace{P}^\infty}{\cdot}(\interp{e_i}{\dbrace{P}^\infty}{\cdot})\]. 
Assume however that it is {\em not} the case that
\[\interp{{\tt B}}{\dbrace{P}^\infty}{\cdot}(\interp{e\;\oln{e}{n}}{\dbrace{P}^\infty}{\cdot})\]
There are two cases for the base constract @B@:
\begin{itemize}
  \item Let us now consider the case where @B@ = $\{ x \mid e_p \}$. By adequacy it must
  be that: $P |- e\;\ol{e} \Downarrow w \curly S_1$ for some $w$ and set $S_1$ and moreover
  $P |- e_p[e\;ol{e}/x] \Downarrow \{ @BAD@, False \} \curly S_2$ for some set $S_2$. 

  Of course the following lemma is true:
  \begin{lemma}\label{lem:curly} 
    If $P |- e \Downarrow w \curly S$ then $S$ is a finite set. Moreover, 
    for every $e' \in S$ there exists $w$ such that $P |- e' \Downarrow w$.
  \end{lemma}
  Moreover we have:
  \begin{lemma}\label{lem:bot-not-redex} 
     If $P |- e \Downarrow w \curly S$ then 
     $\bot \notin \interp{S}{\dbrace{P}^{\infty}}{\cdot}$. 
  \end{lemma}
  \begin{proof} If $\bot \in \interp{S}{\dbrace{P}^{\infty}}{\cdot}$ then there exists
  a term $e \in S$ such that $\interp{e}{\dbrace{P}^{\infty}}{\cdot} = \bot$. This means
  that $P |- e \not\Downarrow$ but that is a contradiction to $e \in S$ by 
  Lemma~\ref{lem:curly}.
  \end{proof}

  Let us now define the {\em minimal sets} operationally and denotationally:

  \[\begin{array}{lcl}
           M        & \triangleq & S_1 \cup S_2 \\
           {\cal M} & \triangleq & \interp{S_1\cup S_2}{\dbrace{P}^{\infty}}{\cdot}
  \end{array}\]
  Consider now the function $\mu : D_\infty -> D_\infty$ defined as: 
  \[\begin{array}{lcl} 
        \mu(d) & \triangleq & \left\{ \begin{array}{ll} 
                   d           & \text{when } \unroll(d) = \ret(\inj{bad}(1)) \\
                   d           & \text{when } d \in \Min \\ 
                   \bot        & \text{otherwise } 
                                      \end{array}\right.
  \end{array}\] 
  In other words $\mu(\cdot)$ conflates all the non-interesting values to $\bot$. 
  Now we may consider the {\em set} which is the image of $D_\infty$ through $\mu$: 
  \[ D_\infty^\mu  \triangleq \mu(D_\infty) \] 

  Notice that this set is {\em finite} with cardinality at most $card(M) + 2$. Also, 
  we treat this is a {\em set}. Although $D_\infty$ has a domain structure, we do not 
  care about $D_\infty^\mu$ being a domain. 

  Now, in this $D_\infty^\mu$ we may redefine the interpretation of first-order constants
  and variable symbols in our theories, using ${\cal I}^\mu$ below:


  {\setlength{\arraycolsep}{2pt}  
  \[\begin{array}{rcl}
     \mlinterp{f_{ptr}} & = & \mu(\dbrace{P}^{\infty}(f)) \\  
 %% \roll(\ret(\inj{->}(\dlambda d_1 @.@ \ldots  \\
 %%                       &   & \quad \roll(\ret(\inj{->}(\dlambda d_n @.@ \\ 
 %%                       &   & \quad\quad\text{ if there exist } \oln{e}{n} \text{ s.t. } f[\taus]\;\ol{e} \in M \\ 
 %%                       &   & \quad\quad\quad\text{ and } \interp{e_i}{\dbrace{P}^\infty}{\cdot} = d_i\text{ then } \\
 %%                       &   & \quad\quad\quad\quad \mu(\dapp(\dbrace{P}^{\infty}(f),\oln{d}{n})) \\ 
 %%                       &   & \quad\quad\text{ else } \bot)))\ldots))) \\ \\ 
   \mlinterp{f^{n}}  & = & \dlambda (d {:} \prod_{n}D_{\infty}^\mu) @.@  \\
                       %% &   & \quad\quad\text{ if there exist } \oln{e}{n} \text{ s.t. } f[\taus]\;\ol{e} \in M \\ 
                       %% &   & \quad\quad\quad\text{ and } \interp{e_i}{\dbrace{P}^\infty}{\cdot} = \pi_i(d)\text{ then } \\
                       &   & \quad\quad (\mu\cdot\dapp)(\mu(\dbrace{P}^{\infty}(f)),\oln{\pi_i(d)}{i \in 1..n})) \\
                       %% &   & \quad\quad\text{ else } \bot \\ \\ 

   \mlinterp{app}     & = & \dlambda (d {:} D_{\infty}^\mu \times D_{\infty}^\mu) @.@ \\ \
                      &   & \quad\qquad \mu(\dapp(\pi_1(d),\pi_2(d))) \\ \\

   \mlinterp{K^{\ar}}     & = & \dlambda (d {:} \prod_{\ar}D_{\infty}^\mu) @.@ \mu(\roll(\ret(\inj{K}(d)))) \\ 
   \mlinterp{\sel{K}{i}} & = & \dlambda (d {:} D_{\infty}^{\mu}) @.@ \mu(\roll(\bind_g(\unroll(d)))) \\ 
     \text{where } g  & = & [\;\bot \\ 
                      &   & ,\;\dlambda d @.@ \unroll(\pi_i(d))  \quad (\text{case for constr. } K) \\ 
                      &   & ,\;\bot \\ 
                      &   & ,\;\ldots\\ 
                      &   & ,\;\bot\; ]
  \end{array}\]}

  Sadly, while the interpretation above is relatively simple, it does not validate the axiom 
  for \textsc{TUCase}. The fact that the denotation of a function application may be in the minimal set, 
  does not guarrantee that evaluation had proceeded along this function and hence the case scrutinee will
  be in the minimal set. This will be true only if we add an intentional test in the interpretation of 
  functions that queries the set $M$. {\bf DV:TODO tomorrow}. 

  {\bf DV: TODO: We need something like the definition below (but not quite, it does not type check yet)}: 
  {\setlength{\arraycolsep}{2pt}  
  \[\begin{array}{rcl}
     \mlinterp{f_{ptr}} & = & \mu(\roll(\ret(\inj{->}(\dlambda d_1 @.@ \ldots  \\
                       &   & \quad \mu(\roll(\ret(\inj{->}(\dlambda d_n @.@ \\ 
                       &   & \quad\quad\text{ if there exist } \oln{e}{n} \text{ s.t. } f[\taus]\;\ol{e} \in M \\ 
                       &   & \quad\quad\quad\text{ and } \interp{e_i}{\dbrace{P}^\infty}{\cdot} = d_i\text{ then } \\
                       &   & \quad\quad\quad\quad \mu(\dapp(\dbrace{P}^{\infty}(f),\oln{d}{n})) \\ 
                       &   & \quad\quad\text{ else } \bot))))\ldots)))) \\ \\ 
   \mlinterp{f^{n}}  & = & \dlambda (d {:} \prod_{n}D_{\infty}^\mu) @.@  \\
                       &   & \quad\quad\text{ if there exist } \oln{e}{n} \text{ s.t. } f[\taus]\;\ol{e} \in M \\ 
                       &   & \quad\quad\quad\text{ and } \interp{e_i}{\dbrace{P}^\infty}{\cdot} = \pi_i(d)\text{ then } \\
                       &   & \quad\quad\quad\quad \mu(\dapp(\dbrace{P}^{\infty}(f),\oln{\pi_i(d)}{i \in 1..n})) \\
                       &   & \quad\quad\text{ else } \bot \\ \\ 

   \mlinterp{app}     & = & \dlambda (d {:} D_{\infty}^\mu \times D_{\infty}^\mu) @.@ \\ \
                      &   & \quad\qquad \mu(\dapp(\pi_1(d),\pi_2(d))) \\ \\

   \mlinterp{K^{\ar}}     & = & \dlambda (d {:} \prod_{\ar}D_{\infty}^\mu) @.@ \mu(\roll(\ret(\inj{K}(d)))) \\ 
   \mlinterp{\sel{K}{i}} & = & \dlambda (d {:} D_{\infty}^{\mu}) @.@ \mu(\roll(\bind_g(\unroll(d)))) \\ 
     \text{where } g  & = & [\;\bot \\ 
                      &   & ,\;\dlambda d @.@ \unroll(\pi_i(d))  \quad (\text{case for constr. } K) \\ 
                      &   & ,\;\bot \\ 
                      &   & ,\;\ldots\\ 
                      &   & ,\;\bot\; ]
  \end{array}\]}

  \item The other case is when $@B@ = \CF$. {\bf TODO}
\end{itemize}







%% \newpage

%% \section{Contract checking soundness} 

%% \section{Contracts}

%% The syntax that we use for contracts is in Figure~\ref{fig:contract-syntax}. 
%% Contracts are typed (here, just monomorphically), and we give an operational 
%% semantics for contract satisfaction in the same figure. 

%% \begin{figure*}\small
%% \[\begin{array}{c} 
%% \ruleform{\Sigma;\Gamma |- \Ct } \\ \\ 
%% \prooftree
%% \Sigma;\Delta,x{:}\tau |- e : \Bool
%% ---------------------------------------{TCBase}
%% \Sigma;\Delta |- \{ (x{:}\tau) \mid e \} : \tau
%% ~~~~ 
%% \begin{array}{c}
%% \Sigma;\Delta |- \Ct_1 : \tau \\
%% \Sigma;\Delta,(x{:}\tau) |- \Ct_2 : \tau' 
%% \end{array}
%% ---------------------------------------{TCArr}
%% \Sigma;\Delta |- (x{:}\Ct_1) -> \Ct_2 : \tau -> \tau'
%% ~~~~ 
%% \Sigma;\Delta |- \Ct_1 : \tau \quad \Sigma;\Delta |- \Ct_2 : \tau 
%% ---------------------------------------{TCConj}
%% \Sigma;\Delta |- \Ct_1 \& \Ct_2 : \tau
%% ~~~~ 
%% \phantom{\Gamma}
%% ---------------------------------------{TCf}
%% \Sigma;\Delta |- \CF : \tau
%% \endprooftree \\ \\ 
%% \ruleform{\Sigma;P |- e \in \Ct} \\ \\
%% \prooftree
%%  P \not|- e \Downarrow 
%% -----------------------------------------------{ECDiv}
%%  \Sigma;P |- e \in \{ (x{:}\tau) \mid e' \}
%%  ~~~~
%%  P |- e'[e/x] \Downarrow \True 
%% -------------------------------------------{ECTrue}
%%  \Sigma;P |- e \in \{ (x{:}\tau) \mid e' \}
%%  ~~~~
%%  P \not|- e'[e/x] \Downarrow 
%%  ------------------------------------------{ECCDiv}
%%  \Sigma;P |- e \in \{ (x{:}\tau) \mid e' \} 
%%  ~~~~~
%%  \begin{array}{c} 
%%  \Sigma;\cdot |- \Ct_1 : \tau \\
%%  \text{for all } u, \Sigma;\cdot |- u : \tau ==> \Sigma;P |- e\;u \in \Ct_2[u/x]
%%  \end{array}
%%  --------------------------------------------{ECArr}
%%  \Sigma;P |- e \in (x{:}\Ct_1) -> \Ct_2 
%%  ~~~~
%%  \begin{array}{c}
%%  \Sigma,\cdot |- e : \tau  \\ 
%%  e \in \Ecf \quad \text{(See Section~\ref{sect:cf})}
%%  %% \text{for all } u, (\Sigma;\cdot |- u : \tau -> \Bool) /\ (@BAD@ \notin u) ==> \neg (P |- u\;e \Downarrow @BAD@)
%%  \end{array}
%%  --------------------------------------------------------------------------------------------{ECf}
%%  \Sigma;P |- e \in \CF 
%%  ~~~~~ 
%%  \Sigma;P |- e \in \Ct_1 \quad \Sigma;P |- e \in \Ct_2
%%  --------------------------------------------------------------------------------------------{ECConj}
%%  \Sigma;P |- e \in \Ct_1 \& \Ct_2
%% \endprooftree
%% \end{array}\]
%% \caption{Contract syntax and semantics}\label{fig:contract-syntax}
%% \end{figure*}







%% \section{Induction and admissibility}
%% {\bf TODO} 


%% \section{Minimization}
%% {\bf TODO} 

%% \section{Some ideas}
%% Sometimes the $\CF$ contract stands in our way e.g. for library stuff. It might 
%% be interesting to explore some user-defined pragma to side-step the $\bad$ case
%% in some pattern matches (i.e. make it on demand, pretty much as $F^{\star}$ does, where
%% only the user's assertions matter.
%% %% \acks
%% %% Acknowledgements here

\end{document}
