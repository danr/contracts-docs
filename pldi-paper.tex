\documentclass[preprint,nocopyrightspace]{sigplanconf}

\usepackage{techreport}


\begin{document}
\preprintfooter{\textbf{--- EARLY DRAFT to PLDI 2013 ---}}

\renewcommand{\textfraction}{0.1}
\renewcommand{\topfraction}{0.95}
\renewcommand{\dbltopfraction}{0.95}
\renewcommand{\floatpagefraction}{0.8}
\renewcommand{\dblfloatpagefraction}{0.8}

%%% Extra definitions -- move to techreport at some point (carefully to not break that!)
\newcommand{\Ct}{{\tt C}}
\newcommand{\CF}{{\tt CF}}
\newcommand{\True}{\textit{True}}
\newcommand{\False}{\textit{False}}
\newcommand{\Bool}{\mathop{Bool}}
\newcommand{\ys}{\ol{y}}
\newcommand{\Th}[2]{{\cal T}_{#1,#2}}
\newcommand{\Ecf}{\textsc{Ecf}}
\newcommand{\oln}[2]{\ol{#1}^{#2}}
\newcommand{\tmar}[2]{\mathop{tmar}_{#1}(#2)}
\newcommand{\tyar}[2]{\mathop{tyar}_{#1}(#2)}
\newcommand{\ar}{n}
\newcommand{\lcf}[1]{\textsf{cf}(#1)}
\newcommand{\lcfZ}{\textsf{cf}}
\newcommand{\lncf}[1]{\neg\textsf{cf}(#1)}
\newcommand{\unr}{\mathop{unr}}
\newcommand{\bad}{\mathop{bad}}
\newcommand{\sel}[2]{\mathop{sel\_#1\!_{#2}}}
\newcommand{\ctrans}[3]{{\cal C}\{\!\!\{#3\}\!\!\}}
\newcommand{\etrans}[3]{{\cal E}\{\!\!\{#3\}\!\!\}}
\newcommand{\utrans}[3]{{\cal U}(#3)\{\!\!\{#2\}\!\!\}}

% Get rid of this -- just temporay
\newcommand{\uutrans}[3]{{\cal U}\{\!\!\{#3\}\!\!\}}

\newcommand{\dtrans}[2]{{\cal D}\{\!\!\{#2\}\!\!\}}
\newcommand{\ptrans}[2]{{\cal P}\{\!\!\{#2\}\!\!\}}
%% Gadgets of domain theory



\newcommand{\rollK}{\mathsf{roll}}
\newcommand{\unrollK}{\mathsf{unroll}}
\newcommand{\bindK}{\mathsf{bind}}
\newcommand{\retK}{\mathsf{ret}}
\newcommand{\injK}[2]{\mathsf{#1}(#2)}
\newcommand{\injKZ}[1]{\mathsf{#1}}
\newcommand{\injFun}[1]{\mathsf{Fun}(#1)}
\newcommand{\injBad}{\mathsf{Bad}}

% \newcommand{\roll}[1]{\rollK(#1)}
% \newcommand{\unroll}[1]{\unrollK(#1)}
% \newcommand{\bind}[2]{\bindK_{#1}(#2)}
% \newcommand{\ret}[1]{\retK(#1)}
\newcommand{\roll}[1]{#1}
\newcommand{\unroll}[1]{#1}
\newcommand{\bind}[2]{#1(#2)}
\newcommand{\ret}[1]{#1}
\newcommand{\inj}[2]{{#1}(#2)}

\newcommand{\dlambda}{\mathsf{\lambda}}
\newcommand{\curry}{\mathsf{curry}}
\newcommand{\eval}{\mathsf{eval}}
\newcommand{\uncurry}{\mathsf{incurry}}
\newcommand{\dapp}{\mathsf{app}}

\newcommand{\unitcpo}{{\sf{\bf 1}}}
\newcommand{\VarCpo}{\textit{Var}}
\newcommand{\FVarCpo}{\textit{FunVar}}
\newcommand{\interp}[3]{[\![#1]\!]_{\langle {#2},{#3}\rangle}}
\newcommand{\dbrace}[1]{[\![#1]\!]}
\newcommand{\linterp}[1]{{\cal I}(#1)}
\newcommand{\lassign}[1]{\mu(#1)}
\newcommand{\elab}[1]{\rightsquigarrow \formula{#1}}
\newcommand{\Fcf}{F_{\lcfZ}}
\newcommand{\definable}[1]{{\mathop{def}}(#1)}
\newcommand{\curly}{\rightsquigarrow}
\newcommand{\Min}{\cal M}
\newcommand{\mlinterp}[1]{{\cal I}^{min}(#1)}

\renewcommand{\Th}{{\cal T}}

\newcommand{\theLang}{\lambda_{\sf HALO}}

%% \title{MIN}
\title{A new approach to static contract checking for higher-order lazy programs}

\authorinfo{Dimitrios Vytiniotis \\ Simon Peyton Jones}
           {Microsoft Research}{}

\authorinfo{Dan Ros\'{e}n \\ Koen Claessen}
           {Chalmers University}{}
%% \authorinfo{Nathan Collins}
%%            {Portland State University}{}
\maketitle
\makeatactive

\begin{abstract}
Even well-typed programs can go wrong, by encountering a pattern-match
failure, or simply returning the wrong answer.  An
increasingly-popular response is to allow programmers to write
\emph{contracts} that express semantic properties, such as
crash-freedom or some useful post-condition.
We study the \emph{static verification} of such contracts.
Our main contribution is a novel translation to first-order logic
of both Haskell programs, and contracts written in Haskell,
all justified by denotational semantics. This translation enables us to prove
that functions satisfy their contracts using an off-the-shelf first-order logic
theorem prover.
\end{abstract}

\section{Introduction}\label{s:intro}
  Haskell programmers enjoy the benefits of strong static types and purity:
static types eliminate many bugs early on in the development cycle, and purity
simplifies equational reasoning about programs. Despite these benefits, however,
bugs may still remain in purely functional code and programs often
crash if applied to the wrong arguments.
For example, consider these Haskell definitions:
\begin{code}
  f xs = head (reverse (True : xs))
  g xs = head (reverse xs)
\end{code}
Both @f@ and @g@ are well typed (and hence do not ``go wrong'' in Milner's
sense), but @g@ will crash when applied to the empty list, whereas @f@
cannot crash regardless of its arguments.
To distinguish the two we need reasoning that goes well beyond
that typically embodied in a standard type system.

Many variations of {\em dependent type systems}~\cite{norell:thesis,Xi:2007:DMA:1230756.1230759,fstar} or
{\em refinement type systems}~\cite{Rondon:2008:LT:1375581.1375602,Knowles+:sage}
have been proposed to address this problem, each offering different degrees of
expressiveness or automation.
Another line of work aiming to address this challenge, studied by many researchers
as well~\cite{Findler:2002:CHF:581478.581484,Blume:2006:SCM:1166013.1166016,Knowles+:sage,Siek06gradualtyping,Wadler:2009:WPC:1532974.1532976}, allows programmers to annotate
functions with {\em contracts}, which are forms of behavioural specifications.
For instance, we might write the following contract for
@reverse@:
\[ @reverse@ \in (xs : \CF) -> \{ ys \mid @null@\;xs\; @<=>@\; @null@\;ys \}  \]
%% \begin{code}
%%   reverse ::: xs:CF -> { ys | null xs <=> null ys }
%% \end{code}
This contract annotation asserts that if @reverse@ is applied to a
crash-free (@CF@) argument list @xs@ then the result @ys@ will be empty (@null@)
if and only if @xs@ is empty. What is a crash-free argument?
Since we are using lazy semantics, a list could contain cons-cells that yield
errors when evaluated, and the @CF@ precondition asserts that the input list
is not one of those.
Notice also that @null@ and @<=>@ are just ordinary Haskell functions, perhaps
written by the programmer, even though they appear inside contracts.

With this property of @reverse@ in hand, we might
hope to prove that @f@ satisfies the contract
\[ @f@ \in \CF -> \CF \]
But how do we verify that @reverse@ and @f@ satisfy the claimed
contracts? Contracts are often tested dynamically, but
our plan here is different: we want to verify contracts \emph{statically}
and \emph{automatically}.

It should be clear that there is a good deal of logical reasoning to do,
and a now-popular approach is to delegate the task to an off-the-shelf theorem
prover such as Z3~\cite{z3citation} or Vampire~\cite{vampire}, or search for
counterexamples with a finite model finder~\cite{paradox}.
With that in mind, we make the following new contributions:

\begin{itemize}
  \item We give a translation of Haskell programs to first-order logic (FOL) theories.
        It turns out that lazy programs (as opposed to
        strict ones!) have a very natural translation into first-order logic
        (Section~\ref{ssect:trans-fol}).
  \item We give a translation of contracts to FOL formulae, and an axiomatisation of
        the language semantics in FOL
        (Section~\ref{s:contracts-fol}).
  \item Our main contribution is to show that if we can prove the formula
        that arises from a contract translation
        for a given program, then the program does indeed satisfy this contract. Our proof
        uses the novel idea of employing the denotational
        semantics as a first-order model (Section~\ref{ssect:denot}).
  \item We show how to use this translation in practice for static contract checking with
        a FOL theorem prover (Section~\ref{sect:soundness}),
        and how to prove goals by induction (Section~\ref{sect:extensions}).
% \item \dr{TODO} The other main contribution is a sound and complete
%   heuristic called ``Minimization'', that indeed minimizes the search
%   space for theorem provers (Section~\ref{sect:min}), which by
%   practical experience yields finite counterexamples in many cases,
%   and greatly enhances up proving times. We also investigate how to
%   express it using triggers.
\end{itemize}

This work is a first step towards practical contract checking
for Haskell programs, laying out the theoretical foundations for further engineering
and experimentation. Nevertheless, we have already implemented a prototype for Haskell
programs that uses GHC as a front-end. We have evaluated the practicality of our approach
on many examples, including lazy and higher-order programs, and goals that require
induction. We report this initial encouraging evaluation in
Section~\ref{sect:implementation}.

To our knowledge no one has previously presented a translation of lazy higher-order programs to
first-order logic, in a provably sound way with respect to a denotational
semantics. Furthermore, our approach to static contract checking is
distinctly different to previous work: instead of wrapping and
symbolic execution~\cite{xu+:contracts,Xu:2012:HCC:2103746.2103767},
we harness purity and laziness to directly use the denotational semantics
of programs and contracts and discharge the obligations with a
FOL theorem prover, side-stepping
the wrapping process. Instead of generating verification conditions by pushing
pre- and post- conditions through a program, we directly ask a theorem prover to prove
a contract for the FOL encoding of a program.
We discuss related work in Section~\ref{sect:related}.

%% \newpage
%%   \item The translation
%% For this paper we focus on the translation, but to substantiate the practicality


%% \end{itemize}


%% \begin{itemize}
%% \item We show how to translate Haskell terms
%% into First Order Logic (FOL) (Section~\ref{ssect:trans-fol}).
%% It may appear surprising that this
%% is even possible, since Haskell is a higher order language.  Although
%% the basic idea of the translation is folklore in the community,
%% we believe that this paper is the first to explain it explicitly.

%% \item We also show how to translate \emph{contracts} into FOL
%%       (Section~\ref{s:contracts-fol}),
%%       a translation that is rather less obvious.

%% \item We give a proof based on denotational semantics
%% that if the FOL prover discharges a
%% suitable theorem about the translated Haskell term and contract,
%% then indeed the original Haskell term satisfies that contract (Section~\ref{s:xxx}).

%% %\item It is one thing to make a sound translation, and quite another
%% %to produce FOL terms that the FOL prover can actually prove anything
%% %about --- a common experience is that it goes out to lunch instead.  We
%% %describe a number of techniques that dramatically improve
%% %theorem-proving times, moving them from infeasible to feasible (Section\ref{xxx}).

%% \item For this paper we focus on the
%% translation, but we have also implemented a static contract checker
%% for Haskell itself, by using GHC as a front end.  We have evaluated
%% the practicality of this approach on many examples, including lazy and
%% higher-order programs, as we describe in Section~\ref{xxx}.  \spj{I'd like
%% to say something more substantial here.}
%% \end{itemize}





















% % \section{Checking Haskell contracts in practice}\label{s:examples}
% %   TODO

\subsection{A high-level overview of the system}\label{ssect:schematic}


\section{A higher-order lazy language and its contracts}\label{sect:language}
  To formalise the ideas behind our implementation, we define a
tiny source language $\theLang$:
a polymorphic, higher-order, call-by-name $\lambda$-calculus with
algebraic datatypes, pattern matching, and recursion.
Our actual implementation treats all of Haskell, by using GHC as a front
end to translate Haskell into language $\theLang$.

\kc{I think we should add an example here, of a Haskell program and its translation into FOL. Suggestion: map.}

\subsection{Syntax of $\theLang$} \label{s:syntax}

\begin{figure}
\[\begin{array}{l}
\begin{array}{lrll}
\multicolumn{4}{l}{\text{Programs, definitions, and expressions}} \\
P   & ::= & d_1 \ldots d_n \\
d   & ::= & f\; \ol{a} \; \ol{(x\!:\!\tau)} = u \\
u   & ::= & \multicolumn{2}{l}{e \; \mid \; @case@\;e\;@of@\;\ol{K\;\ol{y} -> e}} \\
e  & ::=  & x            & \text{Variables} \\
   & \mid & f[\ol{\tau}] & \text{Function variables} \\
   & \mid & K[\ol{\tau}](\ol{e}) & \text{Data constructors (saturated)} \\
   & \mid & e\;e         & \text{Applications} \\
   & \mid & @BAD@        & \text{Runtime error} \\
\end{array}\\ \\
\begin{array}{lrll}
\multicolumn{3}{l}{\text{Syntax of closed values}} \\
 v,w & ::= & K^\ar[\ol{\tau}](\oln{e}{\ar}) \;\mid\; f^\ar[\ol{\tau}]\;\oln{e}{m < \ar} \;\mid\; @BAD@ \\ \\
\end{array}
\\
\begin{array}{lrll}
\multicolumn{3}{l}{\text{Contracts}} \\
 \Ct & ::=  & \{ x \mid e \}        & \text{Base contracts}  \\
     & \mid &  (x : \Ct_1) -> \Ct_2      & \text{Arrow contracts} \\
     & \mid & \Ct_1 \& \Ct_2             & \text{Conjunctions}   \\
     & \mid & \CF                        & \text{Crash-freedom}   \\
\end{array}
\\ \\
\begin{array}{lrll}
\multicolumn{3}{l}{\text{Types}} \\
\tau,\sigma & ::=  & T\;\taus & \text{Datatypes} \\
            & \mid & a \mid \tau -> \tau
\end{array}
\\ \\
\begin{array}{lrll}
\multicolumn{3}{l}{\text{Type environments and signatures}} \\
\Gamma & ::=  & \cdot \mid \Gamma,x \\
\Delta & ::=  & \cdot \mid \Delta,a \mid \Delta,x{:}\tau \\
\Sigma & ::=  & \cdot \mid \Sigma,T{:}n \mid \Sigma,f{:}\forall\ol{a} @.@ \tau \mid \Sigma,K^{\ar}{:}\forall\ol{a} @.@ \oln{\tau}{\ar} -> @T@\;\as
\end{array}
\\ \\
\begin{array}{lrll}
\multicolumn{3}{l}{\text{Auxiliary functions}} \\
%% constrs(\Sigma,T) & = & \{ K \mid (K{:}\forall \as @.@ \taus -> T\;\as) \in \Sigma \} \\
(\cdot)^{-}            & = & \cdot \\
(\Delta,a)^{-}         & = & \Delta^{-} \\
(\Delta,(x{:}\tau))^{-} & = & \Delta^{-},x
%% \tyar{D}{f} & = & n & \\
%%             & \multicolumn{3}{l}{\text{when}\; (f |-> \Lambda\oln{a}{n} @.@ \lambda\ol{x{:}\tau} @.@ u) \in D} \\
%% \tmar{D}{f} & = & n & \\
%%             & \multicolumn{3}{l}{\text{when}\; (f |-> \Lambda\ol{a} @.@ \lambda\oln{x{:}\tau}{n} @.@ u) \in D}
\end{array}
\end{array}\]
\caption{Syntax of $\theLang$ and its contracts}\label{fig:syntax}
\end{figure}

Figure~\ref{fig:syntax} presents the syntax of $\theLang$.  A program
$P$ consists of a set of recursive function definitions $d_1 \ldots
d_n$. Each definition has a left hand side that binds its type-variable and
term-variable parameters;
if $f$ has $n$ term-variable parameters we say that
it has arity $n$, and sometimes write it $f^n$.
The right hand side $u$ of a definition is either a @case@ expression or a
@case@-free expression $e$.  A @case@-free expression consists of
variables $x$, function variables $f[\taus]$ fully applied to their
type arguments, applications $e_1\;e_2$, data constructor applications
$K[\taus](\ol{e})$. As a notation, we use
$\oln{x}{n}$ for sequences of elements of size $n$. When $n$ is
omitted $\ol{x}$ has a size which is implied by the context or is not
interesting.

A program \emph{crashes} if it evaluates the special value @BAD@.
For example, we assume that the standard Haskell function @error@
simply invokes @BAD@, thus:
\begin{code}
  error :: String -> a
  error s = BAD
\end{code}
Moreover, we assume that all incomplete pattern-matches are completed, with the
missing case yielding @BAD@.  For example:
\begin{code}
  head :: [a] -> [a]
  head (x:xs) = x
  head []     = BAD
\end{code}
In our context, @BAD@ is our way to saying what it means for a program to ``go wrong'',
and verification amounts to proving that a program cannot invoke @BAD@.

Our language embodies several convenient syntactic constraints:
(i)~$\lambda$ abstractions occur only at the top-level,
(ii)~@case@-expressions can only immediately follow a function
definition, and (iii) constructors are fully applied.
Any Haskell program can be transformed into this restricted
form by lambda-lifting, @case@-lifting, and eta-expansion respectively,
and our working prototype does just this.
However this simpler language is extremely
convenient for the translation of programs to first-order logic.

$\theLang$ is an explicitly-typed language, and we assume the existence
of a typing relation $\Sigma |- P$, which checks that a program
conforms to the definitions in the signature $\Sigma$. A signature
$\Sigma$ (Figure~\ref{fig:syntax}) records the declared data types,
data constructors and types of functions in the program $P$. The
well-formedness of expressions is checked with a typing relation
$\Sigma;\Delta |- u : \tau$, where $\Delta$ is a typing environment,
also in Figure~\ref{fig:syntax}.  We do not give the details of the
typing relation since it is standard.
Our technical development and analysis in the following sections
assume that the program has been checked for type errors.
The typing judgement should check that all pattern matches are
exhaustive; as mentioned above, missing cases should return @BAD@.

The syntax of closed values is also given in Figure~\ref{fig:syntax}. Since we do not
have arbitrary $\lambda$-abstractions, values can only be partial function applications
$f^\ar[\ol{\tau}]\;\oln{e}{m < \ar}$, data constructor applications $K[\tau](\ol{e})$,
and the error term @BAD@.

The operational semantics of $\theLang$ is entirely standard, and we do not give it here.
We write $P |- u \Downarrow v$ to mean ``in program P, right-hand side $u$ evaluates to
value $v$'',

% -------------------- Omitted section on operational semantics ------------
% \subsection{Operational semantics of $\theLang$}
%
% \spj{Why do we give an operational as well as denotational semantics?}
% The big-step operational semantics of our language is given in
% Figure~\ref{fig:opsem}, which contains no surprises. One interesting
% detail of big-step semantics is that they do not distinguish between non-termination
% and ``getting stuck'', meaning that if $P \not|- e \Downarrow v$ then $e$ could either diverge or its
% evaluation could get stuck. We return to this convenient for our purposes form of operational
% semantics later. \spj{Where later? Also this sentence is hard to parse; indeed I'm not quite
% sure what it means.}
% \begin{figure}
% \[\begin{array}{c}
% \ruleform{P |- u \Downarrow v} \\
% \prooftree
% \begin{array}{c} \ \\
% \end{array}
% -------------------------------------{EVal}
% P |- v \Downarrow v
% ~~~~~
% \begin{array}{c}
% (f \;\ol{a}\;\oln{(x{:}\tau)}{m} = u) \in P \\
% P |- u[\ol{\tau}/\ol{a}][\ol{e}/\ol{x}] \Downarrow v
% \end{array}
% -------------------------------------{EFun}
% P |- f[\ol{\tau}]\;\oln{e}{m} \Downarrow v
% ~~~~~
% \begin{array}{c}
% P |- e_1 \Downarrow v_1 \\
% P |- v_1\;e_2 \Downarrow w
% \end{array}
% ------------------------------------------------{EApp}
% P |- e_1\;e_2 \Downarrow w
% ~~~~~
% \begin{array}{c}
% P |- e_1 \Downarrow @BAD@
% \end{array}
% ------------------------------------------------{EBadApp}
% P |- e_1\;e_2 \Downarrow @BAD@
% ~~~~~
% %% \endprooftree \\ \\
% %% \ruleform{P |- u \Downarrow v} \\ \\
% %% \prooftree
% %% P |- e \Downarrow v
% %% -------------------------------------{EUTm}
% %% P |- e \Downarrow v
% %% ~~~~
% \begin{array}{c}
% P |- e \Downarrow K_i[\ol{\sigma}_i](\ol{e}_i) \quad
% P |- e'_i[\ol{e}_i/\ol{y}_i] \Downarrow w
% \end{array}
% ------------------------------------{ECase}
% P |- @case@\;e\;@of@\;\ol{K\;\ol{y} -> e'} \Downarrow w
% ~~~~~
% \begin{array}{c}
% P |- e \Downarrow @BAD@ \\
% \end{array}
% ------------------------------------{EBadCase}
% P |- @case@\;e\;@of@\;\ol{K\;\ol{y} -> e'} \Downarrow @BAD@
% %% \begin{array}{c}
% %% (f |-> \Lambda\ol{a} @.@ \lambda\oln{x{:}\tau}{m} @.@ @case@\;e\;@of@\;\ol{K\;\ol{y} -> e'}) \in D \\
% %% D |- e[\ol{\tau}/\ol{a}][\ol{e}/\ol{x}] \Downarrow @BAD@ \\
% %% \end{array}
% %% -------------------------------------{EBadCase}
% %% D |- f[\ol{\tau}]\;\oln{e}{m} \Downarrow @BAD@
% \endprooftree
% \end{array}\]
% \caption{Operational semantics of $\theLang$}\label{fig:opsem}
% \end{figure}
% %% We can state some standard properties of the typing and evaluation relation.
% %% \begin{lemma}[Subject reduction]
% %% Assume $\Sigma |- P$ and $\Sigma;\cdot |- e : \tau$
% %% If $P |- e \Downarrow w$ then $P |- value(w)$ and $\Sigma;\cdot |- w : \tau$.
% %% \end{lemma}
% The operational semantics of Figure~\ref{fig:opsem} has the possibility of non-determinism because
% of the overlapping of several rules for applications. But that is not a problem, as we can prove that evaluation
% is deterministic using the following two lemmas.
% \begin{lemma}[Value determinacy]
% If $\Sigma;\cdot |- v : \tau$ and
% $\Sigma |- P$ and $P |- v \Downarrow w$ then $ v = w $.
% \end{lemma}
% \begin{lemma}[Determinacy of evaluation]
% If $\Sigma;\cdot |- e : \tau$ and
% $\Sigma |- P$ and $\Sigma;\cdot |- e \Downarrow v_1$ and $\Sigma;\cdot |- e \Downarrow v_2$ then
% $v_1 = v_2$.
% \end{lemma}
% Finally, big-step soundness asserts that an expression that evaluates results in a
% well-typed value.
% \begin{lemma}[Big-step soundness]
% If $\Sigma;\cdot |- e : \tau$ and
% $\Sigma |- P$ and $\Sigma;\cdot |- e \Downarrow v$ then $\Sigma;\cdot |- v : \tau$.
% \end{lemma}

\subsection{Contracts}

% \begin{figure}
% \[\begin{array}{lcl}
% e \in \{ x \mid e_p\} & <=> &  e \not\Downarrow \text{ or } e_p[e/x] \not\Downarrow \text{ or} \\
%                         &     &  e_p[e/x] \Downarrow True \\
% e \in (x{:}\Ct_1) -> \Ct_2 & <=> &
%                         \forall e' \in \Ct_1.\; (e\;e') \in \Ct_2[e'/x] \\
% e \in \Ct_1 \& \Ct_2 & <=> & e \in \Ct_1 \text{ and } e \in \Ct_2 \\
% e \in \CF            & <=> & \forall {\cal C}. BAD \not\in {\cal C} \Rightarrow e \not\Downarrow BAD
% \end{array}
% \]
% \caption{Operational semantics of contracts} \label{f:contract-spec-op}
% \end{figure}

We now turn our attention to contracts. The syntax of contracts
is given in Figure~\ref{fig:syntax} and includes base contracts
$\{ x \mid e \}$, arrow contracts $(x : \Ct_1) -> \Ct_2$, conjunctions
$\Ct_1 \& \Ct_2$ and crash-freedom $\CF$. Previous work~\cite{xu+:contracts}
includes other constructs as well, but the constructs we give here are enough to verify
many programs and exhibit the interesting theoretical and practical problems.

We write $e \in C$ to mean ``the expression $e$ satisfies the contract $C$'', and similarly
for functions $f$.  We will say what contracts mean formally in
Section~\ref{s:den-sem-contracts}.  However, here is their informal meaning:
\begin{itemize}
\item $e \in \{x | e'\}$ means that $e$ does not evaluate to a value
  or $e[e'/x]$ evaluates to @True@ or does not evaluate to a value.
Notice that $e'$ is an arbitrary expression
(in our implementation, arbitrary Haskell expressions),
rather than being restricted to some well behaved meta-language.  This
is great for the programmer because the language and its library
functions is familiar, but it poses a challenge for verification
because these expressions in contracts may themselves diverge or
crash.
\item $e \in (x:\Ct_1) \rightarrow \Ct_2$ means that whenever $e'$ satisfies $\Ct_1$, it
is the case that $(e\,e')$ satisfies $\Ct_2[e'/x]$.
\item $e \in \Ct_1 \& \Ct_2$ means that $e$ satisfies both $\Ct_1$ and $\Ct_2$.
\item $e \in \CF$ means that $e$ is \emph{crash free}; that is $e$ does not
crash regardless of what context it is plugged into (see Section~\ref{s:cf-fol}).
\end{itemize}

% -----------------------------------------------------------------
\section{Translating $\theLang$ to first-order logic} \label{ssect:trans-fol}

Our goal is to answer the question ``does expression $e$ satisfy
contract $C$?''.  Our plan is to translate the expression and the
contract into a first-order logic (FOL) term and formula respectively, and get a standard
FOL prover to do the heavy lifting. In this section we formalise our translation,
and describe how we use it to verify contracts.

\subsection{The FOL language}

\begin{figure}
\[\begin{array}{c}
\begin{array}{lrll}
\multicolumn{3}{l}{\text{Terms}} \\
  s,t & ::=  & x                          & \text{Variables} \\
      & \mid & f(\ol{t})                  & \text{Function applications} \\
      & \mid & K(\ol{t})                  & \text{Constructor applications} \\
      & \mid & \sel{K}{i}(t)              & \text{Constructor selectors} \\
      & \mid & f_{ptr} \mid app(t,s)       & \text{Pointers and application} \\
      & \mid & \unr \mid \bad             & \text{Unreachable, bad} \\ \\
\multicolumn{3}{l}{\text{Formulae}} \\
 \phi & ::=  & \lcf{t}    & \text{Crash-freedom} \\
%%      & \mid & \lncf{t}   & \text{Can provably cause crash} \\
      & \mid & t_1 = t_2  & \text{Equality} \\
      & \mid & \phi \land \phi \mid \phi \lor \phi \mid \neg \phi \\
      & \mid & \forall x @.@ \phi \mid \exists x @.@ \phi \\ \\
\end{array}
\\
\multicolumn{1}{l}{\text{Abbreviations}} \\
\begin{array}{rcl}
app(t,\oln{s}{n}) & = & (\ldots(app(t,s_1),\ldots s_n) \\
\phi_1 \Rightarrow \phi_2 & = & \neg \phi_1 \lor \phi_2
\end{array}
\end{array}\]
\caption{Syntax of FOL}\label{fig:fol-image}
\end{figure}

We begin with the syntax of the FOL language, which is given in
Figure~\ref{fig:fol-image}. There are two syntactic forms,
\emph{terms} and \emph{formulae}. Terms include saturated function applications
$f(\ol{t})$, saturated constructor applications $K(\ol{t})$, and variables. They
also include, for each data constructor $K^\ar$ in the signature
$\Sigma$ with arity $\ar$ a set of {\em selector functions}
$\sel{K}{i}(t)$ for $i \in 1 \ldots \ar$.  The terms $app(t,s)$ and
$f_{ptr}$ concern the higher-order aspects of $\theLang$
(namely un-saturated applications), which we
discuss in Section~\ref{s:hof}.  Finally we introduce two new
syntactic constructs $\unr$ and $\bad$. As an abbreviation we often use
$app(t,\ol{s})$ for the sequence of applications to each $s_i$, as
Figure~\ref{fig:fol-image} shows.

A formula $\phi$ in Figure~\ref{fig:fol-image} is just a first-order logic
formula, augmented with a predicate $\lcf{t}$ for crash-freedom, which
we discuss in Section~\ref{s:cf-fol}.

\subsection{Translation of expressions to FOL}\label{ssect:trans-exprs}

% ---------------------------------------------------
\begin{figure}\small
\setlength{\arraycolsep}{2pt}
\[\begin{array}{c}
\ruleform{\ptrans{\Sigma}{P} = \formula{\phi} } \quad
\ptrans{\Sigma}{\ol{d}} = \bigwedge \ol{\dtrans{\Sigma}{d}}
\\ \\
\ruleform{\dtrans{\Sigma}{d} = \formula{\phi}} \\ \\
\begin{array}{rcl}
  \dtrans{\Sigma}{f \;\ol{a}\;\ol{(x{:}\tau)} = u}
    & =     & \formula{\forall \ol{x} @.@\, \utrans{\sigma}{u}{f(\ol{x})}} \\
    & \land & \formula{\forall \ol{x} @.@\, f(\ol{x}) = app(f_{ptr},\xs)} \\
\end{array}
\\ \\
\ruleform{\utrans{\Sigma}{u}{s} = \formula{\phi}} \\ \\
\begin{array}{rcl}
\utrans{\Sigma}{e}{s}
  & = & \formula{(s = \etrans{\Sigma}{\Gamma}{e})} \\
\multicolumn{3}{l}{\utrans{\Sigma}
    {@case@\;e\;@of@\;\ol{K\;\ol{y} -> e'}}{s}} \\
\multicolumn{3}{l}{
\quad
  \begin{array}[t]{rl}
    = & \formula{(t = \bad => s = bad)} \\
    \land & \formula{(\forall \ol{y} @.@ t = K_1(\ol{y}) => s = \etrans{\Sigma}{\Gamma}{e'_1})\;\land \ldots}  \\
    \land & \formula{(t{\neq}\bad\;\land\;
                 t{\neq}K_1(\oln{{\sel{K_1}{i}}(t)}{})\;\land\;\ldots => s=\unr)} \\
    \mbox{where} & t  =  \etrans{\Sigma}{\Gamma}{e}
 \end{array}
}
\end{array}
\\ \\
\ruleform{\etrans{\Sigma}{\Gamma}{e} = \formula{t} } \\ \\
\begin{array}{rcl}
\etrans{\Sigma}{\Gamma}{x} & = & \formula{x} \\
\etrans{\Sigma}{\Gamma}{f[\ol{\tau}]} & = & \formula{f_{ptr}} \\
\etrans{\Sigma}{\Gamma}{K[\ol{\tau}](\ol{e})} & = & \formula{K(\ol{\etrans{\Sigma}{\Gamma}{e}})} \\
\etrans{\Sigma}{\Gamma}{e_1\;e_2} & = & \formula{app(\etrans{\Sigma}{\Gamma}{e_1},
                                                     \etrans{\Sigma}{\Gamma}{e_2})} \\
\etrans{\Sigma}{\Gamma}{@BAD@} & = & \formula{\bad}
\end{array}
\\ \\
\ruleform{\ctrans{\Sigma}{\Gamma}{e \in \Ct} = \formula{\phi}} \\ \\
\begin{array}{rcl}
\ctrans{\Sigma}{\Gamma}{e \in \{(x{:}\tau) \mid e' \}}
  & = & \formula{t{=}\unr} \\
  & \lor & \formula{t'[t/x]{=}\unr} \\
  & \lor & \formula{t'[t/x]{=}\True} \\
  & \mbox{where} &
    \begin{array}[t]{rcl}
      t  & = & \etrans{\Sigma}{\Gamma}{e} \; \text{and} \; t' = \etrans{\Sigma}{\Gamma}{e'}
    \end{array}
\\
\ctrans{\Sigma}{\Gamma}{e \in (x{:}\Ct_1) -> \Ct_2}
  & = & \formula{\forall x @.@ \ctrans{\Sigma}{\Gamma,x}{x \in \Ct_1}
                          \Rightarrow \ctrans{\Sigma}{\Gamma,x}{e\;x \in \Ct_2}}
\\
\ctrans{\Sigma}{\Gamma}{e \in \Ct_1 \& \Ct_2}
   & = & \formula{ \ctrans{\Sigma}{\Gamma}{e \in \Ct_1} /\ \ctrans{\Sigma}{\Gamma}{e \in \Ct_2}}
\\
\ctrans{\Sigma}{\Gamma}{e \in \CF} & = & \formula{\lcf{\etrans{\Sigma}{\Gamma}{e}}}
\end{array}
\end{array}\]
\caption{Translation of programs and contracts to logic}
   \label{fig:etrans}\label{fig:contracts-minless}
\end{figure}
% ---------------------------------------------------
\begin{figure}\small
\setlength{\arraycolsep}{1pt}
\[\begin{array}{c}
\ruleform{\text{Theory}\,\Th} \\
\begin{array}{ll}
\multicolumn{2}{l}{\text{Axioms for $bad$ and $unr$}} \\
 \textsc{AxAppBad}  & \formula{\forall x @.@ app(\bad,x){=}\bad}  \\
 \textsc{AxAppUnr}  & \formula{\forall x @.@ app(\unr,x){=}\unr}    \\
 \textsc{AxDisjBU} & \formula{\bad \neq \unr}  \\
\\
\multicolumn{2}{l}{\mbox{Axioms for data constructors}} \\
 \textsc{AxDisjC} & \formula{\forall \oln{x}{n}\oln{y}{m} @.@ K(\ol{x}) \neq J(\ol{y})} \\
                  & \text{ for every } (K{:}\forall\as @.@ \oln{\tau}{n} -> T\;\as) \in \Sigma \\
                  & \text{ and } (J{:}\forall\as @.@ \oln{\tau}{m} -> S\;\as) \in \Sigma \\
 \textsc{AxDisjCBU} & \formula{(\forall \oln{x}{n} @.@ K(\ol{x}) \neq \unr \; \land \; K(\ol{x}) \neq \bad)} \\
                  & \text{ for every } (K{:}\forall\as @.@ \oln{\tau}{n} -> T\;\as) \in \Sigma \\
 \textsc{AxInj}   & \formula{\forall \oln{y}{n} @.@ \sel{K}{i}(K(\ys)) = y_i} \\
                  & \text{for every } K^\ar \in \Sigma \text{ and } i \in 1..n \\
\\
\multicolumn{2}{l}{\mbox{Axioms for crash-freedom}} \\
 \textsc{AxCfC}  & \formula{\forall \oln{x}{n} @.@ \lcf{K(\ol{x})} <=> \bigwedge\lcf{\ol{x}}} \\
                 & \text{ for every } (K{:}\forall\as @.@ \oln{\tau}{n} -> T\;\as) \in \Sigma \\
 \textsc{AxCfBU} & \formula{\lcf{\unr} /\ \lncf{\bad}} \\
\end{array}
\end{array}\]
\caption{Theory $\Th$: axioms of the FOL constants}\label{fig:prelude} \label{fig:data-cons}
\end{figure}

% ---------------------------------------------------
What exactly does it mean to translate an expression to first-order logic?
We are primarily interested in reasoning about equality, so we might
hope for this informal guiding principle:

\begin{quote}
If we can prove\footnote{From an appropriate axiomatisation of the semantics of
programs.} in FOL that $\etrans{}{}{e_1} = \etrans{}{}{e_2}$ then
$e_1$ and $e_2$ are semantically equivalent.
\end{quote}

where $\etrans{}{}{e}$ is the translation of $e$ to a FOL term. That is, we can
reason about the equality of Haskell terms by translating them into FOL, and then using
a FOL theorem prover. The formal statement of this property is Corollary~\ref{cor:guiding-principle}

The translation of programs, definitions, and expressions to FOL
is given in Figure~\ref{fig:etrans}.
The function $\ptrans{}{P}$ translates a program to a conjunction of formulae,
one for each definition $d$, using $\dtrans{}{d}$ to translate
each definition.
The first clause in $\cal D$s right-hand side uses $\cal U$ to
translate the right hand side $u$, and quantifies over the $\ol{x}$.
We will deal with the second clause of $\cal D$
in Section~\ref{s:hof}.

Ignoring @case@ for now (which we discuss in Section~\ref{s:case-fol}),
the formula $\utrans{}{e}{f(\ol{x})}$
simply asserts the equality $f(\ol{x}) = \etrans{}{}{e}$.
That is, we use a new function $f$ in the logic for each function definition in the
program, and assert that any application of $f$ is equal to (the logical translation of)
$f$'s right hand side. Notice that we erase type arguments in the translation
since they do not affect the semantics. You might think that the translation
$f(\ol{x}) = \etrans{}{}{e}$ is entirely obvious but, surprisingly, it is only correct
because we are in a call-by-name setting. The same equation is problematic in a
call-by-value setting -- an issue
we return to towards the end of Section~\ref{s:soundness}.

Lastly $\etrans{}{}{e}$ deals with expressions.  We will deal with
functions and application shortly
(Section~\ref{s:hof}), but the other equations for $\etrans{}{}{e}$
are straightforward.  Notice that $\etrans{}{}{@BAD@} = bad$, and recall
that @BAD@ is the $\theLang$-term used for an inexhaustive @case@ or a call
to @error@.  It follows from our guiding principle
that for any $e$, if we manage to prove in FOL that $ \etrans{}{}{e} = bad $,
then the source program $e$
must be semantically equivalent to @BAD@, meaning that it definitely
crashes.

\subsection{Translating higher-order functions} \label{s:hof}

If $\theLang$ was
a first-order language, the translation of function calls would be easy:
$$
\etrans{\Sigma}{\Gamma}{f[\ol{\tau}]\;\ol{e}} = \formula{f(\ol{\etrans{}{}{e}})} \\
$$
At first it might be surprising that we can also translate a \emph{higher-order} language
$\theLang$ into first order logic, but in fact it is easy to do so, as
Figure~\ref{fig:etrans} shows.  We introduce into the logic
(a) a single new function $app$, and (b) a nullary constant $f_{ptr}$ for each function $f$
(see Figure~\ref{fig:fol-image}).
Then, the equations for $\etrans{}{}{e}$ translate application in $\theLang$ to
a use of $app$ in FOL, and any mention of function $f$ in $\theLang$ to a use
of $f_{ptr}$ in the logic.  For example:
$$
\etrans{}{}{@map f xs@} = app( app( @map@_{ptr}, @f@_{ptr}), @xs@)
$$
assuming that @map@ and @f@ are top-level functions in the $\theLang$-program, and
@xs@ is a local variable.  Once enough $app$ applications stack up, so that
$@map@_{ptr}$ is applied to two arguments, we can invoke the @map@ function directly
in the logic, an idea we express with the following axiom:
$$
\forall x y.\;app(app(@map@_{ptr}, x), y) = @map@(x,y)
$$
These axioms, one for each function $f$, are generated by the second
clause of the rules for $\dtrans{}{d}$ in Figure~\ref{fig:etrans}.
(The notation $app(f,\ol{x})$ is defined in Figure~\ref{fig:fol-image}.)
You can think of $@map@_{ptr}$ as a ``pointer to'', or ``name of'' of, @map@.
The $app$ axiom for @map@ translates a saturated use of @map@'s pointer into
a call of @map@ itself.

This translation of higher-order functions to first-order logic may be
easy, but it is not complete. In particular, in first-order logic we
can only reason about functions with a concrete first-order
representation (i.e. the functions that we already have and their
partial applications) but, for example, lambda expressions cannot be
created during proof time. Luckily, the class of properties we reason
about (contracts) never require us to do so.

\subsection{Data types and {\tt case} expressions} \label{s:case-fol}

The second equation for $\utrans{}{u}{s}$ in Figure~\ref{fig:etrans} deals with
@case@ expressions, by generating a conjunction of formulae, as follows:
\begin{itemize}
\item If the scrutinee $t$ is $bad$ (meaning that evaluating it invokes @BAD@) then
the result $s$ of the @case@ expression is also $bad$.  That is, @case@ is strict in
its scrutinee.
\item If the scrutinee is an application of one of the constructors $K_i$ mentioned
in one of the @case@ alternatives, then the result $s$ is equal to the corresponding
right-hand side, $e'_i$, after quantifying the variables $\ol{y}$ bound by the @case@ alternative.
\item Otherwise the result is $unr$.
The bit before the implication $\Rightarrow$ is just the
negation of the previous preconditions; the formula
  $t{\neq}K_1(\oln{{\sel{K}{1}}(t)}{})$
is the clumsy FOL way to say ``$t$ is not built with constructor $K_1$''.
\end{itemize}
Why do we need the last clause? Consider the function @not@:
\begin{code}
  not :: Bool -> Bool
  not True = False
  not False = True
\end{code}
Suppose we claim that $@not@ \in @CF@ \rightarrow @CF@$, which is patently true.
But if we lack the last clause above, the claim
is \emph{not} true in every model;
for example @not@ might crash when given the (ill-typed but crash-free)
argument @3@.  The third clause above excludes this possibility by
asserting that the result of @not@ is the special crash-free constant $unr$
if the scrutinee is ill-typed (i.e. not $bad$ and not built with the
constructors of the type).  This is the whole reason we need $unr$ in the first place.
In general, if $\etrans{}{}{e} = unr$ is provable in the logic, then $e$ is ill-typed,
or divergent.

Of course, we also need to axiomatise the behaviour of data constructors and
selectors, which is done in Figure~\ref{fig:data-cons}:
\begin{itemize}
\item \textsc{AxDisjCBU} explains that a term headed by a data constructor cannot
also be $bad$ or $unr$.
\item \textsc{AxInj} explains how the selectors $\sel{K}{i}$ work.
\item \textsc{AxDisjC} tells the prover that all data constructors are pairwise disjoint.
There are a quadratic number of such axioms, which presents a scaling problem.
For this reason FOL provers sometimes offer a built-in notion of data constructors,
so this is not a problem in practice, but we ignore this pragmatic issue here.
\end{itemize}

\kc{Something needs to be said about recursion. Higher-order functions are not the problem, but recursive definitions are. What we get in FOL are all fixpoints, but our denotational semantics only wants least fixpoints. Luckily, many things we want to prove hold for all fixpoints, but sometimes we need to invoke induction, which is not explicitly part of the translation. Translating all this into a higher-order logic (such as Isabelle) would allow us to express that we mean the least fixpoint.}

\subsection{Translation of contracts to FOL} \label{s:contracts-fol}

Now that we know how to translate \emph{programs} to first order
logic, we turn our attention to translating \emph{contracts}.  We do
not translate a contract \emph{per se}; rather we translate the claim
$e \in \Ct$.  Once we have translated $e \in \Ct$ to a first-order logic
formula, we can ask a prover to prove it using axioms from the translation of
the program, or axioms from Figure~\ref{fig:data-cons}. If successful, we can
claim that indeed $e$ does satisfy $C$.  Of course that needs proof,
which we address in Section~\ref{s:soundness}.

Figure~\ref{fig:contracts-minless} presents the translation
$\ctrans{}{\Gamma}{e \in \Ct}$; there are four equations corresponding
to the syntax of contracts in Figure~\ref{fig:syntax}.
The last three cases are delightfully simple and direct.  Conjunction of contracts
turns into conjunction in the logic; a dependent function contract turns
into universal quantification and implication; and the claim that $e$ is
crash-free turns into a use of the special term $\lcf{t}$ in the logic.
We discuss crash-freedom in Section~\ref{s:cf-fol}.

The first equation, for predicate contracts $e \in \{x \mid e' \}$,
is sightly more complicated.
The first clause $t=unr$, together with the axioms for $unr$ in Figure~\ref{fig:prelude}, ensures
that $unr$ satisfies every contract.
The second and third say that the contract holds if $e'$ diverges or is semantically
equal to @True@.  The choices embodied in this rule were discussed at length
in earlier work \cite{xu+:contracts} and we do not rehearse it here.

\subsection{Crash-freedom} \label{s:cf-fol}

The claim $e \in @CF@$, pronounced ``$e$ is crash-free'', means that $e$ cannot
crash \emph{regardless of context}.  So, for example @(BAD, True)@ is not crash-free
because it can crash if evaluated in the context @fst (BAD, True)@.  Of course,
the context itself should not be the source of the crash; for example @(True,False)@ is
crash-free even though @BAD (True,False)@ will crash.

We use the FOL term $\lcf{t}$ to assert that $t$ is crash-free. The axioms for $\lcfZ$
are given in Figure~\ref{fig:prelude}.  \textsc{AxCfC} says that a data constructor application
is crash-free if and only iff its arguments are crash-free.  \textsc{AxCfBU} says that
$unr$ is crash-free, and that $bad$ is not.  That turns out to be all that we need.

\subsection{Summary}

That completes our formally-described --- but so far only informally-justified --- translation
from a $\theLang$ program and a set of contract claims, into first-order logic.
To a first approximation, we can now hand the generated axioms from the program and
our axiomatisation to an FOL theorem prover and ask it to use them to prove the
translation of the contract claims.




\section{Soundness through denotational semantics}
   \label{sect:contracts}\label{ssect:denot}
  Our account so far has been largely informal.  How can we be sure
that if the FOL prover says ``Yes!  This FOL formula is provable'', then
the corresponding $\theLang$ program indeed satisfies the claimed contract?

To prove this claim we take a denotational approach.
Most of what follows is an adaptation of well-known techniques to
our setting and there are no surprises --- we refer the reader
to~\cite{winskel} or~\cite{benton+:coq-domains} for a short and modern
exposition of the standard methodology.

\subsection{Technical preliminaries}

We will assume a program $P$, well-formed in a signature $\Sigma$, so
that $\Sigma |- P$.
Given a signature $\Sigma$ we define a strict
bi-functor $F$ on complete partial orders (cpos), below:
%% For a well-formed signature $\Sigma$, we define the strict bi-functor on cpos, below,
%% assuming that $K_1\ldots K_k$ are all the constructors in $\Sigma$:
\[\begin{array}{lclll}
  F(D^{-},D^{+}) & = & ( \quad{\prod_{\ar_1}{D^{+}}} & K_1^{\ar_1} \in \Sigma \\
               & + & \;\quad\ldots                    & \ldots \\
               & + & \;\quad{\prod_{\ar_k}{D^{+}}} & K_k^{\ar_k} \in \Sigma \\
               & + & \;\quad(D^{-} =>_c D^{+}) \\
               & + & \;\quad\unitcpo_{bad} \quad )_{\bot}
\end{array}\]
\spj{Why bi-functor? Why not just functor?}
The bi-functor $F$ is the lifting of a big sum: that sum consists of
(i) products, one for each possible constructor (even across different data types), (ii) the continuous
function space from $D^{-}$ to $D^{+}$, and (iii) a unit cpo to denote @BAD@ values.
The notation $\prod_{n}{D}$ abbreviates $n$-ary products of cpos (the unit cpo $\unitcpo$ if $n = 0$).
The product and sum constructions are standard, but note that we use their non-strict versions.
The notation $C =>_c D$ denotes the cpo
induced by the space of continuous functions from the cpo $C$ to the cpo $D$. We use
the notation $\unitcpo_{bad}$ to
denote a single-element cpo -- the $bad$ subscript is just there for readability.
The notation $D_\bot$ is {\em lifting}.
% , which is a monad, equipped with the following two continuous
% functions.
% \[\begin{array}{l}
%    \retK   : D =>_c D_\bot \\
%    \bindK_{f : D =>_c E_\bot} : D_\bot =>_c E_\bot
% \end{array}\]
% with the obvious definitions.

Observe that we have dropped all type information from the source
language. The elements of the products corresponding to data
constructors are simply $D^{+}$ (instead of more a precise description
from type information) and the return types of data constructors are
similarly ignored. This is not to say that a more type-rich
denotational semantics is not possible (or desirable even) but this
simple denotational semantics turns out to be sufficient for
formalisation and verification.

%% for $\lambda$-abstractions and @BAD@. Observe that we have
%% Moreover, the following continuous operations are defined:
%% \[\begin{array}{l}
%%    \curry_{f : D\times E =>_c F} : D =>_c (E =>_C F) \\
%%    \eval : (E =>_c D)\times E =>_c D
%% \end{array}\]
%% for any cpos $D, E, F$.

Now we can define $D_{\infty}$ as the solution to this recursive domain equation
$$D_{\infty} \approx F( D_{\infty}, D_{\infty})$$
We can show that $D_{\infty}$ exists using the
standard {\em embedding-projection} pairs methodology. Moreover, we define the
value domain $V_{\infty}$ thus:
    \[V_{\infty} = \begin{array}[t]{ll}
             \quad\;{\prod_{\ar_1}{D_{\infty}}} & K_1^{\ar_1} \in \Sigma \\
             \; + \;\ldots                    & \ldots \\
             \; + \;{\prod_{\ar_k}{D_{\infty}}} & K_k^{\ar_k} \in \Sigma \\
             \; + \;(D_{\infty} =>_c D_{\infty}) \\
             \; + \;\unitcpo_{bad} \quad
    \end{array}\]
The following continuous functions also exist:
% , each being the inverse of the
% other (i.e. composing to the identity function on the corresponding cpo):
\[\begin{array}{rcl}
  \retK   & : & D =>_c D_\bot \\
  \bindK_{f : D =>_c E_\bot} & : & D_\bot =>_c E_\bot \\
  \rollK & : & (V_{\infty})_\bot =>_c D_{\infty} \\
  \unrollK & : &D_{\infty} =>_c (V_{\infty})_\bot
\end{array}\]
However in what follows we will always elide these functions to reduce clutter.

To denote elements of $V_{\infty}$ we use the following notation.
\begin{itemize}
\item $\injK{K}{d_1, \ldots, d_n}$ denotes the injection of
the $n$-ary product of $D_{\infty}$ into the component of the sum
$V_{\infty}$ corresponding to the $n$-ary constructor $K$.
\item $\injFun{d}$ is the injection of
an element of $D_{\infty} =>_c D_{\infty}$ into the function component of $V_{\infty}$
\item $\injBad$ is the unit injection into $V_{\infty}$.
\end{itemize}

%% We summarize the (standard) construction needed for the proof of Lemma~\ref{lem:rec-solution} based on embedding-projection pairs,
%% because some of its details will be useful later. Consider the chain of cpos $D_i$ defined as:
%% \[\begin{array}{lcl}
%%    D_0 & = & \{\bot\} \\
%%    D_{i+1} & = & F_{\Sigma}(D_i,D_i)
%% \end{array}\]
%% and moreover consider the corresponding {\em embeddings} $e_i : D_i =>_c D_{i+1}$ and
%% {\em projections} $p_i : D_{i+1} =>_c D_i$ defined as:
%% \[\begin{array}{lcl}
%%    e_0 & = & \dlambda d @.@ \bot_{D_1} \\
%%    p_0 & = & \dlambda d @.@ \bot_{D_0} \\
%%    e_{i+1} & = & F_{\Sigma}(p_i,e_i) \\
%%    p_{i+1} & = & F_{\Sigma}(e_i,p_i)
%% \end{array}\]
%% The following is an easy fact to prove:
%% \begin{lemma}
%% For every $i$ and $x$ element of $D_i$ we have $p_i\cdot e_i(x) = x$. For
%% every $y$ element of $D_{i+1}$ we have that $e_i\ cdot p_i(y) \sqsubseteq y$.
%% \end{lemma}
%% Consider now the cpo defined by the carrier set
%%    \[ \{ x \in \Pi_{i \in \omega}D_i \;\mid\; x_n = p_n(x_{n+1}) \} \]
%% and the pointwise order induced by the order in each $D_i$, and $\bot$ element the
%% infinite tuple of the corresponding $\bot$ elements. This cpo {\em is} going to be the
%% set $D_{\infty}$. To prove this we need som more definitions. Let $j_{n,m} : D_n =>_c D_m$
%% be defined as:
%% \[\begin{array}{lcl}
%%    j_{n,m}(d) & = & \left\{\begin{array}{ll}
%%                              e_{m-1}\cdot\ldots \cdot e_n(d) & n < m \\
%%                              d                        & n = m \\
%%                              p_{n-1}\cdot\ldots \cdot p_n(d) & n > m
%%                           \end{array}\right.
%% \end{array}\]
%% and define $j_i : D_i =>_c D_\infty$ as:
%% \[\begin{array}{lcl}
%%    j_i(d) & = & \langle j_{i,0}(d),j_{i,1}(d),\ldots \\
%% \end{array}\]
%% We can easily show that $j_i$ and $\pi_i$ (the $i$-th projection from a tuple) form
%% an embedding-projection pair:
%% \begin{lemma}
%% For all $i$ and $x$ element of $D_{\infty}$ it is $j_i(\pi_i(x)) \sqsubseteq x$.
%% For all $y$ element of $D_{i}$ it is $\pi_i(j_i(y)) = y$.
%% \end{lemma}
%% The most important theorem is that the limit of $j_i\cdot\pi_i$ is the identity.
%% \begin{lemma}\label{lem:id-sqcup}
%% For all $x$ elements of $D_{\infty}$ it is $\sqcup(j_i\cdot\pi_i)(x) = x$.
%% \end{lemma}
%% \begin{proof} We have one direction by the previous lemma and least upper bounds.
%% So the hard direction is to show that $\sqcup(j_i\cdot\pi_i)(x) \sqsupseteq x$.
%% We know that:
%% \[       \pi_n(j_n(\pi_n(d))) = \pi_n(d) \]
%% by unfolding definitions and we know that $\sqcup(j_n\cdot\pi_n) \sqsupseteq j_n\cdot\pi_n$
%% since it is a least upper bound. By monotonicity we then get that
%% \[      \pi_n(\sqcup(j_n\cdot\pi_n)(d)) \sqsupseteq \pi_n(d) \]
%% but that holds for ever $n$ which means that:
%% \[       \sqcup(j_n\cdot\pi_n)(d) \sqsupseteq d \]
%% since $\sqsubseteq$ on $D_\infty$ is defined by the conjuction of the
%% pointwise $\sqsubseteq$ on each $D_i$.
%% \end{proof}
%% Now let us consider the chain
%% \[ F_{\Sigma}(D_0,D_0) \quad F_{\Sigma}(D_1,D_1) \quad \ldots \]
%% and the pointed cpo $F_{\Sigma}(D_{\infty},D_{\infty})$ we have that:
%% \[\begin{array}{lcl}
%%     F_{\Sigma}(p_i,e_i)   & : & F(D_i,D_i) =>_c F(D_{i+1},D_{i+1}) \\
%%     F_{\Sigma}(\pi_i,j_i) & : & F(D_i,D_i) =>_c F(D_{\infty},D_{\infty})
%% \end{array}\]
%% we will show that there exists an isomorphism $D_\infty \cong F_{\Sigma}(D_\infty,D_\infty)$.

%% \begin{lemma} Define functions $\roll = \sqcup(j_{i+1}\cdot F_{\Sigma}(j_i,\pi_i))$ and
%% $\unroll = \sqcup (F_{\Sigma}(\pi_i,j_i)\cdot \pi_{i+1})$. They form the required isomorphism
%% $D_\infty \cong F_{\Sigma}(D_\infty,D_\infty)$.
%% \end{lemma}
%% This lemma concludes the proof of Lemma~\ref{lem:rec-solution}.

%% The following fact will be extremely useful in establishing the existence of solutions
%% to recursive equations {\em over} the recursively defined domain via approximating the
%% denotations. Let us call $\rho_i = j_i\cdot\pi_i$.

%% \begin{theorem}\label{lem:min-inv-reqs} The following are true:
%% \begin{itemize}
%%    \item $\unroll \cdot \rho_{i+1} \cdot \roll = F_{\Sigma}(\rho_i,\rho_i)$
%%    \item $\sqcup\rho_i(d) = d$, for all elements $d$ of $D_{\infty}$.
%% \end{itemize}
%% \end{theorem}


\subsection{Denotational semantics of expressions and programs}

\begin{figure}
$$
\setlength{\arraycolsep}{2pt}
\begin{array}{c}
\begin{array}{rcl}
\multicolumn{3}{l}{\interp{e}{\cdot}{\cdot} : (\FVarCpo =>_c D_{\infty}) \times (\VarCpo =>_c D_{\infty}) =>_c D_{\infty}} \\
  \interp{x}{\sigma}{\rho} & = & \rho(x) \\
  \interp{f\;[\taus]}{\sigma}{\rho} & = & \sigma(f) \\
  \interp{K\;[\taus]\;(\ol{e})}{\sigma}{\rho} & = & \roll{\ret{\injK{K}{\ol{\interp{e}{\sigma}{\rho}}}}} \\
  \interp{e_1\;e_2}{\sigma}{\rho} & = & \dapp(\interp{e_1}{\sigma}{\rho}, \interp{e_2}{\sigma}{\rho}) \\
  \interp{@BAD@}{\sigma}{\rho} & = & \roll{\ret{\injBad}} \\[2mm]
\hline \\
\multicolumn{3}{l}{\interp{u}{\cdot}{\cdot} : (\FVarCpo =>_c D_{\infty}) \times (\VarCpo =>_c D_{\infty}) =>_c D_{\infty}} \\[1mm]
  \interp{e}{\sigma}{\rho} & = & \interp{e}{\sigma}{\rho} \\[1mm]
  \interp{@case@\;e\;@of@ \ol{ K\;\ys -> e_K}}{\sigma}{\rho}
          & = & \unroll{\interp{e_K}{\sigma}{\rho, \ol{y |-> d}}} \\
          & & \quad \text{if $\unroll{\interp{e}{\sigma}{\rho}} = \injK{K}{\ol{d}}$} \\
          & & \quad \text{and $K$ is a @case@ branch} \\
          & = & \injBad \quad \text{if}\; \unroll{\interp{e}{\sigma}{\rho}} = \injBad \\
          & = & \bot \quad \text{otherwise}
\end{array} \\
\hline \\
\begin{array}{rcl}
\multicolumn{3}{l}{\dbrace{P} : (\FVarCpo =>_c D_{\infty}) =>_c (\FVarCpo =>_c D_{\infty})}  \\[1mm]
\dbrace{P}_{\sigma} f & = & \roll{\ret{\injFun{\dlambda d_1 @.@ \ldots
       \roll{\ret{\injFun{\dlambda d_n @.@ \interp{u}{\sigma}{\ol{x |-> d}}}}}\ldots}}} \\
 && \quad \text{ if } (f\;\ol{a}\;\ol{x} = u) \in P \\
          & = & \bot \quad \text{otherwise}
\end{array}
\end{array}
$$
\caption{Denotational semantics of $\theLang$} \label{f:denot}
\end{figure}

Figure~\ref{f:denot} gives the denotational interpretations
of expressions $e$, right hand sides $u$, and programs $P$, in
terms of the domain-theoretic language and combinators we have defined.

First, the denumerable set of term variable names $x_1,\ldots$ induces a discrete
cpo $\VarCpo$  and the denumerable set of function variable names $f_1,\ldots$ induces a discrete
cpo $\FVarCpo$. We define, {\em semantic term environments} to be the cpo $(\VarCpo =>_c D_{\infty})$,
and {\em semantic function environments} to be the cpo $(\FVarCpo =>_c D_{\infty})$.

Figure~\ref{f:denot} defines the denotational semantics of expressions
$\dbrace{e}$ as a continuous map from these two environments to $D_{\infty}$.
It is entirely straightforward except for application, which depends on
the continuous function $\dapp : D_\infty \times D_\infty =>_c D_\infty$,
defined thus\footnote{
A small technical remark: we write
the definition with pattern matching notation
$\dapp(d,a)$ (instead of using $\pi_1$ for projecting
out $d$ and $\pi_2$ for projecting out $a$) but that is fine,
since $\times$ is not a lifted construction.
Also note that we are, as advertised, suppressing uses
of $\bindK$, $\rollK$, etc.
}:
{\setlength{\arraycolsep}{2pt}
\[\begin{array}{rcll}
  \dapp(d,a) & = & d_f(a)    & \text{if}\; d = \injFun{d_f} \\
             & = & \injBad & \text{if}\; d = \injBad \\
             & = & \bot    & \text{otherwise}
\end{array}\]}%
That is, application applies the payload $d_f$
if the function $d$ comes
from the appropriate component of $V_{\infty}$, propagates $\injBad$,
and otherwise returns $\bot$.  %% Indeed application, partial
%% application, and currying are all definable in cpos of continuous
%% functions, so we will be using $\lambda$-calculus notation for our
%% domain theory, as above.

The semantics of right-hand sides $\dbrace{u}$ is defined similarly.  The
semantics of a @case@ expression is the semantics of the matching branch,
if one exists. Otherwise, like application, it propagates $\injBad$.
In all other cases we return $\bot$, not $\injBad$;
all the missing cases can only be constructors
of different datatypes than the datatype that $K$ belongs to, because
all @case@ expressions are complete (Section~\ref{s:syntax}).
This treatment corresponds directly to our treatment of $unr$ in Section~\ref{s:case-fol}.

Finally, Figure~\ref{f:denot} gives the semantics of a program $P$, which should
be read recalling its syntax in Figure~\ref{fig:syntax}.
Since $\dbrace{P})$ is continuous, its limit exists and is an element of the
cpo $\FVarCpo =>_c D_{\infty}$.

\begin{definition}
We will refer to the limit of the $\dbrace{P}$ as $\dbrace{P}^{\infty}$ in what follows.
Moreover, to reduce notational overhead below, for a program with no free variables we
will use notation $\dbrace{e}$ to mean $\interp{e}{\dbrace{P}^\infty}{\cdot}$, and
$\dbrace{e}_\rho$ to mean $\interp{e}{\dbrace{P}^\infty}{\rho}$
\end{definition}

% ------------------ omit this -------------------
\begin{comment}
Types do not matter at all for our denotational semantics.
\begin{lemma}[Type irrelevance]
It is the case that $\interp{u}{\sigma}{\rho} = \interp{u[\ol{\tau}/\as]}{\sigma}{\rho}$
for any type substitution of variable $\as$ to types $\taus$.
%% and $\interp{e}{\sigma}{\rho} = \interp{e[\ol{\tau}/\as]}{\sigma}{\rho}$.
\end{lemma}
%% \begin{proof} Straightforward induction. \end{proof}
The following is an essential lemma for establishing the soundness of denotational semantics:
\begin{lemma}[Substitutivity]
If $\Sigma;\Delta,x{:}\tau |- e : \tau$ and $\rho$ is a semantic environment
and $\Sigma;\Delta |- e' : \tau'$ then
\[ \interp{e}{\sigma}{\rho,x |-> \interp{e'}{\sigma}{\rho}} = \interp{e[e'/x]}{\sigma}{\rho} \]
and if $\Sigma;\Delta,x{:}\tau |- u : \tau$ then
\[ \interp{u}{\sigma}{\rho,x |-> \interp{e'}{\sigma}{\rho}} = \interp{u[e'/x]}{\sigma}{\rho} \]
\end{lemma}
\end{comment}
% ------------------ omit this -------------------
Although we have not presented a formal operational semantics, we state the usual
soundness and adequacy results:
\begin{theorem}[Soundness and adequacy]\label{thm:adequacy}
Assume $\Sigma |- P$ and $u$ with no free term variables. Then (i)~if $P |- u \Downarrow v$ then $\dbrace{u} = \dbrace{v} $; and (ii)~if $\unroll{\dbrace{e}} \neq \bot$, then
$\exists v$ such that $P |- e \Downarrow v$.
\end{theorem}
\begin{comment}
The proof of this theorem is also standard domain theory so we only sketch the
high-level road-map: The proof proceeds by defining a {\em logical relation} between
semantics and syntax, via the use of a bi-functor on admissible relations between elements
of $D_\infty$ and closed expressions, and using minimal
invariance~\cite{pitts-rel-domains} to show that this
bi-functor has a fixpoint. Adequacy then follows from the {\em fundamental theorem} of this
logical relation, which asserts that every expression is related to its denotation.
\end{comment}
%% To do this we define a {\em logical relation} first between semantics
%% and syntax. Let $Rel \subseteq D_\infty \times Expr$ be the space of
%% {\em admissible} and {\em equality-respecting} relations between
%% denotations and closed (non-necessarily well-typed) terms. Some explanations:
%% \begin{itemize}
%%   \item $R \in Rel$ is {\em admissible} iff whenever
%%   $R(d_i,e)$ for every element of a chain $d_1\ldots$ then also $R(\sqcup_{\omega}d_i,e)$.
%%   \item $R \in Rel$ is {\em equality-respecting} iff for every
%%   $R(d,e)$ and $d' = d$ (according to the equality on $D_{\infty}$) it also is
%%   $R(d',e)$.
%% \end{itemize}

%% Let use define the following bi-functor on the space of $Rel$ relations:
%% {\setlength{\arraycolsep}{2pt}
%% \[\begin{array}{lcl}
%%    F_{P}(R^{-},R^{+}) & = & \{ (d,e)\;\mid\;\forall \ol{d} @.@ \unroll(d) = \ret(\inj{K_1^\ar}\langle\oln{d}{\ar}\rangle) ==> \\
%%                    &   & \quad \exists \oln{e}{\ar} @.@ P |- e \Downarrow K_1[\taus](\ol{e}) \land (d_i,e_i) \in R^{+} \} \\
%%                    & \cup & \ldots \\
%%                    & \cup & \{ (d,e)\;\mid\;\forall d_0 @.@ \unroll(d) = \ret(\inj{->}(d_0)) ==> \\
%%                    &   & \quad \exists v @.@ P |- e \Downarrow v \;\land \\
%%                    &  & \quad\quad \forall (d',e') \in R^{-} @.@ (\dapp(d,d'),v\;e') \in R^{+} \}  \\
%%                    & \cup & \{ (d,e)\;\mid\; \unroll(d) = \ret(\inj{bad}(1)) ==> \\
%%                    &   & \quad P |- e \Downarrow @BAD@ \}
%% \end{array}\]}

%% \begin{lemma} There exists negative and a positive fixpoint of $F_{P}$ and they coincide: let us call this
%% $F_{P}^\infty$ -- it is isomorphic to $F_{P}(F_P^\infty,F_P^\infty)$.
%% \end{lemma}
%% \begin{proof}
%% We can follow the standard roadmap described in the work of Pitts to show this, taking
%% advantage of the approximation on every element of $D_{\infty}$ given in
%% Lemma~\ref{lem:min-inv-reqs}.
%% \end{proof}

%% \begin{lemma}\label{lem:bot-in-fix}
%% $(\roll(\bot),e) \in F_{P}^\infty$. \end{lemma}

%% \begin{lemma}\label{lem:eval-respecting}
%% If $(d,e) \in F_{P}^\infty$ and $P |- e \Downarrow v$ then $(d,v) \in F_{P}^\infty$.
%% Moreover, if $(d,v) \in F_{P}^\infty$ and $P |- e \Downarrow v$ then $(d,e) \in F_{P}^\infty$.
%% \end{lemma}
%% \begin{proof}
%% For the first part,
%% if $(d,e) \in F_{P}^\infty$ then $(d,e) \in F_{P}(F_{P}^\infty,F_{P}^\infty)$.
%% By the definition of $F_{P}(\cdot,\cdot)$ and by rule \rulename{EVal} the
%% result follows. The second part is a similar case analysis.
%% \end{proof}

%% \begin{lemma}[Fundamental theorem for expressions]\label{lem:fund-thm-exp}
%% For all $\sigma$ such that $(\sigma(f),f\;[\taus]) \in F_P^\infty$ and
%% all $\rho$ and vectors of closed terms $\ol{e}$ such that $(\rho(x_i),e_i) \in F_P^\infty$
%% and all $e$ with free variables in $\ol{x}$ it must be the case
%% that $(\interp{e}{\sigma}{\rho},e[\ol{e}/\ol{x}]) \in F_P^\infty$.
%% \end{lemma}
%% \begin{proof} The proof is by induction on $e$.
%% \begin{itemize}
%%   \item Case $e = x_i$ for some $x_i \in \ol{x}$ follows by the assumptions.
%%   \item Case $e = f\;[\taus]$ for some $f$ follows by assumptions.
%%   \item Case $e = K^\ar[\taus](\oln{e'}{\ar})$. By induction hypothesis we
%%   have that for each $e'_i$ it is $(\interp{e'_i}{\sigma}{\rho},e'_i[\ol{e}/\ol{x}]) \in F_P^\infty$ and
%%   by using rule \rulename{EVal} we are done since
%%       \[ \interp{K^{\ar}[\taus](\oln{e'}{\ar})}{\sigma}{\rho} = \roll(\ret(\inj{K}(\langle\ol{\interp{e'_i}{\sigma}{\rho}}\rangle))) \]
%%   \item Case $e = @BAD@$ follows by unfolding definitions.
%%   \item Case $e = e_1\;e_2$. We need to show that
%%      \[ (\interp{e_1\;e_2}{\sigma}{\rho},e_1[\ol{e}/\ol{x}]\;e_2[\ol{e}/\ol{x}]) \in F_P^\infty \]
%%   By induction hypothesis we have that
%%   \begin{eqnarray}
%%      (\interp{e_1}{\sigma}{\rho},e_1[\ol{e}/\ol{x}]) \in F_P^\infty \label{eqn:e1} \\
%%      (\interp{e_2}{\sigma}{\rho},e_2[\ol{e}/\ol{x}]) \in F_P^\infty \label{eqn:e2}
%%   \end{eqnarray}
%%   Equation~\ref{eqn:e1} gives four cases: First, if $\interp{e_1}{\sigma}{\rho} = \roll(\bot)$ then we
%%   are done since $\dapp(\bot,\_) = \roll(\bot)$ and $(\roll(\bot), e_1\;e_2) \in F_P^\infty$ by Lemma~\ref{lem:bot-in-fix}.
%%   Second, if $\interp{e_1}{\sigma}{\rho} = \roll(\ret(\inj{K}(\langle\ol{d}\rangle)))$ for some constructor
%%   $K$ then $\dapp(\interp{e_1}{\sigma}{\rho},\_) = \roll(\bot)$ and by similar reasoning as above we are done.
%%   Third, if $\interp{e_1}{\sigma}{\rho} = \roll(\ret(\inj{bad}(1)))$ then it must be that $P |- e_1[\ol{e}/\ol{x}] \Downarrow @BAD@$ by
%%   induction hypothesis, and by rule \rulename{EBadApp} we know that $P |- e_1[\ol{e}/\ol{x}]\;e_2[\ol{e}/\ol{x}] \Downarrow @BAD@$ hence,
%%   by Lemma~\ref{lem:eval-respecting} we are done. The final case is the interesting one, where
%%   $\interp{e_1}{\sigma}{\rho} = \roll(\ret(\inj{->}(d_0)))$ in which case by induction hypothesis we know that
%%   $(d_0(\interp{e_2}{\sigma}{\rho}), v\;e_2[\ol{e}/\ol{x}]) \in F_P^\infty$ for $P |- e_1[\ol{e}/\ol{x}] \Downarrow v$. But we know that
%%   $v\;e_2[\ol{e}/\ol{x}]$ evaluates to a value {\em iff} $e_1[\ol{e}/\ol{x}]\;e_2[\ol{e}/\ol{x}]$ evaluates to a value and by
%%   Lemma~\ref{lem:eval-respecting} we are done.
%% \end{itemize}
%% \end{proof}


%% \begin{lemma}[Fundamental theorem for top-level expressions]\label{lem:fund-thm-case}
%% For all $\sigma$ such that $(\sigma(f),f\;[\taus]) \in F_P^\infty$ and
%% all $\rho$ and vectors of closed terms $\ol{e}$ such that $(\rho(x_i),e_i) \in F_P^\infty$
%% and all $u$ with free variables in $\ol{x}$ it must be the case
%% that $(\interp{u}{\sigma}{\rho},u[\ol{e}/\ol{x}]) \in F_P^\infty$.
%% \end{lemma}
%% \begin{proof} By induction on $u$. If $u$ is a term $e$ then we are immediately done
%% by Lemma~\ref{lem:fund-thm-exp}. If $u = @case@\;e\;@of@\;\ol{K\;\ys -> e'}$ then the
%% result follows by appealing to the induction hypothesis for $e$ and performing a case
%% analysis on $\interp{e}{\sigma}{\rho}$ -- in the interesting case we appeal further to
%% Lemma~\ref{lem:fund-thm-exp} for a matching $e_K$ and the evaluation-respecting lemma,
%% Lemma~\ref{lem:eval-respecting}.
%% \end{proof}

%% Finally, for the recursive functions environment $P$ we prove the following.
%% \begin{lemma} For any $f$, $(\dbrace{P}^\infty(f),f\;[\taus]) \in F_P^\infty$. \end{lemma}
%% \begin{proof}
%% Since $F_P^\infty$ is itself an {\em admissible} relation, we need only prove that:
%% \[ \forall i @.@ \forall f @.@ (\dbrace{P}^i(f),f\;[\taus]) \in F_P^\infty \]
%% which we do by induction on $i$. For $i = 0$ we are immediately done by Lemma~\ref{lem:bot-in-fix}.
%% Let us assume that the property is true for $i$. We must show it is true for $i+1$. That is,
%% we must show that $(\dbrace{P}^{i+1}(f),f\;[\taus]) \in F_P^\infty$. Hence, if $f$ has arity $n$, by
%% the definition of $\dbrace{P}$ and the definition of the logical relation it is enough to show that
%% for all $(\oln{d}{n},\oln{e}{n}) \in F_P^\infty$ it must be the case that
%% \[    (\interp{u}{\dbrace{P}^i}{\ol{x |-> d}}, u[\ol{e}/\ol{x}]) \in F_P^\infty \]
%% for $f |-> (\Lambda\as @.@ \lambda\oln{x{:}\tau}{n} @.@ u) \in P$. But that follows by
%% Lemma~\ref{lem:fund-thm-case}, since by induction hypothesis it is the case that for
%% every $f$ we have $(\dbrace{P}^i(f),f\;[\ol{\tau}]) \in F_P^\infty$.
%% \end{proof}

%% \begin{corollary}\label{cor:fund-thm-top}
%% For every closed expression $e$ in $P$ (not-necessarily well-typed) we have that
%% $(\interp{e}{\dbrace{P}^\infty}{\cdot}, e) \in F_P^\infty$.
%% \end{corollary}

%% From this corollary, adequacy follows by unfolding definitions.

%% \begin{corollary}[Model-based-reasoning]
%% If $\Sigma |- P$ and and $e_1$ contains no free term variables and $e_2$ contains no free term
%% variables.

%% $\Sigma;\cdot |- e_1 : \tau$ and $\Sigma;\cdot |- e_2 : \tau$,
%% then for every closed $e$ such that $\Sigma |- P$ and $\Sigma;\cdot |- e : \tau -> Bool$,
%% if $\interp{e_1}{\dbrace{P}^\infty}{\cdot} = \interp{e_2}{\dbrace{P}^\infty}{\cdot}$ then
%% $P |- e\;e_1 \Downarrow$ iff $P |- e\;e_2 \Downarrow$.
%% \end{corollary}
%% \begin{proof}
%% For one direction assume that $P |- e\;e_1 \Downarrow w$, hence by computational soundness it must be that
%% $\interp{e\;e_1}{\dbrace{P}^\infty}{\cdot} = \roll(\ret(d))$. By assumptions we must also
%% have that $\interp{e\;e_2}{\dbrace{P}^\infty}{\cdot} = \roll(\ret(d))$. By the fundamental theorem
%% we know that
%% \[ (\interp{e\;e_2}{\dbrace{P}^\infty}{\cdot}, e\;e_2) \in F_{P}^\infty \]
%% and hence $P |- e\;e_2 \Downarrow$. The other direction is symmetric.
%% \end{proof}

\subsection{Denotational semantics of contracts} \label{s:den-sem-contracts}

\begin{figure}
$$
\setlength{\arraycolsep}{1pt}
\begin{array}{rcl}
\multicolumn{3}{c}{
\ruleform{\dbrace{\Ct}_{\rho} \subseteq D_{\infty}} }
\\ \\
\dbrace{x \mid e}_\rho
  & =  & \{ d \mid \unroll{d} = \bot \, \lor \, \unroll{\dbrace{e}_{\rho,x|->d}}
                \in \{ \ret{\injKZ{True} , \bot} \} \}
\\[1em]
\dbrace{(x{:}\Ct_1) -> \Ct_2}_{\rho}
 & = & \{ d \mid
           \forall d' \!\in\! \dbrace{\Ct_1}_\rho.
           \dapp(d,d') \in \dbrace{\Ct_2}_{\rho,x|->d'}
           \}
\\[1em]
\dbrace{\Ct_1 \& \Ct_2}_\rho
 & = & \{ d | d \in \dbrace{\Ct_1}_\rho /\ d \in \dbrace{\Ct_2}_\rho \}
\\[1em]
\dbrace{\CF}_\rho & = &  \Fcf^{\infty}  \\
\multicolumn{3}{l}{\text{where}} \\
   F_{\lcfZ}^{\infty} & = & \{ \bot \} \\
                   & \cup & \{\;\injK{K}{\ol{d}} \mid K^n \in \Sigma,\; d_i \in F_{\lcfZ}^{\infty} \} \\
                   & \cup & \{\;\injFun{d} \mid \forall d' \in F_{\lcfZ}^{\infty}.\; d(d') \in F_{\lcfZ}^{\infty} \}
\end{array}
$$
\caption{Denotations of contracts} \label{f:den-sem-contracts}
\end{figure}

Now we are ready to say formally what it means for a function to satisfy a contract.
We define the semantics of a contract as the set of denotations that satisfy it:
\[              \dbrace{\Ct}_\rho \subseteq D_\infty  \]
where $\Ct$ is a contract with free term variables in the semantic environment $\rho$.
Figure~\ref{f:den-sem-contracts} gives the definition of this function.
A base contract $\{x \mid e\}$ is satisfied by $\bot$ or
or by a computation that causes the predicate $e$ to become $\bot$
or return \True\footnote{
In previous
work \cite{xu+:contracts} the base contract also
required {\em crash-freedom}.
We changed this choice only for reasons of taste; both choices
are equally straightforward technically.}.
%% The reason for introducing the
%% possibility of $\bot$ either for the expression or for the predicate
%% is associated with {\em admissibility} of induction, a topic that we
%% return to in Section~\ref{sect:induction}.
The denotation of an arrow contract, and of conjunction, are both straightforward.

The $\CF$ contract is a little harder. Intuitively an expression is crash-free iff it cannot
crash if plugged into an arbitrary crash-free context. Of course this is a
self-referential definition so how do we know it makes sense? The original paper
\cite{xu+:contracts} specified that an expression is crash free iff it
cannot crash when plugged into a context that {\em syntactically} does not contain the
@BAD@ value. This is a reasonable definition in the operational semantics world, but
here we can do better because we are working with elements of $D_\infty$. Using
techniques developed by Pitts~\cite{pitts-rel-domains} we can define crash-freedom denotationally as the greatest solution
to the recursive equation for $F_{\lcfZ}^{\infty}$ in Figure~\ref{f:den-sem-contracts}. Technically,
since the equation involves mixed-variance recursion, to show that such a fixpoint exists we have
to use minimal invariance.
In addition we get the following, which will be useful later on for induction:
\begin{lemma}\label{lem:cf-admissible}
$\bot \in F_{\lcfZ}$ and $F_{\lcfZ}^{\infty}$ is admissible, that is if all elements of a chain are in
$F_{\lcfZ}^{\infty}$ then so is its limit.
\end{lemma}

%% Using the techniques developed by Pitts we may define a recursive predicate for crash-freedom,
%% using the following strict bi-functor on admissible sets of
%% denotations $S^{-},S^{+} \subseteq D_\infty$.
%% {\setlength{\arraycolsep}{2pt}
%% \[\begin{array}{rcl}
%%    F_{\lcfZ}(S^{-},S^{+}) & = & \{\;d\;\mid\;\unroll{d} \neq \ret{\inj{bad} 1} \text{ and } \\
%%                       &    & \quad \text{for all } \ol{d}, \\
%%                       &    & \quad\quad\text{ if } \unroll{d} = \ret{\inj{K_1^\ar}\langle\oln{d}{\ar}\rangle} \\
%%                       &    & \quad\quad\text{ then } \ol{d} \in S^{+} \} \\
%%                    & \cup & \ldots \\
%%                    & \cup & \{\;d\;\mid\;\unroll{d} \neq \ret{\inj{bad} 1} \text{ and } \\
%%                    &      & \quad \text{for all } d_0, \\
%%                    &      & \quad\quad\text{ if } \unroll{d} = \ret{\inj{->}(d_0)} \text{ then } \\
%%                    &      & \quad\quad\text{ for all }\;d' \in S^{-} \text{ it is } d_0(d') \in S^{+} \}  \\
%% \end{array}\]}
%% The intersection of admissible sets is admissible so the $\Fcf$ bi-functor has a negative and positive fixpoint, and by minimal
%% invariance they coincide (one direction follows by Tarski-Knaster, the other can be inductively proved using the fact that the
%% lub of the chain of embedding-projections is the identity and the fact that this
%% functor preserves admissibility for the covariant argument). Let us call this admissible set $\Fcf^{\infty} \subseteq D_{\infty}$.

%% We may now define the interpretation of $\CF$ contracts using this fixpoint:
%% \[     \dbrace{\CF}_\rho(d) \text{ iff }  d \in \Fcf^{\infty}    \]

%% %% We may also define the denotational semantics of contracts, below. We assume again
%% %% that there is a program $P$, well-formed in a signature $\Sigma$. In the definition,
%% %% $\rho$ is a semantic environment.

%% %% \begin{definition}[Denotational semantics of contracts]
%% %% \[\begin{array}{l}
%% %%     \dbrace{x \mid e}_\rho(d) \text{ iff } \\
%% %%         \quad \unroll(d) = \bot \text{ or }
%% %%         \unroll(\interp{e}{\dbrace{P}^\infty}{\rho,x|->d}) = \bot\;\text{ or } \\
%% %%         \quad \unroll(\interp{e}{\dbrace{P}^\infty}{\rho \uplus x|->d}) = \ret(\inj{\True} 1) \\ \\
%% %%     \dbrace{(x{:}\Ct_1) -> \Ct_2}_{\rho}(d) \text{ iff } \\
%% %%         \quad \text{for all } d_x \in D_\infty \\
%% %%         \quad\quad \text{if }
%% %%                      \dbrace{\Ct_1}_\rho(d_x)\text{ then }
%% %%                      \dbrace{\Ct_2}_{\rho,x|->d_x}(\dapp(d,d_x)) \\ \\
%% %%     \dbrace{\CF}_\rho(d) \text{ iff }  d \in \Fcf^{\infty} \\  \\
%% %%     \dbrace{\Ct_1 \& \Ct_2}_\rho(d) \text{ iff }
%% %%        \dbrace{\Ct_1}_\rho(d) \text{ and }
%% %%        \dbrace{\Ct_2}_\rho(d)
%% %% \end{array}\]
%% %% For a closed contract $\Ct$ we will use notation
%% %% $\dbrace{\Ct}$ for its denotation in the empty
%% %% semantic environment.
%% %% \end{definition}

%% %% To the extend that in the end we are only interested in base contracts, giving a
%% %% denotational semantics of full-higher-order contracts is not really interesting
%% %% but we do this anyway. For a given denotation $d$, we define the
%% %% predicate $\interp{\Ct}{\dbrace{P}^\infty}{\rho}(d)$ by recursion on the structure
%% %% of the contract $\Ct$, such that:


\subsection{Soundness of the logic translation}  \label{s:soundness}

We have developed a formal semantics for expressions as well as contracts, so it is time we
see how we can use this semantics to show that our translation to first-order logic is sound
with respect to this semantics.

Our plan is to give an interpretation (in the FOL sense of the term) to our
translated FOL terms, using the carrier set $D_\infty$ as our model.
Happily this is straightforward to do:
\[\begin{array}{rcl}
   \linterp{f(\ol{t})} & = & \dapp(\dbrace{f},\ol{\linterp{t}}) \\
   \linterp{app(t_1,t_2)}     & = & \dapp(\linterp{t_1}, \linterp{t_2}) \\
   \linterp{f_{ptr}}  & = & \dbrace{f} \\
   \linterp{K(\ol{t})} & = & \roll{\ret{\injK{K}{\ol{\linterp{t}}}}} \\
   \linterp{\sel{K}{i}(t)} & = &  d_i \quad \text{if}\; \linterp{t} = \injK{K}{\ol{d}} \\
                           & = & \bot \quad \text{otherwise} \\
  \linterp{unr}       & = & \bot \\
  \linterp{bad}       & = & \injBad
\end{array}\]

The essential soundness theorem that states that our interpretation makes sense is
the following.
\begin{theorem}[Interpretation respects denotations]\label{thm:interp-respect}
Assume that $\Sigma |- P$ and expression $e$ does not contain any free variables.
Then, if $\etrans{}{\cdot}{e} = t$ then $\linterp{t} = \dbrace{e}$.
\end{theorem}
The proof is an easy induction on the size of the term $e$.

Our soundness results are expressed with the following theorem

\begin{theorem}\label{thm:models-inf}
If $\Sigma |- P$ then $\langle D_{\infty},{\cal I}\rangle \models \Th \land \ptrans{\Sigma}{P}$
\end{theorem}
%% \begin{theorem}\label{thm:models-defs}
%% If $\Sigma |- P$ then $\langle D_{\infty},{\cal I}\rangle \models \dtrans{\Sigma}{P}$.
%% \end{theorem}
%% \begin{theorem}\label{thm:models-cf} $\langle D_\infty,{\cal I}\rangle \models \Th_\lcfZ$.
%% \end{theorem}
As a corollary we get our ``guiding principle'' from the introduction.
\begin{corollary}\label{cor:guiding-principle}
Assume that $\Sigma |- P$ and $e_1$ and $e_2$ contain no free term variables. The following
are true:
\begin{itemize*}
  \item $\dbrace{e_1} = \dbrace{e_2}$ iff ${\cal I}(\etrans{}{}{e_1}) = {\cal I}(\etrans{}{}{e_2})$.
  \item If $\Th \land \ptrans{\Sigma}{P} |- \etrans{}{}{e_1} = \etrans{}{}{e_2}$ then $\dbrace{e_1} = \dbrace{e_2}$.
\end{itemize*}
\end{corollary}
\begin{proof} The first part follows directly from Theorem~\ref{thm:interp-respect}.
For the second part the left-hand side implies that $\etrans{}{}{e_1}$ and $\etrans{}{}{e_2}$ are
equal in all models of $\Th \land \ptrans{\Sigma}{P}$, in particular (using Theorem~\ref{thm:models-inf})
by $\langle D_{\infty},{\cal I}\rangle$ and by the first part the case is finished.
\end{proof}


\begin{theorem}\label{thm:den-contr-satisfaction} Assume that $e$ and $\Ct$ contain no free
term variables. Then the FOL translation of the claim $e \in \Ct$ holds in the model
if and only if the denotation of $e$ is in the semantics of $\Ct$.  Formally:
$$\langle D_\infty,{\cal I}\rangle \models \ctrans{}{\cdot}{e \in \Ct}
  \;\; \Leftrightarrow \;\; \dbrace{e} \in \dbrace{\Ct}
$$
\end{theorem}

% What does the last result tell us about the actual execution of the program $e$? In this
% paper we have elided a discussion about the operational semantics but we discuss this
% in further detail in Section~\ref{sect:discussion}.

\paragraph{Completeness of axiomatisation}

The $D_\infty$ domain has a complex structure and there are many more facts that
hold about elements of $D_\infty$ that are not reflected in any of our axioms in $\Th$.
For instance, here are some admissible axioms that are valid:
\[\begin{array}{l}
    \formula{\forall \oln{x}{n} @.@ app(f_{ptr},\xs) \neq \unr}
    \formula{\land\; app(f_{ptr},\xs) \neq \bad} \\
    \formula{\quad\land\; \forall \oln{y}{k} @.@ app(f_{ptr},\xs) \neq K(\ol{y})} \\
    \text{ for every } (f\;\ol{a}\;\oln{x}{m} = u) \in P
    \text{ and } K \in \Sigma \text{ with } m > n
%% \\
%%     \text{ and every } (K{:}\forall\as @.@ \oln{\tau}{k} -> T\;\as) \in \Sigma \text{ and } m > n
\end{array}\]
These axioms assert that partial applications cannot be equated to
any constructor, $\bot$ nor $\injBad$. If the reader is worried that without a
complete formalisation of all equalities of $D_\infty$ it is impossible to prove any
programs correct, we would like to reassure them that that is not the case as we
shall see in the next section.

\paragraph{Lazy semantics simplifies the translation}
We have mentioned previously (Section~\ref{ssect:trans-exprs}) that the
laziness of $\theLang$ helps in keeping the translation simple, and here we
explain why.

Whenever we use universal quantification in the logic, we really quantify over
{\em any} denotation, including $\bot$ and $\injBad$. In a call-by-name language,
given a function @f x = True@, the axiom $\forall x @.@ f(x) = \injKZ{True}$
is true in the intended denotational model. However, in a call-by-value setting, $x$
is allowed to be interpreted as $\bot$. That means that the unguarded axiom is
actually not true, because $f\,\bot \not= \injKZ{True}$.
Instead we need the following annoying variation:
\[  \forall x @.@ x \neq bad \land x \neq \unr => f(x) = t \]
Moreover, the axioms for the $app(\cdot,\cdot)$ combinator have to be modified
to perform checks that the argument is not $\bot$ or $\injBad$ before actually
calling a function.   In a call-by-name language these guards are needed
only for @case@ and the function part of $app$.

These complications lead to a more complex first-order theory in the call-by-value
case.

%% value, including $\bot$. Very rarely do we have to add a condition that
%% a value is not equal to $\bot$. Only when a function is explicitly strict in a particular argument (by doing pattern matching) do we have to do this. Also, function application in the language corresponds directly to term construction in the logic.

%% To deal with a (pure) strict language, we would have two choices: Either (1) we explicitly make each function strict in all its arguments, by adding extra axioms for each function definition that express what happens when you apply the function to $\bot$, and by adding extra conditions to all the other function axioms that exclude $\bot$; or (2) we use a monadic translation (or some variant thereof), basically turning each application site into a definedness check of the arguments. Neither method leads to a simple first-order theory that is easy to reason about by automatic provers.

%% \spj{What else should we say here.  Something about recursion?  About multiple functions?}




%% \subsection{Denotational versus operational semantics for contracts}
%% TODO -- I have just dumpted material here.

%% We have the rather obvious theorem below.

%% \begin{theorem}[Soundness and completeness for denotational semantics]
%% Assume a program $P$ with signature $\Sigma$, and expression $e$ and contract $\Ct$
%% such that $fv(e) \cup fv(\Ct) \subseteq dom(P)$. Then
%% $\langle D_\infty,{\cal I}\rangle \models \ctrans{\Sigma}{P}{e \in \Ct}$ iff
%% $\interp{\Ct}{\dbrace{P}^{\infty}}{\cdot}(\interp{e}{\dbrace{P}^\infty}{\cdot})$.
%% \end{theorem}




%% \subsubsection{Contract satisfaction and crash-freedom}\label{sect:cf}

%% We would like to define a set of contract-satisfying denotations and also a set of contract-satisfying terms,
%% characterized by $P |- e \in \Ct$, such that the following claim becomes true:

%% \begin{proposition} Assume that $\Sigma |- P$ and $fv(e) \subseteq dom(P)$, i.e. $e$ is closed.
%% Then: $\langle D_\infty,{\cal I}\rangle \models \ctrans{\Sigma}{\Delta}{e \in \Ct}$ iff $P |- e \in \Ct$.
%% \end{proposition}

%% Now there are several problems with coming up with a good definition of $P |- e \in \Ct$,
%% which we elaborate in the following sections.

%% \subsubsection{Problem I: Crash-freedom}

%% Ideally we would like to define crash-freedom {\em semantically} using the following
%% strict bifunctor on admissible sets $S^{-},S^{+} \subseteq D_{\infty}$.
%% {\setlength{\arraycolsep}{2pt}
%% \[\begin{array}{rcl}
%%    F_{\lcfZ}(S^{-},S^{+}) & = & \{\;d\;\mid\;\unroll(d) \neq \ret(\inj{bad}(1))\;\land\; \\
%%                       &    & \quad \forall \ol{d} @.@ \unroll(d){=}\ret(\inj{K_1^\ar}\langle\oln{d}{\ar}\rangle) ==> \ol{d} \in S^{+} \} \\
%%                    & \cup & \ldots \\
%%                    & \cup & \{\;d\;\mid\;\unroll(d) \neq \ret(\inj{bad}(1))\;\land\; \\
%%                    &      & \quad \forall d_0 @.@ \unroll(d) = \ret(\inj{->}(d_0)) ==> \\
%%                    &      & \quad\quad \forall\;d' \in S^{-} ==> \dapp(d,d') \in S^{+} \}  \\
%% \end{array}\]}
%% The $\Fcf$ bifunctor has a negative and positive fixpoint, and by minimal invariance they coincide (one direction
%% follows by Tarski-Knaster, the other can be inductively proved using the approximations on ever element of $D_{\infty}$ given
%% in Lemma~\ref{lem:min-inv-reqs} and the fact that the lub of the chain of $\rho_i$ is the identity and the fact that this
%% functor preserves admissibility for the positive sets). Let us call this admissible set $\Fcf^{\infty} \subseteq D_{\infty}$.

%% We consider this predicate to be the ``ideal crash-freedom'' -- however it is very difficult to give a 1-1 operational
%% definition. The reason is that the $\Fcf$ functor quantifies in the function case over any $d'$ -- whereas in the operational
%% semantics it is only reasonable that we quantify over all terms (or over terms that do not contain @BAD@) In the absense of
%% full abstraction of the domain (which is plausible, especially if we extend the language with other features) it is unclear
%% what a corresponding predicate would look like in terms of operational semantics.

%% We then go for a simpler predicate, which only characterizes crash-freedom for first-order terms,
%% generate by the following functor on {\em admissible} sets of denotations:
%% {\setlength{\arraycolsep}{2pt}
%% \[\begin{array}{rcl}
%%    G_{\lcfZ}(S^{+}) & = & \{\;d\;\mid\; \unroll(d){=}\ret(\inj{K_1^\ar}\langle\oln{d}{\ar}\rangle) \land \ol{d} \in S^{+} \} \\
%%                   & \cup & \ldots \\
%%                   & \cup & \{\;\bot\;\}
%% \end{array}\]}
%% Notice that if $S$ is admissible then so is $G_{\lcfZ}(S)$.

%% %% The $G_{\lcfZ}$ functor has a fixpoint and it is an admissible relation, and we will use its
%% %% fixpoint $G_{\lcfZ}^\infty$, so now we need to say what $G_{\lcfZ|}^\infty$ means operationally.
%% \begin{lemma} The functor $G_{\lcfZ}$ has a unique fixpoint $G_{\lcfZ}^\infty$ on admissible sets. \end{lemma}
%% \begin{proof}
%% The intersection of admissible sets is admissible. Hence we have a complete join semi-lattice (which induces a
%% complete lattice), so the monotone functor $G_{\lcfZ}$ does have a smallest and a greatest fixpoint call
%% it $G_{\lcfZ}^{min}$ and $G_{\lcfZ}^{max}$. Moreover this fixpoint will be an admissible relation. Now it must be
%% that $G_{\lcfZ}^{min} \subseteq G_{\lcfZ}^{max}$ so we only show next that
%% also $G_{\lcfZ}^{max} \subseteq G_{\lcfZ}^{min}$. To do this we will show that:
%% \[ \forall i. d \in G_{\lcfZ}^{max} ==> \rho_i(d) \in G_{\lcfZ}^{min} \]
%% by induction on $i$. For $i = 0$ it follows since $\rho_0(d) = \bot$. Let us assume
%% that it holds for $i$, we need to show that $\rho_{i+1}(d) \in G_{\lcfZ}(G_{\lcfZ}^{min})$.
%% We know however that $d \in G_{\lcfZ}(G_{\lcfZ}^{max}$ and by simply case analysis and appealing
%% to the induction hypothesis we are done. Finally, by admissibility it must be that
%% $\sqcup\rho_i(d) \in G_{\lcfZ}^{min}$ and by Lemma~\ref{lem:min-inv-reqs} it
%% must be that $d \in G_{\lcfZ}^{min}$. This means that the two fixpoints coincide,
%% hence there is only a unique fixpoint of $G_{\lcfZ}$, call it $G_{\lcfZ}^\infty$.
%% \end{proof}

%% Now, we would like to define operationally the set of {\em crash-free} terms as a set $\Ecf$ of
%% closed terms that satisfies:
%% {\setlength{\arraycolsep}{2pt}
%% \[\begin{array}{rcl}
%%    \Ecf & =    & \{ e \;\mid\; P |- e \Downarrow K[\taus](\ol{e}) /\ \ol{e} \in \Ecf \} \\
%%         & \cup & \ldots \\
%%         &      & \{ e \;\mid\; P \not|- e \Downarrow \}
%% \end{array}\]}%
%% We do not know that the set $\Ecf$ exists, so we have to prove it.
%% \begin{lemma}
%% There exists a largest set that satisfies the $\Ecf$ equation above.
%% \end{lemma}
%% \begin{proof}
%% Define $\Ecf$ to be the set
%% \[ \{ e\;\mid\; \interp{e}{\dbrace{P}^\infty}{\cdot} \in G_{\lcfZ}^{\infty}\} \]
%% It is straightforward (by computational adequacy) to show that it satisfies the $\Ecf$ recursive
%% equation above. For uniqueness, assume any other set $E$ that satisfies the recursive equation
%% above. We can show that $\interp{E}{\dbrace{P}^\infty}{\cdot}$ is a
%% fixpoint of $G_{\lcfZ}$ and since there is only one such fixpoint, this is unique. So we have that:
%% \[\begin{array}{ll}
%%  e \in E & ==> \\
%%  \interp{e}{\dbrace{P}^\infty}{\cdot} \in \interp{E}{\dbrace{P}^\infty}{\cdot} & ==> \\
%%  \interp{e}{\dbrace{P}^\infty}{\cdot} \in G_{\lcfZ}^\infty & ==> \\
%%  e \in \Ecf
%% \end{array}\]
%% \end{proof}
%% %% \begin{lemma}
%% %% If $e \in E$ and $\interp{e}{\dbrace{P}^\infty}{\cdot} = \interp{e'}{\dbrace{P}^\infty}{\cdot}$ then $e' in E$.
%% %% \end{lemma}
%% %% This relies on the fact that
%% %% if $\interp{e}{\dbrace{P}^\infty}{\cdot} \in \interp{E}{\dbrace{P}^\infty}{\cdot}$ then $e \in E$. Why is
%% %% that? Because the assumption means that
%% %% $\interp{e}{\dbrace{P}^\infty}{\cdot} \in \{ d | \exists e' \in E /\ d = \interp{e'}{\dbrace{P}^\infty}{\cdot} \}$
%% %% and hence this means that there exists some $e' \in E $ such that
%% %% $\interp{e}{\dbrace{P}^\infty}{\cdot} = \interp{e'}{\dbrace{P}^\infty}{\cdot}$
%% %% \end{proof}

%% Let us extend the interpretation function above $\linterp{\cdot}$ so that:
%% \[\begin{array}{rcl}
%%    \linterp{\lcfZ}  & = & G_{\lcfZ}^{\infty}
%% \end{array}\]

%% \begin{theorem}
%% If $\Sigma |- P$ then we have that $\langle D_{\infty},{\cal I}\rangle \models \Th{\Sigma}{P}^{\lcfZ}$.
%% \end{theorem}

%% Notice that the axiom:
%% \[  \textsc{AxCfC}  \quad \formula{\forall x y @.@ \lcf{x} /\ \lcf{y} => \lcf{app(x,y)}} \]
%% is {\em not validated} by this interpretation of crash-freedom we have given.


%% \subsubsection{Problem II: the absense of full-abstraction}

%% Unfortunately higher-orderness bites again. Having defined the set $\Ecf$ we might define formally
%% the predicate $P |- e \in \Ct$ where $fv(e) \subseteq dom(P)$ and $fv(\Ct) \subseteq dom(P)$ as
%% follows:
%% {\setlength{\arraycolsep}{2pt}
%% \[\begin{array}{lcl}
%%     P |- e \in \{ x\;\mid\;e_p\} & <=> & P |- e \not\Downarrow \text{ or } P |- e_p[e/x] \not\Downarrow \text{ or} \\
%%                                  &     & P |- e_p[e/x] \Downarrow True \\
%%     P |- e \in (x{:}\Ct_1) -> \Ct_2 & <=> &
%%                                  \text{for all } P' e' \text{ s.t. } fv(e') \subseteq dom(P{\uplus}P')  \\
%%                                    &   &  \text{it is } P\uplus P' |- e\;e' \in \Ct_2[e'/x] \\
%%     P |- e \in \Ct_1 \& \Ct_2 & <=> & P |- e \in \Ct_1 \text{ and } P |- e \in \Ct_2 \\
%%     P |- e \in \CF            & <=> & e \in \Ecf
%% \end{array}\]}

%% Note we made the definition above well-scoped but not necessarily well-typed; let's ignore that for now (making everything
%% well-typed includes extra difficulties in the proof but hopefully not surmountable).

%% The interesting case is the case for arrow contracts, where we have extended the set of definitions $P$ with more
%% definitions $P'$ -- that is to allow for tests $e'$ which can have arbitrary computational power, and not only those
%% that can be constructed in the current environment. That is expected the way we have set up things, so let us examine
%% what happens when we try to prove the proposition below:

%% \begin{proposition} Assume that $\Sigma |- P$ and $fv(e) \subseteq dom(P)$, i.e. $e$ is closed.
%% Then: $\langle D_\infty,{\cal I}\rangle \models \ctrans{\Sigma}{\Delta}{e \in \Ct}$ iff $P |- e \in \Ct$.
%% \end{proposition}

%% {\flushleft{\em Failed proof}:}
%% The base case and the case of $\CF$ follow from computational adequacy so we are good. However
%% let's try to prove the arrow case and in particular the $(<=)$ direction.

%% Let us assume that for all $P'$ and $e'$ such that $fv(e') \subseteq dom(P\uplus P')$ it is the case that
%% $P |- e\;e' \in \Ct_2[e'/x]$. We need to show that $\langle D_\infty,{\cal I}\rangle$ is a model of the
%% formula $\forall x. \ctrans{\Sigma}{x}{x \in \Ct_1} => \ctrans{\Sigma}{x}{e\;x \in \Ct_2}$. Let us fix
%% a denotation $d \in D_{\infty}$ and let us assume
%% that $\langle D_{\infty},{\cal I} \rangle \models \ctrans{\Sigma}{x}{x \in \Ct_1}[d/x]$. However, this does not
%% necessarily mean that we can find a closed $e'$ and $P'$, such
%% that $\interp{e'}{\dbrace{P{\uplus}P'}^\infty}{\cdot} = d$ to be able to use the assumptions, unless some sort
%% of full-abstraction property is true. So we are stuck.

%% Here is a concrete counterexample, based on the lack of full-abstraction due to the {\em parallel or} function.
%% Consider the program $P$ below:
%% \[\begin{array}{lcl}
%% f_\omega & |-> & f_\omega \\
%% f & |-> & \lambda (b{:}Bool) @.@ \lambda (h{:}Bool->Bool->Bool) @.@ \\
%%   &     & \quad @if@\;(h\;True\;b)\;\&\&\;(h\;b\;True)\;\&\& \\
%%   &     & \quad\qquad\qquad not\;(h\;False\;False)\;@then@ \\
%%   &     & \quad\quad @if@\;(h\;True\;f_\omega)\;\&\&\;(h\;f_\omega\;True)\;@then@\;@BAD@ \\
%%   &     & \quad\quad @else@\;True \\
%%   &     & \quad @else@\;True
%% \end{array}\]
%% Consider now the candidate contract for $f$ below:
%% \[ \CF -> (\CF -> \CF -> \CF) -> \CF \]
%% Operationally we may assume a crash-free boolean as well as a function $h$ which is
%% $\CF -> \CF -> \CF$. The first conditional ensures that the function behaves like an ``or'' function or
%% diverges. However if we pass the first conditional,
%% the second conditional will always diverge and hence the contract will be satisfied.

%% However, denotationally it is possible to have a {\em monotone} function $por$ defined as follows:
%% \[\begin{array}{lcl}
%%   por\;\bot\;\bot & = & \bot \\
%%   por\;\bot\;True & = & True \\
%%   por\;True\;\bot & = & True \\
%%   por\;False\;False & = & False
%% \end{array}\]
%% with the rest of the equations (for @BAD@ arguments) induced by monotonicity and whatever boolean value
%% we like when both arguments are @BAD@.

%% Now, this is denotationally a $\CF -> \CF -> \CF$ function, and it will pass the first conditional, but it will
%% also pass the second conditional, yielding @BAD@. Hence denotationally the contract for $f$ {\em does not hold}.

%% So we have a concrete case where the $<=$ direction fails. Because of contra-variance of arrow contracts, it is
%% likely that the $=>$ direction is false as well.


%% %% Now it may be the case that for all denotations that semantically satisfy a contract, these denotations {\em are}
%% %% realizable by a term $e'$ and a context $P'$ but it is not entirely clear how to prove this (or if this is a good
%% %% idea). I am not sure if this is true either.
%% %% The other idea out of this situation is to compile the arrow contract differently by not quantifying over all
%% %% denotations but rather some kind of {\em definable} denotations -- but I do not know how exactly to do this.


%% \paragraph{A way out of this?}
%% Well, if we restrict our higher-order tests to those that can be constructed from our signature then
%% we may define the following:

%% {\setlength{\arraycolsep}{2pt}
%% \[\begin{array}{lcl}
%%     P |- e \in \{ x\;\mid\;e_p\} & <=> & P |- e \not\Downarrow \text{ or } P |- e_p[e/x] \not\Downarrow \text{ or} \\
%%                                  &     & P |- e_p[e/x] \Downarrow True \\
%%     P |- e \in (x{:}\Ct_1) -> \Ct_2 & <=> &
%%                                  \text{for all } e' \text{ s.t. } fv(e') \subseteq dom(P)  \\
%%                                    &   &  \text{it is } P |- e\;e' \in \Ct_2[e'/x] \\
%%     P |- e \in \Ct_1 \& \Ct_2 & <=> & P |- e \in \Ct_1 \text{ and } P |- e \in \Ct_2 \\
%%     P |- e \in \CF            & <=> & e \in \Ecf
%% \end{array}\]}
%% Notice that the difference with the previous version of $P |- e \in \Ct$ is that we {\em do not} extend the
%% definitions $P'$ so we don't get the full power of higher-order tests. We show that {\em in the current signature
%% only} does the program satisfy the contract.


%% Why did we do this change? Because denotationally this is not terribly hard to support -- instead of translating
%% \[\begin{array}{l}
%%   \ctrans{\Sigma}{\Gamma}{e \in (x{:}\Ct_1) -> \Ct_2} =
%%   \formula{\forall x @.@ \ctrans{\Sigma}{\Gamma,x}{x \in \Ct_1} => \ctrans{\Sigma}{\Gamma,x}{e\;x \in \Ct_2}}
%% \end{array}\]
%% we use the following:
%% \[\begin{array}{l}
%%   \ctrans{\Sigma}{\Gamma}{e \in (x{:}\Ct_1) -> \Ct_2} = \\
%%   \qquad\qquad\quad
%% \formula{\forall x @.@ \definable{x} \land \ctrans{\Sigma}{\Gamma,x}{x \in \Ct_1} => \ctrans{\Sigma}{\Gamma,x}{e\;x \in \Ct_2}}
%% \end{array}\]
%% where $\definable{x}$ could be axiomatized as containing all terms
%% made up of the functions in $P$, applications, and data constructors:

%% \[\begin{array}{lll}
%%  \textsc{DefCons} & \formula{\forall \xs @.@ \definable{K(\xs)} <=> \definable{\xs}} \\
%%                         & \text{ for every } (K{:}\forall\as @.@ \oln{\tau}{n} -> T\;\as) \in \Sigma \\
%%  \textsc{DefFun}  & \formula{\definable{f_{ptr}}}  \\
%%                         & \text{ for every } (f |-> \Lambda\as @.@ \lambda\oln{x{:}\tau}{n} @.@ u) \in P \\
%%  \textsc{DefApp}  & \formula{\forall x y @.@ \definable{x}\land\definable{y} => \definable{app(x,y)}}
%% %% \formula{\bad \neq \unr}  \\
%% %%  \textsc{AxDisjB} & \formula{\forall \oln{x}{n}\oln{y}{m} @.@ K(\ol{x}) \neq J(\ol{y})} \\
%% %%                   & \text{ for every } (K{:}\forall\as @.@ \oln{\tau}{n} -> T\;\as) \in \Sigma \\
%% %%                   & \text{ and } (J{:}\forall\as @.@ \oln{\tau}{m} -> S\;\as) \in \Sigma \\
%% %%  \textsc{AxDisjC} & \formula{(\forall \oln{x}{n} @.@ K(\ol{x}) \neq \unr \land K(\ol{x}) \neq \bad)} \\
%% %%                   & \text{ for every } (K{:}\forall\as @.@ \oln{\tau}{n} -> T\;\as) \in \Sigma \\ \\
%% %%  \textsc{AxAppA}  & \formula{\forall \oln{x}{n} @.@ f(\ol{x}) = app(f_{ptr},\xs)} \\
%% %%                   & \text{ for every } (f |-> \Lambda\as @.@ \lambda\oln{x{:}\tau}{n} @.@ u) \in P \\
%% %%  %% \textsc{AxAppB}  & \formula{\forall \oln{x}{n} @.@ K(\ol{x}) = app(\ldots (app(x_K,x_1),\ldots,x_n)\ldots)} \\
%% %%  %%                  & \text{ for every } (K{:}\forall\as @.@ \oln{\tau}{n} -> T\;\as) \in \Sigma \\
%% %%  \textsc{AxAppC}  & \formula{\forall x, app(\bad,x) = \bad \; /\ \; app(\unr,x) = \unr}    \\ \\
%% %%  %% Not needed: we can always extend partial constructor applications to fully saturated and use AxAppC and AxDisjC
%% %%  %% \textsc{AxPartA} & \formula{\forall \oln{x}{n} @.@ app(\ldots (app(x_K,x_1),\ldots,x_n)\ldots) \neq \unr} \\
%% %%  %%                  & \formula{\quad\quad \land\; app(\ldots (app(x_K,x_1),\ldots,x_n)\ldots) \neq \bad} \\
%% %%  %%                  & \text{ for every } (K{:}\forall\as @.@ \oln{\tau}{m} -> T\;\as) \in \Sigma \text{ and } m > n \\
%% %%  \textsc{AxPartB} & \formula{\forall \oln{x}{n} @.@ app(f_{ptr},\xs) \neq \unr} \\
%% %%                   & \formula{\quad\land\; app(f_{ptr},\xs) \neq \bad} \\
%% %%                   & \formula{\quad\land\; \forall \oln{y}{k} @.@ app(f_{ptr},\xs) \neq K(\ol{y})} \\
%% %%                   & \text{ for every } (f |-> \Lambda\as @.@ \lambda\oln{x{:}\tau}{m} @.@ u) \in P  \\
%% %%                   & \text{ and every } (K{:}\forall\as @.@ \oln{\tau}{k} -> T\;\as) \in \Sigma \text{ and } m > n  \\ \\
%% %%  \textsc{AxInj}   & \formula{\forall \oln{y}{n} @.@ \sel{K}{i}(K(\ys)) = y_i} \\
%% %%                   & \text{for every } (K{:}\forall\as @.@ \oln{\tau}{n} -> T\;\as) \in \Sigma \text{ and } i \in 1..n \\ \\
%% %% \end{array} \\
%% %% \ruleform{\Th{\Sigma}{P}^{\lcfZ}} \\ \\
%% %% \begin{array}{lll}
%% %%  \textsc{AxCfA}   & \formula{\lcf{\unr} /\ \lncf{\bad}} \\
%% %%  \textsc{AxCfB}   & \formula{\forall \oln{x}{n} @.@ \lcf{K(\ol{x})} <=> \bigwedge\lcf{\ol{x}}} \\
%% %%                   & \text{ for every } (K{:}\forall\as @.@ \oln{\tau}{n} -> T\;\as) \in \Sigma
%% \end{array}\]


%% In the model, $\definable{\cdot}$ should be possible to define, as a
%% predicate on denotations. The disadvantage to this approach is that
%% arrow contracts will only hold for whatever is in your context, not
%% arbitrary expressions, which might be what we want, but might not be
%% modular enough.

%% The other {\em potential} problem (i.e. I have not yet checked) might be in the
%% proof of admissibility of induction.

%% And yet another potential problem is that as we incrementally extend our signature
%% with new function definitions (and possibly contracts) previously defined contracts
%% may no longer hold. This is pretty bad for modularity.

%% \paragraph{Yet another possible solution}

%% A solution that seems somewhat more modular is based on the observation that,
%% during the evaluation of a program there exists a {\em set} of terms (maybe infinite) that can
%% appear as arguments to other terms or functions. Our idea is to guard the arrow contracts so that
%% we do not quantify over any possible term (or denotation, in the translation) but rather only
%% those that may appear as {\em arguments} in some application. We translate arrow contract as
%% follows:
%% \[\begin{array}{l}
%%   \ctrans{\Sigma}{\Gamma}{e \in (x{:}\Ct_1) -> \Ct_2} = \\
%%   \qquad\qquad\quad
%% \formula{\forall x @.@ arg(x) \land \ctrans{\Sigma}{\Gamma,x}{x \in \Ct_1} => \ctrans{\Sigma}{\Gamma,x}{e\;x \in \Ct_2}}
%% \end{array}\]
%% where $arg(x)$ ensures that $x$ is the denotation of a term that will be passed as an argument to $e$. We'd need to define
%% a similar predicate on the evaluation relation, call it $Arg(e)$ and modify the program translation to thread the $arg(\cdot)$
%% predicate through.



\subsection{Contract checking as satisfiability}\label{sect:soundness}
  Having established the soundness of our translation, it is time
we see in this section how we can use this sound translation to verify a program.
The following theorem is then true:

\begin{theorem}[Soundness]\label{thm:prover-soundness}
Assume that $e$ and $\Ct$ contain only function symbols from $P$ and no free term variables.
Let $\Th_{all} = \Th\;\land\;\ptrans{}{P}$.
If $\Th_{all} \land \neg\ctrans{\Sigma}{P}{e \in \Ct}$ is unsatisfiable
then $\langle D_\infty,{\cal I}\rangle \models \ctrans{\Sigma}{P}{e \in \Ct}$ and
consequently $\dbrace{e} \in \dbrace{\Ct}$.
\end{theorem}
\begin{proof}
If there is no model for this formula then its negation must be valid (true in all models), that
is $ \neg \Th_{all} \lor \ctrans{\Sigma}{P}{e \in \Ct}$ is valid. By completeness
of first-order logic $\Th_{all} |- \ctrans{\Sigma}{P}{e \in \Ct}$. This means
that all models of $\Th_{all}$ validate $\ctrans{\Sigma}{P}{f \in \Ct}$. In particular,
for the denotational model we have that $\langle D_\infty,{\cal I}\rangle \models \Th_{all}$
and hence $\langle D_\infty,{\cal I} \rangle \models \ctrans{\Sigma}{P}{e \in \Ct}$.
Theorem~\ref{thm:den-contr-satisfaction} finishes the proof.
\end{proof}

Hence, to verify a program $e$ satisfies a contract $\Ct$ we need to do the following:
\begin{itemize}
  \item Generate formulae for the theory $\Th\;\land\;\ptrans{}{P}$
  \item Generate the negation of a contract translation: $\neg\ctrans{\Sigma}{P}{e \in \Ct}$
  \item Ask a theorem prover solver for a model for the conjunction of the above formulae
\end{itemize}

\paragraph{Incremental verification}

Theorem~\ref{thm:prover-soundness} gives us a way to check that an expression satisfies a
contract. Assume that we are given a program $P$ with a function $f \in dom(P)$, for which
we have already shown that $\langle D_\infty,{\cal I}\rangle \models \ctrans{\Sigma}{P}{f \in \Ct_f}$.
Suppose next that we are presented with a ``next'' goal, to prove that
$\langle D_\infty,{\cal I}\rangle \models \ctrans{\Sigma}{P}{h \in \Ct_h}$.
We may consider the following three variations of how to do this:

\begin{itemize}
  \item Ask for the unsatisfiability of:
    \[  \Th\; \land \; \ptrans{\Sigma}{P} \; \land \;\neg\ctrans{\Sigma}{P}{h \in \Ct_h} \]
        The soundness of this query follows from Theorem~\ref{thm:prover-soundness} above.

  \item Ask for the unsatisfiability of:
    \[  \Th \; \land\; \ptrans{\Sigma}{P} \; \land \; \ctrans{\Sigma}{P}{f \in \Ct_f} \land \neg \ctrans{\Sigma}{P}{h \in \Ct_h} \]
        This query adds the {\em already proven} contract for $f$ to the theory. If this formula
        is unsatisfiable, then its negation is valid, and we know that the denotational model is
        a model of the theory {\em and} of $\ctrans{\Sigma}{P}{f \in \Ct_f}$ and hence it must also
        be a model of $\ctrans{\Sigma}{P}{h \in \Ct_h}$.
  \item Ask for the unsatisfiability of:
    \[  \Th \; \land \;
        \ptrans{\Sigma}{P \setminus f} \; \land \;
        \ctrans{\Sigma}{P}{f \in \Ct_f} \; \land \;
        \neg\ctrans{\Sigma}{P}{h \in \Ct_h} \]
        This query removes the axioms associated with the \emph{definition} of $f$, leaving
        only its \emph{contract} available.  This makes the proof of $h$'s contract
        insensitive to changes in $f$'s implementation.
        Via a similar reasoning as before, such an invocation
        is sound as well.
\end{itemize}


%% Our final goal is going to show that a program does not crash, that
%% is the final contract will be of the form $e \in \Ct$ where $\Ct$ is
%% going to be some {\em base contract}. Note that by base contract adequacy
%% (Lemma~\ref{lem:base-contract-adequacy}) if we manage to show a base contract
%% denotationally, then the contract holds in operational terms.

%% \clearpage


\section{Induction}\label{sect:induction}
  
An important practical extension is the ability to prove contracts about recursive functions
using induction. For instance, we might want to prove that @length@ satisfies $\CF -> \CF$.
\begin{code}
  length []     = Z
  length (x:xs) = S (length xs)
\end{code}
In the second case we need to show that the result of @length xs@ is crash-free but we do not
have this information so the proof gets stuck, often resulting in the FOL-solver looping.

A naive approach would be to perform induction over the list argument of @length@ -- however
in Haskell datatypes may be lazy infinite streams and ordinary induction is not necessarily
a valid proof principle. Fortunately, we can still appeal to {\em fixpoint induction}. The
fixpoint induction sheme that we use for @length@ above would be to {\em assume} that the
contract holds for the occurence some function @length_rec@ inside the body of its definition,
and then try to prove it for the function:
\begin{code}
  length []     = Z
  length (x:xs) = S (length_rec xs)
\end{code}

Formally, our induction scheme is:
\begin{definition}[Induction sheme]\label{def:induction}
To prove that $\dbrace{g} \in \dbrace{\Ct}$ for a function
$g\;\as\;\ol{x{:}\tau} = e[g]$ (meaning $e$ contains
some occurrences of $g$), we perform the following steps:
\begin{itemize*}
  \item Generate function symbols $g^{\circ}$, $g^{\bullet}$
  \item Generate the theory formula \[ \phi = \Th \land
             \ptrans{}{P \cup g^{\bullet}\;\as\;\ol{x{:}\tau} = e[g^{\circ}]} \]
  \item Prove that the query $\phi \land \ctrans{}{}{g^{\circ} \in \Ct} \land \neg \ctrans{}{}{g^{\bullet} \in \Ct}$
        is unsatisfiable.
\end{itemize*}
%% If it is unsatisfiable then $\dbrace{f} \in \dbrace{\Ct}$.
\end{definition}

Why is this approach sound? The crucial step here is the fact that contracts are admissible predicates.
\begin{theorem}[Contract admissibility]
If $d_i \in \dbrace{\Ct}$ for all elements of a chain $d_1 \sqsubseteq d_2 \sqsubseteq \ldots$ then the limit of the chain
$\sqcup d_i \in \dbrace{\Ct}$. Moreover, $\bot \in \dbrace{\Ct}$.
\end{theorem}
\begin{proof} By induction on the contract $\Ct$; for the $\CF$ case we get the result from Lemma~\ref{lem:cf-admissible}.
For the predicate case we get the result from the fact that the denotations of programs
are continuous in $D_{\infty}$. The arrow case follows by induction.
\end{proof}

We can then prove the soundness of our induction scheme.
\begin{theorem} The induction scheme in Definition~\ref{def:induction} is correct. \end{theorem}
\begin{proof} We need to show that:
$\dbrace{P}^{\infty}(g) \in \dbrace{\Ct}$ and hence, by admissibility it is enough to find
a chain whose limit is $\dbrace{P}^{\infty}(g)$ and such that every element is in $\dbrace{\Ct}$.
Let us consider the chain $\dbrace{P}^{k}(g)$ so that $\dbrace{P}^0(g) = \bot$ and
$\dbrace{P}^{k+1}(g) = \dbrace{P}_{(\dbrace{P}^{k})}(g)$ whose limit is $\dbrace{P}^{\infty}(g)$. We
know that $\bot \in \dbrace{\Ct}$ so, by using contract admissiblity, all we need to show is
that if $\dbrace{P}^{k}(g) \in \dbrace{\Ct}$ then $\dbrace{P}^{k+1}(g) \in \dbrace{\Ct}$.

To show this, we can assume a model where the denotational interpretation ${\cal I}$ has been
extended so that ${\cal I}(g^{\circ}) = \dbrace{P}^{k}(g)$ and ${\cal I}(g^{\bullet}) = \dbrace{P}^{k+1}(g)$.
By proving that the formula
\[ \phi \land \ctrans{}{}{g^{\circ} \in \Ct} \land \neg \ctrans{}{}{g^{\bullet} \in \Ct} \]
is unsatisfiable, since $\langle D_{\infty},{\cal I}\rangle \models \phi$ and
$\langle D_{\infty},{\cal I}\rangle \models \ctrans{}{}{g^{\circ} \in \Ct}$, we learn
that $\langle D_{\infty},{\cal I}\rangle \models \ctrans{}{}{g^{\bullet} \in \Ct}$,
and hence $\dbrace{P}^{k+1}(g) \in \dbrace{\Ct}$.
\end{proof}

Note that contract admissibility is {\em absolutely essential} for the
soundness of our induction scheme, and is not a property that holds of
every predicate on denotations. For example, consider the following
Haskell definition:
\begin{code}
  ones = 1 : ones
  f (S x) = 1 : f x
  f Z     = [0]
\end{code}
Let us try to check if the $\forall x @.@ @f@(x) \neq @ones@$ is true in
the denotational model, using fixpoint induction. The case for $\bot$ holds,
and so does the case for the @Z@ constructor. For the $@S@\;x$ case, we can
assume that $@f@(x) \neq @ones@$ and we can easily prove that this implies that
$@f@(@S@\;x) \neq @ones@$. Nevertheless, the property is {\em not true} -- just pick
a counterexample $\dbrace{@s@}$ where @s = S s@. What happened here is that the property
is denotationally true of all the elements of the following chain
\[ \bot \sqsubseteq \injK{S}{\bot} \sqsubseteq \injK{S}{\injK{S}{\bot}} \sqsubseteq \ldots \]
but is false in the limit of this chain. In other words $\neq$ is not admissible and our
induction scheme is plain nonsense for non-admissible predicates.

Finally, we have observed that for many practical cases, a
straightforward generalization of our lemma above for mutually
recursive definitions is required. Indeed, our tool performs mutual
fixpoint induction when a recursive group of functions is
encountered. We leave it as future work to develop more advanced
techniques such as strengthening of induction hypotheses or
identifying more sophisticated induction schemes.




\section{Implementation and practical experience}\label{sect:implementation}
  Our prototype contract checker is called \textbf{Halo}.
It uses GHC to parse, typecheck, and desugar a Haskell program,
translates it into first order logic (exactly as in Section~\ref{ssect:trans-fol}), and
invokes a FOL theorem prover (Equinox, Z3, Vampire, etc) on the FOL formula.
The desugared Haskell program is expressed in GHC's intermediate language
called Core~\cite{Sulzmann:2007:SFT:1190315.1190324}, an explicitly-typed
variant of System F.  It is straightforward
to translate Core into our language $\theLang$.

\subsection{Expressing contracts in Haskell}

How does the user express contracts?  We write contracts in Haskell
itself, using higher-order abstract syntax and a GADT, in
a manner reminiscent
of the work on {\em typed contracts} for functional
programming~\cite{Hinze:2006:TCF:2100071.2100093}:
\begin{code}
data Contract t where
  (:->) :: Contract a
        -> (a -> Contract b)
        -> Contract (a -> b)
  Pred  :: (a -> Bool) -> Contract a
  CF    :: Contract a
  (:&:) :: Contract a -> Contract a -> Contract a
\end{code}
A value of type @Contract t@ is a
a contract for a function of type @t@.
The connectives are @:->@ for dependent contract function space, @CF@
for crash-freedom, @Pred@ for predication, and
@:&:@ for conjunction.
One advantage of writing contracts as Haskell terms is that
we can use Haskell itself to build new contract combinators.
For example, a useful derived connective is non-dependent function space:
\par {\small
\begin{code}
(-->) :: Contract a -> Contract b -> Contract (a -> b)
c1 --> c2 = c1 :-> (\_ -> c2)
\end{code}
} \par
%As one would expect, @:->@ and @-->@ are right-associve.  We can
%create contract combinators that are always satisfied, and never
%satisfied:
%
%\begin{code}
%any :: Contract a
%any = Pred (\ _ -> True)
%
%never :: Contract a
%never = Pred (error "never!")
%\end{code}

A contract is always associated with a function,
so we pair the two in a @Statement@:
\begin{code}
  data Statement where
      (:::) :: a -> Contract a -> Statement
\end{code}
In our previous mathematical notation we might write the following
contract for @head@:
$$
@head@ \in \CF \;\&\; \{@xs@ \mid @not (null xs)@ \} \rightarrow \CF
$$
Here is how we express the contract as a Haskell definition:
\begin{comment}
head (x:xs) = x
head []     = error "empty list"

not True = False    null [] = True
not False = True    null xs = False

f . g = \x -> f (g x)
\end{comment}
\begin{code}
c_head :: Statement
c_head = head ::: CF :&: Pred (not . null) --> CF
\end{code}
If we put this definition in a file @Head.hs@, together with the supporting
definitions of @head@, @not@, and @null@,
then we can run @halo Head.hs@.
The @halo@ program translates the contract and the supporting function
definitions into
FOL, generates a TPTP file, and invokes a theorem prover.
And indeed @c_head@ is verified by all theorem provers we tried.

For recursive functions @halo@ uses fixpoint induction, as
described in Section~\ref{sect:induction}.

% ------------------------ Omit ----------------------------
\begin{comment}
\subsection{Recursion}

We prove the contract for a recursive function
using fixed point induction (Section~\ref{s:induction}).
For recursive
functions, the tool then gives three TPTP files, one that can be used to try to
prove the contract without induction, one for the base case and step case. The
typical situation is that the one without induction is not provable, because it
lacks the appropriate induction hypothesis. The base case always succeeds
(because $\bot$ satisfies every contract as we have seen in Section~\ref{s:induction})
so it really only serves as a sanity check of the tool.
The induction step case may pass or fail, depending on if the
contract really holds, and if the induction hypothesis is strong
enough, and if we have assumed the right contracts for functions that may be used
in the body of the function that we are working on.

One example of a recursive function in the Prelude is @foldr1@.

\begin{code}
foldr1          :: (a -> a -> a) -> [a] -> a
foldr1 f [x]    =  x
foldr1 f (x:xs) =  f x (foldr1 f xs)
foldr1 _ []     =  error "foldr1: empty list"
\end{code}

We can state that if @foldr1@ is applied to a crash free function, and
a non-empty list, then the result should be crash free as a contract:
\begin{code}
c_foldr = foldr1 ::: (CF --> CF --> CF)
                 --> CF :&: Pred (not . null) --> CF
\end{code}
Our tool proves this contract, but only when recursion is used.
\end{comment}
% ------------------------ End Omit ----------------------------

\subsection{Practical considerations}

To make the theorem prover work as fast as possible we trim the
theories to include only what is needed to prove a
property. Unnecessary function pointers, data types and definitions
for the current goal are not generated.

When proving a series of contracts, it is natural to do so in dependency order.
For example:
\begin{code}
  reverse :: [a] -> [a]
  reverse [] = []
  reverse (x:xs) = reverse xs ++ [x]

  reverse_cf :: Statement
  reverse_cf = reverse ::: CF --> CF
\end{code}
To prove this contract we must first prove that
$@(++)@ \in \CF \rightarrow \CF \rightarrow \CF$;
then we can prove @reverse@'s contract assuming the one for @(++)@.
At the moment, @halo@ asks the programmer to specify which auxiliary contracts
are useful, via a second constructor in the @Statement@ type:
\begin{code}
  reverse_cf = reverse ::: CF --> CF
                       `Using` append_cf
\end{code}


\subsection{Dependent contracts}

@halo@ can prove dependent contracts.
For example:
$$
@filter@ \in (@p@ : \CF \rightarrow \CF) \rightarrow
             \CF \rightarrow \CF \;\&\; \{@ys@ \mid @all p ys@ \}
$$
This contract says that under suitable assumptions of crash-freedom,
the result of @filter@ is both crash-free and satisfies @all p@.
Here @all@ is a standard Haskell function, and @p@ is the functional
argument itself.

In our source-file syntax we use @(:->)@ to bind @p@.
\begin{code}
filter_all :: Statement
filter_all =
  filter ::: (CF --> CF) :-> \p ->
                CF --> (CF :&: Pred (all p))
\end{code}
The contract looks slightly confusing since it uses two ``arrows'', one from @:->@,
and one from the @->@ in the lambda.  This contract is proved by
applying fixed point induction.

\subsection{Higher order functions}

Our tool also deals with (very) higher order functions.
Consider this function @withMany@, taken from the
library @Foreign.Util.Marshal@:
%\footnote{\url{http://hackage.haskell.org/packages/archive/base/latest/doc/html/Foreign-Marshal-Utils.html#v:withMany}}:

\begin{code}
withMany :: (a -> (b -> res) -> res)
         -> [a] -> ([b] -> res) -> res
withMany _       []     f = f []
withMany withFoo (x:xs) f = withFoo x (\x' ->
      withMany withFoo xs (\xs' -> f (x':xs')))
\end{code}

For @withMany@, our tool proves

\[ @withMany@ \in \begin{array}[t]{l} (\CF -> (\CF -> \CF) -> \CF) -> \\
                                      \quad\quad (\CF -> (\CF -> \CF) -> \CF)
                  \end{array}\]

% ------------------------ Omit ----------------------------
\begin{comment}
\subsection{A small case-study about invariants}

We consider a somewhat non-standard way of expressing propositional
logic formulae:

\begin{code}
data Formula = And [Formula]
             | Or  [Formula]
             | Neg (Formula)
             | Implies (Formula) (Formula)
             | Lit Bool
\end{code}

One invariant that we are particularly interested in is that we
should never have two consecutive negations, and that the lists of
@And@ and @Or@ are of length $\ge$ 2. We can express that as an ordinary
Haskell predicate:

\begin{code}
invariant :: Formula -> Bool
invariant f = case f of
  And xs      -> properList xs && all invariant xs
  Or xs       -> properList xs && all invariant xs
  Neg Neg{}   -> False
  Neg x       -> invariant x
  Implies x y -> invariant x && invariant y
  Lit x       -> True

properList :: [a] -> Bool
properList []  = False
properList [_] = False
properList _   = True
\end{code}

Now, we have a recursive function that negates formula:

\begin{code}
neg :: Formula -> Formula
neg (Neg f)         = f
neg (And fs)        = Or (map neg fs)
neg (Or fs)         = And (map neg fs)
neg (Implies f1 f2) = neg f2 `Implies` neg f1
neg (Lit b)         = Lit b
\end{code}

We make a combinator saying what it means to retain a predicate:

\begin{code}
retain :: (a -> Bool) -> Contract (a -> a)
retain p = Pred p :-> \x -> Pred (\r -> p x && p r)
\end{code}

\dr{TODO: explain this. This was DV's brilliant idea but I still don't
  fully understand it} Now, since @neg@ uses @map@, we need to show that
@map@ can retain the invariant. We use @all@, introduced above, for
this:

\begin{code}
map_invariant = map ::: retain invariant -->
                        retain (all invariant)
\end{code}

Explicitly spelling out the definition of @retain@ in the statement
above would be tedious and error-prone, so we see the benefit of being
able to express contracts as a DSL.

We can now express that @neg@ retains the invariant:

\begin{code}
neg_contr = neg ::: retain invariant
  `Using` map_invariant
\end{code}

We use @Using :: Statement -> Statement -> Statement@, another
constructor for @Statement@, which allows us to assume that other
contracts holds, when proving a complicated statement, thus
our Statement data type really looks like this:

\begin{code}
data Statement where
    (:::) :: a -> Contract a -> Statement
    Using :: Statement -> Statement -> Statement
\end{code}

For now, it's the user's responsibility to prove these assumed
contracts (for instance, with our tool!), but one can imagine a more
sophisticated front-end which does this automatically.  Note that
the assumption in @neg_contr@ is necessary. If we remove it, and
use the min-translation, we a finitely counter satisfiable theory.
\end{comment}
% ------------------------ End of omit ----------------------------


% \subsection{Example: shrink}
%
% Recall that @fromJust@ is the partial function @Maybe a -> a@, and consider
% this code:
%
% \begin{code}
% shrink :: (a -> a -> a) -> [Maybe a] -> a
% shrink op []     = error "Empty list!"
% shrink op [x]    = fromJust x
% shrink op (x:xs) = fromJust x `op` shrink op xs
% \end{code}
%
% Is this contract satisfied for it?
% \begin{code}
%     (CF --> CF --> CF) -->
%     (CF :&: Pred nonEmpty :&: Pred (all isJust)) --> CF
% \end{code}

\subsection{Experimental Results}


\newcommand{\timeout}{-}
\newcommand{\tot}{\multicolumn{2}{c}{\timeout}}
\newcommand{\tol}{\multicolumn{2}{c |}{\timeout}}

\begin{figure}

\begin{center}
\begin{restab}

 Description
 & \multicolumn{2}{c | }{equinox}
 & \multicolumn{2}{c | }{Z3}
 & \multicolumn{2}{c | }{vampire}
 & \multicolumn{2}{c}{E}
 \\

\hline

@ack@ $\CF$               &  \tol   & 0&04  &  0&03  &  \tot \\
@all@ $\CF$               &  \tol   & 0&00  &  3&36  &  0&04 \\
@(++)@ $\CF$              &  \tol   & 0&03  &  3&30  &  0&38 \\
@concatMap@ $\CF$         &  \tol   & 0&03  &  6&60  &  \tot \\
@length@ $\CF$            &  0&87   & 0&00  &  0&80  &  0&01 \\

@(+)@ $\CF$               & 44&33   & 0&00  &  3&32  &  0&10 \\
@(*)@ $\CF$               &  6&44   & 0&03  &  3&36  &  \tot \\
@factorial@ $\CF$         &  6&69   & 0&02  &  4&18  & 31&04 \\
@exp@ $\CF$               &  \tol   & 0&03  &  3&36  &  \tot \\
@(*)@ accum $\CF$         &  \tol   & 0&03  &  3&32  &  \tot \\
@exp@ accum $\CF$         &  \tol   & 0&04  &  4&20  &  0&12 \\
@factorial@ accum $\CF$   &  \tol   & 0&03  &  3&32  &  \tot \\

@reverse@ $\CF$           & 13&40   & 0&03  & 28&77  &  \tot \\

% Nats examples
%@double@ $\CF$           &  0&20   & 0&02  &  1&39  &  0&01 \\
%@even@ $\CF$             & 17&47   & 0&00  &  0&01  &  0&00 \\
%@half@ $\CF$             & 39&03   & 0&00  &  3&90  &  0&08 \\

@(++)@/@any@ morphism     &  \tol   & 0&03  &  \tol  &  \tot \\
@filter@ satisfies @all@  &  \tol   & 0&03  &  \tol  &  \tot \\

%@filter@ @all@ only pred &  \tol   & 0&02  &  \tol  &  \tot \\

%iterTree $\CF$           &  \tol   & 0&00  &  0&02  &  0&01 \\
%lefts $\CF$              &  4&88   & 0&00  &  3&31  &  0&02 \\

@iterate@ $\CF$           &  5&54   & 0&00  &  0&00  &  0&00 \\
@repeat@ $\CF$            &  0&06   & 0&00  &  0&00  &  0&01 \\

@foldr1@                  &  \tol   & 0&01  &  1&04  & 24&78 \\
@head@                    & 18&62   & 0&00  &  0&00  &  0&01 \\
@fromJust@                &  0&05   & 0&00  &  0&00  &  0&00 \\

% risers                  &  \tol   & \tol  &  \tol  &  \tot \\
@risersBy@                &  \tol   & \tol  &  1&53  &  \tot \\
% @risersBy@ (@nonEmpty@) &  \tol   & 8&63  &  1&20  &  \tot \\
% This just looks stupid. Better get some inlining going

@shrink@                  &  \tol   & 0&04  &  \tol  &  \tot \\
@withMany@ $\CF$          &  \tol   & 0&00  &  \tol  &  \tot \\

\end{restab}
\end{center}

\caption{
  Theorem prover running time in seconds on some of the problems in the test suite
  on contracts that hold.
  \label{fig:unsres}
}

\end{figure}



We have run @halo@ on a collection of mostly-small tests,
some of which can be
viewed in Figure~\ref{fig:unsres}. Our full testsuite and tables can be downloaded from
\url{https://github.com/danr/contracts/blob/master/tests/BigTestResults.md}.
\dv{Is this right? How many examples did we evaluate on? 10? 20? 300? is it important?}
The test cases include:
\begin{itemize}
  \item Crash-freedom of standard functions
    (@(++)@, @foldr1@, @iterate@, @concatMap@).

  \item Crash-freedom of functions with more complex recursive patterns
        (Ackermann's function, functions with accumulators).

%  \item A library for predicate logic terms with some smart constructors
%        retaining an invariant,
%
%        % \dr{Everything failed from this without min so I just skip it}

  \item Partial functions given appropriate preconditions
        (@foldr1@, @head@, @fromJust@).

  \item The @risers@ example from Catch~\citep{Mitchell:2008:PBE:1411286.1411293}.

  \item Some non-trivial post-conditions, such as the example above with @filter@ and @all@,
        and also $@any p xs || any p ys@ = @any p (xs ++ ys)@$.
\end{itemize}

We tried four theorem provers, Equinox, Z3, Vampire and E, and gave
them 60 seconds for each problem. For our problems, Z3 seems to be the most
successful theorem prover.



\section{Discussion}\label{sect:discussion}
  \paragraph{Contracts that do not hold}
\label{ssect:countersat}

In practice, programmers will often propose contracts that do not hold.
For example, consider the following definitions:
\begin{code}
  length []     = Z
  length (x:xs) = S (length xs)

  isZero Z = True
  isZero _ = False
\end{code}
Suppose that we would like to check the (false) contract:
   \[ @length@ \in \CF -> \{ x \mid @isZero@\;x\} \]
\emph{A satisfiability-based checker
will simply diverge} trying to construct a counter model for the
negation of the above query; we have confirmed that this is indeed the
behaviour of several tools (Z3, Equinox, Eprover).  Why?  When a
counter-model exists, it will include tables for the function symbols
in the formula. Recall that functions in FOL are total over the domain
of the terms in the model. This means that function tables may be {\em
infinite} if the terms in the model are infinite. Several (very
useful!)  axioms such as the discrimination axioms \textsc{AxDisjC}
may in fact force the models to be infinite.

In our example, the table for @length@ is indeed infinite since @[]@ is
always disjoint from @Cons x xs@ for any @x@ and @xs@. Even if there
is a finitely-representable infinite model, the theorem prover may
search forever in the ``wrong corner'' of the model for a
counterexample.

From a practical point of view this is unfortunate; it is not
acceptable for the checker to loop when the programmer writes an
erroneous contract.  Tantalisingly, there exists a very simple
counterexample, e.g. @[Z]@, and that single small example is all the
programmer needs to see the falsity of the contract.

Addressing this problem is a challenging (but essential)
direction for future work, and we are currently
working on a modification of our theory that admits the denotational model, but
also permits {\em finite models} generated from counterexample traces.
%% These ideas are reminiscent to the techniques that the Nitpick \dv{IS this right?} tool
%% uses for generating finite counterexamples in Isabelle. \dv{Someone please check!}.
If the theory can guarantee the existence of a finite model in case of a counterexample,
a finite model checker such as Paradox~\cite{paradox} will be able find it.

%% It is obviouly unacceptable for the system to go into a loop if
%% the programmer writes a bogus contract, and we have promising
%% preliminary results based on so-called ``minimisation'', and
%% finite counter-model generators such as Paradox \cite{koen}, but we
%% leave this for (absolutely essential) future work.

\paragraph{A tighter correspondence to operational semantics?}

Earlier work gave a declarative specfication of contracts using
\emph{operational semantics} \cite{xu+:contracts}.  In this paper we have
instead used a \emph{denotational semantics} for contracts (Figure~\ref{f:den-sem-contracts}).
It is natural to ask whether or not the two semantics are identical.

From computational adequacy, Theorem~\ref{thm:adequacy} we can easily state
the following theorem:
\begin{corollary} Assume that $e$ and $\Ct$ contain no term variables and
assume that $\ctrans{}{\cdot}{e \in \{x \mid e_p\}} = \formula{\phi}$. It is the case
that $\langle D_\infty,{\cal I}\rangle \models \phi$ if and only iff either
$P \not|- e \Downarrow$ or $P \not|- e_p[e/x] \Downarrow$ or $P |- e_p[e/x] \Downarrow \True$. \end{corollary}
Hence, the operational and denotational semantics of \emph{predicate contracts} coincide.
However, the correspondence is not precise for \emph{dependent function contracts}.
Recall the operational definition of contract satisfaction for
a function contract:
\[\begin{array}{l}
   e \in (x{:}\Ct_1) -> \Ct_2 \text{ iff} \\
   \text{for all } e' \text{ such that } (e' \in \Ct_1) \text{ it is } e\;e' \in \Ct_2[e'/x]
\end{array}\]
The denotational specification (Figure~\ref{f:den-sem-contracts})
says that for all denotations $d'$ such that
$d' \in \dbrace{\Ct_1}$, it is the case that
$\dapp(\dbrace{e},d') \in \dbrace{\Ct_2}_{x |->d'}$.

Alas there are {\em more} denotations than images of terms in $D_{\infty}$,
and that breaks the correspondence. Consider the program:
\begin{code}
  loop = loop

  f :: (Bool -> Bool -> Bool) -> Bool
  f h = if (h True True) && (not (h False False))
        then if (h True loop) && (h loop True)
             then BAD else True
        else True
\end{code}
Also consider now this candidate contract for $f$:
\[ @f@ \in (\CF -> \CF -> \CF) -> \CF \]
Under the \emph{operational} definition of contract satisfaction,
@f@ indeed satisfies the contract.
To reach @BAD@ we have to pass both conditionals.
The first ensures that @h@ evaluates at least one of its
arguments, while the second will diverge if either argument is evaluated.
Hence @BAD@ cannot be reached, and the contract is satisfied.

However, \emph{denotationally} it is possible to have the classic
parallel-or function, $por$, defined as
follows:
\[\begin{array}{lcl}
  por\;\bot\;\bot & = & \bot \\
  por\;\bot\;\injKZ{True} & = & \injKZ{True} \\
  por\;\injKZ{True}\;\bot & = & \injKZ{True} \\
  por\;\injKZ{False}\;\injKZ{False} & = & \injKZ{False}
\end{array}\]
We have to define $por$ in the language of denotational semantics, because
we cannot write it in Haskell --- that is the point!  For convenience,
though, we use pattern matching notation instead of our
language of domain theory combinators.
The rest of the equations (for @BAD@ arguments) are induced by
monotonicity and we may pick whatever boolean value we like when both
arguments are @BAD@.

Now, this is denotationally a $\CF -> \CF -> \CF$ function, but it
will pass \emph{both} conditionals,
yielding @BAD@. Hence $app(@f@,por) = \injBad$, and @f@'s contract does not hold.
So we have a concrete case where an expression may satisfy its
contract operationally but not denotationally, because of the usual
loss of full abstraction: there are more tests than programs in the denotational world.
Due to contra-variance
we expect that the other inclusion will fail too.

This is not a serious problem in practice.
After all the two definitions mostly coincide, and
they precisely coincide in the base case.  At the end of the day, we
are interested in whether $@main@ \in \CF$, and we have proven that if
is crash-free denotationally, it is definitely crash-free in any
operationally-reasonable term.

Finally, is it possible to define an operational model for our FOL theory that interpreted
equality as contextual equivalence? Probably this could be made to work, although we believe
that the formal clutter from syntactic manipulation of terms could be worse than the current
denotational approach.


\paragraph{Polymorphic crash-freedom}

Observe that our axiomatisation of crash-freedom in Figure~\ref{fig:prelude}
includes only axioms for data constructors. In fact, our denotational interpretation
$\Fcf^{\infty}$ allows more axioms, such as:
\[\begin{array}{l}
    \forall x y @.@ \lcf{x} \land \lcf{y} => \lcf{app(x,y)}
\end{array}\]
This axiom is useful if we wish to give directly a $\CF$ contract to a value of
arrow type. For instance, instead of specifying that @map@ satisfies the contract
$(\CF -> \CF) -> \CF -> \CF$ one may want to say that it satisfies the contract
$\CF -> \CF -> \CF$. With the latter contract we need the previous axiom to be
able to apply the function argument of @map@ to a crash-free value and get a
crash-free result.

In some situations, the following axiom might be beneficial as well:
\[\begin{array}{l}
    (\forall \xs @.@ \lcf{f(\xs)}) => \lcf{f_{ptr}}
\end{array}\]
If the result of applying a function to any possible argument is crash-free then
so is the function pointer. This allows us to go in the inverse direction as before,
and pass a function pointer to a function that expects a $\CF$ argument. However notice
that this last axiom introduces a quantified assumption, which might lead to significant
efficiency problem.

Ideally we would like to say that $\dbrace{\CF} = \dbrace{\CF \rightarrow \CF}$,
but that is not quite true.  In particular,
\[\begin{array}{l}
   (\forall x @.@ \lcf{app(y,x)}) => \lcf{y}
\end{array}\]
is {\em not} valid in the denotational model. For instance consider the
value $\injK{K}{\injBad}$ for $y$. The left-hand side is going to always
be true, because the application is ill-typed and will yield $\bot$, but $y$
is not itself crash-free.




\section{Related work}\label{sect:related}
  %%
%% Ilya Sergey suggests this article:
%%     Higher-Order Symbolic Execution via Contracts, by
%%     Sam Tobin-Hochstadt David Van Horn (Northeastern University)
%% About static contract checking using abstract interpretation in Racket


There are very few practical tools for the automatic verification of
{\em lazy and higher-order} functional programs.  Furthermore, our
approach of directly translating the denotational semantics of
programs does not appear to be well-explored in the literature.

Catch~\cite{Mitchell:2008:PBE:1411286.1411293} is one of the very few tools that
address the verification of lazy Haskell, and have been evaluated on real programs.
Using static analysis, Catch can detect pattern match failures, and hence
prove that a program cannot crash. Some annotations that describes the set of
constructors that are expected as arguments to each function may be
necessary for the analysis to succeed.
Our aim in this paper is to achieve similar goals, but
moreover to be in a position to assert functional correctness.

Liquid Types~\cite{Rondon:2008:LT:1375581.1375602} is an influential
approach to call-by-value functional program verification.
Contracts are written as refinements in a fixed language of predicates (which may
include recursive predicates) and the extracted conditions are discharged using an
SMT-solver. Because the language of predicates is fixed, predicate abstraction can
very effectively {\em infer} precise refinements, even for recursive functions, and
hence the annotation burden is very low. In our case, since the language of predicates
is, {\em by design}, the very same programming language with the same semantics, inference
of function specifications is harder. The other important difference is that liquid types
requires all {\em uses} of a function to satisfy its precondition, whereas in the semantics
that we have chosen, bad uses are allowed but the programmer gets no guarantees back.
\dv{Todo: Andrey Rybalchenko ``sausage factory''}

Rather different to Liquid Types, Dminor~\cite{Bierman+:subtyping} allows
refinements to be written in the very same programming language that programs are written.
Contrary to our case however, in Dminor
the expressions that refine types must be pure --- that is, terminating --- and have a unique
denotation (e.g. not depending on the store)\dr{what is the store?}.
Driven from a typing relation that includes
logic entailment judgements, verification conditions are extracted and discharged automatically using Z3.
Similar in spirit, other dependent type systems such
as Fstar~\cite{fstar} also extract verification conditions that are discharged
using automated tools
or interactive theorem provers. Hybrid type systems such as Sage~\cite{Knowles+:sage}
attempt to prove as many of the goals statically, and defer the rest as runtime
goals.

Boogie~\cite{boogie} is a verification back end that supports procedures as well as
pure functions.  By using Z3, Boogie verifies
programs written in the BoogiePL intermediate language,
which could potentially be used as the
back end of our translation as well.
Recent work on performing induction on top of an
induction-free SMT solver proposes a ``tactic''
for encoding induction schemes as first-order queries, which is reminiscent of the way
that we perform induction \cite{Leino:2012:AIS:2189257.2189278}.

The recent work on the Leon system~\cite{Suter:2011:SMR:2041552.2041575} presents
an approach to the verification of {\em first-order} and {\em call-by-value}
recursive functional programs, which appears to be very efficient in practice.  It works
by extending SMT with recursive programs and ``control literals'' that guide the pattern
matching search for a counter-model, and is guaranteed to find a model if one exists
(whereas that is not yet the case in our system, as we discussed earlier). It
treats does not include
a $\CF$-analogous predicate, and no special treatment of the $\bot$
value nor pattern match failures seem to be in the scope of that project.
However, it gives a very fast verification framework for partial functional correctness.

The tool Zeno~\cite{zeno} verifies equational properties of functional
programs using Haskell as a front end. Its proof search is based on
induction, equality reasoning and operational semantics. While
guaranteeing termination, it can also start new induction proofs
driven by syntactic heuristics. However, it only considers the finite
and total subset of values, and we want to reason about Haskell
programs as they appear in the wild: possibly non-terminating, with
lazy infinite values, and run time crashes.

% Another tool that proves equational
% properties of Haskell programs under the same assumptions is
% HipSpec~\cite{hipspec} but
% But if we mention HipSpec, we should mention Hip. Then what about Hip?

First-order logic has been used as a target for higher-order languages
in other verification contexts as well.  Users of the interactive
theorem prover Isabelle have for many years had the opportunity to use
automated first-order provers to discharge proof obligations. This
work has recently culminated in the tool Sledgehammer
\cite{Sledgehammer}, which not only uses first-order provers, but also
SMT solvers as back ends.  There has also been a version of the
dependently typed programming language Agda in which proof obligations
could be sent to an automatic first-order prover \cite{AgdaFOL}. Both
of these use a translation from a typed higher-order language of
well-founded definitions to first-order logic. The work in this area
that perhaps comes closest to ours, in that they deal with a lazy,
general recursive language with partial functions, is by
\citet{TypeTheoryFOL}, who use Agda as a logical framework to reason
about general recursive functional programs, and combine interaction
in Agda with automated proofs in first-order logic.

The previous work on static contract checking for Haskell~\cite{xu+:contracts}
was based on {\em wrapping}. A term was effectively wrapped
with an appropriately nested contract test, and symbolic execution or
aggressive inlining was used to show that @BAD@ values could
never be reached in this wrapped term.
In follow-up work, Xu~\cite{Xu:2012:HCC:2103746.2103767} proposes a variation, this time for a
{\em call-by-value} language, which performs symbolic execution along with
a ``logicization'' of the program that can be used (via a theorem prover)
to eliminate paths that can
provably not generate @BAD@ value,. The ``logicization'' of a
program has a similar spirit to our translation to logic,a but it is
not clear which model is intended to prove the soundness of this translation
and justify its axiomatisation.
Furthermore, the logicization of programs is dependent on whether
the resulting formula is going to be used as a goal or assumption in a proof. We believe
that the direct approach proposed in this paper, which is to directly encode the semantics
of programs and contracts, might be simpler. That said, symbolic execution as proposed
in~\cite{Xu:2012:HCC:2103746.2103767} has the significant advantage of querying a
theorem prover on many small goals as symbolic execution proceeds, instead of a
single verification goal in the end. We have some ideas about how to break large
contract negation queries to smaller ones, guided by the symbolic evaluation of
a function, and we plan to integrate this methodology in our tool.

%% \begin{itemize}
%%   \item Contracts in general (Findler Felleisen etc)
%%   \item Xu's 2009
%%   \item Xu's PEPM 2012: Very related
%%   \item Minimization/finite models? Isabelle? (Jasmin's thesis?)
%%   \item Yann Regis-Giannas
%%   \item Xeno (equalities), Hipspec
%%   \item Higher-order model checking
%%   \item Triggers
%%   \item Our approach is reminiscent of appraches from the 80's/90's but which?
%%   \item Treatment of @BAD@ as in Extensible Extensions paper (maybe just a comment is neededed inline)
%%   \item More stuff that Koen knows about??????????
%% \end{itemize}


\section{Future work}\label{sect:future}
  % Integers
% SMT 2.0
% Printing countermodels
% (Typeclasses)

There are several avenues for future work.  In terms of our
implementation, we would like to add support for primitive data types,
such as @Integer@, using theorem provers such as @T-SPASS@ to deal
with the @tff@ (typed first-order arithmetic) extension TPTP. Another
way is to also generate theories in the SMT 2.0 format, understood by
Z3, which has support for integer arithmetic and more. As mentioned in
Section~\ref{ssect:countersat} we have ideas how to generate finite
counter examples for contracts that do not hold, and how they should
be presented to the user. It would also be interesting to see if
\emph{triggers} in SMT 2.0 could also be used to support that goal.
Another important direction is finding ways to split our big
verification goals into smaller ones that can be proven significantly
faster. Finally, we would like to investigate whether we can
automatically strengthen contracts to be used as induction hypotheses
in inductive proofs, deriving information from failed attempts.

%A way of presenting countermodels given by @paradox@ in an easily
%understandable way for the user would be helpful.
% quote Reasoning with Triggers?


\paragraph{Acknowledgements}
Thanks to Richard Eisenberg for helpful feedback and Nathan Collins
for early work on a prototype of our translation.

\bibliographystyle{plainnat}
\bibliography{references}

\appendix

\section{Minimization}
   \newcommand{\ThMin}{\Th^{min}}
\newcommand{\utransmin}[3]{\utrans{#1}{#2}{#3}^{min}}
\newcommand{\dtransmin}[3]{\dtrans{#1}{#2}{#3}^{min}}
\newcommand{\ctransmin}[3]{\ctrans{#1}{#2}{#3}^{min}}
\newcommand{\calI}{{\cal J}}
\newcommand{\SDownarrow}{\downarrow}

So far we have described the basic translation of the denotational semantics, programs, and contracts
to first-order logic. However, to enable verification of contracts in practice we must consider two
important extensions, outlined in the rest of this section.

\subsection{Heuristics for countermodel minimization}\label{sect:minimization}

For a query of the form $\Th_{\infty} \land \ptrans{}{P} \land \neg \ctrans{}{P}{e \in \Ct}$, a theorem prover will search
for a model. When such a model exists, it will include tables for the function symbols in the formula. Notice that functions
in FOL are total over the domain of the terms in the model. This means that function tables may be {\em infinite} if the
terms in the model are infinite. Several (very useful!) axioms such as the discrimination axioms \textsc{AxDisjC} may in
fact force the models to be infinite. For instance consider the following devinitions:
\begin{code}
length [] = Z
length (x:xs) = S (length xs)

isZero Z = True
isZero _ = False
\end{code}
Suppose that we would like to check that
   \[ @length@ \in \CF -> \{ x \mid @isZero@\;x\} \]
which is a falsifiable contract.  A satisfiability-based checker
will simply diverge trying to construct a counter model for the negation of the above query; we
have confirmed that this is indeed the behaviour of several tools (Z3, Equinox, Eprover).
Indeed the table for @length@ is infinite since @[]@ is always disjoint from @Cons x xs@ for
any @x@ and @xs@. Even if there is a finitely representable infinite model there is always the
possibility of the theorem prover searching in the ``wrong corner'' of the model for a
counterexample with no success.

From a practical point of view this is {\em not acceptable}: After all, there exists a very simple
counterexample that demonstrates the problem, e.g. @[Z]@, and we only need the
functions of our program to be defined on a {\em finite} number of values (those that appear
during the evaluation of this problematic counterexample) to be able to demonstrate
the problem. We simply {\em do not care} about values that a function may take outside
the set of expressions that appear during the finite evaluation of a counterexample.

This is a challenge that we solve by modifying our axiomatization of the semantics
and the translation of programs and contracts in a way that it can still admit
the ``proper'' $\langle D_\infty,{\cal I}\rangle$ model, but can also admit
some {\em finite} model in the case that a counterexample exists.

To achieve this effect, we introduce a predicate $min(\cdot)$ that, intuitively, is true
for the terms that have been evaluated during the execution of a counterexample. We use
the name $min$ because the purpose of this predicate is to minimize countermodels. We often
refer to the terms in the $min(\cdot)$ predicate as the {\em set of terms we are interested in}.

\begin{figure}
{\small
\[\setlength{\arraycolsep}{1pt}
\begin{array}{c}
\ruleform{\ThMin} \\ \\
\begin{array}{lll}
 \textsc{AxAppBad}  & \formula{\forall x @.@ app(\bad,x) = \bad} \\
 \textsc{AxAppUnr}  & \formula{\forall x @.@ app(\unr,x) = \unr} \\
 \textsc{AxDisjBU} & \formula{\bad \neq \unr} \\
 \textsc{AxAppMin}& \formula{\highlight{\forall x, min(app(x,y)) => min(x)}} \\ \\

 \textsc{AxDisjC} & \formula{\forall \oln{x}{n}\oln{y}{m} @.@} \\
                  & \formula{\;\;\highlight{min(K(\ol{x}))\;\lor\;min(J(\ol{y}))} =>
                                  K(\ol{x}){\neq}J(\ol{y})} \\
                  & \text{ for every } (K{:}\forall\as @.@ \oln{\tau}{n} -> T\;\as) \in \Sigma \\
                  & \text{ and } (J{:}\forall\as @.@ \oln{\tau}{m} -> S\;\as) \in \Sigma \\
 \textsc{AxDisjCU} & \formula{\forall \oln{x}{n} @.@ \highlight{min(K(\ol{x}))} => K(\ol{x}) \neq unr} \\
 \textsc{AxDisjCB} & \formula{\forall \oln{x}{n} @.@ K(\ol{x}) \neq \bad} \\ \\
 %% \textsc{AxPtr}  & \formula{\forall \oln{x}{n} @.@ \highlight{min(app(f_{ptr},\xs))} => f(\ol{x}) = app(f_{ptr},\xs)} \\
 %%                 & \text{ for every } (f |-> \Lambda\as @.@ \lambda\oln{x{:}\tau}{n} @.@ u) \in P \\
 \textsc{AxInj}   & \formula{\forall \oln{y}{n} @.@ \highlight{min(\sel{K}{i}(K(\ys)))}} \\ %% \;\land\; min(y_i)}} \\
                  & \formula{\quad\qquad\qquad => \sel{K}{i}(K(\ys)) = y_i} \\
                  & \text{for every } (K{:}\forall\as @.@ \oln{\tau}{n} -> T\;\as) \in \Sigma \text{ and } i \in 1..n  \\ \\ \\
 \textsc{AxCfBU}  & \formula{\lcf{\unr} /\ \lncf{\bad}} \\
 \textsc{AxCfMin} & \formula{\highlight{\forall x @.@ \lcf{x} => min(x) \lor x = unr}} \\
 %% \textsc{AxCfB1}   & \formula{\forall \oln{x}{n} @.@ \bigwedge_i (\lcf{x_i}\lor \neg(min(x_i))} => \lcf{K(\ol{x})} \lor \neg(min(K(\ol{x}))) \\
 %%                   & \text{ for every } (K{:}\forall\as @.@ \oln{\tau}{n} -> T\;\as) \in \Sigma \\
 \textsc{AxCfC1} & \formula{\forall \oln{x}{n} @.@ \bigwedge\lcf{\ol{x}}} => \lcf{K(\ol{x})} \\
                 & \text{ for every } (K{:}\forall\as @.@ \oln{\tau}{n} -> T\;\as) \in \Sigma \\
 \textsc{AxCfC2} & \formula{\highlight{min(K(\oln{x}{n}))\land\neg\lcf{K(\oln{x}{n})}}} \\
                 & \formula{\quad\qquad\qquad \highlight{ => \bigvee_i (min(x_i)\land\neg\lcf{x_i})}}
\end{array}
%% \ruleform{\Th_\lcfZ^{min}} \\ \\
%% \begin{array}{lll}
%%  \textsc{AxCfBU}  & \formula{\lcf{\unr} /\ \lncf{\bad}} \\
%%  \textsc{AxCfMin} & \formula{\highlight{\forall x @.@ \lcf{x} => min(x) \lor x = unr}} \\
%%  %% \textsc{AxCfB1}   & \formula{\forall \oln{x}{n} @.@ \bigwedge_i (\lcf{x_i}\lor \neg(min(x_i))} => \lcf{K(\ol{x})} \lor \neg(min(K(\ol{x}))) \\
%%  %%                   & \text{ for every } (K{:}\forall\as @.@ \oln{\tau}{n} -> T\;\as) \in \Sigma \\
%%  \textsc{AxCfC1} & \formula{\forall \oln{x}{n} @.@ \bigwedge\lcf{\ol{x}}} => \lcf{K(\ol{x})} \\
%%                  & \text{ for every } (K{:}\forall\as @.@ \oln{\tau}{n} -> T\;\as) \in \Sigma \\
%%  \textsc{AxCfC2} & \formula{\highlight{min(K(\oln{x}{n}))\land\neg\lcf{K(\oln{x}{n})}}} \\
%%                  & \formula{\quad\qquad\qquad \highlight{ => \bigvee_i (min(x_i)\land\neg\lcf{x_i})}}
%% \end{array}
\end{array}\]}
\caption{A theory with minimization}\label{fig:min-theory}
\end{figure}

\kc{I don't understand why AxInj has min(sel(K(x)) as a condition and not min(K(x)). I seem to recall that both work in practice, but I don't understand why. Where does this turn up in the proof? With the condition as it is now, it is possible to get e.g. min(Just False), min(Just True), Just False = Just True, and min(False), min(True), but obviously not False = True.}

Figure~\ref{fig:min-theory} presents a variation of $\Th$, called $\ThMin$, which includes minimization.
In this figure, we have highlighted the parts where our theory differs compared to $\Th_\infty$ from
Figure~\ref{fig:prelude}. The first interesting axiom is \rulename{AxAppMin} which asserts that, if an
appliction $app(x,y)$ has been evaluated during the execution of a counterexample, then so has $x$. This
is axiom is simple reflecting the evaluation order in an operational semantics. The most crucial axioms for
ensuring that we can still get finite models are given by \rulename{AxDisjC}, which asserts that two constructor
values are disjoint {\em only} if we are interested in one of the two. Hence, if evaluation of a counterexample
has never ``touched'' a particular constructor value, that value will not be in our interesting set, and a model
that we get back from a theorem prover can conflate this value to other values or even the interpetation of $\unr$.
Similarly \rulename{AxDisjCBU} asserts that a constructor value is not $\unr$ nor $\bad$ if it is an interesting value.
Finally, the selector axiom group \rulename{AxInj} has been modified to guard the selector value as one would expect.


\begin{figure}\small
\setlength{\arraycolsep}{2pt}
\[\begin{array}{c}
\ruleform{\utrans{\Sigma}{u}{s} = \formula{\phi}} \\ \\
\begin{array}{rcl}
\utrans{\Sigma}{e}{s}
  & = & \formula{(s = \etrans{\Sigma}{\Gamma}{e})} \\
\multicolumn{3}{l}{\utrans{\Sigma}
    {@case@\;e\;@of@\;\ol{K\;\ol{y} -> e'}}{s}} \\
\multicolumn{3}{l}{
\quad
  \begin{array}[t]{rl}
    =     & \formula{\highlight{min(t)}} \\
    \land & \formula{(t = \bad => s = bad)} \\
    \land & \formula{(\forall \ol{y} @.@ t = K_1(\ol{y}) => s = \etrans{\Sigma}{\Gamma}{e'_1})\;\land \ldots}  \\
    \land & \formula{((t{\neq}\bad\;\land\;
                 t{\neq}K_1(\oln{{\sel{K_1}{i}}(t)}{})\;\land\;\ldots) => s{=}\unr))} \\
    \mbox{where} & t  =  \etrans{\Sigma}{\Gamma}{e}
 \end{array}
}
\end{array}
\\ \\
\ruleform{\dtrans{\Sigma}{d} = \formula{\phi}} \\ \\
\begin{array}{rcl}
  \dtrans{\Sigma}{f \;\ol{a}\;\ol{(x{:}\tau)} = u}
    & =     & \formula{\forall \ol{x} @.@ \highlight{min(f(\ol{x})} => \utrans{\sigma}{u}{f(\ol{x})}} \\
    & \land & \formula{\forall \ol{x} @.@ \highlight{min(f(\ol{x})} => f(\ol{x}) = app(f_{ptr},\xs)} \\
\end{array}
\end{array}\]
\caption{Program translation with minimization}\label{fig:min-def-trans-min}
\end{figure}

%% \begin{figure}\small
%% \[\begin{array}{c}
%% \ruleform{\uutrans{\Sigma}{\Gamma}{t \sim u}^{min} = \formula{\phi}} \\ \\
%% \prooftree
%%    \begin{array}{c}
%%    \etrans{\Sigma}{\Gamma}{e} = \formula{t}
%%    \end{array}
%%    ----------------------------------------{DExp}
%%    \begin{array}{l}
%%    \uutrans{\Sigma}{\Gamma}{s \sim e }^{min} = \formula{\highlight{min(s)} => (s = t)}
%%    \end{array}
%%    ~~~~~
%%   \begin{array}{l}
%%   \etrans{\Sigma}{\Gamma}{e} = \formula{t} \\
%%   %% constrs(\Sigma,T) = \ol{K} \\
%%   \text{for each branch}\;(K\;\oln{y}{l} -> e') \text{ it is }
%%   %% \begin{array}{l}
%%   %%          (K{:}\forall \cs @.@ \oln{\sigma}{l} -> T\;\oln{c}{k}) \in \Sigma \text{ and }
%%            \etrans{\Sigma}{\Gamma,\ol{y}}{e'} = \formula{ t_K }
%%   %% \end{array}
%%   \end{array}
%%   ------------------------------------------{DCase}
%%   {\setlength{\arraycolsep}{1pt}
%%   \begin{array}{l}
%%   \uutrans{\Sigma}{\Gamma}{s \sim @case@\;e\;@of@\;\ol{K\;\ol{y} -> e'}}^{min} = \\
%%   \;\;\formula{ \begin{array}{l}
%%      \highlight{min(s)} => \\
%%      \begin{array}{ll}
%%           ( & \highlight{min(t)}\;\land \\
%%             & (t = \bad => s = \bad)\;\land \\
%%             & (\forall \ol{y} @.@ t = K_1(\ol{y}) => s = t_{K_1})\;\land \ldots \land \\
%%             & (t{\neq}\bad\;\land\;t{\neq}K_1(\oln{{\sel{K_1}{i}}(t)}{})\;\land\;\ldots => s{=}\unr) \\
%%           )
%% %% (t = \bad /\ s = \bad)\;\lor\;(s = \unr)\;\lor \\
%% %%                                 \quad      \bigvee(t = K(\oln{{\sel{K}{i}}(t)}{}) \land
%% %%                                            s = t_K[\oln{\sel{K}{i}(t)}{}/\ol{y}])
%%                    \end{array}
%%      \end{array}}
%%   \end{array}}
%%   %% {       \setlength{\arraycolsep}{2pt}
%%   %% \begin{array}{l}
%%   %% \utrans{\Sigma}{\Gamma}{s \sim @case@\;e\;@of@\;\ol{K\;\ol{y}{->}e'}} = \\
%%   %% \;\;\formula{
%%   %%      \begin{array}{l} (\highlight{s{=}\unr})\;\lor \\
%%   %%                           \;\; (\highlight{min(s) => min(t)}\;\land  \\
%%   %%                           \quad((t = \bad /\ s = \bad)\;\lor \\
%%   %%                           \quad\quad \bigvee(t = K(\oln{{\sel{K}{i}}(t)}{}) \land
%%   %%                                          s = t_K[\oln{\sel{K}{i}(t)}{}/\ol{y}])))
%%   %%                  \end{array}
%%   %%          }
%%   %% \end{array}}
%% \endprooftree
%% \end{array}\]
%% \caption{Program translation with minimization}\label{fig:min-def-trans-min}
%% \end{figure}

The translation of programs to accomodate minimization requires only modification
to the $\uutrans{}{\Gamma}{u}$ judgement, which now become $\uutrans{}{\Gamma}{u}^{min}$. Its definition
is given in Figure~\ref{fig:min-def-trans-min}. Rule \rulename{DExp} is unfolding a function
defininion only if the result of the function is in the $min(\cdot)$ set. Rule \rulename{DCase}
has the same flavor. However if we have $min(s)$ then the focus of evaluation in the counterexample
will move on to the scrutinee of the case expression, and hence we get a $min(t)$ predicate, where the
term $t$ is the FOL translation of the case scrutinee $e$.

An easy property of this translation is that it yields a theory that still admits the denotational model
when we interpret $min$ as the everywhere-true predicate:
\begin{theorem} $\langle D_\infty, {\cal I}\uplus min |-> \lambda d.true\rangle \models \ThMin$. \end{theorem}

However, we shall later prove that, unlike $\Th_\infty$, the new theory $\ThMin$ does admit in many practical
cases finite models along with infinite ones (such as $\langle D_{\infty},{\cal I}\rangle$), and these finite
models are directly induced from finite traces of failed programs.



\subsection{Minimization in contracts}

We have showed that, given a strict trace, our new theory admits a finite model. But what is its proving
power? In order to prove a contract we may have to appeal to some axiom from the theory. But, alas, most
axioms in $\ThMin$ are guarded by $min(\cdot)$ predicates, effectively casting them unusable if we are
trying to prove contracts that arise from the translation in Figure~\ref{fig:contracts-minless}.

Our solution to this problem is to modify the translation of contracts as well to use $min(\cdot)$ predicates.
The modified contract translation as well as an axiomatization of $\Th_\lcfZ^{min}$ is in Figure~\ref{fig:min-typing}.
An important deviation compared to our previous contract translation in Figure~\ref{fig:contracts-minless} is the
splitting of the contract translation to a positive variant $\ctransmin{}{\Gamma}{e \in \Ct}$ and a negative
variant $\ctransmin{}{\Gamma}{e \notin \Ct}$. The reason for this is a subtle interaction of the $min(\cdot)$
predicates.

Our goal will be to find a contradiction to $\ctransmin{}{\Gamma}{e \notin \Ct}$, so we start by explaining
the negative variant. Observe that the formula $\ctransmin{}{\Gamma}{e \notin \Ct}$
always give rise to a conjunction of $min$ predicates as the base cases (rule \rulename{NCBase} and \rulename{NCCf})
assert $min$ predicates. Consider this very simple example:
\begin{code}
g x  = True
\end{code}
and we wish to show that $g$ satisfies $\CF -> \CF$, we may assert the existence of a crash-free $x$, such that
$app(g,x)$ is not crash-free. However, recall that the axioms that we have generated for $g$ are:
\[\begin{array}{l}
   \forall x. min(app(g_{ptr},x)) => app(g_{ptr},x) = g(x) \\
   \forall x. min(g(x)) => g(x) = True
\end{array}\]
This means that, if we have no information that $min(app(g_{ptr},x))$ we are not in a position
to assert that the result is $True$ which is crash-free! In fact for any value $y$ that is not
in the $min$ predicate there is a countermodel, e.g. $g(y) = \bad$. Therefore the formula
$\ctransmin{}{\Gamma}{e \notin \Ct}$ inserts enough $min$ predicates in order to ``drive''
the application of function axioms from the assumptions. In our example this manifests as:
\[\begin{array}{l}
   \ctransmin{}{}{g \notin \CF -> \CF} = \\
  \exists x @.@ \ctransmin{}{}{x \in \CF} \land min(app(g_{ptr},x)) \land \\
  \quad\;\; \neg\lcf{app(g_{ptr},x)}
\end{array}\]
which is now unsatisfiable, as required.

The positive variant $\ctransmin{}{\Gamma}{e \in \Ct}$ is first introduced when
trying to derive a contradiction for an arrow contract (rule \rulename{NCTrans}):
\[\begin{array}{l}
   \ctransmin{\Sigma}{\Gamma}{e \notin (x{:}\Ct_1) -> \Ct_2}  = \\
    \exists x @.@ \highlight{\ctransmin{\Sigma}{\Gamma,x}{x \in \Ct_1}} \land \ctransmin{\Sigma}{\Gamma,x}{e\;x \notin \Ct_2}
\end{array}\]
It corresponds to adding an assumption $\ctransmin{\Sigma}{\Gamma,x}{x \in \Ct_1}$ and trying to
derive a contradiction for the conclusion of the arrow contract.

Designing an effective positive variant for practical verification turns out to be the subject of
delicate design choices. Let us consider the base case \rulename{CBase}. One naive approach would
be to simply assign:
\[\begin{array}{l}
     \ctransmin{}{\Gamma}{e \in \{ x \mid e' } = \\
     \quad (t{=}\unr) \lor (t'[t/x]{=}\unr) \lor (t'[t/x]{=}\True)
\end{array}\]
where $t'$ is the FOL translation of $e'$ and $t$ is the FOL translation of $e$.
However, in many practical examples we may have to use the truth value of a predicate to actually learn something
about its argument. Consider the following example:
\begin{code}
f True  = True
f False = error "bad!"
\end{code}
and suppose we would like to prove that @f@ satisfies the contract $\CF \& \{ y \mid (id\;y) \} -> \CF$. Intuitively,
the precondition should only be satisfied by the $\True$ value or $\unr$. However let us examine what happens. We try
to derive a contradiction from:
\[\begin{array}{l}
    \exists x @.@ \ctransmin{}{\Gamma}{x \in \CF} \land \ctransmin{}{\Gamma}{x \in \{ y \mid (id\;y)\}} \land \\
    \phantom{\exists x \;\;} \ctransmin{}{\Gamma}{f\;x \notin \CF}
\end{array}\]
With the aforementioned definition of positive contracts,
a theorem prover will try to come up with some crash free $x$ such that $app(id,x) = \unr \lor app(id,x) = True$
and such that $f\;x$ is not crash-free. However, observe that the value $False$ will satisfy the above requirements!
We do not have available a predicate $min(app(id,x))$ and hence we cannot reduce $app(id,x)$ to $x$ and learn that $x$
can only be $\unr$ or $True$. Hence the contract has become satisfiable whereas this simple function really does belong
in the denotational semantics of its contract!

The solution to this problem is to state that we are actually {\em interested} in the value of the predicate. Formally,
we add a $min(t'[t/x])$ predicate that will now allow us to use information from the value of the predicate $t'[t/x]$
to learn information about $t$:
\[\begin{array}{l}    \ctransmin{}{\Gamma}{e \in \{ x \mid e' } = \\
   \quad \highlight{min(t'[t/x]} \land \\
   \quad \quad ((t{=}\unr) \lor (t'[t/x]{=}\unr) \lor (t'[t/x]{=}\True))
\end{array}\]
Sadly, this choice has an unfortunate consequence: there are now {\em too many} terms that belong in
the $min$-predicate. Consider for instance that the following positive contract has been produced:
\[\begin{array}{l}
    \ctransmin{}{\Gamma}{e \in \Ct_1 -> \{x \mid e'\}} = \\
        \quad \forall x. \ctransmin{}{\Gamma}{x \notin \Ct_1} \lor (min(t'[t/x]) \land \ldots )
\end{array}\]
where again $t$ is the translation of $e$ and $t'$ is the translation of $e'$. This axiom is problematic
from an efficiency point of view as it asserts that {\em for every} x, either it does not belong in $\Ct_1$
or we have $min(t'[t/x])$. This means that way too many values may start appearing in the $min(\cdot)$ tables,
which in turn can cause uncontrollable instantiation of axioms and generation of new terms.
Indeed a few higher-order examples that require induction demonstrate an unacceptable
degradation in performance using the above variation of the rule.

Observe however that we should never really have to learn the value of a predicate $t'[t/x]$ unless we
are actually interested in the value $t$ above. If not, it is quite unlikely that any information about
$t'[t/x]$ will result in some progress in finding a contradiction. Guided by this intuition, we further modify
the rule \rulename{CBase} to introduce a {\em guard} which allows the rule to trigger only if $t$ itself is
in the $min$ predicate:
\[\begin{array}{l}
    \ctransmin{}{\Gamma}{e \in \{ x \mid e' } = \\
    \highlight{min(t) => } \\
    \quad (\highlight{min(t'[t/x])} \land \\
    \quad\quad \land ((t{=}\unr) \lor (t'[t/x]{=}\unr) \lor (t'[t/x]{=}\True)))
\end{array} \]
and this is our final variant of rule \rulename{CBase}.

Similar design choices apply for $\CF$ contracts: in the negative case we assert that we are interested
in a term $t$ (rule \rulename{NCCf}) and try to derive a contradiction to the negation of crash-freedom.
In the positive case we guard crash-freedom by a $min$ assumption. Some delicate choices however have still
been made in the axiomatization of crash-freedom in Figure~\ref{fig:min-typing}. In particular we have now
two substantially different axioms than we had in the $min$-less world:
\[\begin{array}{lll}
 \textsc{AxCfMin} & \formula{\highlight{\forall x @.@ \lcf{x} => min(x) \lor x = unr}} \\
 %% \textsc{AxCfB1}   & \formula{\forall \oln{x}{n} @.@ \bigwedge_i (\lcf{x_i}\lor \neg(min(x_i))} => \lcf{K(\ol{x})} \lor \neg(min(K(\ol{x}))) \\
 %%                   & \text{ for every } (K{:}\forall\as @.@ \oln{\tau}{n} -> T\;\as) \in \Sigma \\
 \textsc{AxCfC2} & \formula{\highlight{min(K(\oln{x}{n}))\land\neg\lcf{K(\oln{x}{n})}}} \\
                 & \formula{\quad\qquad\qquad \highlight{ => \bigvee_i (min(x_i)\land\neg\lcf{x_i})}}
\end{array}\]
\dv{We need one example here as well.}
Axiom \rulename{AxCfMin} asserts that if we have managed to prove that a value is crash-free
then {\em at some point} we have been interested in that value, or that it is $\unr$. Axiom
\rulename{AxCfC2} asserts that if we are interested in a constructor value $K(\ol{x})$ which is
nevertheless not crash-free, we know that there must exist some $x_i$ which we have been intrested
in, and is nevertheless crash-free.

\kc{I don't get why we have AxCfMin at all. Forcing something to be
  min means that it will be evaluated by the theorem prover. Stating
  that something is crash-free should not necessary lead to the whole
  thing being evaluated. Where is this needed in any proof?}
\dr{It was actually your suggestion Koen. From the google docs:
    We could add an axiom
    CF(x) => min(x)
    this is because we never care about CF-ness for non-min things; we
    might just as well state that everything outside min is not CF. This
    simplifies our axioms: ...
    }


\paragraph{Soundness}
These choices are extremely delicate and have been derived very carefully from our intuitions about
evaluation traces as well as practical experience from examples, but there is actually a very large
design space in the placement of $min$. Fortunately for many variations of these rules, and in particular
for the set that we have chosen to present in this paper, soundness with respect to the denotational
semantics is not difficult to see.

The following lemma connects the positive and negative translations to
the original translation $\ctrans{}{\Gamma}{e \in \Ct}$.

\begin{lemma}\label{lem:contract-min} Assume a model $\langle M,\calI\rangle$ such that $\calI(min)(d)$ holds for every $d \in M$.
Then the following are true, assuming that that $dom(\Gamma) \subseteq dom(\calI)$:
\begin{itemize*}
  \item If $\langle M,\calI\rangle \models \neg \ctransmin{}{\Gamma}{e \notin \Ct}$ then $\langle M,\calI\rangle\models \ctrans{}{\Gamma}{e \in \Ct}$
  \item If $\langle M,\calI\rangle \models \ctrans{}{\Gamma}{e \in \Ct}$ then $\langle M,\calI\rangle \models \ctransmin{}{\Gamma}{e \in \Ct}$.
\end{itemize*}
\end{lemma}
%% \begin{proof} We prove the two cases simultaneously by induction on the structure of the contract $\Ct$:
%% \begin{itemize*}
%%   \item For the first part, the cases for rules \rulename{NCBase}, \rulename{NNCf}, \rulename{CConj} are straightforward. The only
%%         interesting case is the \rulename{NCArr}, where we have that
%%         \[\begin{array}{l}
%%              \neg \ctransmin{}{\Gamma}{e \in (x : \Ct_1) -> \Ct_2}  = \\
%%              \quad\quad \forall x @.@ \neg \ctransmin{}{\Gamma,x}{x \in \Ct_1} \lor \neg \ctransmin{}{\Gamma,x}{e\;x \notin \Ct_2}
%%         \end{array}\]
%%         Pick an $d$ and assume $\ctrans{}{\Gamma,x}{x \in \Ct_1}$ holds in the model extened with $x |-> d$.
%%         By induction hypothesis (second case) it must be the case that $\ctransmin{}{\Gamma,x}{x \in \Ct_1}$ in this model, and hence $\neg \ctransmin{}{\Gamma,x}{e\;x \notin \Ct_2}$.
%%         By induction hypothesis then (first case) we get $\ctrans{}{\Gamma,x}{e\;x \in \Ct_2}$ as required.
%%   \item The second part is symmetric by appealing in the \rulename{CArr} case to the induction hypotheses for both sides.
%% \end{itemize*}
%% \end{proof}

The following is an easy observation.
\begin{lemma}\label{lem:min-model} If $\langle M,\calI\rangle$ is a model of $\Th \land \Th_\lcfZ \land \dtrans{}{P}$ then the same model, extended with
$\calI_m(min)(d) = true$ for every $d \in M$, is a model of $\ThMin \land \Th_\lcfZ^{min} \land \dtrans{}{P}^{min}$.
\end{lemma}
However the next theorem is the most important result about minimization.
\begin{theorem}[Soundness of min translation]\label{thm:min-soundness} If $\ThMin\land \Th_{\lcfZ}^{min} \land \dtrans{}{P}^{min} |- \neg \ctransmin{}{}{e \notin \Ct}$ then
                   $\Th \land \Th_{\lcfZ} \land \dtrans{}{P} |- \ctrans{}{}{e \in \Ct}$ and hence $\dbrace{\Ct}(\dbrace{e})$.
\end{theorem}
\begin{proof}
Pick a model $\langle M, \calI\rangle$ of $\Th \land \Th_{\lcfZ} \land \dtrans{}{P}$ and extend its interpretation
so that $\calI(min)(d) = true$. By Lemma~\ref{lem:min-model} this means that the the extended model is a model of
$\ThMin\land \Th_{\lcfZ}^{min} \land \dtrans{}{P}^{min}$ and hence this is also a model of $\neg \ctransmin{}{}{e \notin \Ct}$.
By Lemma~\ref{lem:contract-min} (first case) this is also a model of $\ctrans{}{}{e \in \Ct}$.
\end{proof}
Theorem~\ref{thm:min-soundness} shows that we may simply generate $\ThMin \land \Th_{\lcfZ}^{min} \land \dtrans{}{P}^{min} \land \neg \ctransmin{}{}{e \notin \Ct}$
and ask a theorem prover for a model of this formula. If it is unsatisfiable, then its negation is valid and from the theorem we learn that $\dbrace{\Ct}(\dbrace{e})$.


\paragraph{Completeness}

Is it true that everything that we can prove using
the $min$-less theory, is also provable using the $min$-enabled translation?
That is most likely not the case, although we are not aware of counterexamples. However, as we show
in the Section~\ref{evaluation} we have identified that our heuristics for minimization work much
better than the $min$-less version: For the 90 small programs in our testsuite, the theorem provers
finish on average faster when a contract is provable, and a finite model checker always finds a finite
model in the case where there exists a counterexample in the $min$-enabled translation.


\paragraph{Finite model guarantees}

\begin{figure}\small
\setlength{\arraycolsep}{2pt}
\[\begin{array}{c}
\ruleform{\ctrans{\Sigma}{\Gamma}{e \notin \Ct} = \formula{\phi}} \\ \\
\begin{array}{rcl}
\ctrans{\Sigma}{\Gamma}{e \in \{(x{:}\tau) \mid e' \}}
  & = & \highlight{min(t) \land min(t'[t/x])} \\
  & \land & t \neq \unr \\
  & \land & t'[t/x] \neq \unr \\
  & \land & t'[t/x] \neq True \\
 \mbox{where} &  &
    \begin{array}[t]{lcl}
      t  & = & \etrans{\Sigma}{\Gamma}{e} \\
      t' & = & \etrans{\Sigma}{\Gamma}{e'}
    \end{array}
\\
\ctrans{\Sigma}{\Gamma}{e \notin (x{:}\Ct_1) -> \Ct_2}
  & = & \formula{\exists x @.@ \ctrans{\Sigma}{\Gamma,x}{x \in \Ct_1}
                          \land \ctrans{\Sigma}{\Gamma,x}{e\;x \notin \Ct_2}}
\\
\ctrans{\Sigma}{\Gamma}{e \notin \Ct_1 \& \Ct_2}
   & = & \formula{ \ctrans{\Sigma}{\Gamma}{e \notin \Ct_1} \lor
                   \ctrans{\Sigma}{\Gamma}{e \notin \Ct_2}}
\\
\ctrans{\Sigma}{\Gamma}{e \notin \CF} & = & \formula{\highlight{min(t)} \land
                                               \neg\lcf{\etrans{\Sigma}{\Gamma}{e}}} \\
\mbox{where}  &  & t = \etrans{\Sigma}{\Gamma}{e} \\ \\
\end{array} \\ \\
\ruleform{\ctrans{\Sigma}{\Gamma}{e \in \Ct} = \formula{\phi}} \\ \\
\begin{array}{rcl}
\ctrans{\Sigma}{\Gamma}{e \in \{(x{:}\tau) \mid e' \}}
  & = & \highlight{min(t) => }  \\
  &   & \begin{array}[t]{l} \highlight{(min(t'[t/x])} \land \\
                           \;\;\begin{array}[t]{ll}
                                      & (t{=}\unr\\
                                 \lor & t'[t/x]{=}\unr \\
                                 \lor & t'[t/x]{=}\True))
                                \end{array}
        \end{array} \\
 \mbox{where}  &  &
    \begin{array}[t]{lcl}
      t  & = & \etrans{\Sigma}{\Gamma}{e} \\
      t' & = & \etrans{\Sigma}{\Gamma}{e'}
    \end{array}
\\
\ctrans{\Sigma}{\Gamma}{e \in (x{:}\Ct_1) -> \Ct_2}
  & = & \formula{\forall x @.@ \ctrans{\Sigma}{\Gamma,x}{x \notin \Ct_1}
                          \lor \ctrans{\Sigma}{\Gamma,x}{e\;x \in \Ct_2}}
\\
\ctrans{\Sigma}{\Gamma}{e \in \Ct_1 \& \Ct_2}
   & = & \formula{ \ctrans{\Sigma}{\Gamma}{e \in \Ct_1} /\ \ctrans{\Sigma}{\Gamma}{e \in \Ct_2}}
\\
\ctrans{\Sigma}{\Gamma}{e \in \CF} & = & \formula{min(t) => \lcf{\etrans{\Sigma}{\Gamma}{e}}} \\
\mbox{where}  &  & t = \etrans{\Sigma}{\Gamma}{e}
\end{array}
\end{array}\]
\caption{Translation of contracts with minimization}\label{fig:min-typing}
\end{figure}



%% \ruleform{\ctransmin{\Sigma}{\Gamma}{e \in \Ct} = \formula{\phi}} \\ \\
%% \prooftree
%%   \begin{array}{c}
%%    \etrans{\Sigma}{\Gamma}{e} = \formula{t} \quad
%%    \etrans{\Sigma}{\Gamma,x}{e'} = \formula{t'}
%%   \end{array}
%%   ------------------------------------------{CBase}
%%   \begin{array}{l}
%%    \ctransmin{\Sigma}{\Gamma}{e \in \{(x{:}\tau) \mid e' \}} = \\
%%   %% \Sigma;\Gamma |- e \in \{(x{:}\tau \mid e' \}
%%    \quad \formula{\highlight{min(t) => (min(t'[t/x])\;}\land} \\
%%    \quad \formula{\quad ((t{=}\unr) \lor (t'[t/x]{=}\unr) \lor (t'[t/x]{=}\True)))}
%%   \end{array}
%%   ~~~~~
%%   \begin{array}{c}
%%   \ctransmin{\Sigma}{\Gamma,x}{x \notin \Ct_1} {=} \formula{\phi_1} \quad
%%   \ctransmin{\Sigma}{\Gamma,x}{e\;x \in \Ct_2} {=} \formula{\phi_2}
%%   \end{array}
%%   ------------------------------------------{CArr}
%%   \begin{array}{l}
%%   \ctransmin{\Sigma}{\Gamma}{e \in (x{:}\Ct_1) -> \Ct_2} =
%%   \formula{\forall x @.@ \phi_1 \lor \phi_2}
%%   \end{array}
%%   ~~~~~
%%   \begin{array}{c}
%%   \ctransmin{\Sigma}{\Gamma}{e \in \Ct_1} = \formula{ \phi_1} \quad
%%   \ctransmin{\Sigma}{\Gamma}{e \in \Ct_2} = \formula{ \phi_2}
%%   \end{array}
%%   ------------------------------------------{CConj}
%%   \ctransmin{\Sigma}{\Gamma}{e \in \Ct_1 \& \Ct_2} = \formula{ \phi_1 /\ \phi_2}
%%   ~~~~~
%%   \etrans{\Sigma}{\Gamma}{e} =  \formula{t}
%%   -------------------------------------------{CCf}
%%   \ctransmin{\Sigma}{\Gamma}{e \in \CF} = \formula{\highlight{min(t)} => \lcf{t}}
%%  \endprooftree  \\ \\
%% \ruleform{\ctransmin{\Sigma}{\Gamma}{e \notin \Ct} = \formula{\phi}} \\ \\
%% \prooftree
%%   \begin{array}{c}
%%    \etrans{\Sigma}{\Gamma}{e} = \formula{t} \quad
%%    \etrans{\Sigma}{\Gamma,x}{e'} = \formula{t'}
%%   \end{array}
%%   ------------------------------------------{NCBase}
%%   \begin{array}{l}
%%    \ctransmin{\Sigma}{\Gamma}{e \notin \{(x{:}\tau) \mid e' \}} = \\
%%    \quad\formula{\highlight{min(t)\;\land\;min(t'[t/x])}\;\land } \\
%%    \quad\formula{\;\;((t{\neq}\unr) \land ((t'[t/x]{\neq}\unr) \land (t'[t/x]{\neq}\True)))}
%%   \end{array}
%%   ~~~~~
%%   \begin{array}{c}
%%   \ctransmin{\Sigma}{\Gamma,x}{x \in \Ct_1} {=} \formula{\phi_1} \quad
%%   \ctransmin{\Sigma}{\Gamma,x}{e\;x \notin \Ct_2} {=} \formula{\phi_2}
%%   \end{array}
%%   ------------------------------------------{NCArr}
%%   \begin{array}{l}
%%   \ctransmin{\Sigma}{\Gamma}{e \notin (x{:}\Ct_1) -> \Ct_2} =
%%   \formula{\exists x @.@ \phi_1 \land \phi_2}
%%   \end{array}
%%   ~~~~~
%%   \begin{array}{c}
%%   \ctransmin{\Sigma}{\Gamma}{e \notin \Ct_1} = \formula{ \phi_1} \quad
%%   \ctransmin{\Sigma}{\Gamma}{e \notin \Ct_2} = \formula{ \phi_2}
%%   \end{array}
%%   ------------------------------------------{NCConj}
%%   \ctransmin{\Sigma}{\Gamma}{e \in \Ct_1 \& \Ct_2} = \formula{ \phi_1 \lor \phi_2}
%%   ~~~~~
%%   \etrans{\Sigma}{\Gamma}{e} =  \formula{t}
%%   -------------------------------------------{NCCf}
%%   \ctransmin{\Sigma}{\Gamma}{e \notin \CF} = \formula{\highlight{min(t)} \land \neg\lcf{t}}
%%  \endprooftree
%% \end{array}\]
%% \caption{Translation of contracts with minimization}\label{fig:min-typing}
%% \end{figure}

Can we {\em guarantee} formally in some cases the existence of a finite model of the theory $\Th_{infty}^{min}$?
The answer is yes for many practically important cases. We may define formally a variation on evaluation,
$P |- e \SDownarrow v$ which is stricter than $P |- e \Downarrow$ in evaluating {\em all} subterms of a term.
For instance it includes a rule:
\[\begin{array}{c}\prooftree
P |- \highlight{\ol{e} \SDownarrow \ol{v}}
-------------------------------------{EValC}
P |- K[\taus](\ol{e}) \SDownarrow K[\taus](\ol{e})
\endprooftree\end{array}\]
Notice that if $P |- e \SDownarrow v$ then $P |- e \Downarrow v$ (A precise formulation of $P |- e \SDownarrow v$
can be found in the Appendix). Observe that despite being stricter, the $\SDownarrow$ relation does not cause more
@BAD@ values to be thrown.

%% Our finite model theorem is the following:
%% \begin{theorem}\label{thm:finite-model} If $P |- e \SDownarrow v$ then there exists
%% a finite set $S^{min}$ and an interpretation ${\cal I}^{min}$ such that
%% $\langle S^{min},{\cal I}^{min}\rangle \models \ThMin$. Moreover
%% $\langle S^{min},{\cal I}^{min}\rangle \models \ptrans{}{P}^{min}$.
%% \end{theorem}
We can, for evaluations in $\SDownarrow$, guarantee the existence of finite models.
\begin{theorem}\label{thm:finite-model} If $P |- e \SDownarrow v$ and
$P |- e'[e/x] \SDownarrow @BAD@$ or $P |- e'[e/x] \SDownarrow False$, then
there exists a finite set $S^{min}$ and an interpretation ${\calI}^{min}$ such that
\[ \langle S^{min},{\calI}^{min}\rangle \models \ThMin \land \Th_{\lcfZ}^{min} \land \ptrans{}{P}^{min} \]
and moreover $\langle S^{min},{\calI}^{min}\rangle \models \ctransmin{}{}{e \notin \{ x \mid e' \}}$.
\end{theorem}
Intuitively the guarantee for a finite model is for a counterexample trace that does not involve some
infinite value (because that would not belong in the $\SDownarrow$ evaluation). We believe that this
is possible to generalize to arbitrary traces though this might require further changes in our
$min$-enabled translation. Although the theorem is not the most general theorem that one could desire,
in practice we have observed that we always get finite countermodels from satisfiable queries in
our testsuite.

%% However, we are interested in finite models of the theory $\Th_\infty^{min}$ and in what
%% follows we show how to construct a finite model of $\ThMin$, starting from an execution
%% trace $P |- e \Downarrow v$ that satisfies certain conditions.

%% Consider the graph $(G,E)$ induced by an execution trace
%% $P |- e \Downarrow v$ with $G$ the set
%%     \[ \{ e \mid P |- e \Downarrow v \text{in the trace}\} \cup
%%            \{ @bad@ \} \cup \{ @bot@ \} \]
%% where @bad@ and @bot@ are two distinguished elements (that will serve as the interpretations
%% of $\bad$ and $\unr$ respectively in this model. We also add blue edges $G$ between
%% between the conclusion and the assumptions of any evaluation rule that has been used
%% in the trace. Next we {\em complete} this graph so that for every node of the
%% form $n : (e_1\;e_2)$, if $P |- e_2 \not\Downarrow$, a directed black edge is added
%% from $e_1\;e_2$ to @bot@, else if $P |- e_2 \Downarrow v_2$ we add a black edge to
%% from $n : (e_1\;e_2)$ to a new node $e_2$ and recursively build the graph
%% $P |- e_2 \Downarrow v_2$. Similarly for every node of the form $n : K[\taus](\oln{e})$.
%% This process is infinite but it has an infinite fixpoint by Tarski-Knaster since we are
%% continuously adding nodes and edges.

%% Next we add red undirected edges, along the evaluation blue edges when the semantics
%% agree. We also add red edge for every two application nodes $n : e_1\;e_2$ and
%% $n' : e_1'\;e_2'$ such that $n_1 : e_1$ and $n_1' : e_1'$ are in the reflexive transitive %% closure of red edges and $n_2
%% we recurse on to
%% $\oln{e}$
%% %%  such that for every rule of the form
%% %% $\frac{e_1 \Downarrow

%% Let us revisit the
%% evaluation relation of Figure~\ref{fig:opsem} and let us refine it with the highlighted
%% parts in Figure~\ref{fig:opsem-strict}. The omitted rules are the same as in
%% Figure~\ref{fig:opsem}, with the only difference that they use $\SDownarrow$ instead
%% of $\Downarrow$.

%% \begin{figure}\small
%% \[\begin{array}{c}
%% \ruleform{P |- u \SDownarrow v} \\ \\
%% \prooftree
%% \begin{array}{c} \ \\
%% \end{array}
%% P |- \highlight{\ol{e} \SDownarrow \ol{v}}
%% -------------------------------------{EValF}
%% P |- f^\ar[\taus]\;\oln{e}{m < \ar} \SDownarrow f^\ar[\taus]\;\oln{e}{m < \ar}
%% ~~~~~
%% P |- \highlight{\ol{e} \SDownarrow \ol{v}}
%% -------------------------------------{EValC}
%% P |- K[\taus](\ol{e}) \SDownarrow K[\taus](\ol{e})
%% ~~~~
%% \phantom{G}
%% -------------------------------------{EValB}
%% P |- @BAD@ \SDownarrow @BAD@
%% ~~~~~
%% \begin{array}{c}
%% (f |-> \Lambda\ol{a} @.@ \lambda\oln{x{:}\tau}{m} @.@ u) \in P \\
%% P |- u[\ol{\tau}/\ol{a}][\ol{e}/\ol{x}] \SDownarrow v \quad
%% \highlight{P |- \ol{e} \SDownarrow \ol{v}}
%% \end{array}
%% -------------------------------------{EFun}
%% P |- f[\ol{\tau}]\;\oln{e}{m} \SDownarrow v
%% ~~~~~
%% \begin{array}{c}
%% P |- e_1 \SDownarrow v_1 \\
%% \highlight{P |- e_2 \SDownarrow v_2} \\
%% P |- v_1\;e_2 \SDownarrow w
%% \end{array}
%% ------------------------------------------------{EApp}
%% P |- e_1\;e_2 \SDownarrow w
%% ~~~~
%% \begin{array}{c}  \ \\
%% P |- e_1 \SDownarrow @BAD@ \\
%% \highlight{P |- e_2 \SDownarrow v_2}
%% \end{array}
%% ------------------------------------------------{EBadApp}
%% P |- e_1\;e_2 \SDownarrow @BAD@
%% \endprooftree \\ \\
%% \text{ .. plus rules for @case@ .. }
%% \end{array}\]
%% \caption{Strict operational semantics}\label{fig:opsem-strct}
%% \end{figure}

%% Observe that $\SDownarrow$ is a stricter version of $\Downarrow$, that
%% can potentially diverge more often than $\Downarrow$ but cannot crash more often,
%% as the results of evaluating expressions under constructors or arguments of
%% applications are not used. In fact the following lemma is a straightforward induction.

%% \begin{lemma}
%% If $P |- e \SDownarrow v$ then $P |- e \Downarrow v$.
%% \end{lemma}

%% Our finite model theorem is then the following:

%% \begin{theorem}\label{thm:finite-model} If $P |- e \SDownarrow v$ then there exists
%% a finite set $S^{min}$ and an interpretation ${\cal I}^{min}$ such that
%% $\langle S^{min},{\cal I}^{min}\rangle \models \ThMin$. Moreover
%% $\langle S^{min},{\cal I}^{min}\rangle \models \ptrans{}{P}^{min}$.
%% \end{theorem}
%% \begin{proof} The reader who is interested in the precise
%% construction of the model from the trace can consult the
%% Appendix.
%% \end{proof}


%% \begin{figure}
%% {\small
%% \[\setlength{\arraycolsep}{1pt}
%% \end{array}\]}
%% \caption{Crash-freedom with minimization}\label{fig:min-theory}
%% \end{figure}





\section{Extensions}
  
The initial contracts language is sufficient to express a wide variety
of properties, but it is easy to extend the language to be able to
declare more properties, and sometimes in a more straightforward way.
This section describes the new constructs.

\subsection{Parameterised Contracts and Local Assumptions}
Can this property that describes that @all p@ is a list homomorphism
be written as a contract?

$$\forall \; @p@ \; @xs@ \; @ys@ \; . \;
    @all p xs && all p ys@ = @all p (xs ++ ys)@$$

First of all, we need to ask ourselves which of the functions this
could be a contract for. A promising candidate seems to be @(++)@. So
will this do for a contract?

\[\begin{array}{rcl}
@(++)@ & \in & \{ @xs@ \mid @CF@ \; \& \; @any p xs@ \} \\
       & \to & \{ @ys@ \mid @CF@ \; \& \; @any p ys@ \} \\
       & \to & \{ @rs@ \mid @CF@ \; \& \; @any p rs@ \}
\end{array}\]

The problem here is that @p@ is a free variable, moreover, it is also
important that $@p@ \in @CF@ \to @CF@$. One could introduce a new function
which takes @p@ as an argument but ignores it as this:

\begin{code}
append_dummy p xs ys = xs ++ ys
\end{code}

Now, the contract can be expressed:

\[\begin{array}{rcl}
@append_dummy@ & \in & ( @p@ : @CF@ \to @CF@ ) \\
               & \to & \{ @xs@ \mid @CF@ \; \& \; @any p xs@ \} \\
               & \to & \{ @ys@ \mid @CF@ \; \& \; @any p ys@ \} \\
               & \to & \{ @rs@ \mid @CF@ \; \& \; @any p rs@ \}
\end{array}\]

This approach has several downsides:
\begin{enumerate}
  \item The @append_dummy@ function is not recursive, but this
  contract needs to be proved with fixed point induction on @(++)@,

  \item Annoying acrobatics involved in introducing a new function to
  get a new parameter,

  \item If we use this contract when proving another contract,
  chances are that @append_dummy@ is not going to be interesting, even
  though the property it expresses is. This could be a problem when
  using the min-translation.
\end{enumerate}

We will later argue that it is benefical to prove contracts for not
only one function, but for any expression. This solves the first entry
in the list above, and partially the second because then we could use
a lambda instead of a new top-level declaration.

However, by just extending the language of statements slightly we will
solve the latter two. We will allow quantification and assumptions in
statetements, so the contract can instead be written as this:

\[\begin{array}{rcl}
\forall \; @p@ \; & . & \; (@p@ \in @CF@ \to @CF@) => \\
@(++)@ & \in & \{ @xs@ \mid @CF@ \; \& \; @any p xs@ \} \\
       & \to & \{ @ys@ \mid @CF@ \; \& \; @any p ys@ \} \\
       & \to & \{ @rs@ \mid @CF@ \; \& \; @any p rs@ \}
\end{array}\]

Thus, statements now contain these two extra constructs:

\[\begin{array}{lrll}
  s,t & ::=  & e \in C                     & \text{Contracts} \\
      & \mid & \highlight{s => t}          & \text{Assumption} \\
      & \mid & \highlight{\forall x @.@ s} & \text{Quantification} \\
      & \mid & s \; \textsf{using} \; t    & \text{Reuse}
\end{array}\]

The translation of statements can be viewed in
Figure~\ref{fig:stmt-trans}.  Fixed point induction can now be
expressed quite elegantly: for a statement $s$ susceptible to fixed
point induciton over $f$, we now instead consider the statement
$s[f^\circ/f] => s[f^\bullet/f]$.

\begin{figure}\small
\setlength{\arraycolsep}{2pt}
\[\begin{array}{c}
\ruleform{\trs{v}{s} = \formula{\phi}} \\ \\
\begin{array}{lcl}
  \trs{-}{e \in C}         & = & \trc{e \notin C} \\
  \trs{+}{e \in C}         & = & \trc{e \in C} \\
  \trs{-}{\forall x @.@ s} & = & \exists x @.@ \trs{-}{s} \\
  \trs{+}{\forall x @.@ s} & = & \forall x @.@ \trs{+}{s} \\
  \trs{-}{s => t}          & = & \trs{+}{s} \land \trs{-}{t} \\
  \trs{+}{s => t}          & = & \trs{-}{s} \lor \trs{+}{t} \\
  \trs{-}{s \; \textsf{using} \; t} & = & \trs{-}{s} \land \trs{+}{t} \\
  \trs{+}{s \; \textsf{using} \; t} & = & \trs{+}{s} \\
\end{array}
\end{array}\]
\caption{
    The translation of statements in positive and negative
    position. Right-nested uses of $\textsf{using}$ are assumed to be
    removed.  \label{fig:stmt-trans}
}
\end{figure}

Another property describable with these statements is the one of
associativity as a contract:

\[\begin{array}{rcl}
\forall \; @z@ \; . \; @z@ : @CF@ => @(+)@
    & \in & ( @x@ \in @CF@ ) \to ( @y@ : @CF@ ) \to \\
    &     & \{ @r@ \mid @CF@ \; \& \; @r + z == x + (y + z)@ \}
\end{array}\]

\subsection{Expressions, not Functions}

If we lift the restriction that contracts is always accompanied by a
function to let contracts express properties about general
expressions, we get a richer language which can express this:

\begin{enumerate}
  \item Contracts that for functions that are partially applied:
    $$@map fromJust@ \in \{ @xs@ \mid @all isJust xs@ \} \to @CF@$$

  \item Repeated variables are allowed, allowing reflexivity and
    idempotence:
    \[\begin{array}{l}
    \forall \; @x@ \; . @x@ \in @CF@ => @x == x@ \in \{ @b@ \mid @CF@ \; \& \; @b@ \} \\
    \forall \; @x@ \; . @x@ \in @CF@ => @x && x@ \in \{ @b@ \mid @CF@ \; \& \; @b == x@ \}
    \end{array}\]

  \item Combined with assumptions, we can now express complex
      properties such as symmetry:

    \[\begin{array}{rcl}
    \forall \; @x@ & . & @x@ \in @CF@ => \\
    \forall \; @y@ & . & @y@ \in @CF@ => \\
                   &   & @x == y@ \in \{ @b@ \mid @CF@ \; \& \; @b@ \} => \\
                   &   & @y == x@ \in \{ @b@ \mid @CF@ \; \& \; @b@ \}    \\
    \end{array}\]

\end{enumerate}

Fixed point induction will be applied to the function at the top of
the rightmost (of $=>$s) statement if it is a recursive function. For
the symmetry example, this means that we would do fixed point induction
over @(==)@.

\subsection{First order equality in contracts}
\dr{This is not implemented, it is something to consider}

The earlier example with associativity of @(++)@ was defined in terms
of some equality function @(==)@. However, these are not very
convenient to prove with. We have to show that they constitute an
equivalence relation, but more serious is that we need to show that
they form a congruence over the functions we are interested in. The
equality in first order logic always has this property; it is
substitutive. And indeed, some properties we show hold up to its
equality, such as associativity:

\[\begin{array}{rcl}
\forall \{ @zs@ \} . @(++)@ & \in & \{ @xs@ \} \to \{ @ys@ \} \to \\
                            &     & \{ @rs@ \mid @rs ++ zs@ = @xs ++ (ys ++ zs)@ \}
\end{array}\]

How can we express this in our DSL? We make a new @Eq@ constructor for @Contract@:

$$@Eq :: (a -> Equality a) -> Contract a@$$,

and we make a new data type Equality:

\begin{code}
data Equality a where
    (:=:) :: a -> a -> Contract a
\end{code}

We now write the property as this:

\begin{code}
app_assoc :: [a] -> Statement
app_assoc zs = (++) :::
    Any :-> \xs -> Any :-> \ys -> Eq
        (\rs -> rs ++ zs :=: xs ++ (ys ++ zs))
\end{code}

Using induction, the step case goes through, but the base case is a
bit more problematic. We then have to prove that we have
$\bot @++ zs@ = @xs ++ (ys ++ zs)@$ for all @xs@, @ys@ and @zs@.
Dimitrios comes to the rescue and demands all contracts to hold in the
base case, so we really add a bottom fall-through for equality.

Formally, we extend contracts with a new construct
\[\begin{array}{lrll}
\multicolumn{3}{l}{\text{Contracts}} \\
 \Ct & ::=  & \cdots                 & \text{Previous constructs} \\
     & \mid & \formula{\{ x \mid e_1 = e_2 \}} & \text{Equality}
\end{array}\]

Translated as follows:

\[\begin{array}{l}
\ctrans{\Sigma}{\Gamma}{e \in \{x \mid e_1 = e_2 \}}
  = \; t{=}\unr \; \lor \; t_1[t/x]{=}t_2[t/x] \\
\quad \text{where} \;
   t = \etrans{\Sigma}{\Gamma}{e}, \;
   t_1 = \etrans{\Sigma}{\Gamma}{e_1} \; \text{and} \;
   t_2 = \etrans{\Sigma}{\Gamma}{e_2}
\end{array}\]

\paragraph{Suggested min translation of equality}

\[\begin{array}{l}
\ctrans{\Sigma}{\Gamma}{e \in \{x \mid e_1 = e_2 \}} \\
\quad = \; \formula{( min(t_1) \lor min(t_2) )} \\ %%  \formula{min(t) \; \land \;}
\quad => (\formula{min(t)} \; \land \; t{=}\unr) \; \lor \; t_1[t/x]{=}t_2[t/x] \\ \\
\ctrans{\Sigma}{\Gamma}{e \notin \{x \mid e_1 = e_2 \}} \\
\quad = \; \formula{(min(t) \; \land \; t{=}unr) \lor neq(t_1[t/x],t_2[t/x]))}
\end{array}\]

Where $neq$ is an apartness relation axiomatised in Figure~\ref{fig:neq-axioms}.

\dr{If removing min, we can also remove neq by replace it to $\neq$}

\begin{figure}
{\small
\[\setlength{\arraycolsep}{1pt}
\begin{array}{c}
 \ruleform{neq} \\ \\
\begin{array}{lll}
 \textsc{NeqIrrRefl} & \forall x @.@ \neg neq(x,x) \\
 \textsc{NeqSym}     & \forall x, y @.@ neq(x,y) => neq(y,x) \\
 \textsc{NeqTrans}   & \forall x, y, z @.@ neq(x,y) => (neq(x,z) \lor neq(y,z)) \\
 \textsc{NeqMin}     & \forall x, y @.@ neq(x,y) => (min(x) \land min(y)) \\
 \textsc{NeqDisj}    & \forall \oln{x}{n}\oln{y}{m} @.@ neq(K(\ol{x}),K(\ol{y})) => \bigvee_i neq(x_i,y_i) \\
                     & \text{ for every } (K{:}\forall\as @.@ \oln{\tau}{n} -> T\;\as) \in \Sigma \\
 \textsc{NeqApp}     & \forall f, g, x, y @.@ neq(app(f,x),app(g,y)) => \\
                     & (neq(f,g) \lor neq(x,y))
\end{array}
\end{array}\]}
\caption{An axiomatisation of neq
    \label{fig:neq-axioms}}
\end{figure}

\paragraph{Equality versus partially applied contracts}

With the exception from the extra $\bot$ guard, the expressibility of
equality and partially applied contracts indeed do overlap. The example
with @map fromJust@ above can be now instead be written:

$$@map@ \in \{ @f@ \mid @f@ = @fromJust@ \} \to \{ @xs@ \mid @all isJust xs@ \} \to @CF@$$

It is a bit clumsy, so we accept both versions.

\subsection{SMT 2.0 and triggers}
\dr{I don't know where this section will fit}
\paragraph{Support for primitive Integers}


\section{Printing finite countermodels}
  
By using the min translation, we enable the possibility for finite
models.  When a contract does not hold under some assumed other
contracts or the induction hypothesis, the theory is satisfiable. When
using the min translation this means we have good chances for finding
a finite countermodel. While we do not have a formal proof for this,
or for the limitiations, practical experience shows that the finite
model finder @paradox@ is very efficient on our problem.

A finite model has some finite domain of elements that we can write
$\mathbf{D} = \{\mathbf{1} , \mathbf{2} , \cdots , \mathbf{n}\}$ for a
domain of $n$ elements. We also have an interpretation for every
function and predicate in the theory. Or predicates are @min@ and
@CF@, but we have many uses for FOL functions (and constants), namely,
they correspond to:

\begin{itemize}
    \item Original functions a from the GHC Core
    \item Constructor
    \item Projections
    \item Pointers
    \item Skolem variables from translated contracts
    \item The @app@ table
\end{itemize}

We consider a subtraction example:

\begin{code}
    (-) :: Nat -> Nat -> Nat
    x      - Zero   = x
    Zero   - _      = error "Negative Nat!"
    Succ x - Succ y = x - y
\end{code}

Let's explore the counterexample we get with this contract:
$$@(-)@ \in \{ @CF@ -> @CF@ -> @CF@ \}$$

Two skolem constants, @x@ and @y@ are introduced for the first and
second argument respectively, for @(-)@.  Running @paradox@ on the
translated file with fixed point induction gives us a five element
countermodel. The relevant entries for the constants are:

\[\begin{array}{rcl}
@x@ & = & \mathbf{3} \\
@y@ & = & \mathbf{4}
\end{array}\]

How can we know what these elements $\mathbf{3}$ and $\mathbf{4}$ are?
A first glance at the entry for @min@ shows that $\mathbf{2}$ and
$\mathbf{5}$ are the only non-min values, so we ought to find some
constructor that correspond to $\mathbf{3}$ and $\mathbf{4}$. The
constructor tables for BAD, UNR, Zero and Succ look like this:

\[\begin{array}{lcl}
@BAD@ & = & \mathbf{1} \\
@UNR@ & = & \mathbf{2} \\
\\
@Zero@ & = & \mathbf{3} \\
\\
@Succ@(\mathbf{1}) & = & \mathbf{5} \\
@Succ@(\mathbf{2}) & = & \mathbf{2} \\
@Succ@(\mathbf{3}) & = & \mathbf{4} \\
@Succ@(\mathbf{4}) & = & \mathbf{5} \\
@Succ@(\mathbf{5}) & = & \mathbf{5} \\
\end{array}\]

Interesting! It seems like $@x@ = @Zero@$ and $@y@ = @Succ @\mathbf{3}$.
Futhermore, @UNR@ is $\mathbf{2}$, which was not in the @min@-set, so
we haven't ever started evaluating it (phew!). Notice, too, that @x@
and @y@ are not @BAD@. But let's have a look at the table for
@Succ@. There are three elements, $\mathbf{1}$, $\mathbf{4}$ and
$\mathbf{5}$, that go to $\mathbf{2}$ - can then this element be
@Succ@ of two different things? Naturally not, the missing piece is
the projection table (recall that constructors are not everywhere
injective when we do not have min). Here it is for @Succ@:
\[\begin{array}{lcl}
@Succ@_0(\mathbf{1}) & = & \mathbf{3} \\
@Succ@_0(\mathbf{2}) & = & \mathbf{3} \\
@Succ@_0(\mathbf{3}) & = & \mathbf{2} \\
@Succ@_0(\mathbf{4}) & = & \mathbf{3} \\
@Succ@_0(\mathbf{5}) & = & \mathbf{5} \\
\end{array}\]

Now we need to match up this table with the one for $@Succ@$. We
search for elements $x$ with $@Succ@_0(@Succ@(x)) = x$. Those
elements are the only we can say are really constructed with $@Succ@$.
Let's put this in a little table for our five elements:
\[\begin{array}{ccc}
x          & @Succ@(x)  & @Succ@_0(@Succ@(x)) \\
\mathbf{1} & \mathbf{5} & \mathbf{5} \\
\mathbf{2} & \mathbf{2} & \mathbf{3} \\
\mathbf{3} & \mathbf{4} & \mathbf{3} \\
\mathbf{4} & \mathbf{5} & \mathbf{5} \\
\mathbf{5} & \mathbf{5} & \mathbf{5} \\
\end{array}\]

In the example at hand, we get these two: $\mathbf{5} = @Succ@(\mathbf{5})$
and $\mathbf{3} = @Succ@(\mathbf{4})$. The latter is indeed @y@, and
we have that $@y@ = @Succ@(\mathbf{4}) = @Succ Zero@$, since
$\mathbf{4}$ is @Zero@, as seen above.

And that is our counterexample - the contract breaks by running
@Zero - Succ Zero@. In general, we can find representatives of
skolem variables with concrete types by this algorithm:

\paragraph{Finding representative constructors}
For each domain element $\mathbf{d}$, this is the procedure to see
which constructors it is equal to.  For a constructor @K@ with
arguments $\mathbf{d_1} \, \cdots \, \mathbf{d_n}$, check in the
tables for @K@, @K_1@, \ldots, @K_n@ if

\[\begin{array}{lcl}
@K@(\mathbf{d_1},\cdots,\mathbf{d_n}) & = & \mathbf{d}   \\
@K@_1(\mathbf{d})                    & = & \mathbf{d_1} \\
@K@_n(\mathbf{d})                    & = & \mathbf{d_n} \\
\end{array}\]

Then we know that $\mathbf{d} = @K @\mathbf{d_1} \, \cdots \, \mathbf{d_n}$.
After all such $\mathbf{d}$ has been determined, constructors and
projections functions are not needed any more.

\subsection{Types}

In the discussion with the subtraction example above, we did not
consider the types of elements - and while there were @Bool@s in the
theory, I did not say anything about them. But it turns out that our
element for @x@, namely $\mathbf{3}$, is also equal to @True@!  This
behaviour - indeed a bit weird at a first glance - is actually
expected since never add any axioms saying that constructors of
different types are unequal - such that $@True@ \neq @Zero@$. This is
deliberate - we will never need to tell these two values apart in a
proof.

So, when printing the values of skolem variables we need to consider
their type.  When printing a skolem variable $x$ in a type $\tau$,
we will only consider constructors of the correct type:

\[\begin{array}{rcl}
\mathsf{show}_{\tau}(x)
    & = & \{ @C@ \, (\mathsf{show}_{\tau_1}(x_1)) \, \cdots \, (\mathsf{show}_{\tau_n}(x_n)) \\
    &   & \mid @C@ : \tau_1 -> \cdots -> \tau_n -> \tau_r \\
    &   & , x = @C@ \, x_1 \, \cdots \, x_n \\
    &   & , \exists \sigma . \sigma\tau_r = \tau \} \\
\end{array}\]

The base case is nullary constructors, or to use another skolem
variable of the right type (but we should avoid printing @x@ = @x@).
The $\sigma$ is a substitution and allows us to pick a more specific
constructor, in particular, to allow us to use
$@UNR@ : \forall \alpha . \alpha$ when, say, a @Bool@ is expected.

If we cannot find a good representative for an element $\mathbf{d}$,
it is printed as a metavariable as @?d@. Futhermore, if $\mathbf{d}$
is not in the min set, it is simply written as $\cdots$.


\section{Inlining}
  
The inliner inlines non-recursive, non-casing functions, when inlining
the function would not insert a new lambda.  When dealing with
contracts, it allows inserting a lambda after the @:->@ because it uses
higher-order abstract syntax.

By using the inliner, we can now without hassle use GHC's
optimiser. It is now on by default. It chops up contracts into several
top-level functions, but the inliner then restores the structure by
making it one big definition. Before, the translator would carry
around a context and buggily inline sometimes, but now this is
separated.

When using optimisation, we also get rid of the nasty bug when an expression
is cased on twice, with default branches. Without optimisation, you can get
unsoundness with functions pattern-matching on two arguments.

We have two uses of inlining so far:

\paragraph{Reassembling statements} When using optimisation, the
contracts get splitted up into several entries. For example,
a contract like @map_cf = map ::: (CF --> CF) --> CF --> CF@
might look like this in the optimised GHC core:

\begin{code}
map_cf = (:::) map a3_rke

a3_rke = (-->) a3_rp7 CF

a3_rp7 = (-->) a3_rzt a3_rke

a3_rzt = (-->) CF CF
\end{code}

We could translate these definitions to our internal representation of
contracts, carrying around the right hand sides of expressions, and
this is what we did before. This has a few complications. For one,
it might not look as regular as above: sometimes we get a redefinition
of @(:->)@ to some new variable name and the implementation got really
messy. Further, we want to know what functions are used in a contract,
including inside @Pred@, and this is just one function call if the
statement is reassembled, rather than trying to figure this out while
walking around definitions.

\paragraph{Destroying sharing}
When translating function definitions, we don't want to have
unnecessary function around. An especially annoying pattern is:

\begin{code}
foo x = case x of
    K1 -> lvl_fk4
    K2 y -> foo y

lvl_fk4 = BAD
\end{code}

With endless variantions, where @lvl_fk4@ can get a real-world token,
and so on. Introducing these extra symbols give an extra overhead for
theorem provers.

A function that is neither recursive nor using case is a good
candidate for removal.

We such top-level lifted CAFs for sharing, but we should
try to destroy all sharing since everything is memoized in
theorem provers anyway. So we don't want to see:

\begin{code}
zero = Zero
one = Succ zero
two = Succ one
\end{code}

Such things should always be inlined to not introduce new constants.
For instance, @eprover@ is sometimes confused by such constants.



\section{Problems with DEFAULT}
  
The translation also handles @DEFAULT@ patterns. Here is one example:

\begin{code}
isVar x = case x of
    DEFAULT -> False
    Var     -> True
\end{code}

Understandably, we translate this to
$$\forall x . isVar(var) = true \land (x \neq var => isVar(x) = false)$$

So far so good. What is a bit more curios is if you do a case on the
\emph{same expression} twice, and use default patterns both times.
This happens very often: after desugaring, functions that pattern matches
on two arguments like @zip@ and @drop@, are translated this way, but let's
look at a minimal example, a somewhat weird implementation of boolean or,
and a simple contract for it:

\begin{code}
(||) :: Bool -> Bool -> Ok
True  || y    = True
x     || True = True
False || y    = False

or_cf = (||) ::: CF --> CF --> CF
\end{code}

The or function is desugared like this to GHC Core, and running
the simple expression optimiser:
\begin{code}
lvl_lb7 :: Bool
lvl_lb7 = Control.Exception.Base.patError
    @ GHC.Types.Bool "DefaultBug.hs:\
            \(18,1)-(20,21)|function ||"

|| :: Bool -> Bool -> Bool
|| =
  \ (x :: Bool) (y :: Bool) ->
    case x of {
      __DEFAULT ->
        case y of _ {
          __DEFAULT ->
            case x of _ {
              __DEFAULT -> lvl_lb7;
              False -> False
            };
          True -> True
        };
      True -> True
    }
\end{code}

So how is this translated? Let's look at the branch that goes to
@lvl_lb7@. The first @case@ on @x@ says that it must be different from
@True@, @UNR@ and @Bad@. Ignoring the case on @y@ for now, the second
says that @x@ must be different from @False@ this time, in addition to
@UNR@ and @Bad@. In fact, we end up with something like this:

\[\begin{array}{rl}
\forall @x@ , \; @y@ \; . \;
    & @x@ \neq @True@ \land \\
    & @x@ \neq @False@ \land \\
    & @x@ \neq @UNR@ \land \\
    & @x@ \neq @BAD@ \land \\
    & @y@ \neq @True@ \land \\
    & @y@ \neq @UNR@ \land \\
    & @y@ \neq @BAD@ => ((@x || y@) = @lvl_lb7@ (= @BAD@))
\end{array}\]

Which might at first look ok, what happens if we run @Zero || Zero@?
Then we get that @Zero || Zero = BAD@, becase something ill-typed causes
a pattern match failure that doesn't exist for well typed arguments.
This makes this property satisfiable!

What we want is to make ill-typed arguments go to @UNR@. By using the
optimiser, this is correctly translated. Same goes for more complicated
functions like @zip@.


\section{Fixed point induction and splitting goals}
  
\dr{First a small introduction, then the technique is put it more formally.
    This section also introduces the exact way I translate functions, and
    how contracts are skolemised.}

We can make things a lot easier for the theorem provers by giving them
smaller theories to consider, and sometimes really trimming down the
theories is the only way to make it tractable for them. This is why we
never add unnecessary axioms for data types or declarations. But we
can do better that that. This section explains a way to follow the
pattern matchings in a recursive function to make a separate theory
for every right hand side of a function. To illustrate, we will
use the @filter@ function:

\begin{code}
    filter = \p ys -> case ys of
        (:) x xs -> case p x of
            True -> (:) x (filter p xs)
            False -> filter p xs
        [] -> []
\end{code}

To prove a contract with induction on this function, we instead
consider this:

\begin{code}
    filter' = \p ys -> case ys of
        x:xs -> case p x of
            True -> x:filter_rec p xs
            False -> filter_rec p xs
        [] -> []
\end{code}

As usual, we assert the induction hypothesis for @filter_rec@, and
want to prove the contracts for @filter'@ above. Let's just pick
a simple contract, namely $@filter@ \in (@CF@ -> @CF@) -> @CF@ -> @CF@$.
We could just dump the definition of @filter'@ and the contract
and hope the theorem prover will find its way, but we can generate
a theory for every right hand side in the definition above.

Let's look at the @True@ alternative in the case on @p x@. The
current contstraints we have from casing is that $@ys@ = @x:xs@$,
and $@p x@ = @True@$. We have two case scrutinees behind us, so
we also have a ``min set'' consisting of @ys@ and @p x@.

Let's translate the @filter@ contract, with skolem constants for
$p$ and $ys$:

\[\begin{array}{rl}
      & (min(p) => (\forall x . min(app(p,x)) => @CF@(app(p,x)))) \\
\land & (min(ys) => @CF@(ys)) \\
\land & (min(@filter'@(p,ys)) \land \neg @CF@(@filter'@(p,ys)))
\end{array}\]

We add the constraints and the min set-too:

$$ys = x@:@xs \land app(p,x) = @True@ \land min(ys) \land min(app(p,x)) $$

The benefit now is that we do not need to add the \emph{entire}
definition of @filter@ to the theory, but only this branch:

\[\begin{array}{rl}
\forall \; p \; x \; xs . & min(@filter'@(p,x@:@xs)) \land app(p,x) = @True@ => \\
                          & @filter'@(p,x@:@xs) = x @:@ @filter_rec@(p,xs)
\end{array}\]

To sum up, when we translate a declaration, for each alternative we
record the ``min set'' of expressions cased on, the collected
constraints, and of course the right hand side. When we then translate
a contract, we translate the constraints with the skolemised variables
in the contract, and add $min$ of everything in the ``min-set''.

\subsection{Nested case translation through constraints}

The implementation supports translating nested cases, if they all are
on the top level, so the second @case@ in @filter@ is OK:

\begin{code}
    filter = \p ys -> case ys of
        (:) x xs -> case p x of
            True -> (:) x (filter p xs)
            False -> filter p xs
        [] -> []
\end{code}

While the case here is not:
\begin{code}
    filterSwitch = \b p q xs ->
                 filter (case b of
                             True -> p
                             False -> q) xs
\end{code}

The ``inner'' case will get lifted out to an own definition.

Formally, the rules for translating nested cases is presented in
Figure~\ref{fig:nested-case-trans}. The different arguments to
$\mathcal{U}$ are the quantified variables $\ol{q}$, the constraints
$\ol{c}$, and what function we are defining $s$, plus the current
expression $u$.  Constraints are either equalities from matching a
scrutinee to an expression, or an inequality coming from the @UNR@
``branch''.

One possible optimisation is that when the scrutinee
expression is a variable rather than an expression, it can be
substituted in the quantified variables, the constraints and the given
expression. This is what is implemented, but it can be toggled off
with the flag @--var-scrut-constr@. Another relevant option is
@--case-lift-inner@, which makes all nested cases
depth two or greated can be forced to be lifted to the top level,
making the translation equal to what is written in
Figure~\ref{fig:etrans}.

\newcommand{\ut}[4]{{\cal U}^{#1}_{#2}(#3)\{\!\!\{#4\}\!\!\}}

\begin{figure}\small
\setlength{\arraycolsep}{2pt}
\[\begin{array}{c}
\ruleform{\dtrans{\Sigma}{d} = \formula{\phi}} \\ \\
\begin{array}{rcl}
  \dtrans{\Sigma}{f \;\ol{x}\; = u}
    & = & \ut{\ol{x}}{\epsilon}{f(\ol{x})}{u} \\
\end{array} \\
\mbox{{\footnotesize (pointer axiom omitted)}}
\\ \\
\ruleform{\ut{\ol{q}}{\ol{c}}{s}{u} = \formula{\phi}} \\ \\
\begin{array}{rcl}
\ut{\ol{q}}{\ol{c}}{s}{e}
  & = & \formula{\forall \ol{q} @.@
    (min(s) \land \ol{c} \land s = \etrans{\Sigma}{\Gamma}{e})} \\
\multicolumn{3}{l}{\ut{\ol{q}}{\ol{c}}{s}
    {@case@\;e\;@of@\;\ol{K\;\ol{y} -> e'}}} \\
\multicolumn{3}{l}{
\quad
  \begin{array}[t]{rl}
    = & \formula{\ut{\ol{x} \smallsmile \ol{y}}{(t = K_1(\ol{y})) , \ol{c}}{s}{e'_1}} \land \ldots  \\
    \land & \formula{ \ut{\ol{x}}{(t \neq K_1(\oln{{\sel{K_1}{i}}(t)}{})) , \ldots \smallsmile \ol{c}}{s}{@UNR@} } \\
    \land & \formula{min(s) => min(t)} \\
    \mbox{where} & t  =  \etrans{\Sigma}{\Gamma}{e}
 \end{array}
}
\end{array}
\end{array}\]
\caption{Nested case translation of programs.
For simplicity, every case is assumed to have a $\mathtt{BAD} -> \mathtt{BAD}$ branch.
\label{fig:nested-case-trans}
}
\end{figure}

\subsection{Avoiding nested quantifiers in contract translations}

Given a contract $f \notin (x_1 : C_1) -> \ldots -> (x_n : C_n) -> C$ to translate,
we can can translate it as this:

$$\exists x_1 @.@ \ctrans{}{}{x_1 \in C_1} \land \cdots \land
  \exists x_n @.@ \ctrans{}{}{x_n \in C_n} \land
                  \ctrans{}{}{f \, x_1 \, \ldots \, x_n \notin C}
$$

However, we can pull the quantifiers to the top level:

$$\exists x_1 \; \ldots \; x_n
          @.@ \ctrans{}{}{x_1 \in C_1} \land \cdots \land
 %             \ctrans{}{}{x_n \in C_n} \land
              \ctrans{}{}{e \, x_1 \, \ldots \, x_n \notin C}
$$

Now, because this is on the top level of a translated contract,
$x_1 \ldots x_n$ can be skolemised. Futhermore, this is a conjunction,
so we can put the formulae in different clauses, aiding the theorem provers
(this can be tested with @--no-skolemisation@ and @--no-pull-quants@).

\subsection{Translating split goals}

\newcommand{\ud}[4]{{\dot\mathcal{U}}^{#1}_{#2}(#3)\{\!\!\{#4\}\!\!\}}

Again, we assume that we have a contract
$f \notin (x_1 : C_1) -> \ldots -> (x_n : C_n) -> C$.
Let's change the translation for this function: the idea is to make a
theory for every branch of the cases for a function @f@ we want to
prove a contract for.  The new translation will be called
$\dot\mathcal{U}$.  This function will now return a set of clauses:
let us call the translation of the theory without $f$, and the
contract (using $x_1 \; \ldots \; x_n$ as skolem variables), besides
$f$, for $\mathcal{T}$. For every set of clauses
$A \in \ud{\epsilon}{\epsilon}{f(x_1,\ldots,x_n)}{u}$, we will query a theorem prover if
$\mathcal{T} \cup A$ is unsatisfiable.

Instead of putting matches in constraints as conditionals, we put them
in a positive position for every theory. But we need to keep track of
everything scrutineed upon, so we make a new argument, the
``min~set''. However, the quantified variables will now be skolemised,
so they are not needed any more.

The precise rules for $\dot\mathcal{U}$ are given in
\ref{fig:nested-case-trans}. It also gives a way to recursively
go down in functions that $f$ calls. We can limit the times it is entered
by not revisiting any function twice (in particular not $f$), or limit
the depth or the width of the generated functions.

\dr{As a side note, since $\mathcal{U}$ and $\dot\mathcal{U}$ are so similar,
       they share the same code (in @halo/src/Halo/Binds.hs@ and
       @src/Contracts/@ @Trans.hs@). Even more of a side note, the expression
       translator $\etrans{}{}{e}$ is in @halo/src/Halo/ExprTrans.hs@.
   }

\begin{figure}\small
\setlength{\arraycolsep}{2pt}
\[\begin{array}{c}
\ruleform{\ut{\ol{m}}{\ol{c}}{s}{u} = \formula{\phi}} \\ \\
\begin{array}{rcl}
\ud{\ol{m}}{\ol{c}}{s}{e}
  & = & \formula{\{ \ol{min(m)} \land \ol{c} \land s = \etrans{\Sigma}{\Gamma}{e}) \}}  \\
\multicolumn{3}{l}{\ud{\ol{m}}{\ol{c}}{s}
    {@case@\;e\;@of@\;\ol{K\;\ol{y} -> e'}}} \\
\multicolumn{3}{l}{
\quad
  \begin{array}[t]{rl}
    =    & \formula{\ud{t, \ol{m}}{(t = K_1(\ol{y})) , \ol{c}}{s}{e'_1}} \cup \ldots  \\
    \cup & \formula{\ud{t, \ol{m}}{(t \neq K_1(\oln{{\sel{K_1}{i}}(t)}{}) , \ldots \smallsmile \ol{c}}{s}{@UNR@} } \\
    \mbox{where} & t  =  \etrans{\Sigma}{\Gamma}{e}
 \end{array}
}
\end{array}
\\ \\
\ruleform{\text{Recursive base case:}}
\\ \\
\begin{array}{rcl}
e & = & f \, e_1 \, \ldots e_n \\
f & = & \lambda y_1 \ldots y_n @.@ e' \\
\ud{\ol{m}}{\ol{c}}{s}{e} & = & \ud{\ol{m}}{\ol{c}}{s}{e'[e_1/y_1,\ldots]}
\end{array}
\end{array}\]
\caption{
Translation that creates several theories for a contract for a given function.
The number of times the recursive base case should be taken can limited by either some number or that
it never revisits a function. \dr{and it is not yet implemented}
\label{fig:nested-case-trans}
}
\end{figure}


\section{A finite model construction}\label{sect:finite-model-proof}


We now present the proof of Theorem~\ref{thm:finite-model}.
Consider the alternative presentation of $\SDownarrow$ below.
%% \begin{figure}\small
\[\begin{array}{c}
\ruleform{P |- u \SDownarrow v} \\ \\
\prooftree
\begin{array}{c}
P |- e_1 \SDownarrow f^\ar[\taus]\;\oln{e}{\ar-1} \quad
(f |-> \Lambda\as @.@ \oln{x{:}\tau}{\ar} @.@ u) \in P \\
P |- u[\ol{e},e_2/\xs] \SDownarrow w \quad \highlight{P |- e_2 \SDownarrow v_2}
\end{array}
-------------------------------------{EValF}
P |- e_1\;e_2 \SDownarrow w 
~~~~~
 \ar \geq 1
----------------------------{FVal}
P |- f^\ar[\taus] \SDownarrow f 
~~~~ 
(f{|->}\Lambda\as @.@ u) \in P \quad P |- u{\SDownarrow}v
-------------------------------------{FCaf}
P |- f^0[\taus] \SDownarrow v 
~~~~~
\begin{array}{c}
P |- e_1 \SDownarrow f^\ar[\taus]\;\oln{e}{< \ar-1} \quad
\highlight{P |- e_2 \SDownarrow v_2 }
\end{array}
-------------------------------------{EValP}
P |- e_1\;e_2 \SDownarrow f^\ar[\taus]\;\oln{e}{< \ar-1}\;e_2
~~~~~
\begin{array}{c} 
P |- e_1 \SDownarrow @BAD@ \quad
\highlight{P |- e_2 \SDownarrow v_2}
\end{array}
------------------------------------------------{EBadApp}
P |- e_1\;e_2 \SDownarrow @BAD@
~~~~~ 
P |- \highlight{\ol{e} \SDownarrow \ol{v}}
-------------------------------------{EValC}
P |- K[\taus](\ol{e}) \SDownarrow K[\taus](\ol{e})
~~~~
\phantom{G}
-------------------------------------{EValB}
P |- @BAD@ \SDownarrow @BAD@
\endprooftree \\ \\ 
\text{ ... plus rules for @case@ ... } 
\end{array}\]
%% \caption{Semi-strict operational semantics}\label{fig:opsem-semi}
%% \end{figure}

Let us consider a derivation tree $D_{0} @::@ P e \SDownarrow v$ and let us consider the set:
\[ S_0 = \{ e' \mid \exists D @.@ D @::@ P |- e'{\SDownarrow}w \text{ and } D{\sqsubseteq}D_0 \} \cup \{ @BAD@ \} \]
where with the notation $D_1 \sqsubseteq D_2$ we mean that $D_1$ is a sub-derivation of 
the derivation $D_2$. 

Let us create the set $S^{min}$ as follows:
\[         S^{min} = S_0 \cup \{ \bot \} \] 
where $\bot$ is a distinguished element. We refer to the elements of $S^{min}$ as $\mu$ (either and $e$ or $\bot$).
%% The edges of the graph are created with the following 
%% two rules:
%% \begin{itemize*}
%%    \item For every derivation rule of the form 
%%          \[\begin{array}{c}\prooftree
%%                D_1 @::@ e_1 \SDownarrow v_1 \ldots D_n @::@ e_1 \SDownarrow v_1
%%                ---------------------------------------{}
%%                D @::@ e \SDownarrow v  
%%          \endprooftree\end{array}\] 
%%          in $D_{0}$ then we add edges that connect $D$ to each of $e_1 \ldots e_n$.
%%    \item For every application node $D @::@ e_1\;e_2$ we add a {\em directed edge} from 
%%          the node to the node of $e_2$ (which, by the evaluation relation $\SDownarrow$ must also exist
%%          in the graph and by determinacy of $\SDownarrow$ is unique).
%%    \item For every data constructor node $D @::@ K[\taus](\ol{e})$ we add {\em labelled directed edges}
%%          from the node to the node of each $e_i \in \ol{e}$. The labels on the edges are just the indices.
%% \end{itemize*}

Consider the following equivalence relation on elements of $S^{min}$: 
\[\begin{array}{c}\prooftree
            P |- e \SDownarrow v
      ---------------------------{EqEval}
             e \equiv v 
      ~~~~ 
      \begin{array}{c}
           e_1 \equiv e_1' \quad e_2 \equiv e_2' 
      \end{array}
      ---------------------------------------------------{AppCong}
          e_1\;e_2 \equiv e_1' e_2' 
      ~~~~~ 
           e_i \equiv e_i'
      ---------------------------------------------------{ConCong}
           K[\taus](\ol{e}) \equiv K[\taus'](\ol{e}')
      ~~~~~
         \phantom{e}
      ---------------------------------------------------{Refl}
        \mu \equiv \mu
      ~~~~ 
        e_1 \equiv e_2
      ------------------------{Sym}
        e_2 \equiv e_1
      ~~~~\hspace{-5pt}
        e_1 \equiv e_2 \;\; e_2 \equiv e_3
      \hspace{-2pt}------------------------{Trans}
        e_1 \equiv e_3
\endprooftree\end{array}\]

The following lemma is true:
\begin{lemma}\label{lem:equiv-shapes} If $e_1 \equiv e_2$ then:
\begin{itemize*}
  \item If $P |- e_1 \SDownarrow f^\ar[\taus]\;\oln{e_1}{m}$ with $m < \ar$ then $P |- e_2 \SDownarrow f^\ar[\taus]\;\oln{e_2}{m}$ and $e_{1i} \equiv e_{2i}$.
  \item If $P |- e_1 \SDownarrow K[\taus](\ol{e_1})$ then $P |- e_2 \SDownarrow K[\taus'](\ol{e_2})$ and and $e_{1i} \equiv e_{2i}$.
\end{itemize*}
\end{lemma}
In what follows we will use the set $S^{min}$ as the carrier set of our first-order model, we will 
interpret equality as $\equiv$ and the first-order language of our signature as follows.


\[\setlength{\arraycolsep}{2pt}
\begin{array}{rcl}
   \mlinterp{f^{\ar}}(\mu_1,\ldots,\mu_n) & = & 
       \multicolumn{1}{l}{\text{If there exists } (e_1\;e_2) \in S^{min}} \\
   & & \multicolumn{1}{l}{\text{such that } e_1{\SDownarrow}f\;[\taus]\;\oln{e}{\ar-1}} \\
   & & \multicolumn{1}{l}{\text{and } \mu_i \equiv e_i} \\
   & & \multicolumn{1}{l}{\text{and } \mu_{\ar} \equiv e_2 \text{ then } (e_1\;e_2) \text{ else } \bot} \\
   \mlinterp{f_{ptr}} & = & \text{If there exists } f \in S^{min} \\
                        &   & \text{then } f \text{ else } \bot \\
  \mlinterp{app}(\mu_1,\mu_2) & = & \text{If there exists } (e_1\;e_2) \in S^{min} \\ 
                         &   & \text{such that } \mu_1 \equiv e_1 \\ 
                         &   & \text{and } \mu_2 \equiv e_2 \text{ then } (e_1\;e_2) \text{ else } \bot \\
  \mlinterp{K^\ar}(\mu_1,\ldots,\mu_\ar) & = & \text{If there exists } (K[\taus](\ol{e})) \in S^{min} \\
                                    &  & \text{such that } \mu_i' \equiv e_i \text{ then } (K[\taus](\ol{e})) \\
                                    &  & \text{else } \bot \\
  \mlinterp{\sel{K}{i}}(\mu) & = & \text{If there exists } (K[\taus](\ol{e})) \in S^{min} \\ 
                             &   & \text{such that } \mu \equiv (K[\taus](\ol{e})) \text{ then } e_i \\ 
                             &   & \text{else } \bot \\ 
%% \dapp(\dbrace{f},\oln{d}{\ar}) \\ 
%%    \linterp{app}(d_1,d_2)     & = & \dapp(d_1,d_2) \\
%%    \linterp{f_{ptr}}  & = & \dbrace{f} \\
%%    \linterp{K^{\ar}}(d_1,\ldots,d_\ar) & = & \roll(\ret(\inj{K}\langle d_1,\ldots,d_\ar\rangle)) \\ 
%%    \linterp{\sel{K}{i}}(d) & = &  \roll(\bind_g(\unroll(d))) \\ 
%%      \text{where } g  & = & [\;\bot \\ 
%%                       &   & ,\;\dlambda d @.@ \unroll(\pi_i(d))  \quad (\text{case for K}) \\ 
%%                       &   & ,\;\bot \\
%%                       &   & ,\;\ldots\\
%%                       &   & ,\;\bot\; ] \\
  \mlinterp{bad}       & = & @BAD@ \\
  \mlinterp{unr}       & = & \bot \\ \\ \\ 
  \mlinterp{min}(\mu)  & = & \mu \neq \bot \\ 
  \mlinterp{cf}(\mu)   & = & \text{There exists } e \text{ such that } \mu = e \\ 
                       &   & \text{and } \dbrace{e} \in F_\lcfZ^{\infty} 
\end{array}\]

First of all, we have to prove that $\mlinterp{\cdot}$ above is a function and not a relation. 
This is easy to do by observing that the interpretation can only return 
different terms that are nevertheless equated by $\equiv$. Secondly, we must prove that the interpreted 
functions are congruent over $\equiv$. That is also straightforward.

\begin{lemma}\label{lem:min-interp}
If $e \in S^{min}$ and $\etrans{}{\cdot}{e} = t$ then $\mlinterp{t} \equiv e$.
\end{lemma}
\begin{proof} Straightforward induction over the structure of $e$ and by observing 
that in any term $e \in S^{min}$, all structural subterms of $e$ are also in $S^{min}$.
\end{proof}

Moreover, we can show that $\langle S^{min},{\cal I}^{min}\rangle$ is a model of $\ThMin$, by showing that
it validates all the axioms. 
\begin{theorem}
$\langle S^{min},{\cal I}^{min}\rangle \models \ThMin$.
\end{theorem}
\begin{proof} We have to show that each of the axioms in $\ThMin$ is valid in this model. The only interesting 
axioms is \rulename{AxAppMin}:
\[ \forall \ol{x} @.@ min(app(f_{ptr},\xs)) => f(\ol{x}) = app(f_{ptr},\xs) \]
Let us pick elements $\mu_1\ldots\mu_n \in S^{min}$. Since $min(app(f_{ptr},\xs))$ it must be that they are actually
all expressions, $e_1\ldots e_n$. Moreover we have the following chain for the left-hand-side: 
\[\begin{array}{lcl}
     f & \equiv & e_1^{\star} \\ 
     e^{\star}_1\;e_1 & \equiv & e^{\star}_2 \\ 
     e^{\star}_2\;e_2 & \equiv & e^{\star}_3 \\ 
              & \ldots & 
\end{array}\] 
Such that the left-hand-side is $e^{\star}_n\;e_n$. Now, by Lemma~\ref{lem:equiv-shapes} 
we have that $e^{\star}_1 \SDownarrow f$, $e^{\star}_2 \SDownarrow f\;e_1'$ for some $e_1 \equiv e_1'$ and so on, 
so that eventually we have that $e^{\star}_n \SDownarrow f\;e_1'\ldots e_{n-1}'$ for equivalent $e_i \equiv e_i'$. 
Applying the interpretation of $f$ for the right-hand-side finishes the case.
\end{proof} 


\section{Min in terms of unreachable}

 In fact we may take
one step further and equate all the non-interesting values of the domain to $\bot$.

To achieve this effect, we update our Prelude theory axioms as follows:
{\small
\[\setlength{\arraycolsep}{1pt}
\begin{array}{c}
%% \ruleform{\Th{\Sigma}{P}} \\ \\
\begin{array}{lll}
 \textsc{AxDisjA} & \formula{\bad \neq \unr}  \\
 \textsc{AxDisjB} & \formula{\forall \oln{x}{n}\oln{y}{m} @.@} \\
                  & \formula{\;\;\highlight{K(\ol{x}){\neq}\unr\;\lor\;J(\ol{y}){\neq}\unr} =>
                                  K(\ol{x}){\neq}J(\ol{y})} \\
                  & \text{ for every } (K{:}\forall\as @.@ \oln{\tau}{n} -> T\;\as) \in \Sigma \\
                  & \text{ and } (J{:}\forall\as @.@ \oln{\tau}{m} -> S\;\as) \in \Sigma \\
 %% \textsc{AxDisjCUnr} & \formula{\forall \oln{x}{n} @.@ \highlight{\neg min(\unr)}} \\
 %%                  & \text{ for every } (K{:}\forall\as @.@ \oln{\tau}{n} -> T\;\as) \in \Sigma \\ \\
 \textsc{AxDisjCBad} & \formula{\forall \oln{x}{n} @.@ K(\ol{x}) \neq \bad} \\
                  & \text{ for every } (K{:}\forall\as @.@ \oln{\tau}{n} -> T\;\as) \in \Sigma \\ \\

 \textsc{AxAppA}  & \formula{\forall \oln{x}{n} @.@ f(\ol{x}) = app(f_{ptr},\xs)} \\
                  & \text{ for every } (f |-> \Lambda\as @.@ \lambda\oln{x{:}\tau}{n} @.@ u) \in P \\
 %% \textsc{AxAppB}  & \formula{\forall \oln{x}{n} @.@ K(\ol{x}) = app(\ldots (app(x_K,x_1),\ldots,x_n)\ldots)} \\
 %%                  & \text{ for every } (K{:}\forall\as @.@ \oln{\tau}{n} -> T\;\as) \in \Sigma \\
 \textsc{AxAppC}  & \formula{\forall x, app(\bad,x) = \bad \; /\ \; app(\unr,x) = \unr}    \\ \\
 %% Not needed: we can always extend partial constructor applications to fully saturated and use AxAppC and AxDisjC
 %% \textsc{AxPartA} & \formula{\forall \oln{x}{n} @.@ app(\ldots (app(x_K,x_1),\ldots,x_n)\ldots) \neq \unr} \\
 %%                  & \formula{\quad\quad \land\; app(\ldots (app(x_K,x_1),\ldots,x_n)\ldots) \neq \bad} \\
 %%                  & \text{ for every } (K{:}\forall\as @.@ \oln{\tau}{m} -> T\;\as) \in \Sigma \text{ and } m > n \\
 %% \textsc{AxPartB} & \formula{\forall \oln{x}{n} @.@ app(f_{ptr},\xs) \neq \unr} \\
 %%                  & \formula{\quad\land\; app(f_{ptr},\xs) \neq \bad} \\
 %%                  & \formula{\quad\land\; \forall \oln{y}{k} @.@ app(f_{ptr},\xs) \neq K(\ol{y})} \\
 %%                  & \text{ for every } (f |-> \Lambda\as @.@ \lambda\oln{x{:}\tau}{m} @.@ u) \in P  \\
 %%                  & \text{ and every } (K{:}\forall\as @.@ \oln{\tau}{k} -> T\;\as) \in \Sigma \text{ and } m > n  \\ \\
 \textsc{AxInj}   & \formula{\forall \oln{y}{n} @.@ \highlight{K(\ys) \neq \unr\;\land\; y_i \neq \unr}} \\
                  & \formula{\quad\qquad\qquad => \sel{K}{i}(K(\ys)) = y_i} \\
                  & \text{for every } (K{:}\forall\as @.@ \oln{\tau}{n} -> T\;\as) \in \Sigma \text{ and } i \in 1..n \\ \\
 \textsc{AxCfA}   & \formula{\lcf{\unr} /\ \lncf{\bad}} \\
 \textsc{AxCfB1}  & \formula{\forall \oln{x}{n} @.@ \bigwedge\lcf{\ol{x}}} => \lcf{K(\ol{x})} \\
                  & \text{ for every } (K{:}\forall\as @.@ \oln{\tau}{n} -> T\;\as) \in \Sigma \\
 \textsc{AxCfB2}  & \formula{\forall \oln{x}{n} @.@ \lcf{K(\ol{x})}\;\highlight{\land\;K(\ol{x}) \neq \unr} => \bigwedge\lcf{\ol{x}}} \\
                  & \text{ for every } (K{:}\forall\as @.@ \oln{\tau}{n} -> T\;\as) \in \Sigma
\end{array}
\end{array}\]}


\begin{figure}\small
\[\begin{array}{c}
\ruleform{\utrans{\Sigma}{\Gamma}{t \sim u} = \formula{\phi}} \\ \\
\prooftree
   \begin{array}{c} \ \\ \ \\
   \etrans{\Sigma}{\Gamma}{e} = \formula{t}
   \end{array}
   ----------------------------------------{TUTm}
   \begin{array}{l}
   \utrans{\Sigma}{\Gamma}{s \sim e } = \formula{(s = t) \lor \highlight{\neg min(s)}} \ \\ \ \\ \ \\
   \end{array}
   ~~~~~
  \begin{array}{l}
  \etrans{\Sigma}{\Gamma}{e} = \formula{t} \quad
  constrs(\Sigma,T) = \ol{K} \\
  \text{for each branch}\;(K\;\oln{y}{l} -> e') \\
  \begin{array}{l}
           (K{:}\forall \cs @.@ \oln{\sigma}{l} -> T\;\oln{c}{k}) \in \Sigma \text{ and }
           \etrans{\Sigma}{\Gamma,\ol{y}}{e'} = \formula{ t_K }
  \end{array}
  \end{array}
  ------------------------------------------{TUCase}
  {\setlength{\arraycolsep}{1pt}
  \begin{array}{l}
  \utrans{\Sigma}{\Gamma}{s \sim @case@\;e\;@of@\;\ol{K\;\ol{y} -> e'}} = \\
  \;\;\formula{ \begin{array}{l}
     \highlight{min(s)} => \\
     \begin{array}{ll}
          ( & \highlight{min(t)}\;\land \\
            & (t = \bad => s = \bad)\;\land \\
            & (\forall \ol{y} @.@ t = K_1(\ol{y}) => s = t_{K_1})\;\land \ldots \land \\
            & (t \neq \bad\;\land\;t \neq K_1(\oln{{\sel{K_1}{i}}(t)}{})\;\land\;\ldots => s = \unr) \\
          )
%% (t = \bad /\ s = \bad)\;\lor\;(s = \unr)\;\lor \\
%%                                 \quad      \bigvee(t = K(\oln{{\sel{K}{i}}(t)}{}) \land
%%                                            s = t_K[\oln{\sel{K}{i}(t)}{}/\ol{y}])
                   \end{array}
     \end{array}}
  \end{array}}
  %% {       \setlength{\arraycolsep}{2pt}
  %% \begin{array}{l}
  %% \utrans{\Sigma}{\Gamma}{s \sim @case@\;e\;@of@\;\ol{K\;\ol{y}{->}e'}} = \\
  %% \;\;\formula{
  %%      \begin{array}{l} (\highlight{s{=}\unr})\;\lor \\
  %%                           \;\; (\highlight{min(s) => min(t)}\;\land  \\
  %%                           \quad((t = \bad /\ s = \bad)\;\lor \\
  %%                           \quad\quad \bigvee(t = K(\oln{{\sel{K}{i}}(t)}{}) \land
  %%                                          s = t_K[\oln{\sel{K}{i}(t)}{}/\ol{y}])))
  %%                  \end{array}
  %%          }
  %% \end{array}}
\endprooftree
\end{array}\]
\caption{Minimality-enabled definition translation}\label{fig:min-def-trans-min}
\end{figure}



We will explain the modifications to the axiomatization in more detail in later sections.
%% In other words, we ensure that constructor applications are disjoint
%% only for values we are interested in. We will explain each axiom separately later.
%% Intuitively we wish to equate all terms that we are not interested in to $\unr$. We
%% can never be interested in $\unr$ in the intended model because that means that during
%% the evaluation of a term, which completed, we encountered a divergent term -- clearly a
%% contradiction!
What about function definitions? Figure~\ref{fig:etrans} has to be modified slightly as well,
as Figure~\ref{fig:min-def-trans} shows.

\begin{figure}\small
\[\begin{array}{c}
\ruleform{\utrans{\Sigma}{\Gamma}{t \sim u} = \formula{\phi}} \\ \\
\prooftree
   \begin{array}{c} \ \\ \ \\
   \etrans{\Sigma}{\Gamma}{e} = \formula{t}
   \end{array}
   ----------------------------------------{TUTm}
   \begin{array}{l}
   \utrans{\Sigma}{\Gamma}{s \sim e } = \formula{(s = t) \lor \highlight{s = \unr}} \ \\ \ \\ \ \\
   \end{array}
   ~~~~~
  \begin{array}{l}
  \etrans{\Sigma}{\Gamma}{e} = \formula{t} \quad
  constrs(\Sigma,T) = \ol{K} \\
  \text{for each branch}\;(K\;\oln{y}{l} -> e') \\
  \begin{array}{l}
           (K{:}\forall \cs @.@ \oln{\sigma}{l} -> T\;\oln{c}{k}) \in \Sigma \text{ and }
           \etrans{\Sigma}{\Gamma,\ol{y}}{e'} = \formula{ t_K }
  \end{array}
  \end{array}
  ------------------------------------------{TUCase}
  {\setlength{\arraycolsep}{1pt}
  \begin{array}{l}
  \utrans{\Sigma}{\Gamma}{s \sim @case@\;e\;@of@\;\ol{K\;\ol{y} -> e'}} = \\
  \;\;\formula{ \begin{array}{l}
     \highlight{s = \unr}\;\lor \\
     \begin{array}{ll}
          ( & \highlight{(t \neq \unr)}\;\land \\
            & (t = \bad => s = \bad)\;\land \\
            & (\forall \ol{y} @.@ t = K_1(\ol{y}) => s = t_{K_1})\;\land \ldots \land \\
            & (t = \bad\;\lor\;t = K_1(\oln{{\sel{K_1}{i}}(t)}{})\;\lor\;\ldots) \\
          )
%% (t = \bad /\ s = \bad)\;\lor\;(s = \unr)\;\lor \\
%%                                 \quad      \bigvee(t = K(\oln{{\sel{K}{i}}(t)}{}) \land
%%                                            s = t_K[\oln{\sel{K}{i}(t)}{}/\ol{y}])
                   \end{array}
     \end{array}}
  \end{array}}
  %% {       \setlength{\arraycolsep}{2pt}
  %% \begin{array}{l}
  %% \utrans{\Sigma}{\Gamma}{s \sim @case@\;e\;@of@\;\ol{K\;\ol{y}{->}e'}} = \\
  %% \;\;\formula{
  %%      \begin{array}{l} (\highlight{s{=}\unr})\;\lor \\
  %%                           \;\; (\highlight{min(s) => min(t)}\;\land  \\
  %%                           \quad((t = \bad /\ s = \bad)\;\lor \\
  %%                           \quad\quad \bigvee(t = K(\oln{{\sel{K}{i}}(t)}{}) \land
  %%                                          s = t_K[\oln{\sel{K}{i}(t)}{}/\ol{y}])))
  %%                  \end{array}
  %%          }
  %% \end{array}}
\endprooftree
\end{array}\]
\caption{Minimality-enabled definition translation}\label{fig:min-def-trans}
\end{figure}

%% \\ \\
%% \ruleform{ \Dtrans{\Sigma}{P} = \formula{\phi}} \\ \\
%% \prooftree
%%      \begin{array}{l}
%%        \text{for each} (f |-> \Lambda\oln{a}{n} @.@ \lambda\oln{x{:}\tau}{m} @.@ u) \in P \\
%%           \quad \utrans{\Sigma}{\ol{x}}{f(\ol{x}) \sim u} = \formula{\phi}
%%      \end{array}
%%      --------------------{TDefs}
%%      \Dtrans{\Sigma}{P} = \bigwedge_{P} \formula{\forall \ol{x} @.@ \phi}
%% \endprooftree

Now operationally we may instrument the evaluation relation to keep track of the set of
closed terms that appear during evaluation. The instrumented relation appears in
Figure~\ref{fig:opsem-instrumented}. Observe that if $P |- e \Downarrow w \curly S$ then
$S$ is a {\em finite set} of terms.


\begin{figure}\small
\[\begin{array}{c}
\ruleform{P |- e \Downarrow v \curly S} \\ \\
\prooftree
\begin{array}{c} \ \\
\end{array}
%% \begin{array}{c}
%% (f |-> \Lambda\ol{a} @.@ \lambda\oln{x{:}\tau}{m} @.@ u) \in P \\
%% P |- e_1 \Downarrow f\;[\taus]\;\oln{e}{m-1} \curly S_1 \\
%% P |- u[\ol{\tau}/\ol{a}][\ol{e},e_2/\ol{x}] \Downarrow w \curly S
%% \end{array}
%% ------------------------------------{EBeta}
%% P |- e_1\;e_2 \Downarrow w
  S = heads(v)
-------------------------------------{EVal}
P |- v \Downarrow v \curly S
~~~~
\begin{array}{c}
(f |-> \Lambda\ol{a} @.@ \lambda\oln{x{:}\tau}{m} @.@ u) \in P \\
P |- u[\ol{\tau}/\ol{a}][\ol{e}/\ol{x}] \Downarrow v \curly S_1 \\
S_2 = heads(f[\ol{\tau}]\;\oln{e}{m})
\end{array}
-------------------------------------{EFun}
P |- f[\ol{\tau}]\;\oln{e}{m} \Downarrow v \curly S_1 \cup S_2
~~~~~
\begin{array}{c}
P |- e_1 \Downarrow v_1 \curly S_1 \quad
P |- v_1\;e_2 \Downarrow w \curly S_2
\end{array}
------------------------------------------------{EApp}
P |- e_1\;e_2 \Downarrow w \curly S_1 \cup S_2 \cup \{ e_1\;e_2 \}
~~~~~
\begin{array}{c}
P |- e_1 \Downarrow @BAD@ \curly S
\end{array}
------------------------------------------------{EBadApp}
P |- e_1\;e_2 \Downarrow @BAD@ \curly S \cup \{ e_1\;e_2 \}
\endprooftree \\ \\
\ruleform{heads(e) = S} \\ \\
\begin{array}{lcl}
   heads(f\;[\ol{\tau}]) & = & \{ f\;[\ol{\tau}] \} \\
   heads(e_1\;e_2)       & = & \{ e_1\;e_2 \} \cup heads(e_1) \\
   heads(\_)            & = & \emptyset
\end{array} \\ \\
\ruleform{P |- u \Downarrow v \curly S} \\ \\
\prooftree
P |- e \Downarrow v \curly S
-------------------------------------{EUTm}
P |- e \Downarrow v \curly S
~~~~~
\begin{array}{c}
P |- e \Downarrow K_i[\ol{\sigma}_i](\ol{e}_i) \curly S_1 \quad
P |- e'_i[\ol{e}_i/\ol{y}_i] \Downarrow w \curly S_2
\end{array}
------------------------------------{ECase}
P |- @case@\;e\;@of@\;\ol{K\;\ol{y} -> e'} \Downarrow w \curly S_1 \cup S_2
~~~~~
\begin{array}{c}
P |- e \Downarrow @BAD@ \curly S \\
\end{array}
------------------------------------{EBadCase}
P |- @case@\;e\;@of@\;\ol{K\;\ol{y} -> e'} \Downarrow @BAD@ \curly S
%% \begin{array}{c}
%% (f |-> \Lambda\ol{a} @.@ \lambda\oln{x{:}\tau}{m} @.@ @case@\;e\;@of@\;\ol{K\;\ol{y} -> e'}) \in D \\
%% D |- e[\ol{\tau}/\ol{a}][\ol{e}/\ol{x}] \Downarrow @BAD@ \\
%% \end{array}
%% -------------------------------------{EBadCase}
%% D |- f[\ol{\tau}]\;\oln{e}{m} \Downarrow @BAD@
\endprooftree
\end{array}\]
\caption{Redex-instrumented operational semantics}\label{fig:opsem-instrumented}
\end{figure}

\section{The intended min-imal model}


Our goal is then going to be to establish the following result, stated in non-technical terms:
\begin{quote}
If there exists a counterexample to a contract, then the negation of the contract-translation
formula is satisfiable not only on $\langle D_\infty,{\cal I}\rangle$ but it also has a {\em finite}
model. That finite model is a model of our minimality-enabled theory.
\end{quote}

We start unfolding the story. For a given program $P$ in a signature $\Sigma$ we have already
shown how to construct $D_\infty$ and how to give interpretations ${\cal I}$ to a first-order
vocabulary. Let us assume that the program and signature contains a polymorpic $undefined$
function, for convenience $undefined |-> udefined$. This is a realistic assumption to make
(e.g. it comes in the standard Haskell prelude).

Assume now that we are given a formula $\phi$ defined as:
\[  \phi = \ctrans{\Sigma}{\cdot}{e \in \Ct_1 -> \ldots \Ct_n -> @B@} \]
for @B@ a base contract. Assume moreover that there exist $\oln{e}{n}$, closed for the
program $P$, such that for each $e_i$ it is true that:
\[\interp{\Ct_i}{\dbrace{P}^\infty}{\cdot}(\interp{e_i}{\dbrace{P}^\infty}{\cdot})\].
Assume however that it is {\em not} the case that
\[\interp{{\tt B}}{\dbrace{P}^\infty}{\cdot}(\interp{e\;\oln{e}{n}}{\dbrace{P}^\infty}{\cdot})\]
There are two cases for the base constract @B@:
\begin{itemize}
  \item Let us now consider the case where @B@ = $\{ x \mid e_p \}$. By adequacy it must
  be that: $P |- e\;\ol{e} \Downarrow w \curly S_1$ for some $w$ and set $S_1$ and moreover
  $P |- e_p[e\;ol{e}/x] \Downarrow \{ @BAD@, False \} \curly S_2$ for some set $S_2$.

  Of course the following lemma is true:
  \begin{lemma}\label{lem:curly}
    If $P |- e \Downarrow w \curly S$ then $S$ is a finite set. Moreover,
    for every $e' \in S$ there exists $w$ such that $P |- e' \Downarrow w$.
  \end{lemma}
  Moreover we have:
  \begin{lemma}\label{lem:bot-not-redex}
     If $P |- e \Downarrow w \curly S$ then
     $\bot \notin \interp{S}{\dbrace{P}^{\infty}}{\cdot}$.
  \end{lemma}
  \begin{proof} If $\bot \in \interp{S}{\dbrace{P}^{\infty}}{\cdot}$ then there exists
  a term $e \in S$ such that $\interp{e}{\dbrace{P}^{\infty}}{\cdot} = \bot$. This means
  that $P |- e \not\Downarrow$ but that is a contradiction to $e \in S$ by
  Lemma~\ref{lem:curly}.
  \end{proof}

  Let us now define the {\em minimal sets} operationally and denotationally:

  \[\begin{array}{lcl}
           M        & \triangleq & S_1 \cup S_2 \\
           {\cal M} & \triangleq & \interp{S_1\cup S_2}{\dbrace{P}^{\infty}}{\cdot}
  \end{array}\]
  Consider now the function $\mu : D_\infty -> D_\infty$ defined as:
  \[\begin{array}{lcl}
        \mu(d) & \triangleq & \left\{ \begin{array}{ll}
                   d           & \text{when } \unroll(d) = \ret(\inj{bad}(1)) \\
                   d           & \text{when } d \in \Min \\
                   \bot        & \text{otherwise }
                                      \end{array}\right.
  \end{array}\]
  In other words $\mu(\cdot)$ conflates all the non-interesting values to $\bot$.
  Now we may consider the {\em set} which is the image of $D_\infty$ through $\mu$:
  \[ D_\infty^\mu  \triangleq \mu(D_\infty) \]

  Notice that this set is {\em finite} with cardinality at most $card(M) + 2$. Also,
  we treat this is a {\em set}. Although $D_\infty$ has a domain structure, we do not
  care about $D_\infty^\mu$ being a domain.

  Now, in this $D_\infty^\mu$ we may redefine the interpretation of first-order constants
  and variable symbols in our theories, using ${\cal I}^\mu$ below:


  {\setlength{\arraycolsep}{2pt}
  \[\begin{array}{rcl}
     \mlinterp{f_{ptr}} & = & \mu(\dbrace{P}^{\infty}(f)) \\
 %% \roll(\ret(\inj{->}(\dlambda d_1 @.@ \ldots  \\
 %%                       &   & \quad \roll(\ret(\inj{->}(\dlambda d_n @.@ \\
 %%                       &   & \quad\quad\text{ if there exist } \oln{e}{n} \text{ s.t. } f[\taus]\;\ol{e} \in M \\
 %%                       &   & \quad\quad\quad\text{ and } \interp{e_i}{\dbrace{P}^\infty}{\cdot} = d_i\text{ then } \\
 %%                       &   & \quad\quad\quad\quad \mu(\dapp(\dbrace{P}^{\infty}(f),\oln{d}{n})) \\
 %%                       &   & \quad\quad\text{ else } \bot)))\ldots))) \\ \\
   \mlinterp{f^{n}}  & = & \dlambda (d {:} \prod_{n}D_{\infty}^\mu) @.@  \\
                       %% &   & \quad\quad\text{ if there exist } \oln{e}{n} \text{ s.t. } f[\taus]\;\ol{e} \in M \\
                       %% &   & \quad\quad\quad\text{ and } \interp{e_i}{\dbrace{P}^\infty}{\cdot} = \pi_i(d)\text{ then } \\
                       &   & \quad\quad (\mu\cdot\dapp)(\mu(\dbrace{P}^{\infty}(f)),\oln{\pi_i(d)}{i \in 1..n})) \\
                       %% &   & \quad\quad\text{ else } \bot \\ \\

   \mlinterp{app}     & = & \dlambda (d {:} D_{\infty}^\mu \times D_{\infty}^\mu) @.@ \\ \
                      &   & \quad\qquad \mu(\dapp(\pi_1(d),\pi_2(d))) \\ \\

   \mlinterp{K^{\ar}}     & = & \dlambda (d {:} \prod_{\ar}D_{\infty}^\mu) @.@ \mu(\roll(\ret(\inj{K}(d)))) \\
   \mlinterp{\sel{K}{i}} & = & \dlambda (d {:} D_{\infty}^{\mu}) @.@ \mu(\roll(\bind_g(\unroll(d)))) \\
     \text{where } g  & = & [\;\bot \\
                      &   & ,\;\dlambda d @.@ \unroll(\pi_i(d))  \quad (\text{case for constr. } K) \\
                      &   & ,\;\bot \\
                      &   & ,\;\ldots\\
                      &   & ,\;\bot\; ]
  \end{array}\]}

  Sadly, while the interpretation above is relatively simple, it does not validate the axiom
  for \textsc{TUCase}. The fact that the denotation of a function application may be in the minimal set,
  does not guarrantee that evaluation had proceeded along this function and hence the case scrutinee will
  be in the minimal set. This will be true only if we add an intentional test in the interpretation of
  functions that queries the set $M$. {\bf DV:TODO tomorrow}.

  {\bf DV: TODO: We need something like the definition below (but not quite, it does not type check yet)}:
  {\setlength{\arraycolsep}{2pt}
  \[\begin{array}{rcl}
     \mlinterp{f_{ptr}} & = & \mu(\roll(\ret(\inj{->}(\dlambda d_1 @.@ \ldots  \\
                       &   & \quad \mu(\roll(\ret(\inj{->}(\dlambda d_n @.@ \\
                       &   & \quad\quad\text{ if there exist } \oln{e}{n} \text{ s.t. } f[\taus]\;\ol{e} \in M \\
                       &   & \quad\quad\quad\text{ and } \interp{e_i}{\dbrace{P}^\infty}{\cdot} = d_i\text{ then } \\
                       &   & \quad\quad\quad\quad \mu(\dapp(\dbrace{P}^{\infty}(f),\oln{d}{n})) \\
                       &   & \quad\quad\text{ else } \bot))))\ldots)))) \\ \\
   \mlinterp{f^{n}}  & = & \dlambda (d {:} \prod_{n}D_{\infty}^\mu) @.@  \\
                       &   & \quad\quad\text{ if there exist } \oln{e}{n} \text{ s.t. } f[\taus]\;\ol{e} \in M \\
                       &   & \quad\quad\quad\text{ and } \interp{e_i}{\dbrace{P}^\infty}{\cdot} = \pi_i(d)\text{ then } \\
                       &   & \quad\quad\quad\quad \mu(\dapp(\dbrace{P}^{\infty}(f),\oln{\pi_i(d)}{i \in 1..n})) \\
                       &   & \quad\quad\text{ else } \bot \\ \\

   \mlinterp{app}     & = & \dlambda (d {:} D_{\infty}^\mu \times D_{\infty}^\mu) @.@ \\ \
                      &   & \quad\qquad \mu(\dapp(\pi_1(d),\pi_2(d))) \\ \\

   \mlinterp{K^{\ar}}     & = & \dlambda (d {:} \prod_{\ar}D_{\infty}^\mu) @.@ \mu(\roll(\ret(\inj{K}(d)))) \\
   \mlinterp{\sel{K}{i}} & = & \dlambda (d {:} D_{\infty}^{\mu}) @.@ \mu(\roll(\bind_g(\unroll(d)))) \\
     \text{where } g  & = & [\;\bot \\
                      &   & ,\;\dlambda d @.@ \unroll(\pi_i(d))  \quad (\text{case for constr. } K) \\
                      &   & ,\;\bot \\
                      &   & ,\;\ldots\\
                      &   & ,\;\bot\; ]
  \end{array}\]}

  \item The other case is when $@B@ = \CF$. {\bf TODO}
\end{itemize}

\end{document}

%% \begin{abstract}
%% The Glasgow Haskell Compiler is an optimizing
%% compiler that expresses and manipulates first-class equality proofs in
%% its intermediate language.  We describe a simple, elegant technique that
%% exploits these equality proofs to support \emph{deferred type errors}.
%% The technique requires us to treat equality proofs as possibly-divergent
%% terms; we show how to do so without losing either soundness or
%% the zero-overhead cost model that the programmer expects.
%% \end{abstract}

%% \category{D.3.3}{Language Constructs and Features}{Abstract data types}
%% \category{F.3.3}{Studies of Program Constructs}{Type structure}

%% \terms{Design, Languages}

%% \keywords{Type equalities, Deferred type errors, System FC}

%% \section{Denotational semantics}


%% \begin{lemma}[Evaluation preserves equality]
%% If $\Sigma;\cdot |- e : \tau \rightsquigarrow t$ and
%%    $\Sigma |- D \rightsquigarrow \phi_{\Sigma,D}$ and
%%    $D |- e \Downarrow w$ then
%%    $\Sigma;\cdot |- w : \tau \rightsquigarrow s$ and $\Th{\Sigma}{D} /\ \phi_{\Sigma,D} |- t = s$.
%% \end{lemma}
%% \begin{proof} By induction on the evaluation $\Sigma |- e \Downarrow w$. \end{proof}


%% \begin{lemma}[Logic deduces sound value equalities]
%% Assume that $\Sigma;\cdot |- w : \tau \rightsquigarrow t$ and
%% $D |- value(w)$ and $\Sigma |- D \rightsquigarrow \phi_{\Sigma,D}$.
%% Then
%% \begin{enumerate*}
%%   \item If $\Th{\Sigma}{D} /\ \phi_{\Sigma,D} |- t = \bad$ then $w = @BAD@$.
%%   \item If $\Th{\Sigma}{D} /\ \phi_{\Sigma,D} |- t = K(\ol{t})$ then $w = K[\taus](\ol{e})$, such
%%         that $\Sigma;\cdot |- \ol{e : \tau} \rightsquigarrow \ol{s}$, and $\Th{\Sigma}{D} /\ \phi_{\Sigma,D} |- \ol{t = s}$.
%%   \item $\Th{\Sigma}{D} /\ \phi_{\Sigma,D} |- t \neq \unr$.
%% \end{enumerate*}
%% \end{lemma}
%% \begin{proof}
%% The proof of all three cases is by inversion on the $D |- value(w)$ derivation,
%% apealling to the disjointness axioms.
%% %% \begin{enumerate*}
%% %%   \item By inversion on the $D |- value(w)$ derivation. In the case of \rulename{VBad} we are done.
%% %%   The case of \rulename{VFun} cannot happen, by the axiom set \rulename{AxPartB}. The case of \rulename{VCon}
%% %%   cannot happen either: If the application is saturated then \rulename{AxDisjC} shows it is impossible; if it
%% %%   is not saturated we can always extend it and use \rulename{AxAppC} and \rulename{AxDisjC}.
%% %%   \item Again by inversion on $D |- value(w)$ derivation. The case of \rulename{VBad} is easy. The case for
%% %%   \rulename{VCon} follows by injectivity of constructors. The case of \rulename{VFun} can't happen by
%% %%   \rulename{AxPartB}.
%% %%   \item Direct inversion on $D |- value(w)$, and using disjointness axioms.
%% %% \end{enumerate*}
%% \end{proof}

%% Basic soundness will be stated as follows.
%% \begin{theorem}
%% If we have that
%% \begin{enumerate*}
%%   \item $\Sigma;\cdot |- e : \tau$ and $\Sigma;\cdot |- \Ct : \tau$
%%   \item $\Sigma |- D \rightsquigarrow \phi_{\Sigma,D}$
%%   \item $\Sigma;\cdot |- e \in \Ct \rightsquigarrow \phi$
%% \end{enumerate*}
%% and $\Th{\Sigma}{D} /\ \phi_{\Sigma,D} /\ \neg \phi$ is unsatisfiable then $\Sigma;D |- e \in \Ct$.
%% \end{theorem}
%% \begin{proof}
%%  {\bf TODO}
%% \end{proof}

%% A remark: a formula $\phi$ is unsatisfiable iff $\neg \phi$ is valid in FOL. Hence, if
%% $\Th{\Sigma}{D} /\ \phi_{\Sigma,D} /\ \neg \phi$ is unsatisfiable then
%% $\neg (\Th{\Sigma}{D} /\ \phi_{\Sigma,D}) \lor \phi$ must be valid, and by completeness of FOL,
%% $\Th{\Sigma}{D} /\ \phi_{\Sigma,D} |- \phi$.

%% \section{Denotational semantics as FOL models}



%% \begin{figure}\small
%% \[\begin{array}{c}
%% %% \ruleform{ \dtrans{\Sigma}{d} = \formula{\phi} } \\ \\
%% %% \prooftree
%% %%   \begin{array}{c}
%% %%   (f{:}\forall\oln{a}{n} @.@ \oln{\tau}{m} -> \tau) \in \Sigma \quad
%% %%   \etrans{\Sigma}{\ol{a},\ol{x{:}\tau}}{e} = \formula{t}
%% %%   \end{array}
%% %%   -------------------------------------------------------------------{TFDef}
%% %%   \dtrans{\Sigma}{(f |-> \Lambda\oln{a}{n} @.@ \lambda\oln{x{:}\tau}{m} @.@ e)} =  \formula{ (\forall x @.@ f(\oln{x}{m}) = t) }
%% %%   ~~~~~
%% %%   \begin{array}{l}
%% %%   (f{:}\forall\oln{a}{n} @.@ \oln{\tau}{m} -> \tau) \in \Sigma \quad
%% %%   \etrans{\Sigma}{\ol{a},\ol{x{:}\tau}}{e} = \formula{t} \\
%% %%   constrs(\Sigma,T) = \ol{K} \\
%% %%   \text{for each branch}\;(K\;\oln{y}{l} -> e') \\
%% %%   \quad \begin{array}{l}
%% %%            (K{:}\forall \cs @.@ \oln{\sigma}{l} -> T\;\oln{c}{k}) \in \Sigma \\
%% %%            \etrans{\Sigma}{\ol{a},\ol{x{:}\tau},\ol{y{:}\sigma[\taus/\cs]}}{e'} = \formula{ t_K }
%% %%         \end{array}
%% %%   \end{array}
%% %%   -------------------------------------------------------------------{TCaseDef}
%% %%   \begin{array}{l}
%% %%    \dtrans{\Sigma}{(f |-> \Lambda\oln{a}{n} @.@ \lambda\oln{x{:}\tau}{m} @.@ @case@\;e\;@of@\;\ol{K\;\ol{y} -> e'})} = \\
%% %%    \quad \formula{ \begin{array}{lll} \forall \oln{x}{m} @.@ & \hspace{-7pt} (t = \bad /\ f(\ol{x}) = \bad)\; \lor \\
%% %%                                                                     & \hspace{-7pt}(f(\ol{x}) = \unr)\;\lor \\
%% %%                                                                     & \hspace{-7pt}(\bigvee(t = K(\oln{{\sel{K}{i}}(t)}{i\in 1..l})\;/\ \\
%% %%                                                                     & \hspace{-5pt}\quad f(\ol{x}) = t_K[\oln{\sel{K}{i}(t)}{i\in 1..l}/\ol{y}]))
%% %%                                                  \end{array}
%% %%                         }
%% %% \end{array}
%% %% \endprooftree  \\ \\
%% \end{array}\]
%% \caption{Definition elaboration to FOL}\label{fig:typing}
%% \end{figure}

%% {\bf DV: So basically this is Simon's strategy of side-stepping the
%% lack of full abstraction and the associated problems with it: In
%% the end of the day we only care about base contracts, in fact
%% really only about the contract ``is this program crash-free'', so
%% we don't have to make a big fuss about higher-order contracts and
%% their operational semantics. We have to motivate it carefully and
%% also be clear that for the intellectually curious reader who really
%% wants to know what statement we have proved for a function contract
%% when the prover says ``unsat'' we might want to give a full
%% definition of the denotational meaning of contracts including both
%% base and higher-order. I think we do not have the time luxury to
%% look for more elaborate solutions (such as definable denotations
%% and all that crazy stuff) to match the operational and the
%% denotational semantics for higher-order contracts. Fullstop.}

%% \section{Minimizing countermodels}




%% \newpage

%% \section{Contract checking soundness}

%% \section{Contracts}

%% The syntax that we use for contracts is in Figure~\ref{fig:contract-syntax}.
%% Contracts are typed (here, just monomorphically), and we give an operational
%% semantics for contract satisfaction in the same figure.

%% \begin{figure*}\small
%% \[\begin{array}{c}
%% \ruleform{\Sigma;\Gamma |- \Ct } \\ \\
%% \prooftree
%% \Sigma;\Delta,x{:}\tau |- e : \Bool
%% ---------------------------------------{TCBase}
%% \Sigma;\Delta |- \{ (x{:}\tau) \mid e \} : \tau
%% ~~~~
%% \begin{array}{c}
%% \Sigma;\Delta |- \Ct_1 : \tau \\
%% \Sigma;\Delta,(x{:}\tau) |- \Ct_2 : \tau'
%% \end{array}
%% ---------------------------------------{TCArr}
%% \Sigma;\Delta |- (x{:}\Ct_1) -> \Ct_2 : \tau -> \tau'
%% ~~~~
%% \Sigma;\Delta |- \Ct_1 : \tau \quad \Sigma;\Delta |- \Ct_2 : \tau
%% ---------------------------------------{TCConj}
%% \Sigma;\Delta |- \Ct_1 \& \Ct_2 : \tau
%% ~~~~
%% \phantom{\Gamma}
%% ---------------------------------------{TCf}
%% \Sigma;\Delta |- \CF : \tau
%% \endprooftree \\ \\
%% \ruleform{\Sigma;P |- e \in \Ct} \\ \\
%% \prooftree
%%  P \not|- e \Downarrow
%% -----------------------------------------------{ECDiv}
%%  \Sigma;P |- e \in \{ (x{:}\tau) \mid e' \}
%%  ~~~~
%%  P |- e'[e/x] \Downarrow \True
%% -------------------------------------------{ECTrue}
%%  \Sigma;P |- e \in \{ (x{:}\tau) \mid e' \}
%%  ~~~~
%%  P \not|- e'[e/x] \Downarrow
%%  ------------------------------------------{ECCDiv}
%%  \Sigma;P |- e \in \{ (x{:}\tau) \mid e' \}
%%  ~~~~~
%%  \begin{array}{c}
%%  \Sigma;\cdot |- \Ct_1 : \tau \\
%%  \text{for all } u, \Sigma;\cdot |- u : \tau ==> \Sigma;P |- e\;u \in \Ct_2[u/x]
%%  \end{array}
%%  --------------------------------------------{ECArr}
%%  \Sigma;P |- e \in (x{:}\Ct_1) -> \Ct_2
%%  ~~~~
%%  \begin{array}{c}
%%  \Sigma,\cdot |- e : \tau  \\
%%  e \in \Ecf \quad \text{(See Section~\ref{sect:cf})}
%%  %% \text{for all } u, (\Sigma;\cdot |- u : \tau -> \Bool) /\ (@BAD@ \notin u) ==> \neg (P |- u\;e \Downarrow @BAD@)
%%  \end{array}
%%  --------------------------------------------------------------------------------------------{ECf}
%%  \Sigma;P |- e \in \CF
%%  ~~~~~
%%  \Sigma;P |- e \in \Ct_1 \quad \Sigma;P |- e \in \Ct_2
%%  --------------------------------------------------------------------------------------------{ECConj}
%%  \Sigma;P |- e \in \Ct_1 \& \Ct_2
%% \endprooftree
%% \end{array}\]
%% \caption{Contract syntax and semantics}\label{fig:contract-syntax}
%% \end{figure*}

%% \section{Induction and admissibility}
%% {\bf TODO}


%% \section{Minimization}
%% {\bf TODO}

%% \section{Some ideas}
%% Sometimes the $\CF$ contract stands in our way e.g. for library stuff. It might
%% be interesting to explore some user-defined pragma to side-step the $\bad$ case
%% in some pattern matches (i.e. make it on demand, pretty much as $F^{\star}$ does, where
%% only the user's assertions matter.
%% %% \acks
%% %% Acknowledgements here

