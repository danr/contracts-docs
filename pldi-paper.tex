\documentclass[preprint,nocopyrightspace]{sigplanconf}

\usepackage{hcc-techreport}


\begin{document}
\preprintfooter{\textbf{--- EARLY DRAFT to PLDI 2013 ---}}

\renewcommand{\textfraction}{0.1}
\renewcommand{\topfraction}{0.95}
\renewcommand{\dbltopfraction}{0.95}
\renewcommand{\floatpagefraction}{0.8}
\renewcommand{\dblfloatpagefraction}{0.8}

%%% Extra definitions -- move to hcc-techreport at some point (carefully to not break that!)
\newcommand{\Ct}{{\tt C}}
\newcommand{\CF}{{\tt CF}}
\newcommand{\True}{\textit{True}}
\newcommand{\False}{\textit{False}}
\newcommand{\Bool}{\mathop{Bool}}
\newcommand{\ys}{\ol{y}}
\newcommand{\Th}[2]{{\cal T}_{#1,#2}}
\newcommand{\Ecf}{\textsc{Ecf}}
\newcommand{\oln}[2]{\ol{#1}^{#2}}
\newcommand{\tmar}[2]{\mathop{tmar}_{#1}(#2)}
\newcommand{\tyar}[2]{\mathop{tyar}_{#1}(#2)}
\newcommand{\ar}{n}
\newcommand{\lcf}[1]{\textsf{cf}(#1)}
\newcommand{\lcfZ}{\textsf{cf}}
\newcommand{\lncf}[1]{\neg\textsf{cf}(#1)}
\newcommand{\unr}{\mathop{unr}}
\newcommand{\bad}{\mathop{bad}}
\newcommand{\sel}[2]{\mathop{sel\_#1\!_{#2}}}
\newcommand{\ctrans}[3]{{\cal C}\{\!\!\{#3\}\!\!\}}
\newcommand{\etrans}[3]{{\cal E}\{\!\!\{#3\}\!\!\}}
\newcommand{\utrans}[3]{{\cal U}(#3)\{\!\!\{#2\}\!\!\}}

% Get rid of this -- just temporay
\newcommand{\uutrans}[3]{{\cal U}\{\!\!\{#3\}\!\!\}}

\newcommand{\dtrans}[2]{{\cal D}\{\!\!\{#2\}\!\!\}}
\newcommand{\ptrans}[2]{{\cal P}\{\!\!\{#2\}\!\!\}}
%% Gadgets of domain theory



\newcommand{\rollK}{\mathsf{roll}}
\newcommand{\unrollK}{\mathsf{unroll}}
\newcommand{\bindK}{\mathsf{bind}}
\newcommand{\retK}{\mathsf{ret}}
\newcommand{\injK}[2]{\mathsf{#1}(#2)}
\newcommand{\injKZ}[1]{\mathsf{#1}}
\newcommand{\injFun}[1]{\mathsf{Fun}(#1)}
\newcommand{\injBad}{\mathsf{Bad}}

% \newcommand{\roll}[1]{\rollK(#1)}
% \newcommand{\unroll}[1]{\unrollK(#1)}
% \newcommand{\bind}[2]{\bindK_{#1}(#2)}
% \newcommand{\ret}[1]{\retK(#1)}
\newcommand{\roll}[1]{#1}
\newcommand{\unroll}[1]{#1}
\newcommand{\bind}[2]{#1(#2)}
\newcommand{\ret}[1]{#1}
\newcommand{\inj}[2]{{#1}(#2)}

\newcommand{\dlambda}{\mathsf{\lambda}}
\newcommand{\curry}{\mathsf{curry}}
\newcommand{\eval}{\mathsf{eval}}
\newcommand{\uncurry}{\mathsf{incurry}}
\newcommand{\dapp}{\mathsf{app}}

\newcommand{\unitcpo}{{\sf{\bf 1}}}
\newcommand{\VarCpo}{\textit{Var}}
\newcommand{\FVarCpo}{\textit{FunVar}}
\newcommand{\interp}[3]{[\![#1]\!]_{\langle {#2},{#3}\rangle}}
\newcommand{\dbrace}[1]{[\![#1]\!]}
\newcommand{\linterp}[1]{{\cal I}(#1)}
\newcommand{\lassign}[1]{\mu(#1)}
\newcommand{\elab}[1]{\rightsquigarrow \formula{#1}}
\newcommand{\Fcf}{F_{\lcfZ}}
\newcommand{\definable}[1]{{\mathop{def}}(#1)}
\newcommand{\curly}{\rightsquigarrow}
\newcommand{\Min}{\cal M}
\newcommand{\mlinterp}[1]{{\cal I}^{min}(#1)}

\renewcommand{\Th}{{\cal T}}

\newcommand{\theLang}{\lambda_{\sf HALO}}

%% \title{MIN}
\title{A new approach to static contract checking for higher-order lazy programs}

\authorinfo{Dimitrios Vytiniotis \\ Simon Peyton Jones}
           {Microsoft Research}{}

\authorinfo{Dan Ros\'{e}n \\ Koen Claessen}
           {Chalmers University}{}
%% \authorinfo{Nathan Collins}
%%            {Portland State University}{}
\maketitle
\makeatactive

\begin{abstract}
Even well-typed programs can go wrong, by encountering a pattern-match
failure, or simply returning the wrong answer.  An
increasingly-popular response is to allow programmers to write
\emph{contracts} that express semantic properties, such as
crash-freedom or some useful post-condition.
We study the \emph{static verification} of such contracts.
Our main contribution is a novel translation to first-order logic
of both Haskell programs, and contracts written in Haskell,
all justified by denotational semantics. This translation enables us to prove
that functions satisfy their contracts using an off-the-shelf first-order logic
theorem prover.
\end{abstract}

\section{Introduction}\label{s:intro}
  \input{hcc-introduction}

% % \section{Checking Haskell contracts in practice}\label{s:examples}
% %   \input{hcc-examples}

\section{A higher-order lazy language and its contracts}\label{sect:language}
  \input{hcc-language}

\section{Soundness through denotational semantics}
   \label{sect:contracts}\label{ssect:denot}
  \input{hcc-contracts}

\subsection{Contract checking as satisfiability}\label{sect:soundness}
  \input{hcc-verification}

\section{Minimization}
   \input{hcc-min}

\section{Induction}\label{sect:induction}
  \input{induction}

\section{Implementation and practical experience}\label{sect:implementation}
  \input{hcc-implementation}

\section{Extensions}
  
The initial contracts language is sufficient to express a wide variety
of properties, but it is easy to extend the language to be able to
declare more properties, and sometimes in a more straightforward way.
This section describes the new constructs.

\subsection{Parameterised Contracts and Local Assumptions}
Can this property that describes that @all p@ is a list homomorphism
be written as a contract?

$$\forall \; @p@ \; @xs@ \; @ys@ \; . \;
    @all p xs && all p ys@ = @all p (xs ++ ys)@$$

First of all, we need to ask ourselves which of the functions this
could be a contract for. A promising candidate seems to be @(++)@. So
will this do for a contract?

\[\begin{array}{rcl}
@(++)@ & \in & \{ @xs@ \mid @CF@ \; \& \; @any p xs@ \} \\
       & \to & \{ @ys@ \mid @CF@ \; \& \; @any p ys@ \} \\
       & \to & \{ @rs@ \mid @CF@ \; \& \; @any p rs@ \}
\end{array}\]

The problem here is that @p@ is a free variable, moreover, it is also
important that $@p@ \in @CF@ \to @CF@$. One could introduce a new function
which takes @p@ as an argument but ignores it as this:

\begin{code}
append_dummy p xs ys = xs ++ ys
\end{code}

Now, the contract can be expressed:

\[\begin{array}{rcl}
@append_dummy@ & \in & ( @p@ : @CF@ \to @CF@ ) \\
               & \to & \{ @xs@ \mid @CF@ \; \& \; @any p xs@ \} \\
               & \to & \{ @ys@ \mid @CF@ \; \& \; @any p ys@ \} \\
               & \to & \{ @rs@ \mid @CF@ \; \& \; @any p rs@ \}
\end{array}\]

This approach has several downsides:
\begin{enumerate}
  \item The @append_dummy@ function is not recursive, but this
  contract needs to be proved with fixed point induction on @(++)@,

  \item Annoying acrobatics involved in introducing a new function to
  get a new parameter,

  \item If we use this contract when proving another contract,
  chances are that @append_dummy@ is not going to be interesting, even
  though the property it expresses is. This could be a problem when
  using the min-translation.
\end{enumerate}

We will later argue that it is benefical to prove contracts for not
only one function, but for any expression. This solves the first entry
in the list above, and partially the second because then we could use
a lambda instead of a new top-level declaration.

However, by just extending the language of statements slightly we will
solve the latter two. We will allow quantification and assumptions in
statetements, so the contract can instead be written as this:

\[\begin{array}{rcl}
\forall \; @p@ \; & . & \; (@p@ \in @CF@ \to @CF@) => \\
@(++)@ & \in & \{ @xs@ \mid @CF@ \; \& \; @any p xs@ \} \\
       & \to & \{ @ys@ \mid @CF@ \; \& \; @any p ys@ \} \\
       & \to & \{ @rs@ \mid @CF@ \; \& \; @any p rs@ \}
\end{array}\]

Thus, statements now contain these two extra constructs:

\[\begin{array}{lrll}
  s,t & ::=  & e \in C                     & \text{Contracts} \\
      & \mid & \highlight{s => t}          & \text{Assumption} \\
      & \mid & \highlight{\forall x @.@ s} & \text{Quantification} \\
      & \mid & s \; \textsf{using} \; t    & \text{Reuse}
\end{array}\]

The translation of statements can be viewed in
Figure~\ref{fig:stmt-trans}.  Fixed point induction can now be
expressed quite elegantly: for a statement $s$ susceptible to fixed
point induciton over $f$, we now instead consider the statement
$s[f^\circ/f] => s[f^\bullet/f]$.

\begin{figure}\small
\setlength{\arraycolsep}{2pt}
\[\begin{array}{c}
\ruleform{\trs{v}{s} = \formula{\phi}} \\ \\
\begin{array}{lcl}
  \trs{-}{e \in C}         & = & \trc{e \notin C} \\
  \trs{+}{e \in C}         & = & \trc{e \in C} \\
  \trs{-}{\forall x @.@ s} & = & \exists x @.@ \trs{-}{s} \\
  \trs{+}{\forall x @.@ s} & = & \forall x @.@ \trs{+}{s} \\
  \trs{-}{s => t}          & = & \trs{+}{s} \land \trs{-}{t} \\
  \trs{+}{s => t}          & = & \trs{-}{s} \lor \trs{+}{t} \\
  \trs{-}{s \; \textsf{using} \; t} & = & \trs{-}{s} \land \trs{+}{t} \\
  \trs{+}{s \; \textsf{using} \; t} & = & \trs{+}{s} \\
\end{array}
\end{array}\]
\caption{
    The translation of statements in positive and negative
    position. Right-nested uses of $\textsf{using}$ are assumed to be
    removed.  \label{fig:stmt-trans}
}
\end{figure}

Another property describable with these statements is the one of
associativity as a contract:

\[\begin{array}{rcl}
\forall \; @z@ \; . \; @z@ : @CF@ => @(+)@
    & \in & ( @x@ \in @CF@ ) \to ( @y@ : @CF@ ) \to \\
    &     & \{ @r@ \mid @CF@ \; \& \; @r + z == x + (y + z)@ \}
\end{array}\]

\subsection{Expressions, not Functions}

If we lift the restriction that contracts is always accompanied by a
function to let contracts express properties about general
expressions, we get a richer language which can express this:

\begin{enumerate}
  \item Contracts that for functions that are partially applied:
    $$@map fromJust@ \in \{ @xs@ \mid @all isJust xs@ \} \to @CF@$$

  \item Repeated variables are allowed, allowing reflexivity and
    idempotence:
    \[\begin{array}{l}
    \forall \; @x@ \; . @x@ \in @CF@ => @x == x@ \in \{ @b@ \mid @CF@ \; \& \; @b@ \} \\
    \forall \; @x@ \; . @x@ \in @CF@ => @x && x@ \in \{ @b@ \mid @CF@ \; \& \; @b == x@ \}
    \end{array}\]

  \item Combined with assumptions, we can now express complex
      properties such as symmetry:

    \[\begin{array}{rcl}
    \forall \; @x@ & . & @x@ \in @CF@ => \\
    \forall \; @y@ & . & @y@ \in @CF@ => \\
                   &   & @x == y@ \in \{ @b@ \mid @CF@ \; \& \; @b@ \} => \\
                   &   & @y == x@ \in \{ @b@ \mid @CF@ \; \& \; @b@ \}    \\
    \end{array}\]

\end{enumerate}

Fixed point induction will be applied to the function at the top of
the rightmost (of $=>$s) statement if it is a recursive function. For
the symmetry example, this means that we would do fixed point induction
over @(==)@.

It turns out that contracts as they were before this section can now
be described in terms of explicit quantification and
assumptions. Indeed, the following two are equivalent:

\[\begin{array}{l}
@f@ \in (x_1 : C_1) \to \cdots \to (x_n : C_n) \to C
\\ \qquad\qquad <=> \\
\forall \; x_1 . x_1 \in C_1 => \cdots =>
\forall \; x_n . x_n \in C_n =>
    @f@ \; x_1 \; \cdots \; x_n \in C
\end{array}\]

It is up to the users of to use whichever they find more useful.

\subsection{First order equality in contracts}
\dr{This is not implemented, it is something to consider}

The earlier example with associativity of @(++)@ was defined in terms
of some equality function @(==)@. However, these are not very
convenient to prove with. We have to show that they constitute an
equivalence relation, but more serious is that we need to show that
they form a congruence over the functions we are interested in. The
equality in first order logic always has this property; it is
substitutive. And indeed, some properties we show hold up to its
equality, such as associativity:

\[\begin{array}{rcl}
\forall \{ @zs@ \} . @(++)@ & \in & \{ @xs@ \} \to \{ @ys@ \} \to \\
                            &     & \{ @rs@ \mid @rs ++ zs@ = @xs ++ (ys ++ zs)@ \}
\end{array}\]

How can we express this in our DSL? We make a new @Eq@ constructor for @Contract@:

$$@Eq :: (a -> Equality a) -> Contract a@$$,

and we make a new data type Equality:

\begin{code}
data Equality a where
    (:=:) :: a -> a -> Contract a
\end{code}

We now write the property as this:

\begin{code}
app_assoc :: [a] -> Statement
app_assoc zs = (++) :::
    Any :-> \xs -> Any :-> \ys -> Eq
        (\rs -> rs ++ zs :=: xs ++ (ys ++ zs))
\end{code}

Using induction, the step case goes through, but the base case is a
bit more problematic. We then have to prove that we have
$\bot @++ zs@ = @xs ++ (ys ++ zs)@$ for all @xs@, @ys@ and @zs@.
Dimitrios comes to the rescue and demands all contracts to hold in the
base case, so we really add a bottom fall-through for equality.

Formally, we extend contracts with a new construct
\[\begin{array}{lrll}
\multicolumn{3}{l}{\text{Contracts}} \\
 \Ct & ::=  & \cdots                 & \text{Previous constructs} \\
     & \mid & \formula{\{ x \mid e_1 = e_2 \}} & \text{Equality}
\end{array}\]

Translated as follows:

\[\begin{array}{l}
\ctrans{\Sigma}{\Gamma}{e \in \{x \mid e_1 = e_2 \}}
  = \; t{=}\unr \; \lor \; t_1[t/x]{=}t_2[t/x] \\
\quad \text{where} \;
   t = \etrans{\Sigma}{\Gamma}{e}, \;
   t_1 = \etrans{\Sigma}{\Gamma}{e_1} \; \text{and} \;
   t_2 = \etrans{\Sigma}{\Gamma}{e_2}
\end{array}\]

\paragraph{Suggested min translation of equality}

\[\begin{array}{l}
\ctrans{\Sigma}{\Gamma}{e \in \{x \mid e_1 = e_2 \}} \\
\quad = \; \formula{( min(t_1) \lor min(t_2) )} \\ %%  \formula{min(t) \; \land \;}
\quad => (\formula{min(t)} \; \land \; t{=}\unr) \; \lor \; t_1[t/x]{=}t_2[t/x] \\ \\
\ctrans{\Sigma}{\Gamma}{e \notin \{x \mid e_1 = e_2 \}} \\
\quad = \; \formula{(min(t) \; \land \; t{=}unr) \lor neq(t_1[t/x],t_2[t/x]))}
\end{array}\]

Where $neq$ is an apartness relation axiomatised in Figure~\ref{fig:neq-axioms}.

\dr{If removing min, we can also remove neq by replace it to $\neq$}

\begin{figure}
{\small
\[\setlength{\arraycolsep}{1pt}
\begin{array}{c}
 \ruleform{neq} \\ \\
\begin{array}{lll}
 \textsc{NeqIrrRefl} & \forall x @.@ \neg neq(x,x) \\
 \textsc{NeqSym}     & \forall x, y @.@ neq(x,y) => neq(y,x) \\
 \textsc{NeqTrans}   & \forall x, y, z @.@ neq(x,y) => (neq(x,z) \lor neq(y,z)) \\
 \textsc{NeqMin}     & \forall x, y @.@ neq(x,y) => (min(x) \land min(y)) \\
 \textsc{NeqDisj}    & \forall \oln{x}{n}\oln{y}{m} @.@ neq(K(\ol{x}),K(\ol{y})) => \bigvee_i neq(x_i,y_i) \\
                     & \text{ for every } (K{:}\forall\as @.@ \oln{\tau}{n} -> T\;\as) \in \Sigma \\
 \textsc{NeqApp}     & \forall f, g, x, y @.@ neq(app(f,x),app(g,y)) => \\
                     & (neq(f,g) \lor neq(x,y))
\end{array}
\end{array}\]}
\caption{An axiomatisation of neq
    \label{fig:neq-axioms}}
\end{figure}

\paragraph{Equality versus partially applied contracts}

With the exception from the extra $\bot$ guard, the expressibility of
equality and partially applied contracts indeed do overlap. The example
with @map fromJust@ above can be now instead be written:

$$@map@ \in \{ @f@ \mid @f@ = @fromJust@ \} \to \{ @xs@ \mid @all isJust xs@ \} \to @CF@$$

It is a bit clumsy, so we accept both versions.

\subsection{SMT 2.0 and triggers}
\dr{I don't know where this section will fit}
\paragraph{Support for primitive Integers}


\section{Discussion}\label{sect:discussion}
  \input{hcc-discussion}

\section{Related work}\label{sect:related}
  There are very few practical tools for the automatic
verification of arbitrary {\em lazy and higher-order} functional programs, though 
the automated verification of higher-order programs, at least for restricted 
strongly-normalizing languages, has been studied before, for instance by the ACL2 
community. Furthermore, our approach of directly translating the denotational 
semantics of programs does not appear to be well-explored in the literature.

Catch~\cite{Mitchell:2008:PBE:1411286.1411293} is one of the very few tools that
address the verification of lazy Haskell, and have been evaluated on real programs.
Using static analysis, Catch can detect pattern match failures, and hence
prove that a program cannot crash. Some annotations that describes the set of
constructors that are expected as arguments to each function may be
necessary for the analysis to succeed. 
Our aim in this paper is to achieve similar goals, but 
moreover to be in a position to assert functional correctness.

Liquid Types~\cite{Rondon:2008:LT:1375581.1375602} is an influential
approach to call-by-value functional program verification. 
Contracts are written as refinements in a fixed language of predicates (which may
include recursive predicates) and the extracted conditions are discharged using an
SMT-solver. Because the language of predicates is fixed, predicate abstraction can
very effectively {\em infer} precise refinements, even for recursive functions, and
hence the annotation burden is very low. In our case, since the language of predicates
is, {\em by design}, the very same programming language with the same semantics, inference
of function specifications is harder. The other important difference is that liquid types
requires all {\em uses} of a function to satisfy its precondition, whereas in the semantics
that we have chosen, bad uses are allowed but the programmer gets no guarantees back.
\dv{Todo: Andrey Rybalchenko ``sausage factory''}

Rather different to Liquid Types, Dminor~\cite{Bierman+:subtyping} allows 
refinements to be written in the very same programming language that programs are written.
Contrary to our case however, in Dminor
the expressions that refine types must be pure --- that is, terminating --- and have a unique
denotation (e.g. not depending on the store)\dr{what is the store?}. 
Driven from a typing relation that includes
logic entailment judgements, verification conditions are extracted and discharged automatically using Z3.
Similar in spirit, other dependent type systems such
as Fstar~\cite{fstar} also extract verification conditions that are discharged
using automated tools
or interactive theorem provers. Hybrid type systems such as Sage~\cite{Knowles+:sage}
attempt to prove as many of the goals statically, and defer the rest as runtime
goals.

Boogie~\cite{boogie} is a verification back end that supports procedures as well as
pure functions.  By using Z3, Boogie verifies
programs written in the BoogiePL intermediate language, 
which could potentially be used as the
back end of our translation as well. 
Recent work on performing induction on top of an
induction-free SMT solver proposes a ``tactic''
for encoding induction schemes as first-order queries, which is reminiscent of the way
that we perform induction \cite{Leino:2012:AIS:2189257.2189278}.

The recent work on the Leon system~\cite{Suter:2011:SMR:2041552.2041575} presents
an approach to the verification of {\em first-order} and {\em call-by-value}
recursive functional programs, which appears to be very efficient in practice.  It works
by extending SMT with recursive programs and ``control literals'' that guide the pattern
matching search for a counter-model, and is guaranteed to find a model if one exists
(whereas that is not yet the case in our system, as we discussed earlier). It 
treats does not include
a $\CF$-analogous predicate, and no special treatment of the $\bot$
value or pattern match failures seems to be in the scope of that project.
However, it gives a very fast verification framework for partial functional correctness.

The tool Zeno~\cite{zeno} verifies equational properties of functional
programs using Haskell as a front end. Its proof search is based on
induction, equality reasoning and operational semantics. While
guaranteeing termination, it can also start new induction proofs
driven by syntactic heuristics. However, it only considers the finite and total
subset of values, and we want to reason about Haskell
programs as they appear in the wild: possibly non-terminating, with
lazy infinite values, and run time crashes.

% Another tool that proves equational
% properties of Haskell programs under the same assumptions is
% HipSpec~\cite{hipspec} but
% But if we mention HipSpec, we should mention Hip. Then what about Hip?

First-order logic has been used as a target for higher-order languages
in other verification contexts as well.  Users of the interactive
theorem prover Isabelle have for many years had the opportunity to use
automated first-order provers to discharge proof obligations. This
work has recently culminated in the tool
Sledgehammer \cite{Sledgehammer}, which not only uses first-order
provers, but also SMT solvers as back ends.  There has also been a version
of the dependently typed programming language Agda in which proof
obligations could be sent to an automatic first-order
prover \cite{AgdaFOL}. Both of these use a translation from a
typed higher-order language of well-founded definitions to first-order
logic. The work in this area that comes very close to ours, in
that they deal with a lazy, general recursive language with partial
functions, is by \citet{TypeTheoryFOL}, who use Agda as a logical
framework to reason about general recursive functional programs, and
combine interaction in Agda with automated proofs in first-order
logic.

There exists also work on translating the semantics of programs or some of
their properties in higher-order logics, such as the calculus of inductive
constructions, for instance the work on Characteristic Formulae~\cite{char-form} 
and the work on Hoare-logic VCC-based verification~\cite{regis-gianas-pottier-08}. The former
is interpreting a program as a higher-order predicate transformer
whereas the latter introduces a higher-order typed Hoare-logic and
extracts verification conditions which are checked in Coq or an
off-the shelf theorem prover. We aim to stay within the realm of
first-order logic to exploit automation offered by the advances in FOL
theorem provers and model finders. At the same time we do lose
expressivity, as our predicate language are only plain-old Haskell
functions. On the other hand this gives us
admissibility practically for free even in the absense of inductive
types. But even in the case of the previous work of
R\'{e}gis-Gianas and Pottier, positivity conditions have to be imposed on
datatypes to make sure that the logical specification can be embedded
in a logic built in CIC. A final point about that work
is that contract preconditions are enforced at every call site of 
a function, a design choice that we find appealing and might want 
to retrofit in our system, and one that also appears in Liquid types.
In the higher-order logic world, the recent work of 
Huffman~\cite{Huffman:2012:FVM:2364527.2364532} can be used to translate Haskell 
programs and reason about their semantics in HOLCF; the main application being
the verification of properties of monad transformers.

The previous work on static contract checking for Haskell~\cite{xu+:contracts}
was based on {\em wrapping}. A term was effectively wrapped
with an appropriately nested contract test, and symbolic execution or 
aggressive inlining was used to show that @BAD@ values could
never be reached in this wrapped term.
In follow-up work, Xu~\cite{Xu:2012:HCC:2103746.2103767} proposes a variation, this time for a
{\em call-by-value} language, which performs symbolic execution along with
a ``logicization'' of the program that can be used (via a theorem prover)
to eliminate paths that can
provably not generate @BAD@ value,. The ``logicization'' of a
program has a similar spirit to our translation to logic.
Furthermore, the logicization of programs is dependent on whether
the resulting formula is going to be used as a goal or assumption in a proof. 
Improving Xu's work, Tobin-Hochstadt and Van Horn~\cite{hochstadt-horn} propose
a system in the same space of symbolic execution but they enrich the 
language of contracts to arbitrary potentially divergent or crashing functions.
We believe that the approach proposed in this paper, which is to directly 
encode the semantics of programs, might be simpler to specify and reason about.
That said, symbolic execution has the advantage of querying a
theorem prover on many small goals as symbolic execution proceeds, instead of a
single verification goal in the end. We have some ideas about how to break large
contract negation satisfiability queries to smaller ones, guided by the symbolic 
evaluation of a function, and we consider integrating this methodology in our tool.

Finally, higher-order model checking~\cite{koba-ppdp09,koba-popl09} aims at verifying
properties about the execution of functional programs (for instance it can also verify
temporal properties of programs) based on a translation of programs to higher-order 
recursion schemes that generate potential execution trees, and the use of algorithms 
that can decide properties on these trees. Higher-order model checking is an area of 
active research aiming to improve the efficiency and applicability of the original 
approach.

%% \begin{itemize}
%%   \item Contracts in general (Findler Felleisen etc)
%%   \item Xu's 2009
%%   \item Xu's PEPM 2012: Very related
%%   \item Minimization/finite models? Isabelle? (Jasmin's thesis?)
%%   \item Yann Regis-Giannas
%%   \item Xeno (equalities), Hipspec
%%   \item Higher-order model checking
%%   \item Triggers
%%   \item Our approach is reminiscent of appraches from the 80's/90's but which?
%%   \item Treatment of @BAD@ as in Extensible Extensions paper (maybe just a comment is neededed inline)
%%   \item More stuff that Koen knows about??????????
%% \end{itemize}


\section{Future work}\label{sect:future}
  \input{hcc-future}

\paragraph{Acknowledgements}
Thanks to Richard Eisenberg for helpful feedback and Nathan Collins
for early work on a prototype of our translation.

\bibliographystyle{plainnat}
\bibliography{hcc-popl}

\appendix

\section{A finite model construction}\label{sect:finite-model-proof}
\input{hcc-finite-model-proof}

\section{Min in terms of unreachable}

 In fact we may take
one step further and equate all the non-interesting values of the domain to $\bot$.

To achieve this effect, we update our Prelude theory axioms as follows:
{\small
\[\setlength{\arraycolsep}{1pt}
\begin{array}{c}
%% \ruleform{\Th{\Sigma}{P}} \\ \\
\begin{array}{lll}
 \textsc{AxDisjA} & \formula{\bad \neq \unr}  \\
 \textsc{AxDisjB} & \formula{\forall \oln{x}{n}\oln{y}{m} @.@} \\
                  & \formula{\;\;\highlight{K(\ol{x}){\neq}\unr\;\lor\;J(\ol{y}){\neq}\unr} =>
                                  K(\ol{x}){\neq}J(\ol{y})} \\
                  & \text{ for every } (K{:}\forall\as @.@ \oln{\tau}{n} -> T\;\as) \in \Sigma \\
                  & \text{ and } (J{:}\forall\as @.@ \oln{\tau}{m} -> S\;\as) \in \Sigma \\
 %% \textsc{AxDisjCUnr} & \formula{\forall \oln{x}{n} @.@ \highlight{\neg min(\unr)}} \\
 %%                  & \text{ for every } (K{:}\forall\as @.@ \oln{\tau}{n} -> T\;\as) \in \Sigma \\ \\
 \textsc{AxDisjCBad} & \formula{\forall \oln{x}{n} @.@ K(\ol{x}) \neq \bad} \\
                  & \text{ for every } (K{:}\forall\as @.@ \oln{\tau}{n} -> T\;\as) \in \Sigma \\ \\

 \textsc{AxAppA}  & \formula{\forall \oln{x}{n} @.@ f(\ol{x}) = app(f_{ptr},\xs)} \\
                  & \text{ for every } (f |-> \Lambda\as @.@ \lambda\oln{x{:}\tau}{n} @.@ u) \in P \\
 %% \textsc{AxAppB}  & \formula{\forall \oln{x}{n} @.@ K(\ol{x}) = app(\ldots (app(x_K,x_1),\ldots,x_n)\ldots)} \\
 %%                  & \text{ for every } (K{:}\forall\as @.@ \oln{\tau}{n} -> T\;\as) \in \Sigma \\
 \textsc{AxAppC}  & \formula{\forall x, app(\bad,x) = \bad \; /\ \; app(\unr,x) = \unr}    \\ \\
 %% Not needed: we can always extend partial constructor applications to fully saturated and use AxAppC and AxDisjC
 %% \textsc{AxPartA} & \formula{\forall \oln{x}{n} @.@ app(\ldots (app(x_K,x_1),\ldots,x_n)\ldots) \neq \unr} \\
 %%                  & \formula{\quad\quad \land\; app(\ldots (app(x_K,x_1),\ldots,x_n)\ldots) \neq \bad} \\
 %%                  & \text{ for every } (K{:}\forall\as @.@ \oln{\tau}{m} -> T\;\as) \in \Sigma \text{ and } m > n \\
 %% \textsc{AxPartB} & \formula{\forall \oln{x}{n} @.@ app(f_{ptr},\xs) \neq \unr} \\
 %%                  & \formula{\quad\land\; app(f_{ptr},\xs) \neq \bad} \\
 %%                  & \formula{\quad\land\; \forall \oln{y}{k} @.@ app(f_{ptr},\xs) \neq K(\ol{y})} \\
 %%                  & \text{ for every } (f |-> \Lambda\as @.@ \lambda\oln{x{:}\tau}{m} @.@ u) \in P  \\
 %%                  & \text{ and every } (K{:}\forall\as @.@ \oln{\tau}{k} -> T\;\as) \in \Sigma \text{ and } m > n  \\ \\
 \textsc{AxInj}   & \formula{\forall \oln{y}{n} @.@ \highlight{K(\ys) \neq \unr\;\land\; y_i \neq \unr}} \\
                  & \formula{\quad\qquad\qquad => \sel{K}{i}(K(\ys)) = y_i} \\
                  & \text{for every } (K{:}\forall\as @.@ \oln{\tau}{n} -> T\;\as) \in \Sigma \text{ and } i \in 1..n \\ \\
 \textsc{AxCfA}   & \formula{\lcf{\unr} /\ \lncf{\bad}} \\
 \textsc{AxCfB1}  & \formula{\forall \oln{x}{n} @.@ \bigwedge\lcf{\ol{x}}} => \lcf{K(\ol{x})} \\
                  & \text{ for every } (K{:}\forall\as @.@ \oln{\tau}{n} -> T\;\as) \in \Sigma \\
 \textsc{AxCfB2}  & \formula{\forall \oln{x}{n} @.@ \lcf{K(\ol{x})}\;\highlight{\land\;K(\ol{x}) \neq \unr} => \bigwedge\lcf{\ol{x}}} \\
                  & \text{ for every } (K{:}\forall\as @.@ \oln{\tau}{n} -> T\;\as) \in \Sigma
\end{array}
\end{array}\]}


\begin{figure}\small
\[\begin{array}{c}
\ruleform{\utrans{\Sigma}{\Gamma}{t \sim u} = \formula{\phi}} \\ \\
\prooftree
   \begin{array}{c} \ \\ \ \\
   \etrans{\Sigma}{\Gamma}{e} = \formula{t}
   \end{array}
   ----------------------------------------{TUTm}
   \begin{array}{l}
   \utrans{\Sigma}{\Gamma}{s \sim e } = \formula{(s = t) \lor \highlight{\neg min(s)}} \ \\ \ \\ \ \\
   \end{array}
   ~~~~~
  \begin{array}{l}
  \etrans{\Sigma}{\Gamma}{e} = \formula{t} \quad
  constrs(\Sigma,T) = \ol{K} \\
  \text{for each branch}\;(K\;\oln{y}{l} -> e') \\
  \begin{array}{l}
           (K{:}\forall \cs @.@ \oln{\sigma}{l} -> T\;\oln{c}{k}) \in \Sigma \text{ and }
           \etrans{\Sigma}{\Gamma,\ol{y}}{e'} = \formula{ t_K }
  \end{array}
  \end{array}
  ------------------------------------------{TUCase}
  {\setlength{\arraycolsep}{1pt}
  \begin{array}{l}
  \utrans{\Sigma}{\Gamma}{s \sim @case@\;e\;@of@\;\ol{K\;\ol{y} -> e'}} = \\
  \;\;\formula{ \begin{array}{l}
     \highlight{min(s)} => \\
     \begin{array}{ll}
          ( & \highlight{min(t)}\;\land \\
            & (t = \bad => s = \bad)\;\land \\
            & (\forall \ol{y} @.@ t = K_1(\ol{y}) => s = t_{K_1})\;\land \ldots \land \\
            & (t \neq \bad\;\land\;t \neq K_1(\oln{{\sel{K_1}{i}}(t)}{})\;\land\;\ldots => s = \unr) \\
          )
%% (t = \bad /\ s = \bad)\;\lor\;(s = \unr)\;\lor \\
%%                                 \quad      \bigvee(t = K(\oln{{\sel{K}{i}}(t)}{}) \land
%%                                            s = t_K[\oln{\sel{K}{i}(t)}{}/\ol{y}])
                   \end{array}
     \end{array}}
  \end{array}}
  %% {       \setlength{\arraycolsep}{2pt}
  %% \begin{array}{l}
  %% \utrans{\Sigma}{\Gamma}{s \sim @case@\;e\;@of@\;\ol{K\;\ol{y}{->}e'}} = \\
  %% \;\;\formula{
  %%      \begin{array}{l} (\highlight{s{=}\unr})\;\lor \\
  %%                           \;\; (\highlight{min(s) => min(t)}\;\land  \\
  %%                           \quad((t = \bad /\ s = \bad)\;\lor \\
  %%                           \quad\quad \bigvee(t = K(\oln{{\sel{K}{i}}(t)}{}) \land
  %%                                          s = t_K[\oln{\sel{K}{i}(t)}{}/\ol{y}])))
  %%                  \end{array}
  %%          }
  %% \end{array}}
\endprooftree
\end{array}\]
\caption{Minimality-enabled definition translation}\label{fig:min-def-trans-min}
\end{figure}



We will explain the modifications to the axiomatization in more detail in later sections.
%% In other words, we ensure that constructor applications are disjoint
%% only for values we are interested in. We will explain each axiom separately later.
%% Intuitively we wish to equate all terms that we are not interested in to $\unr$. We
%% can never be interested in $\unr$ in the intended model because that means that during
%% the evaluation of a term, which completed, we encountered a divergent term -- clearly a
%% contradiction!
What about function definitions? Figure~\ref{fig:etrans} has to be modified slightly as well,
as Figure~\ref{fig:min-def-trans} shows.

\begin{figure}\small
\[\begin{array}{c}
\ruleform{\utrans{\Sigma}{\Gamma}{t \sim u} = \formula{\phi}} \\ \\
\prooftree
   \begin{array}{c} \ \\ \ \\
   \etrans{\Sigma}{\Gamma}{e} = \formula{t}
   \end{array}
   ----------------------------------------{TUTm}
   \begin{array}{l}
   \utrans{\Sigma}{\Gamma}{s \sim e } = \formula{(s = t) \lor \highlight{s = \unr}} \ \\ \ \\ \ \\
   \end{array}
   ~~~~~
  \begin{array}{l}
  \etrans{\Sigma}{\Gamma}{e} = \formula{t} \quad
  constrs(\Sigma,T) = \ol{K} \\
  \text{for each branch}\;(K\;\oln{y}{l} -> e') \\
  \begin{array}{l}
           (K{:}\forall \cs @.@ \oln{\sigma}{l} -> T\;\oln{c}{k}) \in \Sigma \text{ and }
           \etrans{\Sigma}{\Gamma,\ol{y}}{e'} = \formula{ t_K }
  \end{array}
  \end{array}
  ------------------------------------------{TUCase}
  {\setlength{\arraycolsep}{1pt}
  \begin{array}{l}
  \utrans{\Sigma}{\Gamma}{s \sim @case@\;e\;@of@\;\ol{K\;\ol{y} -> e'}} = \\
  \;\;\formula{ \begin{array}{l}
     \highlight{s = \unr}\;\lor \\
     \begin{array}{ll}
          ( & \highlight{(t \neq \unr)}\;\land \\
            & (t = \bad => s = \bad)\;\land \\
            & (\forall \ol{y} @.@ t = K_1(\ol{y}) => s = t_{K_1})\;\land \ldots \land \\
            & (t = \bad\;\lor\;t = K_1(\oln{{\sel{K_1}{i}}(t)}{})\;\lor\;\ldots) \\
          )
%% (t = \bad /\ s = \bad)\;\lor\;(s = \unr)\;\lor \\
%%                                 \quad      \bigvee(t = K(\oln{{\sel{K}{i}}(t)}{}) \land
%%                                            s = t_K[\oln{\sel{K}{i}(t)}{}/\ol{y}])
                   \end{array}
     \end{array}}
  \end{array}}
  %% {       \setlength{\arraycolsep}{2pt}
  %% \begin{array}{l}
  %% \utrans{\Sigma}{\Gamma}{s \sim @case@\;e\;@of@\;\ol{K\;\ol{y}{->}e'}} = \\
  %% \;\;\formula{
  %%      \begin{array}{l} (\highlight{s{=}\unr})\;\lor \\
  %%                           \;\; (\highlight{min(s) => min(t)}\;\land  \\
  %%                           \quad((t = \bad /\ s = \bad)\;\lor \\
  %%                           \quad\quad \bigvee(t = K(\oln{{\sel{K}{i}}(t)}{}) \land
  %%                                          s = t_K[\oln{\sel{K}{i}(t)}{}/\ol{y}])))
  %%                  \end{array}
  %%          }
  %% \end{array}}
\endprooftree
\end{array}\]
\caption{Minimality-enabled definition translation}\label{fig:min-def-trans}
\end{figure}

%% \\ \\
%% \ruleform{ \Dtrans{\Sigma}{P} = \formula{\phi}} \\ \\
%% \prooftree
%%      \begin{array}{l}
%%        \text{for each} (f |-> \Lambda\oln{a}{n} @.@ \lambda\oln{x{:}\tau}{m} @.@ u) \in P \\
%%           \quad \utrans{\Sigma}{\ol{x}}{f(\ol{x}) \sim u} = \formula{\phi}
%%      \end{array}
%%      --------------------{TDefs}
%%      \Dtrans{\Sigma}{P} = \bigwedge_{P} \formula{\forall \ol{x} @.@ \phi}
%% \endprooftree

Now operationally we may instrument the evaluation relation to keep track of the set of
closed terms that appear during evaluation. The instrumented relation appears in
Figure~\ref{fig:opsem-instrumented}. Observe that if $P |- e \Downarrow w \curly S$ then
$S$ is a {\em finite set} of terms.


\begin{figure}\small
\[\begin{array}{c}
\ruleform{P |- e \Downarrow v \curly S} \\ \\
\prooftree
\begin{array}{c} \ \\
\end{array}
%% \begin{array}{c}
%% (f |-> \Lambda\ol{a} @.@ \lambda\oln{x{:}\tau}{m} @.@ u) \in P \\
%% P |- e_1 \Downarrow f\;[\taus]\;\oln{e}{m-1} \curly S_1 \\
%% P |- u[\ol{\tau}/\ol{a}][\ol{e},e_2/\ol{x}] \Downarrow w \curly S
%% \end{array}
%% ------------------------------------{EBeta}
%% P |- e_1\;e_2 \Downarrow w
  S = heads(v)
-------------------------------------{EVal}
P |- v \Downarrow v \curly S
~~~~
\begin{array}{c}
(f |-> \Lambda\ol{a} @.@ \lambda\oln{x{:}\tau}{m} @.@ u) \in P \\
P |- u[\ol{\tau}/\ol{a}][\ol{e}/\ol{x}] \Downarrow v \curly S_1 \\
S_2 = heads(f[\ol{\tau}]\;\oln{e}{m})
\end{array}
-------------------------------------{EFun}
P |- f[\ol{\tau}]\;\oln{e}{m} \Downarrow v \curly S_1 \cup S_2
~~~~~
\begin{array}{c}
P |- e_1 \Downarrow v_1 \curly S_1 \quad
P |- v_1\;e_2 \Downarrow w \curly S_2
\end{array}
------------------------------------------------{EApp}
P |- e_1\;e_2 \Downarrow w \curly S_1 \cup S_2 \cup \{ e_1\;e_2 \}
~~~~~
\begin{array}{c}
P |- e_1 \Downarrow @BAD@ \curly S
\end{array}
------------------------------------------------{EBadApp}
P |- e_1\;e_2 \Downarrow @BAD@ \curly S \cup \{ e_1\;e_2 \}
\endprooftree \\ \\
\ruleform{heads(e) = S} \\ \\
\begin{array}{lcl}
   heads(f\;[\ol{\tau}]) & = & \{ f\;[\ol{\tau}] \} \\
   heads(e_1\;e_2)       & = & \{ e_1\;e_2 \} \cup heads(e_1) \\
   heads(\_)            & = & \emptyset
\end{array} \\ \\
\ruleform{P |- u \Downarrow v \curly S} \\ \\
\prooftree
P |- e \Downarrow v \curly S
-------------------------------------{EUTm}
P |- e \Downarrow v \curly S
~~~~~
\begin{array}{c}
P |- e \Downarrow K_i[\ol{\sigma}_i](\ol{e}_i) \curly S_1 \quad
P |- e'_i[\ol{e}_i/\ol{y}_i] \Downarrow w \curly S_2
\end{array}
------------------------------------{ECase}
P |- @case@\;e\;@of@\;\ol{K\;\ol{y} -> e'} \Downarrow w \curly S_1 \cup S_2
~~~~~
\begin{array}{c}
P |- e \Downarrow @BAD@ \curly S \\
\end{array}
------------------------------------{EBadCase}
P |- @case@\;e\;@of@\;\ol{K\;\ol{y} -> e'} \Downarrow @BAD@ \curly S
%% \begin{array}{c}
%% (f |-> \Lambda\ol{a} @.@ \lambda\oln{x{:}\tau}{m} @.@ @case@\;e\;@of@\;\ol{K\;\ol{y} -> e'}) \in D \\
%% D |- e[\ol{\tau}/\ol{a}][\ol{e}/\ol{x}] \Downarrow @BAD@ \\
%% \end{array}
%% -------------------------------------{EBadCase}
%% D |- f[\ol{\tau}]\;\oln{e}{m} \Downarrow @BAD@
\endprooftree
\end{array}\]
\caption{Redex-instrumented operational semantics}\label{fig:opsem-instrumented}
\end{figure}

\section{The intended min-imal model}


Our goal is then going to be to establish the following result, stated in non-technical terms:
\begin{quote}
If there exists a counterexample to a contract, then the negation of the contract-translation
formula is satisfiable not only on $\langle D_\infty,{\cal I}\rangle$ but it also has a {\em finite}
model. That finite model is a model of our minimality-enabled theory.
\end{quote}

We start unfolding the story. For a given program $P$ in a signature $\Sigma$ we have already
shown how to construct $D_\infty$ and how to give interpretations ${\cal I}$ to a first-order
vocabulary. Let us assume that the program and signature contains a polymorpic $undefined$
function, for convenience $undefined |-> udefined$. This is a realistic assumption to make
(e.g. it comes in the standard Haskell prelude).

Assume now that we are given a formula $\phi$ defined as:
\[  \phi = \ctrans{\Sigma}{\cdot}{e \in \Ct_1 -> \ldots \Ct_n -> @B@} \]
for @B@ a base contract. Assume moreover that there exist $\oln{e}{n}$, closed for the
program $P$, such that for each $e_i$ it is true that:
\[\interp{\Ct_i}{\dbrace{P}^\infty}{\cdot}(\interp{e_i}{\dbrace{P}^\infty}{\cdot})\].
Assume however that it is {\em not} the case that
\[\interp{{\tt B}}{\dbrace{P}^\infty}{\cdot}(\interp{e\;\oln{e}{n}}{\dbrace{P}^\infty}{\cdot})\]
There are two cases for the base constract @B@:
\begin{itemize}
  \item Let us now consider the case where @B@ = $\{ x \mid e_p \}$. By adequacy it must
  be that: $P |- e\;\ol{e} \Downarrow w \curly S_1$ for some $w$ and set $S_1$ and moreover
  $P |- e_p[e\;ol{e}/x] \Downarrow \{ @BAD@, False \} \curly S_2$ for some set $S_2$.

  Of course the following lemma is true:
  \begin{lemma}\label{lem:curly}
    If $P |- e \Downarrow w \curly S$ then $S$ is a finite set. Moreover,
    for every $e' \in S$ there exists $w$ such that $P |- e' \Downarrow w$.
  \end{lemma}
  Moreover we have:
  \begin{lemma}\label{lem:bot-not-redex}
     If $P |- e \Downarrow w \curly S$ then
     $\bot \notin \interp{S}{\dbrace{P}^{\infty}}{\cdot}$.
  \end{lemma}
  \begin{proof} If $\bot \in \interp{S}{\dbrace{P}^{\infty}}{\cdot}$ then there exists
  a term $e \in S$ such that $\interp{e}{\dbrace{P}^{\infty}}{\cdot} = \bot$. This means
  that $P |- e \not\Downarrow$ but that is a contradiction to $e \in S$ by
  Lemma~\ref{lem:curly}.
  \end{proof}

  Let us now define the {\em minimal sets} operationally and denotationally:

  \[\begin{array}{lcl}
           M        & \triangleq & S_1 \cup S_2 \\
           {\cal M} & \triangleq & \interp{S_1\cup S_2}{\dbrace{P}^{\infty}}{\cdot}
  \end{array}\]
  Consider now the function $\mu : D_\infty -> D_\infty$ defined as:
  \[\begin{array}{lcl}
        \mu(d) & \triangleq & \left\{ \begin{array}{ll}
                   d           & \text{when } \unroll(d) = \ret(\inj{bad}(1)) \\
                   d           & \text{when } d \in \Min \\
                   \bot        & \text{otherwise }
                                      \end{array}\right.
  \end{array}\]
  In other words $\mu(\cdot)$ conflates all the non-interesting values to $\bot$.
  Now we may consider the {\em set} which is the image of $D_\infty$ through $\mu$:
  \[ D_\infty^\mu  \triangleq \mu(D_\infty) \]

  Notice that this set is {\em finite} with cardinality at most $card(M) + 2$. Also,
  we treat this is a {\em set}. Although $D_\infty$ has a domain structure, we do not
  care about $D_\infty^\mu$ being a domain.

  Now, in this $D_\infty^\mu$ we may redefine the interpretation of first-order constants
  and variable symbols in our theories, using ${\cal I}^\mu$ below:


  {\setlength{\arraycolsep}{2pt}
  \[\begin{array}{rcl}
     \mlinterp{f_{ptr}} & = & \mu(\dbrace{P}^{\infty}(f)) \\
 %% \roll(\ret(\inj{->}(\dlambda d_1 @.@ \ldots  \\
 %%                       &   & \quad \roll(\ret(\inj{->}(\dlambda d_n @.@ \\
 %%                       &   & \quad\quad\text{ if there exist } \oln{e}{n} \text{ s.t. } f[\taus]\;\ol{e} \in M \\
 %%                       &   & \quad\quad\quad\text{ and } \interp{e_i}{\dbrace{P}^\infty}{\cdot} = d_i\text{ then } \\
 %%                       &   & \quad\quad\quad\quad \mu(\dapp(\dbrace{P}^{\infty}(f),\oln{d}{n})) \\
 %%                       &   & \quad\quad\text{ else } \bot)))\ldots))) \\ \\
   \mlinterp{f^{n}}  & = & \dlambda (d {:} \prod_{n}D_{\infty}^\mu) @.@  \\
                       %% &   & \quad\quad\text{ if there exist } \oln{e}{n} \text{ s.t. } f[\taus]\;\ol{e} \in M \\
                       %% &   & \quad\quad\quad\text{ and } \interp{e_i}{\dbrace{P}^\infty}{\cdot} = \pi_i(d)\text{ then } \\
                       &   & \quad\quad (\mu\cdot\dapp)(\mu(\dbrace{P}^{\infty}(f)),\oln{\pi_i(d)}{i \in 1..n})) \\
                       %% &   & \quad\quad\text{ else } \bot \\ \\

   \mlinterp{app}     & = & \dlambda (d {:} D_{\infty}^\mu \times D_{\infty}^\mu) @.@ \\ \
                      &   & \quad\qquad \mu(\dapp(\pi_1(d),\pi_2(d))) \\ \\

   \mlinterp{K^{\ar}}     & = & \dlambda (d {:} \prod_{\ar}D_{\infty}^\mu) @.@ \mu(\roll(\ret(\inj{K}(d)))) \\
   \mlinterp{\sel{K}{i}} & = & \dlambda (d {:} D_{\infty}^{\mu}) @.@ \mu(\roll(\bind_g(\unroll(d)))) \\
     \text{where } g  & = & [\;\bot \\
                      &   & ,\;\dlambda d @.@ \unroll(\pi_i(d))  \quad (\text{case for constr. } K) \\
                      &   & ,\;\bot \\
                      &   & ,\;\ldots\\
                      &   & ,\;\bot\; ]
  \end{array}\]}

  Sadly, while the interpretation above is relatively simple, it does not validate the axiom
  for \textsc{TUCase}. The fact that the denotation of a function application may be in the minimal set,
  does not guarrantee that evaluation had proceeded along this function and hence the case scrutinee will
  be in the minimal set. This will be true only if we add an intentional test in the interpretation of
  functions that queries the set $M$. {\bf DV:TODO tomorrow}.

  {\bf DV: TODO: We need something like the definition below (but not quite, it does not type check yet)}:
  {\setlength{\arraycolsep}{2pt}
  \[\begin{array}{rcl}
     \mlinterp{f_{ptr}} & = & \mu(\roll(\ret(\inj{->}(\dlambda d_1 @.@ \ldots  \\
                       &   & \quad \mu(\roll(\ret(\inj{->}(\dlambda d_n @.@ \\
                       &   & \quad\quad\text{ if there exist } \oln{e}{n} \text{ s.t. } f[\taus]\;\ol{e} \in M \\
                       &   & \quad\quad\quad\text{ and } \interp{e_i}{\dbrace{P}^\infty}{\cdot} = d_i\text{ then } \\
                       &   & \quad\quad\quad\quad \mu(\dapp(\dbrace{P}^{\infty}(f),\oln{d}{n})) \\
                       &   & \quad\quad\text{ else } \bot))))\ldots)))) \\ \\
   \mlinterp{f^{n}}  & = & \dlambda (d {:} \prod_{n}D_{\infty}^\mu) @.@  \\
                       &   & \quad\quad\text{ if there exist } \oln{e}{n} \text{ s.t. } f[\taus]\;\ol{e} \in M \\
                       &   & \quad\quad\quad\text{ and } \interp{e_i}{\dbrace{P}^\infty}{\cdot} = \pi_i(d)\text{ then } \\
                       &   & \quad\quad\quad\quad \mu(\dapp(\dbrace{P}^{\infty}(f),\oln{\pi_i(d)}{i \in 1..n})) \\
                       &   & \quad\quad\text{ else } \bot \\ \\

   \mlinterp{app}     & = & \dlambda (d {:} D_{\infty}^\mu \times D_{\infty}^\mu) @.@ \\ \
                      &   & \quad\qquad \mu(\dapp(\pi_1(d),\pi_2(d))) \\ \\

   \mlinterp{K^{\ar}}     & = & \dlambda (d {:} \prod_{\ar}D_{\infty}^\mu) @.@ \mu(\roll(\ret(\inj{K}(d)))) \\
   \mlinterp{\sel{K}{i}} & = & \dlambda (d {:} D_{\infty}^{\mu}) @.@ \mu(\roll(\bind_g(\unroll(d)))) \\
     \text{where } g  & = & [\;\bot \\
                      &   & ,\;\dlambda d @.@ \unroll(\pi_i(d))  \quad (\text{case for constr. } K) \\
                      &   & ,\;\bot \\
                      &   & ,\;\ldots\\
                      &   & ,\;\bot\; ]
  \end{array}\]}

  \item The other case is when $@B@ = \CF$. {\bf TODO}
\end{itemize}

\end{document}

%% \begin{abstract}
%% The Glasgow Haskell Compiler is an optimizing
%% compiler that expresses and manipulates first-class equality proofs in
%% its intermediate language.  We describe a simple, elegant technique that
%% exploits these equality proofs to support \emph{deferred type errors}.
%% The technique requires us to treat equality proofs as possibly-divergent
%% terms; we show how to do so without losing either soundness or
%% the zero-overhead cost model that the programmer expects.
%% \end{abstract}

%% \category{D.3.3}{Language Constructs and Features}{Abstract data types}
%% \category{F.3.3}{Studies of Program Constructs}{Type structure}

%% \terms{Design, Languages}

%% \keywords{Type equalities, Deferred type errors, System FC}

%% \section{Denotational semantics}


%% \begin{lemma}[Evaluation preserves equality]
%% If $\Sigma;\cdot |- e : \tau \rightsquigarrow t$ and
%%    $\Sigma |- D \rightsquigarrow \phi_{\Sigma,D}$ and
%%    $D |- e \Downarrow w$ then
%%    $\Sigma;\cdot |- w : \tau \rightsquigarrow s$ and $\Th{\Sigma}{D} /\ \phi_{\Sigma,D} |- t = s$.
%% \end{lemma}
%% \begin{proof} By induction on the evaluation $\Sigma |- e \Downarrow w$. \end{proof}


%% \begin{lemma}[Logic deduces sound value equalities]
%% Assume that $\Sigma;\cdot |- w : \tau \rightsquigarrow t$ and
%% $D |- value(w)$ and $\Sigma |- D \rightsquigarrow \phi_{\Sigma,D}$.
%% Then
%% \begin{enumerate*}
%%   \item If $\Th{\Sigma}{D} /\ \phi_{\Sigma,D} |- t = \bad$ then $w = @BAD@$.
%%   \item If $\Th{\Sigma}{D} /\ \phi_{\Sigma,D} |- t = K(\ol{t})$ then $w = K[\taus](\ol{e})$, such
%%         that $\Sigma;\cdot |- \ol{e : \tau} \rightsquigarrow \ol{s}$, and $\Th{\Sigma}{D} /\ \phi_{\Sigma,D} |- \ol{t = s}$.
%%   \item $\Th{\Sigma}{D} /\ \phi_{\Sigma,D} |- t \neq \unr$.
%% \end{enumerate*}
%% \end{lemma}
%% \begin{proof}
%% The proof of all three cases is by inversion on the $D |- value(w)$ derivation,
%% apealling to the disjointness axioms.
%% %% \begin{enumerate*}
%% %%   \item By inversion on the $D |- value(w)$ derivation. In the case of \rulename{VBad} we are done.
%% %%   The case of \rulename{VFun} cannot happen, by the axiom set \rulename{AxPartB}. The case of \rulename{VCon}
%% %%   cannot happen either: If the application is saturated then \rulename{AxDisjC} shows it is impossible; if it
%% %%   is not saturated we can always extend it and use \rulename{AxAppC} and \rulename{AxDisjC}.
%% %%   \item Again by inversion on $D |- value(w)$ derivation. The case of \rulename{VBad} is easy. The case for
%% %%   \rulename{VCon} follows by injectivity of constructors. The case of \rulename{VFun} can't happen by
%% %%   \rulename{AxPartB}.
%% %%   \item Direct inversion on $D |- value(w)$, and using disjointness axioms.
%% %% \end{enumerate*}
%% \end{proof}

%% Basic soundness will be stated as follows.
%% \begin{theorem}
%% If we have that
%% \begin{enumerate*}
%%   \item $\Sigma;\cdot |- e : \tau$ and $\Sigma;\cdot |- \Ct : \tau$
%%   \item $\Sigma |- D \rightsquigarrow \phi_{\Sigma,D}$
%%   \item $\Sigma;\cdot |- e \in \Ct \rightsquigarrow \phi$
%% \end{enumerate*}
%% and $\Th{\Sigma}{D} /\ \phi_{\Sigma,D} /\ \neg \phi$ is unsatisfiable then $\Sigma;D |- e \in \Ct$.
%% \end{theorem}
%% \begin{proof}
%%  {\bf TODO}
%% \end{proof}

%% A remark: a formula $\phi$ is unsatisfiable iff $\neg \phi$ is valid in FOL. Hence, if
%% $\Th{\Sigma}{D} /\ \phi_{\Sigma,D} /\ \neg \phi$ is unsatisfiable then
%% $\neg (\Th{\Sigma}{D} /\ \phi_{\Sigma,D}) \lor \phi$ must be valid, and by completeness of FOL,
%% $\Th{\Sigma}{D} /\ \phi_{\Sigma,D} |- \phi$.

%% \section{Denotational semantics as FOL models}



%% \begin{figure}\small
%% \[\begin{array}{c}
%% %% \ruleform{ \dtrans{\Sigma}{d} = \formula{\phi} } \\ \\
%% %% \prooftree
%% %%   \begin{array}{c}
%% %%   (f{:}\forall\oln{a}{n} @.@ \oln{\tau}{m} -> \tau) \in \Sigma \quad
%% %%   \etrans{\Sigma}{\ol{a},\ol{x{:}\tau}}{e} = \formula{t}
%% %%   \end{array}
%% %%   -------------------------------------------------------------------{TFDef}
%% %%   \dtrans{\Sigma}{(f |-> \Lambda\oln{a}{n} @.@ \lambda\oln{x{:}\tau}{m} @.@ e)} =  \formula{ (\forall x @.@ f(\oln{x}{m}) = t) }
%% %%   ~~~~~
%% %%   \begin{array}{l}
%% %%   (f{:}\forall\oln{a}{n} @.@ \oln{\tau}{m} -> \tau) \in \Sigma \quad
%% %%   \etrans{\Sigma}{\ol{a},\ol{x{:}\tau}}{e} = \formula{t} \\
%% %%   constrs(\Sigma,T) = \ol{K} \\
%% %%   \text{for each branch}\;(K\;\oln{y}{l} -> e') \\
%% %%   \quad \begin{array}{l}
%% %%            (K{:}\forall \cs @.@ \oln{\sigma}{l} -> T\;\oln{c}{k}) \in \Sigma \\
%% %%            \etrans{\Sigma}{\ol{a},\ol{x{:}\tau},\ol{y{:}\sigma[\taus/\cs]}}{e'} = \formula{ t_K }
%% %%         \end{array}
%% %%   \end{array}
%% %%   -------------------------------------------------------------------{TCaseDef}
%% %%   \begin{array}{l}
%% %%    \dtrans{\Sigma}{(f |-> \Lambda\oln{a}{n} @.@ \lambda\oln{x{:}\tau}{m} @.@ @case@\;e\;@of@\;\ol{K\;\ol{y} -> e'})} = \\
%% %%    \quad \formula{ \begin{array}{lll} \forall \oln{x}{m} @.@ & \hspace{-7pt} (t = \bad /\ f(\ol{x}) = \bad)\; \lor \\
%% %%                                                                     & \hspace{-7pt}(f(\ol{x}) = \unr)\;\lor \\
%% %%                                                                     & \hspace{-7pt}(\bigvee(t = K(\oln{{\sel{K}{i}}(t)}{i\in 1..l})\;/\ \\
%% %%                                                                     & \hspace{-5pt}\quad f(\ol{x}) = t_K[\oln{\sel{K}{i}(t)}{i\in 1..l}/\ol{y}]))
%% %%                                                  \end{array}
%% %%                         }
%% %% \end{array}
%% %% \endprooftree  \\ \\
%% \end{array}\]
%% \caption{Definition elaboration to FOL}\label{fig:typing}
%% \end{figure}

%% {\bf DV: So basically this is Simon's strategy of side-stepping the
%% lack of full abstraction and the associated problems with it: In
%% the end of the day we only care about base contracts, in fact
%% really only about the contract ``is this program crash-free'', so
%% we don't have to make a big fuss about higher-order contracts and
%% their operational semantics. We have to motivate it carefully and
%% also be clear that for the intellectually curious reader who really
%% wants to know what statement we have proved for a function contract
%% when the prover says ``unsat'' we might want to give a full
%% definition of the denotational meaning of contracts including both
%% base and higher-order. I think we do not have the time luxury to
%% look for more elaborate solutions (such as definable denotations
%% and all that crazy stuff) to match the operational and the
%% denotational semantics for higher-order contracts. Fullstop.}

%% \section{Minimizing countermodels}




%% \newpage

%% \section{Contract checking soundness}

%% \section{Contracts}

%% The syntax that we use for contracts is in Figure~\ref{fig:contract-syntax}.
%% Contracts are typed (here, just monomorphically), and we give an operational
%% semantics for contract satisfaction in the same figure.

%% \begin{figure*}\small
%% \[\begin{array}{c}
%% \ruleform{\Sigma;\Gamma |- \Ct } \\ \\
%% \prooftree
%% \Sigma;\Delta,x{:}\tau |- e : \Bool
%% ---------------------------------------{TCBase}
%% \Sigma;\Delta |- \{ (x{:}\tau) \mid e \} : \tau
%% ~~~~
%% \begin{array}{c}
%% \Sigma;\Delta |- \Ct_1 : \tau \\
%% \Sigma;\Delta,(x{:}\tau) |- \Ct_2 : \tau'
%% \end{array}
%% ---------------------------------------{TCArr}
%% \Sigma;\Delta |- (x{:}\Ct_1) -> \Ct_2 : \tau -> \tau'
%% ~~~~
%% \Sigma;\Delta |- \Ct_1 : \tau \quad \Sigma;\Delta |- \Ct_2 : \tau
%% ---------------------------------------{TCConj}
%% \Sigma;\Delta |- \Ct_1 \& \Ct_2 : \tau
%% ~~~~
%% \phantom{\Gamma}
%% ---------------------------------------{TCf}
%% \Sigma;\Delta |- \CF : \tau
%% \endprooftree \\ \\
%% \ruleform{\Sigma;P |- e \in \Ct} \\ \\
%% \prooftree
%%  P \not|- e \Downarrow
%% -----------------------------------------------{ECDiv}
%%  \Sigma;P |- e \in \{ (x{:}\tau) \mid e' \}
%%  ~~~~
%%  P |- e'[e/x] \Downarrow \True
%% -------------------------------------------{ECTrue}
%%  \Sigma;P |- e \in \{ (x{:}\tau) \mid e' \}
%%  ~~~~
%%  P \not|- e'[e/x] \Downarrow
%%  ------------------------------------------{ECCDiv}
%%  \Sigma;P |- e \in \{ (x{:}\tau) \mid e' \}
%%  ~~~~~
%%  \begin{array}{c}
%%  \Sigma;\cdot |- \Ct_1 : \tau \\
%%  \text{for all } u, \Sigma;\cdot |- u : \tau ==> \Sigma;P |- e\;u \in \Ct_2[u/x]
%%  \end{array}
%%  --------------------------------------------{ECArr}
%%  \Sigma;P |- e \in (x{:}\Ct_1) -> \Ct_2
%%  ~~~~
%%  \begin{array}{c}
%%  \Sigma,\cdot |- e : \tau  \\
%%  e \in \Ecf \quad \text{(See Section~\ref{sect:cf})}
%%  %% \text{for all } u, (\Sigma;\cdot |- u : \tau -> \Bool) /\ (@BAD@ \notin u) ==> \neg (P |- u\;e \Downarrow @BAD@)
%%  \end{array}
%%  --------------------------------------------------------------------------------------------{ECf}
%%  \Sigma;P |- e \in \CF
%%  ~~~~~
%%  \Sigma;P |- e \in \Ct_1 \quad \Sigma;P |- e \in \Ct_2
%%  --------------------------------------------------------------------------------------------{ECConj}
%%  \Sigma;P |- e \in \Ct_1 \& \Ct_2
%% \endprooftree
%% \end{array}\]
%% \caption{Contract syntax and semantics}\label{fig:contract-syntax}
%% \end{figure*}

%% \section{Induction and admissibility}
%% {\bf TODO}


%% \section{Minimization}
%% {\bf TODO}

%% \section{Some ideas}
%% Sometimes the $\CF$ contract stands in our way e.g. for library stuff. It might
%% be interesting to explore some user-defined pragma to side-step the $\bad$ case
%% in some pattern matches (i.e. make it on demand, pretty much as $F^{\star}$ does, where
%% only the user's assertions matter.
%% %% \acks
%% %% Acknowledgements here

