There are very few practical tools for the automatic
verification of arbitrary {\em lazy and higher-order} functional programs, though 
the automated verification of higher-order programs, at least for restricted 
strongly-normalizing languages, has been studied before, for instance by the ACL2 
community. Furthermore, our approach of directly translating the denotational 
semantics of programs does not appear to be well-explored in the literature.

Catch~\cite{Mitchell:2008:PBE:1411286.1411293} is one of the very few tools that
address the verification of lazy Haskell, and have been evaluated on real programs.
Using static analysis, Catch can detect pattern match failures, and hence
prove that a program cannot crash. Some annotations may be necessary for the 
analysis to succeed. Our aim in this paper is to achieve similar goals, and
moreover to be in a position to assert functional correctness.

Liquid Types~\cite{Rondon:2008:LT:1375581.1375602} is an influential
approach to call-by-value functional program verification. 
Contracts are written as refinements in a fixed language of predicates (which may
include recursive predicates) and the extracted conditions are discharged using an
SMT-solver. Because the language of predicates is fixed, predicate abstraction can
very effectively {\em infer} precise refinements, even for recursive functions, and
hence the annotation burden is very low. In our case, since the language of predicates
is, {\em by design}, the very same programming language with the same semantics, inference
of function specifications is harder. The other important difference is that liquid types
requires all {\em uses} of a function to satisfy its precondition, whereas in the semantics
that we have chosen, bad uses are allowed but the programmer gets no guarantees back.
\dv{Todo: Andrey Rybalchenko ``sausage factory''}

Rather different to Liquid Types, Dminor~\cite{Bierman+:subtyping} allows 
refinements to be written in the very same programming language.
Contrary to our case however, in Dminor
the expressions that refine types must be pure --- that is, terminating --- and have a unique
denotation (e.g. not depending on the store)\dr{what is the store?}. 
Driven from a typing relation that includes
logic entailment judgements, verification conditions are extracted and discharged automatically using Z3.
Similar in spirit, the Fstar~\cite{fstar} compiler also extracts verification conditions 
that are discharged using Z3 or interactive theorem provers. Hybrid type systems such as 
Sage~\cite{Knowles+:sage} attempt to prove as many of the goals statically, and defer the 
rest as runtime tests.

Boogie~\cite{boogie} is a verification back end that supports procedures as well as
pure functions.  By using Z3, Boogie verifies
programs written in the BoogiePL intermediate language, 
which could potentially be used as the
back end of our translation as well. 
Recent work on performing induction on top of an
induction-free SMT solver proposes a ``tactic''
for encoding induction schemes as first-order queries, which is reminiscent of the way
that we perform induction \cite{Leino:2012:AIS:2189257.2189278}.

The recent work on the Leon system~\cite{Suter:2011:SMR:2041552.2041575} presents
an approach to the verification of {\em first-order} and {\em call-by-value}
recursive functional programs, which appears to be very efficient in practice.  It works
by extending SMT with recursive programs and ``control literals'' that guide the pattern
matching search for a counter-model, and is guaranteed to find a model if one exists
(whereas that is not yet the case in our system, as we discussed earlier). It
does not include a special treatment of the $\bot$ value nor pattern match failures 
seem to be in the scope of that project, but rather partial functional correctness.

The tool Zeno~\cite{zeno} verifies equational properties of functional
programs. Its proof search is based on induction, equality reasoning and 
operational semantics. While guaranteeing termination, it can also start 
new induction proofs driven by syntactic heuristics. However, it only considers 
finite and total subsets of values, and we want to reason about Haskell
programs as they appear in the wild: possibly non-terminating, with
lazy infinite values, and run time crashes.

% Another tool that proves equational
% properties of Haskell programs under the same assumptions is
% HipSpec~\cite{hipspec} but
% But if we mention HipSpec, we should mention Hip. Then what about Hip?

First-order logic has been used as a target for higher-order languages
in other verification contexts as well.  Users of the interactive
theorem prover Isabelle have for many years had the opportunity to use
automated first-order provers to discharge proof obligations. This
work has recently culminated in the tool
Sledgehammer \cite{Sledgehammer}, which not only uses first-order
provers, but also SMT solvers as back ends.  There has also been a version
of the dependently typed programming language Agda in which proof
obligations could be sent to an automatic first-order
prover \cite{AgdaFOL}. Both of these use a translation from a
typed higher-order language of well-founded definitions to first-order
logic. A work that comes very close to ours, in
that they deal with a lazy, general recursive language with partial
functions, is by \citet{TypeTheoryFOL}, who use Agda as a logical
framework to reason about general recursive functional programs, and
combine interaction in Agda with automated proofs in FOL.

There exists more work on translating the semantics of programs or their properties in 
higher-order logics (e.g. CIC), for instance the work on Characteristic Formulae~\cite{char-form} 
and the work on Hoare-logic VCC-based verification~\cite{regis-gianas-pottier-08}. The former
is interpreting a program as a higher-order predicate transformer
whereas the latter introduces a higher-order typed Hoare-logic and
extracts verification conditions which are proved interactively in Coq or with 
an automated theorem prover. We aim to stay within the realm of
first-order logic to exploit automation offered by the advances in FOL
theorem provers and model finders. At the same time we do lose
expressivity, as our predicate language are only plain-old Haskell
functions. On the other hand this gives us
admissibility practically for free even in the absense of inductive
types. But even in the case of the previous work of
R\'{e}gis-Gianas and Pottier, positivity conditions have to be imposed on
datatypes to make sure that the logical specification can be embedded
in a logic built in CIC. Like Liquid types, that work enforces contract
preconditions in all call sites. Finally, in the higher-order logic world, the 
recent work of Huffman~\cite{Huffman:2012:FVM:2364527.2364532} can be used to 
translate Haskell programs and reason about their semantics in HOLCF; the main 
application being the verification of monad transformers.

The previous work on static contract checking for Haskell~\cite{xu+:contracts}
was based on {\em wrapping}. A term was effectively wrapped
with an appropriately nested contract test, and symbolic execution or 
aggressive inlining was used to show that @BAD@ values could
never be reached in this wrapped term.
In follow-up work, Xu~\cite{Xu:2012:HCC:2103746.2103767} proposes a variation, this time for a
{\em call-by-value} language, which performs symbolic execution along with
a ``logicization'' of the program that can be used (via a theorem prover)
to eliminate paths that can
provably not generate @BAD@ value,. The ``logicization'' of a
program has a similar spirit to our translation to logic.
Furthermore, the logicization of programs is dependent on whether
the resulting formula is going to be used as a goal or assumption in a proof. 
Improving Xu's work, Tobin-Hochstadt and Van Horn~\cite{hochstadt-horn} propose
a system in the same space of symbolic execution but they enrich the 
language of contracts to handle arbitrary, potentially divergent or crashing functions.
We believe that our approach, which is to directly 
encode the semantics of programs, might be simpler to specify and reason about.
That said, symbolic execution has the advantage of querying a
theorem prover on many small goals, instead of a
single big goal. We have some ideas about how to break large
satisfiability queries to smaller ones, guided by the symbolic 
evaluation of functions, and we consider integrating this methodology.

Finally, higher-order model checking~\cite{koba-ppdp09,koba-popl09} aims at verifying
properties about the execution of functional programs (for instance it can also verify
temporal properties of programs) based on a translation of programs to higher-order 
recursion schemes. Higher-order model checking is an area of active research, aiming to 
improve the efficiency and applicability of the original approach.

%% \begin{itemize}
%%   \item Contracts in general (Findler Felleisen etc)
%%   \item Xu's 2009
%%   \item Xu's PEPM 2012: Very related
%%   \item Minimization/finite models? Isabelle? (Jasmin's thesis?)
%%   \item Yann Regis-Giannas
%%   \item Xeno (equalities), Hipspec
%%   \item Higher-order model checking
%%   \item Triggers
%%   \item Our approach is reminiscent of appraches from the 80's/90's but which?
%%   \item Treatment of @BAD@ as in Extensible Extensions paper (maybe just a comment is neededed inline)
%%   \item More stuff that Koen knows about??????????
%% \end{itemize}
