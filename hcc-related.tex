There are very few practical tools for the automatic
verification of {\em lazy and higher-order} functional programs.
Furthermore, our approach of directly translating the denotational semantics of
programs does not appear to be well-explored in the literature.

Catch~\cite{Mitchell:2008:PBE:1411286.1411293} is one of the very few tools that
have been evaluated at scale, and address the verification of lazy Haskell
programs. Using static analysis, Catch can detect pattern match failures, and hence
prove that a program cannot crash. Some specification which describes the set of
constructors that are expected as arguments to each function might be required.
necessary for the analysis to succeed. Our aim with this work is to achieve similar and
moreover be in a position to also assert functional correctness specification.

Liquid Types~\cite{Rondon:2008:LT:1375581.1375602} has been an influential
approach to call-by-value functional program verification. In Liquid Types,
contracts are written as refinements in a fixed language of predicates (which may
include recursive predicates) and the extracted conditions are discharged using an
SMT-solver. Because the language of predicates is fixed, predicate abstraction can
very effectively {\em infer} precise refinements, even for recursive functions, and
hence the annotation burden is very low. In our case, since the language of predicates
is the very same programming language with the same semantics {\em by design}, inference
of function specifications is harder. The other important deviation is that liquid types
requires all {\em uses} of a function to satisfy its precondition whereas in the semantics
that we have chosen, bad uses are allowed but the programmer gets no guarantees back.
\dv{Todo: Andrey Rybalchenko ``sausage factory''}

Rather different to Liquid Types, allowing refinements to be written
in the very same programming language that programs are written, is the
Dminor~\cite{Bierman+:subtyping} approach. Contrary to our case however, in Dminor
the expressions that refine type must be pure, that is, terminating and having a unique
denotation (e.g. not dependent on the store). Driven from a typing relation that includes
logic entailment judgements, verification conditions are extracted and discharged automatically using Z3.
Similar in spirit, other dependent type systems such
as Fstar~\cite{fstar} extract verification conditions which can be discharged
using automated tools
or interactive theorem provers. Hybrid type systems such as Sage~\cite{Knowles+:sage}
will also attempt to prove as many of the goals statically and defer the rest as runtime
goals.

Boogie~\cite{boogie} is a verification back end which supports procedures as well as
pure functions, and axiomatisation of theories and could potentially be used as the
back end of our translation as well. Boogie verify programs written in the BoogiePL
programming languages using Z3. Recent work on performing induction on-top of an
induction-free SMT solver~\cite{Leino:2012:AIS:2189257.2189278} proposes a ``tactic''
for encoding induction schemes as first-order queries, which is reminiscent of the way
that we propose to perform induction.

The recent work on the Leon system~\cite{Suter:2011:SMR:2041552.2041575} presents
an effective approach to the verification of {\em first-order} and {\em call-by-value}
recursive functional programs which appears to be very efficient in practice: it works
by extending SMT with recursive programs and ``control literals'' that guide the pattern
matching search for a counter-model, and is guaranteed to find a model if one exists
(whereas that is not the case in our system, as we discussed earlier). It does not include
a $\CF$-analogous predicate and is call-by-value, though no special treatment of the $\bot$
value nor pattern match failures seem to be in the scope of that project, which, unsurprisingly
yields a very fast verification framework for partial functional correctness.

The tool Zeno~\cite{zeno} verifies equational properties of functional
programs using Haskell as a front end. Its proof search is based on
induction, equality reasoning and operational semantics. And while
guaranteeing termination, it can also start new induction proofs
driven by syntactic heuristics. It only considers the finite and total
subset of values, however, while we want to reason about Haskell
programs as they appear in the wild, possibly non-terminating, with
lazy infinite values and run time crashes.

% Another tool that proves equational
% properties of Haskell programs under the same assumptions is
% HipSpec~\cite{hipspec} but
% But if we mention HipSpec, we should mention Hip. Then what about Hip?

First-order logic has been used as a target for higher-order languages in other verification contexts as well.
The interactive theorem prover Isabelle for many years has had the opportunity to use
automated first-order provers to discharge proof obligations. This work has recently culminated into the tool
Sledgehammer \cite{Sledgehammer}, which not only uses first-order provers, but also SMT solvers as back-ends.
There has been a version of the dependently typed programming language Agda in which
proof obligations could be sent to an automatic first-order prover \cite{AgdaFOL}. Both of these use typed translations from a typed higher-order language of well-founded definitions to first-order logic. The work in this area that perhaps comes closest to ours in that they deal with a lazy, general recursive language with partial functions is by \citet{TypeTheoryFOL}, who use Agda as a logical framework to reason about general recursive functional programs, and combine interaction in Agda with automated proofs in first-order logic.

The previous work on static contract checking for Haskell~\cite{xu+:contracts}
was based on {\em wrapping}. A term was effectively wrapped
with an appropriately pushed contract test, and symbolic execution or aggressive inlining was used to show that @BAD@ values could
never be reached in this wrapped term.
In follow-up work, Xu~\cite{Xu:2012:HCC:2103746.2103767} proposes a variation for a
{\em call-by-value} language this time which performs symbolic execution alongside with
keeping a ``logicization'' of the program which can be used to eliminate paths that can
provably not generate @BAD@ value using a theorem prover. The ``logicization'' of a
program has a similar spirit to our translation to logic but it is unclear what is the
model that is being used to prove the soundness of this translation, nor its
axiomatisation. Furthermore the logicization of programs is dependent on if
the resulting formula is going to be used as a goal or assumption in a proof. We believe
that the direct approach proposed in this paper, which is to directly encode the semantics
of programs and contracts might be simpler. That said, symbolic execution as proposed
in~\cite{Xu:2012:HCC:2103746.2103767} has the significant advantage of querying a
theorem prover on many small goals as symbolic execution proceeds, instead of a
single verification goal in the end. We have some ideas about how to break large
contract negation queries to smaller ones, following the symbolic evaluation of
a function, and we plan to integrate this methodology in our tool as well.

%% \begin{itemize}
%%   \item Contracts in general (Findler Felleisen etc)
%%   \item Xu's 2009
%%   \item Xu's PEPM 2012: Very related
%%   \item Minimization/finite models? Isabelle? (Jasmin's thesis?)
%%   \item Yann Regis-Giannas
%%   \item Xeno (equalities), Hipspec
%%   \item Higher-order model checking
%%   \item Triggers
%%   \item Our approach is reminiscent of appraches from the 80's/90's but which?
%%   \item Treatment of @BAD@ as in Extensible Extensions paper (maybe just a comment is neededed inline)
%%   \item More stuff that Koen knows about??????????
%% \end{itemize}
