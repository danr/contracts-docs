
The inliner inlines non-recursive, non-casing functions, when inlining
the function would not insert a new lambda.  When dealing with
contracts, it allows inserting a lambda after the @:->@ because it uses
higher-order abstract syntax.

By using the inliner, we can now without hassle use GHC's
optimiser. It is now on by default. It chops up contracts into several
top-level functions, but the inliner then restores the structure by
making it one big definition. Before, the translator would carry
around a context and buggily inline sometimes, but now this is
separated.

When using optimisation, we also get rid of the nasty bug when an expression
is cased on twice, with default branches. Without optimisation, you can get
unsoundness with functions pattern-matching on two arguments.

We have two uses of inlining so far:

\paragraph{Reassembling statements} When using optimisation and the
lambda lifter, the contracts get splitted up into several entries. For
example, a contract like @map_cf = map ::: (CF --> CF) --> CF --> CF@
might look like this in the optimised GHC core:

\begin{code}
map_cf = (:::) map a3_rke

a3_rke = (-->) a3_rp7 a3_rzt

a3_rp7 = (-->) CF CF

a3_rzt = (-->) CF CF
\end{code}

We could translate these definitions to our internal representation of
contracts, carrying around the right hand sides of expressions, and
this is what we did before. This has a few complications. For one,
it might not look as regular as above: sometimes we get a redefinition
of @(:->)@ to some new variable name and the implementation got really
messy. Further, we want to know what functions are used in a contract,
including inside @Pred@, and this is just one function call if the
statement is reassembled, rather than trying to figure this out while
walking around definitions.

\paragraph{Destroying sharing}
When translating function definitions, we don't want to have
unnecessary function around. An especially annoying pattern is:

\begin{code}
foo x = case x of
    K1 -> lvl_fk4
    K2 y -> foo y

lvl_fk4 = BAD
\end{code}

With endless variantions, where @lvl_fk4@ can get a real-world token,
and so on. Introducing these extra symbols give an extra overhead for
theorem provers.

A function that is neither recursive nor using case is a good
candidate for removal.

Such CAFs are lifted for sharing, but we should try to destroy all
sharing since everything is memoized in theorem provers anyway. So we
don't want to see:

\begin{code}
zero = Zero
one = Succ zero
two = Succ one
\end{code}

Such things should always be inlined to not introduce new constants -
so whenever @two@ is found in the GHC Core, it's replaced with
@Succ (Succ Zero)@.
There are examples where for instance @eprover@ behaves a lot slower
when this inlining is not done.
