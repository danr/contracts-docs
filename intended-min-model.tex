

Our goal is then going to be to establish the following result, stated in non-technical terms:
\begin{quote}
If there exists a counterexample to a contract, then the negation of the contract-translation
formula is satisfiable not only on $\langle D_\infty,{\cal I}\rangle$ but it also has a {\em finite}
model. That finite model is a model of our minimality-enabled theory.
\end{quote}

We start unfolding the story. For a given program $P$ in a signature $\Sigma$ we have already
shown how to construct $D_\infty$ and how to give interpretations ${\cal I}$ to a first-order
vocabulary. Let us assume that the program and signature contains a polymorpic $undefined$
function, for convenience $undefined |-> udefined$. This is a realistic assumption to make
(e.g. it comes in the standard Haskell prelude).

Assume now that we are given a formula $\phi$ defined as:
\[  \phi = \ctrans{\Sigma}{\cdot}{e \in \Ct_1 -> \ldots \Ct_n -> @B@} \]
for @B@ a base contract. Assume moreover that there exist $\oln{e}{n}$, closed for the
program $P$, such that for each $e_i$ it is true that:
\[\interp{\Ct_i}{\dbrace{P}^\infty}{\cdot}(\interp{e_i}{\dbrace{P}^\infty}{\cdot})\].
Assume however that it is {\em not} the case that
\[\interp{{\tt B}}{\dbrace{P}^\infty}{\cdot}(\interp{e\;\oln{e}{n}}{\dbrace{P}^\infty}{\cdot})\]
There are two cases for the base constract @B@:
\begin{itemize}
  \item Let us now consider the case where @B@ = $\{ x \mid e_p \}$. By adequacy it must
  be that: $P |- e\;\ol{e} \Downarrow w \curly S_1$ for some $w$ and set $S_1$ and moreover
  $P |- e_p[e\;ol{e}/x] \Downarrow \{ @BAD@, False \} \curly S_2$ for some set $S_2$.

  Of course the following lemma is true:
  \begin{lemma}\label{lem:curly}
    If $P |- e \Downarrow w \curly S$ then $S$ is a finite set. Moreover,
    for every $e' \in S$ there exists $w$ such that $P |- e' \Downarrow w$.
  \end{lemma}
  Moreover we have:
  \begin{lemma}\label{lem:bot-not-redex}
     If $P |- e \Downarrow w \curly S$ then
     $\bot \notin \interp{S}{\dbrace{P}^{\infty}}{\cdot}$.
  \end{lemma}
  \begin{proof} If $\bot \in \interp{S}{\dbrace{P}^{\infty}}{\cdot}$ then there exists
  a term $e \in S$ such that $\interp{e}{\dbrace{P}^{\infty}}{\cdot} = \bot$. This means
  that $P |- e \not\Downarrow$ but that is a contradiction to $e \in S$ by
  Lemma~\ref{lem:curly}.
  \end{proof}

  Let us now define the {\em minimal sets} operationally and denotationally:

  \[\begin{array}{lcl}
           M        & \triangleq & S_1 \cup S_2 \\
           {\cal M} & \triangleq & \interp{S_1\cup S_2}{\dbrace{P}^{\infty}}{\cdot}
  \end{array}\]
  Consider now the function $\mu : D_\infty -> D_\infty$ defined as:
  \[\begin{array}{lcl}
        \mu(d) & \triangleq & \left\{ \begin{array}{ll}
                   d           & \text{when } \unroll(d) = \ret(\inj{bad}(1)) \\
                   d           & \text{when } d \in \Min \\
                   \bot        & \text{otherwise }
                                      \end{array}\right.
  \end{array}\]
  In other words $\mu(\cdot)$ conflates all the non-interesting values to $\bot$.
  Now we may consider the {\em set} which is the image of $D_\infty$ through $\mu$:
  \[ D_\infty^\mu  \triangleq \mu(D_\infty) \]

  Notice that this set is {\em finite} with cardinality at most $card(M) + 2$. Also,
  we treat this is a {\em set}. Although $D_\infty$ has a domain structure, we do not
  care about $D_\infty^\mu$ being a domain.

  Now, in this $D_\infty^\mu$ we may redefine the interpretation of first-order constants
  and variable symbols in our theories, using ${\cal I}^\mu$ below:


  {\setlength{\arraycolsep}{2pt}
  \[\begin{array}{rcl}
     \mlinterp{f_{ptr}} & = & \mu(\dbrace{P}^{\infty}(f)) \\
 %% \roll(\ret(\inj{->}(\dlambda d_1 @.@ \ldots  \\
 %%                       &   & \quad \roll(\ret(\inj{->}(\dlambda d_n @.@ \\
 %%                       &   & \quad\quad\text{ if there exist } \oln{e}{n} \text{ s.t. } f[\taus]\;\ol{e} \in M \\
 %%                       &   & \quad\quad\quad\text{ and } \interp{e_i}{\dbrace{P}^\infty}{\cdot} = d_i\text{ then } \\
 %%                       &   & \quad\quad\quad\quad \mu(\dapp(\dbrace{P}^{\infty}(f),\oln{d}{n})) \\
 %%                       &   & \quad\quad\text{ else } \bot)))\ldots))) \\ \\
   \mlinterp{f^{n}}  & = & \dlambda (d {:} \prod_{n}D_{\infty}^\mu) @.@  \\
                       %% &   & \quad\quad\text{ if there exist } \oln{e}{n} \text{ s.t. } f[\taus]\;\ol{e} \in M \\
                       %% &   & \quad\quad\quad\text{ and } \interp{e_i}{\dbrace{P}^\infty}{\cdot} = \pi_i(d)\text{ then } \\
                       &   & \quad\quad (\mu\cdot\dapp)(\mu(\dbrace{P}^{\infty}(f)),\oln{\pi_i(d)}{i \in 1..n})) \\
                       %% &   & \quad\quad\text{ else } \bot \\ \\

   \mlinterp{app}     & = & \dlambda (d {:} D_{\infty}^\mu \times D_{\infty}^\mu) @.@ \\ \
                      &   & \quad\qquad \mu(\dapp(\pi_1(d),\pi_2(d))) \\ \\

   \mlinterp{K^{\ar}}     & = & \dlambda (d {:} \prod_{\ar}D_{\infty}^\mu) @.@ \mu(\roll(\ret(\inj{K}(d)))) \\
   \mlinterp{\sel{K}{i}} & = & \dlambda (d {:} D_{\infty}^{\mu}) @.@ \mu(\roll(\bind_g(\unroll(d)))) \\
     \text{where } g  & = & [\;\bot \\
                      &   & ,\;\dlambda d @.@ \unroll(\pi_i(d))  \quad (\text{case for constr. } K) \\
                      &   & ,\;\bot \\
                      &   & ,\;\ldots\\
                      &   & ,\;\bot\; ]
  \end{array}\]}

  Sadly, while the interpretation above is relatively simple, it does not validate the axiom
  for \textsc{TUCase}. The fact that the denotation of a function application may be in the minimal set,
  does not guarrantee that evaluation had proceeded along this function and hence the case scrutinee will
  be in the minimal set. This will be true only if we add an intentional test in the interpretation of
  functions that queries the set $M$. {\bf DV:TODO tomorrow}.

  {\bf DV: TODO: We need something like the definition below (but not quite, it does not type check yet)}:
  {\setlength{\arraycolsep}{2pt}
  \[\begin{array}{rcl}
     \mlinterp{f_{ptr}} & = & \mu(\roll(\ret(\inj{->}(\dlambda d_1 @.@ \ldots  \\
                       &   & \quad \mu(\roll(\ret(\inj{->}(\dlambda d_n @.@ \\
                       &   & \quad\quad\text{ if there exist } \oln{e}{n} \text{ s.t. } f[\taus]\;\ol{e} \in M \\
                       &   & \quad\quad\quad\text{ and } \interp{e_i}{\dbrace{P}^\infty}{\cdot} = d_i\text{ then } \\
                       &   & \quad\quad\quad\quad \mu(\dapp(\dbrace{P}^{\infty}(f),\oln{d}{n})) \\
                       &   & \quad\quad\text{ else } \bot))))\ldots)))) \\ \\
   \mlinterp{f^{n}}  & = & \dlambda (d {:} \prod_{n}D_{\infty}^\mu) @.@  \\
                       &   & \quad\quad\text{ if there exist } \oln{e}{n} \text{ s.t. } f[\taus]\;\ol{e} \in M \\
                       &   & \quad\quad\quad\text{ and } \interp{e_i}{\dbrace{P}^\infty}{\cdot} = \pi_i(d)\text{ then } \\
                       &   & \quad\quad\quad\quad \mu(\dapp(\dbrace{P}^{\infty}(f),\oln{\pi_i(d)}{i \in 1..n})) \\
                       &   & \quad\quad\text{ else } \bot \\ \\

   \mlinterp{app}     & = & \dlambda (d {:} D_{\infty}^\mu \times D_{\infty}^\mu) @.@ \\ \
                      &   & \quad\qquad \mu(\dapp(\pi_1(d),\pi_2(d))) \\ \\

   \mlinterp{K^{\ar}}     & = & \dlambda (d {:} \prod_{\ar}D_{\infty}^\mu) @.@ \mu(\roll(\ret(\inj{K}(d)))) \\
   \mlinterp{\sel{K}{i}} & = & \dlambda (d {:} D_{\infty}^{\mu}) @.@ \mu(\roll(\bind_g(\unroll(d)))) \\
     \text{where } g  & = & [\;\bot \\
                      &   & ,\;\dlambda d @.@ \unroll(\pi_i(d))  \quad (\text{case for constr. } K) \\
                      &   & ,\;\bot \\
                      &   & ,\;\ldots\\
                      &   & ,\;\bot\; ]
  \end{array}\]}

  \item The other case is when $@B@ = \CF$. {\bf TODO}
\end{itemize}