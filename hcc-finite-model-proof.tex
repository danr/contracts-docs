

We now present the proof of Theorem~\ref{thm:finite-model}.
Consider the alternative presentation of $\SDownarrow$ below.
%% \begin{figure}\small
\[\begin{array}{c}
\ruleform{P |- u \SDownarrow v} \\ \\
\prooftree
\begin{array}{c}
P |- e_1 \SDownarrow f^\ar[\taus]\;\oln{e}{\ar-1} \quad
(f |-> \Lambda\as @.@ \oln{x{:}\tau}{\ar} @.@ u) \in P \\
P |- u[\ol{e},e_2/\xs] \SDownarrow w \quad \highlight{P |- e_2 \SDownarrow v_2}
\end{array}
-------------------------------------{EValF}
P |- e_1\;e_2 \SDownarrow w 
~~~~~
 \ar \geq 1
----------------------------{FVal}
P |- f^\ar[\taus] \SDownarrow f 
~~~~ 
(f{|->}\Lambda\as @.@ u) \in P \quad P |- u{\SDownarrow}v
-------------------------------------{FCaf}
P |- f^0[\taus] \SDownarrow v 
~~~~~
\begin{array}{c}
P |- e_1 \SDownarrow f^\ar[\taus]\;\oln{e}{< \ar-1} \quad
\highlight{P |- e_2 \SDownarrow v_2 }
\end{array}
-------------------------------------{EValP}
P |- e_1\;e_2 \SDownarrow f^\ar[\taus]\;\oln{e}{< \ar-1}\;e_2
~~~~~
\begin{array}{c} 
P |- e_1 \SDownarrow @BAD@ \quad
\highlight{P |- e_2 \SDownarrow v_2}
\end{array}
------------------------------------------------{EBadApp}
P |- e_1\;e_2 \SDownarrow @BAD@
~~~~~ 
P |- \highlight{\ol{e} \SDownarrow \ol{v}}
-------------------------------------{EValC}
P |- K[\taus](\ol{e}) \SDownarrow K[\taus](\ol{e})
~~~~
\phantom{G}
-------------------------------------{EValB}
P |- @BAD@ \SDownarrow @BAD@
\endprooftree \\ \\ 
\text{ ... plus rules for @case@ ... } 
\end{array}\]
%% \caption{Semi-strict operational semantics}\label{fig:opsem-semi}
%% \end{figure}

Let us consider a derivation tree $D_{0} @::@ P e \SDownarrow v$ and let us consider the set:
\[ S_0 = \{ e' \mid \exists D @.@ D @::@ P |- e'{\SDownarrow}w \text{ and } D{\sqsubseteq}D_0 \} \cup \{ @BAD@ \} \]
where with the notation $D_1 \sqsubseteq D_2$ we mean that $D_1$ is a sub-derivation of 
the derivation $D_2$. 

Let us create the set $S^{min}$ as follows:
\[         S^{min} = S_0 \cup \{ \bot \} \] 
where $\bot$ is a distinguished element. We refer to the elements of $S^{min}$ as $\mu$ (either and $e$ or $\bot$).
%% The edges of the graph are created with the following 
%% two rules:
%% \begin{itemize*}
%%    \item For every derivation rule of the form 
%%          \[\begin{array}{c}\prooftree
%%                D_1 @::@ e_1 \SDownarrow v_1 \ldots D_n @::@ e_1 \SDownarrow v_1
%%                ---------------------------------------{}
%%                D @::@ e \SDownarrow v  
%%          \endprooftree\end{array}\] 
%%          in $D_{0}$ then we add edges that connect $D$ to each of $e_1 \ldots e_n$.
%%    \item For every application node $D @::@ e_1\;e_2$ we add a {\em directed edge} from 
%%          the node to the node of $e_2$ (which, by the evaluation relation $\SDownarrow$ must also exist
%%          in the graph and by determinacy of $\SDownarrow$ is unique).
%%    \item For every data constructor node $D @::@ K[\taus](\ol{e})$ we add {\em labelled directed edges}
%%          from the node to the node of each $e_i \in \ol{e}$. The labels on the edges are just the indices.
%% \end{itemize*}

Consider the following equivalence relation on elements of $S^{min}$: 
\[\begin{array}{c}\prooftree
            P |- e \SDownarrow v
      ---------------------------{EqEval}
             e \equiv v 
      ~~~~ 
      \begin{array}{c}
           e_1 \equiv e_1' \quad e_2 \equiv e_2' 
      \end{array}
      ---------------------------------------------------{AppCong}
          e_1\;e_2 \equiv e_1' e_2' 
      ~~~~~ 
           e_i \equiv e_i'
      ---------------------------------------------------{ConCong}
           K[\taus](\ol{e}) \equiv K[\taus'](\ol{e}')
      ~~~~~
         \phantom{e}
      ---------------------------------------------------{Refl}
        \mu \equiv \mu
      ~~~~ 
        e_1 \equiv e_2
      ------------------------{Sym}
        e_2 \equiv e_1
      ~~~~\hspace{-5pt}
        e_1 \equiv e_2 \;\; e_2 \equiv e_3
      \hspace{-2pt}------------------------{Trans}
        e_1 \equiv e_3
\endprooftree\end{array}\]

The following lemma is true:
\begin{lemma}\label{lem:equiv-shapes} If $e_1 \equiv e_2$ then:
\begin{itemize*}
  \item If $P |- e_1 \SDownarrow f^\ar[\taus]\;\oln{e_1}{m}$ with $m < \ar$ then $P |- e_2 \SDownarrow f^\ar[\taus]\;\oln{e_2}{m}$ and $e_{1i} \equiv e_{2i}$.
  \item If $P |- e_1 \SDownarrow K[\taus](\ol{e_1})$ then $P |- e_2 \SDownarrow K[\taus'](\ol{e_2})$ and and $e_{1i} \equiv e_{2i}$.
\end{itemize*}
\end{lemma}
In what follows we will use the set $S^{min}$ as the carrier set of our first-order model, we will 
interpret equality as $\equiv$ and the first-order language of our signature as follows.


\[\setlength{\arraycolsep}{2pt}
\begin{array}{rcl}
   \mlinterp{f^{\ar}}(\mu_1,\ldots,\mu_n) & = & 
       \multicolumn{1}{l}{\text{If there exists } (e_1\;e_2) \in S^{min}} \\
   & & \multicolumn{1}{l}{\text{such that } e_1{\SDownarrow}f\;[\taus]\;\oln{e}{\ar-1}} \\
   & & \multicolumn{1}{l}{\text{and } \mu_i \equiv e_i} \\
   & & \multicolumn{1}{l}{\text{and } \mu_{\ar} \equiv e_2 \text{ then } (e_1\;e_2) \text{ else } \bot} \\
   \mlinterp{f_{ptr}} & = & \text{If there exists } f \in S^{min} \\
                        &   & \text{then } f \text{ else } \bot \\
  \mlinterp{app}(\mu_1,\mu_2) & = & \text{If there exists } (e_1\;e_2) \in S^{min} \\ 
                         &   & \text{such that } \mu_1 \equiv e_1 \\ 
                         &   & \text{and } \mu_2 \equiv e_2 \text{ then } (e_1\;e_2) \text{ else } \bot \\
  \mlinterp{K^\ar}(\mu_1,\ldots,\mu_\ar) & = & \text{If there exists } (K[\taus](\ol{e})) \in S^{min} \\
                                    &  & \text{such that } \mu_i' \equiv e_i \text{ then } (K[\taus](\ol{e})) \\
                                    &  & \text{else } \bot \\
  \mlinterp{\sel{K}{i}}(\mu) & = & \text{If there exists } (K[\taus](\ol{e})) \in S^{min} \\ 
                             &   & \text{such that } \mu \equiv (K[\taus](\ol{e})) \text{ then } e_i \\ 
                             &   & \text{else } \bot \\ 
%% \dapp(\dbrace{f},\oln{d}{\ar}) \\ 
%%    \linterp{app}(d_1,d_2)     & = & \dapp(d_1,d_2) \\
%%    \linterp{f_{ptr}}  & = & \dbrace{f} \\
%%    \linterp{K^{\ar}}(d_1,\ldots,d_\ar) & = & \roll(\ret(\inj{K}\langle d_1,\ldots,d_\ar\rangle)) \\ 
%%    \linterp{\sel{K}{i}}(d) & = &  \roll(\bind_g(\unroll(d))) \\ 
%%      \text{where } g  & = & [\;\bot \\ 
%%                       &   & ,\;\dlambda d @.@ \unroll(\pi_i(d))  \quad (\text{case for K}) \\ 
%%                       &   & ,\;\bot \\
%%                       &   & ,\;\ldots\\
%%                       &   & ,\;\bot\; ] \\
  \mlinterp{bad}       & = & @BAD@ \\
  \mlinterp{unr}       & = & \bot \\ \\ \\ 
  \mlinterp{min}(\mu)  & = & \mu \neq \bot \\ 
  \mlinterp{cf}(\mu)   & = & \text{There exists } e \text{ such that } \mu = e \\ 
                       &   & \text{and } \dbrace{e} \in F_\lcfZ^{\infty} 
\end{array}\]

First of all, we have to prove that $\mlinterp{\cdot}$ above is a function and not a relation. 
This is easy to do by observing that the interpretation can only return 
different terms that are nevertheless equated by $\equiv$. Secondly, we must prove that the interpreted 
functions are congruent over $\equiv$. That is also straightforward.

\begin{lemma}\label{lem:min-interp}
If $e \in S^{min}$ and $\etrans{}{\cdot}{e} = t$ then $\mlinterp{t} \equiv e$.
\end{lemma}
\begin{proof} Straightforward induction over the structure of $e$ and by observing 
that in any term $e \in S^{min}$, all structural subterms of $e$ are also in $S^{min}$.
\end{proof}

Moreover, we can show that $\langle S^{min},{\cal I}^{min}\rangle$ is a model of $\ThMin$, by showing that
it validates all the axioms. 
\begin{theorem}
$\langle S^{min},{\cal I}^{min}\rangle \models \ThMin$.
\end{theorem}
\begin{proof} We have to show that each of the axioms in $\ThMin$ is valid in this model. The only interesting 
axioms is \rulename{AxAppMin}:
\[ \forall \ol{x} @.@ min(app(f_{ptr},\xs)) => f(\ol{x}) = app(f_{ptr},\xs) \]
Let us pick elements $\mu_1\ldots\mu_n \in S^{min}$. Since $min(app(f_{ptr},\xs))$ it must be that they are actually
all expressions, $e_1\ldots e_n$. Moreover we have the following chain for the left-hand-side: 
\[\begin{array}{lcl}
     f & \equiv & e_1^{\star} \\ 
     e^{\star}_1\;e_1 & \equiv & e^{\star}_2 \\ 
     e^{\star}_2\;e_2 & \equiv & e^{\star}_3 \\ 
              & \ldots & 
\end{array}\] 
Such that the left-hand-side is $e^{\star}_n\;e_n$. Now, by Lemma~\ref{lem:equiv-shapes} 
we have that $e^{\star}_1 \SDownarrow f$, $e^{\star}_2 \SDownarrow f\;e_1'$ for some $e_1 \equiv e_1'$ and so on, 
so that eventually we have that $e^{\star}_n \SDownarrow f\;e_1'\ldots e_{n-1}'$ for equivalent $e_i \equiv e_i'$. 
Applying the interpretation of $f$ for the right-hand-side finishes the case.
\end{proof} 
