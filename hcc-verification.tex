We have established in the previous section the soundness of our translation, so it is time 
we see in this section how we can use this sound translation to verify a program. 
The following theorem is then true:

\begin{theorem}[Soundness]\label{thm:prover-soundness}
Assume that $e$ and $\Ct$ contain only function symbols from $P$ and no free term variables.
Let $\Th = \Th_\infty\;\land\;\Th_\lcfZ\;\land\;\ptrans{}{P}$. 
If $\Th \land \neg\ctrans{\Sigma}{P}{e \in \Ct}$ is unsatisfiable 
then $\langle D_\infty,{\cal I}\rangle \models \ctrans{\Sigma}{P}{e \in \Ct}$ and 
consequently $\dbrace{\Ct}(\dbrace{e})$.
\end{theorem}
\begin{proof}
If there is no model for this formula then its negation must be valid (true in all models), that 
is $ \neg \Th \lor \ctrans{\Sigma}{P}{e \in \Ct}$ is valid. By completeness
of first-order logic $\Th |- \ctrans{\Sigma}{P}{e \in \Ct}$. This means 
that all models of $\Th$ validate $\ctrans{\Sigma}{P}{f \in \Ct}$. In particular, 
for the denotational model we have that $\langle D_\infty,{\cal I}\rangle \models \Th$ 
and hence $\langle D_\infty,{\cal I} \rangle \models \ctrans{\Sigma}{P}{e \in \Ct}$. 
Theorem~\ref{thm:den-contr-satisfaction} finishes the proof.
\end{proof}

Hence, to verify a program $e$ satisfies a contract $\Ct$ we need to do the following:
\begin{itemize}
  \item Generate formulae for the theory $\Th_\infty\;\land\;\Th_\lcfZ\;\land\;\ptrans{}{P}$
  \item Generate the negation of a contract translation: $\neg\ctrans{\Sigma}{P}{e \in \Ct}$
  \item Ask a SAT solver for a model for the conjunction of the above formulae
\end{itemize}

\paragraph{Incremental verification}

Theorem~\ref{thm:prover-soundness} gives us a way to check that an expression satisfies a 
contract. Assume that we are given a program $P$ with a function $f \in dom(P)$, for which 
we have already showed that $\langle D_\infty,{\cal I}\rangle \models \ctrans{\Sigma}{P}{f \in \Ct_f}$.
Suppose next that we are presented with a ``next'' goal, to prove that 
$\langle D_\infty,{\cal I}\rangle \models \ctrans{\Sigma}{P}{h \in \Ct}$. 
We may consider the following three variations of how to do this:

\begin{itemize}
  \item Ask for the unsatisfiability of: 
    \[  \Th_\infty \land \Th_\lcfZ \land \dtrans{}{P} \land \neg\ctrans{\Sigma}{P}{h \in \Ct_h} \]
        The soundness of this query follows from Theorem~\ref{thm:prover-soundness} above.

  \item Ask for the unsatisfiability of:
    \[  \Th_\infty \land \Th_\lcfZ \land 
        \dtrans{\Sigma}{P}\;\land\;\ctrans{\Sigma}{P}{f \in \Ct_f} \land \neg \ctrans{\Sigma}{P}{h \in \Ct_h} \]
        This query adds the {\em already proven} contract for $f$ in the theory. If this formula
        is unsatisfiable, then its negation is valid, and we know that the denotational model is 
        a model of the theory {\em and} of $\ctrans{\Sigma}{P}{f \in \Ct_f}$ and hence it must also
        be a model of $\ctrans{\Sigma}{P}{h \in \Ct_h}$.
  \item Ask for the unsatisfiability of:
    \[  \Th_\infty \land \Th_\lcfZ \land 
        \dtrans{\Sigma}{P \setminus f}\land\ctrans{\Sigma}{P}{f \in \Ct_f}\land
        \neg\ctrans{\Sigma}{P}{h \in \Ct_h} \]
        This query removes the axioms associated with the definition of $f$ since we may only have 
        its signature and contract available. Via a similar reasoning as before, such an invocation 
        is sound as well.
\end{itemize}


%% Our final goal is going to show that a program does not crash, that
%% is the final contract will be of the form $e \in \Ct$ where $\Ct$ is
%% going to be some {\em base contract}. Note that by base contract adequacy
%% (Lemma~\ref{lem:base-contract-adequacy}) if we manage to show a base contract 
%% denotationally, then the contract holds in operational terms.

%% \clearpage
